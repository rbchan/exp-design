\documentclass[color=usenames,dvipsnames]{beamer}\usepackage[]{graphicx}\usepackage[]{color}
% maxwidth is the original width if it is less than linewidth
% otherwise use linewidth (to make sure the graphics do not exceed the margin)
\makeatletter
\def\maxwidth{ %
  \ifdim\Gin@nat@width>\linewidth
    \linewidth
  \else
    \Gin@nat@width
  \fi
}
\makeatother

\definecolor{fgcolor}{rgb}{0, 0, 0}
\newcommand{\hlnum}[1]{\textcolor[rgb]{0.69,0.494,0}{#1}}%
\newcommand{\hlstr}[1]{\textcolor[rgb]{0.749,0.012,0.012}{#1}}%
\newcommand{\hlcom}[1]{\textcolor[rgb]{0.514,0.506,0.514}{\textit{#1}}}%
\newcommand{\hlopt}[1]{\textcolor[rgb]{0,0,0}{#1}}%
\newcommand{\hlstd}[1]{\textcolor[rgb]{0,0,0}{#1}}%
\newcommand{\hlkwa}[1]{\textcolor[rgb]{0,0,0}{\textbf{#1}}}%
\newcommand{\hlkwb}[1]{\textcolor[rgb]{0,0.341,0.682}{#1}}%
\newcommand{\hlkwc}[1]{\textcolor[rgb]{0,0,0}{\textbf{#1}}}%
\newcommand{\hlkwd}[1]{\textcolor[rgb]{0.004,0.004,0.506}{#1}}%
\let\hlipl\hlkwb

\usepackage{framed}
\makeatletter
\newenvironment{kframe}{%
 \def\at@end@of@kframe{}%
 \ifinner\ifhmode%
  \def\at@end@of@kframe{\end{minipage}}%
  \begin{minipage}{\columnwidth}%
 \fi\fi%
 \def\FrameCommand##1{\hskip\@totalleftmargin \hskip-\fboxsep
 \colorbox{shadecolor}{##1}\hskip-\fboxsep
     % There is no \\@totalrightmargin, so:
     \hskip-\linewidth \hskip-\@totalleftmargin \hskip\columnwidth}%
 \MakeFramed {\advance\hsize-\width
   \@totalleftmargin\z@ \linewidth\hsize
   \@setminipage}}%
 {\par\unskip\endMakeFramed%
 \at@end@of@kframe}
\makeatother

\definecolor{shadecolor}{rgb}{.97, .97, .97}
\definecolor{messagecolor}{rgb}{0, 0, 0}
\definecolor{warningcolor}{rgb}{1, 0, 1}
\definecolor{errorcolor}{rgb}{1, 0, 0}
\newenvironment{knitrout}{}{} % an empty environment to be redefined in TeX

\usepackage{alltt}
%\documentclass[color=usenames,dvipsnames,handout]{beamer}

\usepackage[sans]{../../lab1}

\hypersetup{pdftex,pdfstartview=FitV}









%% New command for inline code that isn't to be evaluated
\definecolor{inlinecolor}{rgb}{0.878, 0.918, 0.933}
\newcommand{\inr}[1]{\colorbox{inlinecolor}{\texttt{#1}}}
\IfFileExists{upquote.sty}{\usepackage{upquote}}{}
\begin{document}



\begin{frame}[plain]
  \begin{center}
    \Large
    {\color{NavyBlue}{\huge \bf Power Analysis \\}}
    \vspace{2cm}
    \large
    { September 4, 2019} \par
    \vspace{1cm}
    \large
    {FANR 6750: Experimental Methods in Natural Resources
      Research}  \par
  \end{center}
\end{frame}



\section{Motivation}



\begin{frame}[plain]
  \frametitle{Outline}
   \LARGE
   \only<1>{\tableofcontents[hideallsubsections]}
%   \only<2 | handout:0>{\tableofcontents[currentsection,hideallsubsections]}
\end{frame}



\begin{frame}
  \frametitle{Motivation}
  \large
  \begin{quote}
    A statistical test will not be able to detect a true difference if
    the sample size is too small compared with the magnitude of the
    difference.
  \end{quote}
  \uncover<2->{
  \begin{quote}
    Since data are sampled at random, there is always a risk of
    reaching a wrong conclusion, and things can go wrong in two ways.
  \end{quote}
  }
  \uncover<1->{\flushright \footnotesize Dalgaard (2008) \par}
\end{frame}




\section{Type I \& II errors}


\begin{frame}
  \frametitle{Type I \& type II errors}
  {\bf Type I error:} The null hypothesis is correct, but the test rejects it.
  \pause
%  \vfill
  \[
    \alpha = \text{Pr(Type I error)} %= \text{Pr(rejecting a true null hypothesis)}
  \]
  \pause
  \vfill
  {\bf Type II error:} The null hypothesis is wrong, but the test fails to reject it.
%  \pause
%  \[
%    \text{Pr(Type II error)} = \text{Pr(not rejecting a false null hypothesis)}
%  \]
  \pause
  \vfill
  {\bf Power} ($\beta$): The test's ability to reject a null hypothesis that is false.
%  \vfill
  \[
    \beta = 1 - \text{Pr(Type II error)} %The probability of rejecting a
%      false hypothesis.}
  \]
\end{frame}




\begin{frame}
  \frametitle{Type I \& type II errors}
  {%\bf
    The \alert{type I error rate} is set by the scientist.} \par
  \pause
  \vfill
  {%\bf
    The \alert{type II error rate}, and hence the power of the test, depend
    on many factors.}
  \pause
  \vfill
  {%\bf
    In the context of a two-sample $t$ test, power goes up when:} %these factors are:}
%  \begin{enumerate}[\bf (1)]
  \begin{itemize}[<+->]%[\bf (1)]
    \item The magnitude of the difference ($\delta$) increases
    \item The standard deviation (or variance) of the population
      ($\sigma$) decreases
    \item The sample size ($n$) increases
    \item The Type I error rate ($\alpha$) increases
  \end{itemize}
%  \end{enumerate}
\end{frame}




\section{Two-sample $t$ test}



\begin{frame}[fragile]
  \frametitle{Magnitude of the difference ($\delta$)}


\only<1>{\includegraphics[width=\textwidth]{figure/delta1-1}}
%\only<2>{\includegraphics[width=\textwidth]{power_analysis-delta2}}
\end{frame}




\begin{frame}[fragile]
  \frametitle{Magnitude of the difference ($\delta$)}
\includegraphics[width=\textwidth]{figure/delta2-1}
\end{frame}








\begin{frame}[fragile]
  \frametitle{Standard deviation of the population ($\sigma$)}


\only<1>{\includegraphics[width=\textwidth]{figure/sigma1-1}}
%\only<2>{\includegraphics[width=\textwidth]{power_analysis-sigma2}}
\end{frame}




\begin{frame}[fragile]
  \frametitle{Standard deviation of the population ($\sigma$)}
%\only<1>{\includegraphics[width=\textwidth]{power_analysis-sigma1}}
\includegraphics[width=\textwidth]{figure/sigma2-1}
\end{frame}











\begin{frame}[fragile]
  \frametitle{Sample size ($n$)}


\only<1>{\includegraphics[width=\textwidth]{figure/n1-1}}
%\only<2>{\includegraphics[width=\textwidth]{power_analysis-n2}}
\end{frame}







\begin{frame}[fragile]
  \frametitle{Sample size ($n$)}
  \includegraphics[width=\textwidth]{figure/n2-1}
\end{frame}






\begin{frame}[fragile]
  \frametitle{Type I error rate}


\only<1>{\includegraphics[width=\textwidth]{figure/alpha1-1}}
%\only<2>{\includegraphics[width=\textwidth]{power_analysis-alpha2}}
\end{frame}





\begin{frame}[fragile]
  \frametitle{Type I error rate}
  \includegraphics[width=\textwidth]{figure/alpha2-1}
\end{frame}








% \begin{frame}
%   \frametitle{Factors affecting power}
%   \large
%   {%\bf
%     In a two-sample $t$ test, power increases when:}
%   \begin{enumerate}[\bf (1)]
%     \item The difference in means increases
%     \item The standard deviation of the population decreases
%     \item The sample size increases
%     \item The Type I error rate increases
%   \end{enumerate}
% %  \pause
% %  {\bf How can one know power ahead of time?}
% \end{frame}



\begin{frame}[fragile]
  \frametitle{Example in {\bf R}}
  \small
\begin{knitrout}
\definecolor{shadecolor}{rgb}{0.878, 0.918, 0.933}\color{fgcolor}\begin{kframe}
\begin{alltt}
\hlkwd{power.t.test}\hlstd{(}\hlkwc{n}\hlstd{=}\hlnum{5}\hlstd{,} \hlkwc{delta}\hlstd{=}\hlnum{10}\hlstd{,} \hlkwc{sd}\hlstd{=}\hlnum{5}\hlstd{,}
             \hlkwc{sig.level}\hlstd{=}\hlnum{0.05}\hlstd{,} \hlkwc{power}\hlstd{=}\hlkwa{NULL}\hlstd{)}
\end{alltt}
\begin{verbatim}
## 
##      Two-sample t test power calculation 
## 
##               n = 5
##           delta = 10
##              sd = 5
##       sig.level = 0.05
##           power = 0.7905416
##     alternative = two.sided
## 
## NOTE: n is number in *each* group
\end{verbatim}
\end{kframe}
\end{knitrout}
\end{frame}






%\section{Prospective vs retrospective}



\begin{frame}
  \frametitle{When should I do a power analysis?}
  {%\bf
    \centering \large
    \alert{\bf Prospective power analysis} is better than
    \alert{\bf retrospective power analysis} \\}
  \pause
  \vfill
  {\bf
    Retrospective}
  \begin{itemize}
    \item Conducted after experiment
    \item If you failed to reject the null, then your power was likely low
    \item But you can't use this as an excuse!
    \item Only useful as a way of planning a subsequent experiment
  \end{itemize}
  \pause
  \vfill
  {\bf
    Prospective}
  \begin{itemize}
    \item Done before the experiment
    \item Used to determine sample size or power, given $\delta$ and $\sigma$
    \item How can $\delta$ and/or $\sigma$ be known ahead of time?
    \item Requires prior knowledge, perhaps from a pilot study, and
      careful thought about what constitutes a biologically significant
      difference. 
    % \item Requires clear-headed thinking about what consitutes a
    %   biologically significant difference.
  \end{itemize}
\end{frame}



\begin{frame}
  \frametitle{What level of power should I aim for?}
  \large
  {%\bf
    We want power to be as close to 1 as possible.} \par
  \pause
  \vfill %\vspace{1cm}
  {%\bf
    Sometimes it may be prohibitively expensive to obtain a sample
    size large enough to achieve power close to 1.} \par
  \pause
  \vfill %\vspace{1cm}
  {%\bf
    In practice, we are usually satisfied with power $>$0.8. }
\end{frame}




\section{ANOVA}


\begin{frame}
  \frametitle{One-way ANOVA}
%  \large
  {%\bf
    To conduct a power analysis in the context of a one-way ANOVA,
    we need:}
  \begin{itemize}
    \item The among group variance (MSa) instead of the difference in means, and
    \item The within group variance (MSe) instead of the standard deviation
      of the population
  \end{itemize}
  \pause
  \vfill
  {%\bf
    Power goes up when:}
  \begin{itemize}
    \item MSa increases
    \item MSe decreases
    \item Same rules about $n$ and $\alpha$ from before also apply
  \end{itemize}
\end{frame}




\begin{frame}[fragile]
  \frametitle{Example in {\bf R}}
\begin{knitrout}\footnotesize
\definecolor{shadecolor}{rgb}{0.878, 0.918, 0.933}\color{fgcolor}\begin{kframe}
\begin{alltt}
\hlkwd{power.anova.test}\hlstd{(}\hlkwc{groups}\hlstd{=}\hlnum{4}\hlstd{,} \hlkwc{n}\hlstd{=}\hlnum{5}\hlstd{,} \hlkwc{between.var}\hlstd{=}\hlnum{101}\hlstd{,}
                 \hlkwc{within.var}\hlstd{=}\hlnum{20}\hlstd{,} \hlkwc{sig.level}\hlstd{=}\hlnum{0.05}\hlstd{,} \hlkwc{power}\hlstd{=}\hlkwa{NULL}\hlstd{)}
\end{alltt}
\begin{verbatim}
## 
##      Balanced one-way analysis of variance power calculation 
## 
##          groups = 4
##               n = 5
##     between.var = 101
##      within.var = 20
##       sig.level = 0.05
##           power = 0.9999997
## 
## NOTE: n is number in each group
\end{verbatim}
\end{kframe}
\end{knitrout}
\end{frame}








\begin{frame}
  \frametitle{Summary}
%  \begin{itemize}[<+->]
%  \item
  Power analysis let's you determine the necessary sample size
  (or power) for testing an effect size of interest
  \pause
  \vfill
  % \item
  Power is influenced by the magnitude of the effect, the
  standard deviation of the population, the Type I error rate, and
  the sample size
  \pause
  \vfill
  % \item
  Retrospective power analysis isn't useful unless you are
  planning a subsequent experiment
  \pause
  \vfill
  % \item
  {\bf R} has several functions for conducting power analysis,
  but only for simple tests
  % \item
  \pause
  \vfill
  More complicated power analysis can be performed using
  simulation (not covered in this course)
%  \end{itemize}
\end{frame}



\end{document}
