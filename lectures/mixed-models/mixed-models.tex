\documentclass[color=usenames,dvipsnames]{beamer}\usepackage[]{graphicx}\usepackage[]{color}
%% maxwidth is the original width if it is less than linewidth
%% otherwise use linewidth (to make sure the graphics do not exceed the margin)
\makeatletter
\def\maxwidth{ %
  \ifdim\Gin@nat@width>\linewidth
    \linewidth
  \else
    \Gin@nat@width
  \fi
}
\makeatother

\definecolor{fgcolor}{rgb}{0, 0, 0}
\newcommand{\hlnum}[1]{\textcolor[rgb]{0.69,0.494,0}{#1}}%
\newcommand{\hlstr}[1]{\textcolor[rgb]{0.749,0.012,0.012}{#1}}%
\newcommand{\hlcom}[1]{\textcolor[rgb]{0.514,0.506,0.514}{\textit{#1}}}%
\newcommand{\hlopt}[1]{\textcolor[rgb]{0,0,0}{#1}}%
\newcommand{\hlstd}[1]{\textcolor[rgb]{0,0,0}{#1}}%
\newcommand{\hlkwa}[1]{\textcolor[rgb]{0,0,0}{\textbf{#1}}}%
\newcommand{\hlkwb}[1]{\textcolor[rgb]{0,0.341,0.682}{#1}}%
\newcommand{\hlkwc}[1]{\textcolor[rgb]{0,0,0}{\textbf{#1}}}%
\newcommand{\hlkwd}[1]{\textcolor[rgb]{0.004,0.004,0.506}{#1}}%
\let\hlipl\hlkwb

\usepackage{framed}
\makeatletter
\newenvironment{kframe}{%
 \def\at@end@of@kframe{}%
 \ifinner\ifhmode%
  \def\at@end@of@kframe{\end{minipage}}%
  \begin{minipage}{\columnwidth}%
 \fi\fi%
 \def\FrameCommand##1{\hskip\@totalleftmargin \hskip-\fboxsep
 \colorbox{shadecolor}{##1}\hskip-\fboxsep
     % There is no \\@totalrightmargin, so:
     \hskip-\linewidth \hskip-\@totalleftmargin \hskip\columnwidth}%
 \MakeFramed {\advance\hsize-\width
   \@totalleftmargin\z@ \linewidth\hsize
   \@setminipage}}%
 {\par\unskip\endMakeFramed%
 \at@end@of@kframe}
\makeatother

\definecolor{shadecolor}{rgb}{.97, .97, .97}
\definecolor{messagecolor}{rgb}{0, 0, 0}
\definecolor{warningcolor}{rgb}{1, 0, 1}
\definecolor{errorcolor}{rgb}{1, 0, 0}
\newenvironment{knitrout}{}{} % an empty environment to be redefined in TeX

\usepackage{alltt}
%\documentclass[color=usenames,dvipsnames,handout]{beamer}




\usepackage[sans]{../../lab1}
\usepackage{bm}


\hypersetup{pdftex,pdfstartview=FitV}









%% New command for inline code that isn't to be evaluated
\definecolor{inlinecolor}{rgb}{0.878, 0.918, 0.933}
\newcommand{\inr}[1]{\colorbox{inlinecolor}{\texttt{#1}}}
\IfFileExists{upquote.sty}{\usepackage{upquote}}{}
\begin{document}




\begin{frame}[plain]
  \centering \huge
  Linear mixed-effects models \\
  \vspace{1cm}
  \Large
  November 7, 2018 \\
  FANR 6750 \\
  \vfill
  \large
  Richard Chandler and Bob Cooper
\end{frame}


\section{Introduction}



\begin{frame}
  \frametitle{Fixed vs random effects, revisited}
%  \large
  {\bf In the context of mixed-effects models\dots \par}
  \vspace{0.5cm}
{\bf Fixed effect \par}
A parameter that would not change if you repeated your study

\pause
\vspace{1cm}

{\bf Random effect \par}
A parameter that would change if you repeated your study

\pause
\vfill
{\centering \bf
\alert{Random effects will always be associated with a probability
distribution} \par}
\end{frame}



\begin{frame}
  \frametitle{Previous examples}
  \Large
%  \begin{itemize}
%    \item
  ANOVA with random block effects
  \pause
  \vfill
%    \item[]
%    \item
  Nested ANOVA
  \pause
  \vfill
%    \item[]
%    \item
  Split-plot ANOVA
%    \item[]
%    \item Repeated measures ANOVA
%  \end{itemize}
\end{frame}




\begin{frame}
  \frametitle{A simple example}
  \large
  \[
    y_{ij} = \beta_0 + \beta_1 x_{i1} + \gamma_j + \varepsilon_{ij}
  \]
  where
  \[
    \gamma_j \sim \mbox{Normal}(0, \sigma^2_d)
  \]
  and
  \[
    \varepsilon_{ij} \sim \mbox{Normal}(0, \sigma^2)
  \]
  \pause
  \vfill
  \centering
  This amounts to a linear regression with a random intercept \\
\end{frame}






\section{Repeated measures with serial correlation}



%% \begin{frame}
%%   \frametitle{Split-plot model}
%%   The expanded additive model:
%%   \[
%%     y_{ijk} = \beta0 + \beta_1 + \beta_{ij} + \gamma_k +
%%     \alpha\gamma_{ik} + \varphi y_{ijk-1} + \varepsilon_{ijk}
%%   \]
%% \end{frame}



\begin{frame}
  \centering \Huge
  \color{PineGreen}{Mixed effects approach to repeated measures} \par
\end{frame}




\begin{frame}[fragile]
  \frametitle{Repeated measures revisited}
  {\bf Recall the plant data}
  \small
\begin{knitrout}
\definecolor{shadecolor}{rgb}{0.878, 0.918, 0.933}\color{fgcolor}\begin{kframe}
\begin{alltt}
\hlstd{plantData} \hlkwb{<-} \hlkwd{read.csv}\hlstd{(}\hlstr{"plantData.csv"}\hlstd{)}
\hlstd{plantData}\hlopt{$}\hlstd{week} \hlkwb{<-} \hlkwd{factor}\hlstd{(plantData}\hlopt{$}\hlstd{week)}
\hlkwd{head}\hlstd{(plantData)}
\end{alltt}
\begin{verbatim}
##   plant fertilizer week leaves
## 1     1          L    1      4
## 2     1          L    2      5
## 3     1          L    3      6
## 4     1          L    4      8
## 5     1          L    5     10
## 6     2          L    1      3
\end{verbatim}
\end{kframe}
\end{knitrout}
\pause
  {\bf \normalsize Standard split-plot approach}
\begin{knitrout}
\definecolor{shadecolor}{rgb}{0.878, 0.918, 0.933}\color{fgcolor}\begin{kframe}
\begin{alltt}
\hlkwd{library}\hlstd{(nlme)}
\hlstd{lme1} \hlkwb{<-} \hlkwd{lme}\hlstd{(leaves} \hlopt{~} \hlstd{fertilizer}\hlopt{*}\hlstd{week, plantData,}
            \hlkwc{random}\hlstd{=}\hlopt{~}\hlnum{1}\hlopt{|}\hlstd{plant)}
\end{alltt}
\end{kframe}
\end{knitrout}
\end{frame}






\begin{frame}[fragile]
  \frametitle{View the parameter estimates}

\small
\begin{knitrout}\scriptsize
\definecolor{shadecolor}{rgb}{0.878, 0.918, 0.933}\color{fgcolor}\begin{kframe}
\begin{alltt}
\hlkwd{summary}\hlstd{(lme1)}
\end{alltt}
\begin{verbatim}
## Linear mixed-effects model fit by REML
##  Data: plantData
##        AIC      BIC    logLik
##   140.9644 161.2309 -58.48219
##
## Random effects:
##  Formula: ~1 | plant
##         (Intercept) Residual
## StdDev:    1.098863     0.65
##
## Fixed effects: leaves ~ fertilizer * week
##                   Value Std.Error DF   t-value p-value
## (Intercept)         4.6 0.5709641 32  8.056549  0.0000
## fertilizerL         0.0 0.8074652  8  0.000000  1.0000
## week2               2.2 0.4110961 32  5.351547  0.0000
## week3               4.4 0.4110961 32 10.703094  0.0000
## week4               5.4 0.4110961 32 13.135615  0.0000
## week5               7.4 0.4110961 32 18.000657  0.0000
## fertilizerL:week2  -1.0 0.5813777 32 -1.720052  0.0951
## fertilizerL:week3  -1.8 0.5813777 32 -3.096094  0.0041
## fertilizerL:week4  -1.6 0.5813777 32 -2.752084  0.0097
## fertilizerL:week5  -1.4 0.5813777 32 -2.408073  0.0220
\end{verbatim}
\end{kframe}
\end{knitrout}
\end{frame}








\begin{frame}
  \frametitle{Matrix representation}
  \Large
  \[
    {\bf y} = {\bf X}{\bm \beta} + {\bm \gamma} + {\bm \varepsilon}
  \]
  \pause
  \[
    {\bm \gamma} \sim \mbox{Normal}(0, \sigma^2_d)
  \]
  \[
    {\bm \varepsilon} \sim \mbox{Normal}(0, \sigma^2)
  \]
\end{frame}







\begin{frame}[fragile]
  \frametitle{Extract the estimates}
  {\bf Fixed effects (the $\bm \beta$ estimates)}
\begin{knitrout}\tiny
\definecolor{shadecolor}{rgb}{0.878, 0.918, 0.933}\color{fgcolor}\begin{kframe}
\begin{alltt}
\hlkwd{round}\hlstd{(beta} \hlkwb{<-} \hlkwd{fixef}\hlstd{(lme1),} \hlnum{2}\hlstd{)}
\end{alltt}
\begin{verbatim}
##       (Intercept)       fertilizerL             week2             week3 
##               4.6               0.0               2.2               4.4 
##             week4             week5 fertilizerL:week2 fertilizerL:week3 
##               5.4               7.4              -1.0              -1.8 
## fertilizerL:week4 fertilizerL:week5 
##              -1.6              -1.4
\end{verbatim}
\end{kframe}
\end{knitrout}
\pause
\vfill
  {\bf \normalsize Random effects (the $\bm \gamma$ estimates)}
\begin{knitrout}\tiny
\definecolor{shadecolor}{rgb}{0.878, 0.918, 0.933}\color{fgcolor}\begin{kframe}
\begin{alltt}
\hlkwd{round}\hlstd{(gamma} \hlkwb{<-} \hlkwd{ranef}\hlstd{(lme1,} \hlkwc{which}\hlstd{=}\hlnum{1}\hlstd{),} \hlnum{2}\hlstd{)}
\end{alltt}
\begin{verbatim}
##    (Intercept)
## 1        -0.67
## 2        -1.61
## 3         1.38
## 4         1.01
## 5        -0.11
## 6        -0.64
## 7        -1.01
## 8         1.23
## 9         0.11
## 10        0.30
\end{verbatim}
\end{kframe}
\end{knitrout}
\end{frame}



\begin{frame}[fragile]
  \frametitle{Computing the expected values}
  {\bf Ignoring the random effects}
\begin{knitrout}\scriptsize
\definecolor{shadecolor}{rgb}{0.878, 0.918, 0.933}\color{fgcolor}\begin{kframe}
\begin{alltt}
\hlstd{X} \hlkwb{<-} \hlkwd{model.matrix}\hlstd{(}\hlopt{~} \hlstd{fertilizer}\hlopt{*}\hlstd{week, plantData)}
\hlstd{E1} \hlkwb{<-} \hlstd{X} \hlopt \hlstd{beta}
\end{alltt}
\end{kframe}
\end{knitrout}
\pause
  {\bf Including the random effects}
\begin{knitrout}\scriptsize
\definecolor{shadecolor}{rgb}{0.878, 0.918, 0.933}\color{fgcolor}\begin{kframe}
\begin{alltt}
\hlstd{nWeeks} \hlkwb{<-} \hlnum{5}
\hlstd{gamma.long} \hlkwb{<-} \hlkwd{rep}\hlstd{(gamma[[}\hlnum{1}\hlstd{]],} \hlkwc{each}\hlstd{=nWeeks)}
\hlstd{E2} \hlkwb{<-} \hlstd{E1} \hlopt{+} \hlstd{gamma.long}
\end{alltt}
\end{kframe}
\end{knitrout}
\pause
\small
\begin{knitrout}\scriptsize
\definecolor{shadecolor}{rgb}{0.878, 0.918, 0.933}\color{fgcolor}\begin{kframe}
\begin{alltt}
\hlkwd{head}\hlstd{(}\hlkwd{cbind}\hlstd{(plantData,} \hlkwc{E1}\hlstd{=E1,} \hlkwc{E2}\hlstd{=E2))}
\end{alltt}
\begin{verbatim}
##   plant fertilizer week leaves   E1       E2
## 1     1          L    1      4  4.6 3.927090
## 2     1          L    2      5  5.8 5.127090
## 3     1          L    3      6  7.2 6.527090
## 4     1          L    4      8  8.4 7.727090
## 5     1          L    5     10 10.6 9.927090
## 6     2          L    1      3  4.6 2.992492
\end{verbatim}
\end{kframe}
\end{knitrout}
\end{frame}




\begin{frame}[fragile]
  \frametitle{The ANOVA table}
\begin{knitrout}
\definecolor{shadecolor}{rgb}{0.878, 0.918, 0.933}\color{fgcolor}\begin{kframe}
\begin{alltt}
\hlkwd{anova}\hlstd{(lme1)}
\end{alltt}
\begin{verbatim}
##                 numDF denDF  F-value p-value
## (Intercept)         1    32 483.0495  <.0001
## fertilizer          1     8   2.6037  0.1453
## week                4    32 158.2249  <.0001
## fertilizer:week     4    32   3.0059  0.0326
\end{verbatim}
\end{kframe}
\end{knitrout}
\pause
\vfill
\Large
{\centering \bf \color{PineGreen}{Are these $P$-values valid?} \par}
\pause
\vfill
{\centering \bf Not if there is serial correlation \par}
\end{frame}





\begin{frame}
  \frametitle{Auto-regressive models}
  \bf
  An AR($p$) model allows for lag effects of order $p$ \par
  \vfill
  If $p=1$, we say the value of $y$ at time $k$ depends on the value
  of $y$ at time $k-1$. \par
  \pause
  \vfill
  \Large
  \[
    y_{ijk} = \beta_0 + \varphi_1 y_{ijk-1} + \gamma_j + \varepsilon_{ijk}
  \]
\end{frame}




\begin{frame}
  \frametitle{Auto-regressive models}
  \bf
  An AR($p$) model allows for lag effects of order $p$ \par
  \vfill
  If $p=2$, we say the value of $y$ at time $k$ depends on the value
  of $y$ at time $k-1$ and at time $k-2$. \par
  \vfill
  \pause
  \Large
  \[
    y_{ijk} = \beta_0 + \varphi_1 y_{ijk-1} + \varphi_2 y_{ijk-2} + \gamma_j + \varepsilon_{ijk}
  \]
\end{frame}






\begin{frame}[fragile]
  \frametitle{Auto-regressive models}
\vspace{0.5cm}
{\bf Assume an AR(1) correlation structure}
\pause
\begin{knitrout}
\definecolor{shadecolor}{rgb}{0.878, 0.918, 0.933}\color{fgcolor}\begin{kframe}
\begin{alltt}
\hlstd{lme2} \hlkwb{<-} \hlkwd{lme}\hlstd{(leaves} \hlopt{~} \hlstd{fertilizer}\hlopt{*}\hlstd{week, plantData,}
            \hlkwc{random}\hlstd{=}\hlopt{~}\hlnum{1}\hlopt{|}\hlstd{plant,}
            \hlkwc{correlation}\hlstd{=}\hlkwd{corARMA}\hlstd{(}\hlkwc{p}\hlstd{=}\hlnum{1}\hlstd{))}
\end{alltt}
\end{kframe}
\end{knitrout}
\vspace{1cm}
{\bf Assume an AR(3) correlation structure}
\pause
\begin{knitrout}
\definecolor{shadecolor}{rgb}{0.878, 0.918, 0.933}\color{fgcolor}\begin{kframe}
\begin{alltt}
\hlstd{lme3} \hlkwb{<-} \hlkwd{lme}\hlstd{(leaves} \hlopt{~} \hlstd{fertilizer}\hlopt{*}\hlstd{week, plantData,}
            \hlkwc{random}\hlstd{=}\hlopt{~}\hlnum{1}\hlopt{|}\hlstd{plant,}
            \hlkwc{correlation}\hlstd{=}\hlkwd{corARMA}\hlstd{(}\hlkwc{p}\hlstd{=}\hlnum{3}\hlstd{))}
\end{alltt}
\end{kframe}
\end{knitrout}
\end{frame}







\begin{frame}[fragile]
  \frametitle{Autocorrelation function}
\begin{knitrout}\small
\definecolor{shadecolor}{rgb}{0.878, 0.918, 0.933}\color{fgcolor}\begin{kframe}
\begin{alltt}
\hlkwd{plot}\hlstd{(}\hlkwd{ACF}\hlstd{(lme2),} \hlkwc{alpha}\hlstd{=}\hlnum{0.05}\hlstd{)}
\end{alltt}
\end{kframe}
\end{knitrout}
%\begin{center}
\centering
  \includegraphics[width=0.9\textwidth]{figure/plant2-acf-1} \\
%\end{center}
\end{frame}





\begin{frame}[fragile]
  \frametitle{Autocorrelation function}
\begin{knitrout}\small
\definecolor{shadecolor}{rgb}{0.878, 0.918, 0.933}\color{fgcolor}\begin{kframe}
\begin{alltt}
\hlkwd{plot}\hlstd{(}\hlkwd{ACF}\hlstd{(lme3),} \hlkwc{alpha}\hlstd{=}\hlnum{0.05}\hlstd{)}
\end{alltt}
\end{kframe}
\end{knitrout}
%\begin{center}
\centering
  \includegraphics[width=0.9\textwidth]{figure/plant3-acf-1} \\
%\end{center}
\end{frame}




\begin{frame}[fragile]
  \frametitle{Parameter estimates}
%<<summary-lme3,size='tiny'>>=
%summary(lme3)
%@
\begin{knitrout}
\definecolor{shadecolor}{rgb}{0.878, 0.918, 0.933}\color{fgcolor}\begin{kframe}
\begin{alltt}
\hlkwd{summary}\hlstd{(lme3)}
\end{alltt}
\tiny
\begin{verbatim}
## Linear mixed-effects model fit by REML
##  Data: plantData
##        AIC      BIC    logLik
##   142.7793 168.1125 -56.38965
##
## Random effects:
##  Formula: ~1 | plant
##         (Intercept)  Residual
## StdDev:    0.955924 0.8227638
##
## Correlation Structure: ARMA(3,0)
##  Formula: ~1 | plant
##  Parameter estimate(s):
##       Phi1       Phi2       Phi3
##  0.3339878  0.5086039 -0.2573006
## Fixed effects: leaves ~ fertilizer * week
##                   Value Std.Error DF   t-value p-value
## (Intercept)         4.6 0.5640445 32  8.155385  0.0000
## fertilizerL         0.0 0.7976794  8  0.000000  1.0000
## week2               2.2 0.4039417 32  5.446331  0.0000
## week3               4.4 0.3532792 32 12.454737  0.0000
## week4               5.4 0.4867909 32 11.093060  0.0000
## week5               7.4 0.4614437 32 16.036626  0.0000
## fertilizerL:week2  -1.0 0.5712598 32 -1.750517  0.0896
## fertilizerL:week3  -1.8 0.4996123 32 -3.602794  0.0011
## fertilizerL:week4  -1.6 0.6884262 32 -2.324142  0.0266
## fertilizerL:week5  -1.4 0.6525799 32 -2.145331  0.0396
\end{verbatim}
\end{kframe}
\end{knitrout}
\end{frame}




\begin{frame}[fragile]
  \frametitle{ANOVA tables}
  {\bf Model that ignores serial correlation}
\begin{knitrout}\scriptsize
\definecolor{shadecolor}{rgb}{0.878, 0.918, 0.933}\color{fgcolor}\begin{kframe}
\begin{alltt}
\hlkwd{anova}\hlstd{(lme1)}
\end{alltt}
\begin{verbatim}
##                 numDF denDF  F-value p-value
## (Intercept)         1    32 483.0495  <.0001
## fertilizer          1     8   2.6037  0.1453
## week                4    32 158.2249  <.0001
## fertilizer:week     4    32   3.0059  0.0326
\end{verbatim}
\end{kframe}
\end{knitrout}
  {\bf Model that accounts for serial correlation}
\begin{knitrout}\scriptsize
\definecolor{shadecolor}{rgb}{0.878, 0.918, 0.933}\color{fgcolor}\begin{kframe}
\begin{alltt}
\hlkwd{anova}\hlstd{(lme3)}
\end{alltt}
\begin{verbatim}
##                 numDF denDF  F-value p-value
## (Intercept)         1    32 502.8341  <.0001
## fertilizer          1     8   1.6959  0.2291
## week                4    32 107.5671  <.0001
## fertilizer:week     4    32   3.3171  0.0221
\end{verbatim}
\end{kframe}
\end{knitrout}
\end{frame}



\section{Summary}


\begin{frame}
  \frametitle{Summary}
  \large
%  \begin{itemize}[<+->]
%    \item
  Linear mixed-effects models are useful anytime you have
  fixed effects and normally-distriubted random effects \\
  \pause
  \vfill
%    \item
  They allow for (serial, spatial, etc\dots) correlation among
  observations within a level of the grouping variable \\
  \pause
  \vfill
%    \item
  These models can be used for repeated measures data as an
  alternative to MANOVA or the split-plot method with adjusted
  $P$-values \\
  \pause
  \vfill
%    \item
  We do not want you to use this approach for the repeated
  measures questions in the exam. \\
  \pause
  \vfill
%    \item
  But we will expect you to know how to fit linear
      mixed-effects models using {\tt lme}, as we have already done
%  \end{itemize}
\end{frame}










\end{document}








\begin{frame}[fragile]
  \frametitle{Hawk home range size example}
  \scriptsize
\begin{knitrout}
\definecolor{shadecolor}{rgb}{0.878, 0.918, 0.933}\color{fgcolor}\begin{kframe}
\begin{alltt}
\hlstd{hawkData} \hlkwb{<-} \hlkwd{read.csv}\hlstd{(}\hlstr{"hawkData.csv"}\hlstd{)}
\hlstd{hawkData}\hlopt{$}\hlstd{hawk} \hlkwb{<-} \hlkwd{factor}\hlstd{(hawkData}\hlopt{$}\hlstd{hawk)}
\hlstd{hawkData}\hlopt{$}\hlstd{season} \hlkwb{<-} \hlkwd{factor}\hlstd{(hawkData}\hlopt{$}\hlstd{season)}
\hlkwd{head}\hlstd{(hawkData,} \hlnum{15}\hlstd{)}
\end{alltt}
\begin{verbatim}
##    hawk season sex homerange
## 1     4      1   F     127.1
## 2     4      2   F      86.6
## 3     4      3   F     116.0
## 4     4      4   F      78.4
## 5     5      1   F     178.4
## 6     5      2   F      84.7
## 7     5      3   F      48.2
## 8     5      4   F      47.9
## 9     8      1   M     158.8
## 10    8      2   M     173.6
## 11    8      3   M      62.1
## 12    8      4   M      54.7
## 13   12      1   F     241.5
## 14   12      2   F     250.4
## 15   12      3   F     162.9
\end{verbatim}
\end{kframe}
\end{knitrout}
\end{frame}



\begin{frame}[fragile]
  \frametitle{Mixed-effects models}
\begin{knitrout}
\definecolor{shadecolor}{rgb}{0.878, 0.918, 0.933}\color{fgcolor}\begin{kframe}
\begin{alltt}
\hlstd{lme.hawk1} \hlkwb{<-} \hlkwd{lme}\hlstd{(homerange} \hlopt{~} \hlstd{sex}\hlopt{*}\hlstd{season,} \hlkwc{data}\hlstd{=hawkData,}
                 \hlkwc{random}\hlstd{=}\hlopt{~}\hlnum{1}\hlopt{|}\hlstd{hawk)}
\end{alltt}
\end{kframe}
\end{knitrout}
\pause
\begin{knitrout}
\definecolor{shadecolor}{rgb}{0.878, 0.918, 0.933}\color{fgcolor}\begin{kframe}
\begin{alltt}
\hlstd{lme.hawk2} \hlkwb{<-} \hlkwd{lme}\hlstd{(homerange} \hlopt{~} \hlstd{sex}\hlopt{*}\hlstd{season,} \hlkwc{data}\hlstd{=hawkData,}
                 \hlkwc{random}\hlstd{=}\hlopt{~}\hlnum{1}\hlopt{|}\hlstd{hawk,}
                 \hlkwc{correlation}\hlstd{=}\hlkwd{corARMA}\hlstd{(}\hlkwc{p}\hlstd{=}\hlnum{1}\hlstd{))}
\end{alltt}
\end{kframe}
\end{knitrout}
\end{frame}



\begin{frame}[fragile]
  \frametitle{Compare ANOVA tables}
\begin{knitrout}
\definecolor{shadecolor}{rgb}{0.878, 0.918, 0.933}\color{fgcolor}\begin{kframe}
\begin{alltt}
\hlkwd{anova}\hlstd{(lme.hawk1)}
\end{alltt}
\begin{verbatim}
##             numDF denDF  F-value p-value
## (Intercept)     1    14 74.37777  <.0001
## sex             1     9  2.37407  0.1578
## season          3    14  7.28236  0.0035
## sex:season      3    14  0.05892  0.9805
\end{verbatim}
\end{kframe}
\end{knitrout}
\pause
\begin{knitrout}
\definecolor{shadecolor}{rgb}{0.878, 0.918, 0.933}\color{fgcolor}\begin{kframe}
\begin{alltt}
\hlkwd{anova}\hlstd{(lme.hawk2)}
\end{alltt}
\begin{verbatim}
##             numDF denDF  F-value p-value
## (Intercept)     1    14 76.66605  <.0001
## sex             1     9  2.66784  0.1368
## season          3    14  4.92350  0.0154
## sex:season      3    14  0.14536  0.9309
\end{verbatim}
\end{kframe}
\end{knitrout}
\end{frame}



\begin{frame}[fragile]
  \frametitle{Autocorrelation function (ACF)}
  \tiny
\begin{knitrout}
\definecolor{shadecolor}{rgb}{0.878, 0.918, 0.933}\color{fgcolor}\begin{kframe}
\begin{alltt}
\hlkwd{plot}\hlstd{(}\hlkwd{ACF}\hlstd{(lme.hawk1),} \hlkwc{alpha}\hlstd{=}\hlnum{0.05}\hlstd{)}
\end{alltt}
\end{kframe}
\end{knitrout}
\begin{center}
  \includegraphics[width=7cm]{figure/lme-hawk1-1}
\end{center}
\end{frame}



\begin{frame}[fragile]
  \frametitle{Autocorrelation function (ACF)}
  \tiny
\begin{knitrout}
\definecolor{shadecolor}{rgb}{0.878, 0.918, 0.933}\color{fgcolor}\begin{kframe}
\begin{alltt}
\hlkwd{plot}\hlstd{(}\hlkwd{ACF}\hlstd{(lme.hawk2),} \hlkwc{alpha}\hlstd{=}\hlnum{0.05}\hlstd{)}
\end{alltt}
\end{kframe}
\end{knitrout}
\begin{center}
  \includegraphics[width=7cm]{figure/lme-hawk2-1}
\end{center}
\end{frame}











\end{document}


