\message{ !name(lecture-modeling-intro.Rnw)}\documentclass[color=usenames,dvipsnames]{beamer}
%\documentclass[color=usenames,dvipsnames,handout]{beamer}


\usepackage[sans]{../../lab1}
\usepackage{bm}


\hypersetup{pdftex,pdfstartview=FitV}



<<build-fun, include=FALSE, cache=TRUE, eval=FALSE, purl=FALSE>>=
## A function to compile and open the pdf
source("../rnw2pdf.R")
if(1==2) {
    ## Usage:
    rnw2pdf("lecture-modeling-intro") # Don't include the file extension
    rnw2pdf("lecture-modeling-intro", clean=FALSE) # Don't clean intermediate files
    rnw2pdf("lecture-modeling-intro", tangle=TRUE) # If you want the .R file
}
@


<<knitr-theme, include=FALSE, purl=FALSE>>=
##knit_theme$set("navajo-night")
knit_theme$set("edit-kwrite")
@


%% New command for inline code that isn't to be evaluated
\definecolor{inlinecolor}{rgb}{0.878, 0.918, 0.933}
\newcommand{\inr}[1]{\colorbox{inlinecolor}{\texttt{#1}}}






\begin{document}

\message{ !name(lecture-modeling-intro.Rnw) !offset(647) }
res <- persp(forest.seq, elev.seq, matrix(pred2$fit[,1],nseq),
             theta=270, phi=20, col=rgb(0,0,1,0.5), zlab="Expected number of jays",
             ylab="Elevation", xlab="Forest cover", zlim=c(30, 55),
             ticktype="detailed")
points(trans3d(jayData$forest, jayData$elev, jayData$jays, pmat=res),
       col="red", cex=1.5, pch=16)


\message{ !name(lecture-modeling-intro.Rnw) !offset(1344) }

\end{document}

