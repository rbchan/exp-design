\documentclass[color=usenames,dvipsnames]{beamer}\usepackage[]{graphicx}\usepackage[]{color}
%% maxwidth is the original width if it is less than linewidth
%% otherwise use linewidth (to make sure the graphics do not exceed the margin)
\makeatletter
\def\maxwidth{ %
  \ifdim\Gin@nat@width>\linewidth
    \linewidth
  \else
    \Gin@nat@width
  \fi
}
\makeatother

\definecolor{fgcolor}{rgb}{0, 0, 0}
\newcommand{\hlnum}[1]{\textcolor[rgb]{0.69,0.494,0}{#1}}%
\newcommand{\hlstr}[1]{\textcolor[rgb]{0.749,0.012,0.012}{#1}}%
\newcommand{\hlcom}[1]{\textcolor[rgb]{0.514,0.506,0.514}{\textit{#1}}}%
\newcommand{\hlopt}[1]{\textcolor[rgb]{0,0,0}{#1}}%
\newcommand{\hlstd}[1]{\textcolor[rgb]{0,0,0}{#1}}%
\newcommand{\hlkwa}[1]{\textcolor[rgb]{0,0,0}{\textbf{#1}}}%
\newcommand{\hlkwb}[1]{\textcolor[rgb]{0,0.341,0.682}{#1}}%
\newcommand{\hlkwc}[1]{\textcolor[rgb]{0,0,0}{\textbf{#1}}}%
\newcommand{\hlkwd}[1]{\textcolor[rgb]{0.004,0.004,0.506}{#1}}%
\let\hlipl\hlkwb

\usepackage{framed}
\makeatletter
\newenvironment{kframe}{%
 \def\at@end@of@kframe{}%
 \ifinner\ifhmode%
  \def\at@end@of@kframe{\end{minipage}}%
  \begin{minipage}{\columnwidth}%
 \fi\fi%
 \def\FrameCommand##1{\hskip\@totalleftmargin \hskip-\fboxsep
 \colorbox{shadecolor}{##1}\hskip-\fboxsep
     % There is no \\@totalrightmargin, so:
     \hskip-\linewidth \hskip-\@totalleftmargin \hskip\columnwidth}%
 \MakeFramed {\advance\hsize-\width
   \@totalleftmargin\z@ \linewidth\hsize
   \@setminipage}}%
 {\par\unskip\endMakeFramed%
 \at@end@of@kframe}
\makeatother

\definecolor{shadecolor}{rgb}{.97, .97, .97}
\definecolor{messagecolor}{rgb}{0, 0, 0}
\definecolor{warningcolor}{rgb}{1, 0, 1}
\definecolor{errorcolor}{rgb}{1, 0, 0}
\newenvironment{knitrout}{}{} % an empty environment to be redefined in TeX

\usepackage{alltt}
%\documentclass[color=usenames,dvipsnames,handout]{beamer}

%\usepackage[roman]{../../lab1}
\usepackage[sans]{../../lab1}


\hypersetup{pdftex,pdfstartview=FitV}










%% New command for inline code that isn't to be evaluated
\definecolor{inlinecolor}{rgb}{0.878, 0.918, 0.933}
\newcommand{\inr}[1]{\colorbox{inlinecolor}{\texttt{#1}}}
\IfFileExists{upquote.sty}{\usepackage{upquote}}{}
\begin{document}





%\section{Intro}

\begin{frame}[plain]
%  \maketitle
  \LARGE
  \centering \par
  {\bf \color{RoyalBlue}{Lab 4 -- Contrasts, Estimation, and Power Analysis \par}}
  \vspace{1cm}
  \Large
  September 10 \& 11, 2018 \\
  FANR 6750 \par
  \vfill
  \large
  Richard Chandler and Bob Cooper \\
  University of Georgia \\
\end{frame}





\section{Contrasts}


\begin{frame}[plain]
  \frametitle{Today's Topics}
  \Large
  \only<1>{\tableofcontents}%[hideallsubsections]}
  \only<2 | handout:0>{\tableofcontents[currentsection]}%,hideallsubsections]}
\end{frame}



\begin{frame}[fragile]
  \frametitle{Chain Saw Data}
%% {\bf The data as 4 vectors}
%% <<>>=
%% brandA <- c(42,17,24,39,43)
%% brandB <- c(28,50,44,32,61)
%% brandC <- c(57,45,48,41,54)
%% brandD <- c(29,40,22,34,30)
%% @
%% \pause
%% \vspace{0.5cm}
%% {\bf Format as a {\tt data.frame}}
%% <<>>=
%% n <- length(brandA)
%% a <- 4
%% sawData <- data.frame(
%%     y=c(brandA, brandB, brandC, brandD),
%%     Brand=rep(c("A","B","C","D"), each=n))
%% @
\begin{knitrout}\scriptsize
\definecolor{shadecolor}{rgb}{0.878, 0.918, 0.933}\color{fgcolor}\begin{kframe}
\begin{alltt}
\hlstd{sawData} \hlkwb{<-} \hlkwd{read.csv}\hlstd{(}\hlstr{"sawData.csv"}\hlstd{)}
\hlstd{sawData}
\end{alltt}
\begin{verbatim}
##     y Brand
## 1  42     A
## 2  17     A
## 3  24     A
## 4  39     A
## 5  43     A
## 6  28     B
## 7  50     B
## 8  44     B
## 9  32     B
## 10 61     B
## 11 57     C
## 12 45     C
## 13 48     C
## 14 41     C
## 15 54     C
## 16 29     D
## 17 40     D
## 18 22     D
## 19 34     D
## 20 30     D
\end{verbatim}
\end{kframe}
\end{knitrout}
\end{frame}






\begin{frame}[fragile]
  \frametitle{Contrasts}
  {\bf Suppose we want to make 3 {\it a priori} comparisons:}
  \begin{enumerate}[\bf (1)]
    \item Groups A\&D vs B\&C
    \item Groups A vs D
    \item Groups B vs C
  \end{enumerate}
  \pause
  \vspace{0.5cm}
  \begin{center}
    \begin{tabular}{llr}
      \hline
        & Comparison & Null hypothesis \\
      \hline
      1 & AD vs BC & $\frac{\mu_A + \mu_D}{2} - \frac{\mu_B + \mu_C}{2} = 0$ \\
      2 & A vs D & $\mu_A - \mu_D = 0$ \\
      3 & B vs C & $\mu_B - \mu_C = 0$ \\
      \hline
    \end{tabular}
  \end{center}
\end{frame}


\begin{frame}[fragile]
  \frametitle{Constructing contrasts in \R}
  \small
  {\bf Coefficients}
\begin{knitrout}\scriptsize
\definecolor{shadecolor}{rgb}{0.878, 0.918, 0.933}\color{fgcolor}\begin{kframe}
\begin{alltt}
\hlstd{ADvBC} \hlkwb{<-} \hlkwd{c}\hlstd{(}\hlnum{1}\hlopt{/}\hlnum{2}\hlstd{,} \hlopt{-}\hlnum{1}\hlopt{/}\hlnum{2}\hlstd{,} \hlopt{-}\hlnum{1}\hlopt{/}\hlnum{2}\hlstd{,} \hlnum{1}\hlopt{/}\hlnum{2}\hlstd{)}
\hlstd{AvD} \hlkwb{<-} \hlkwd{c}\hlstd{(}\hlnum{1}\hlstd{,} \hlnum{0}\hlstd{,} \hlnum{0}\hlstd{,} \hlopt{-}\hlnum{1}\hlstd{)}
\hlstd{BvC} \hlkwb{<-} \hlkwd{c}\hlstd{(}\hlnum{0}\hlstd{,} \hlnum{1}\hlstd{,} \hlopt{-}\hlnum{1}\hlstd{,} \hlnum{0}\hlstd{)}
\end{alltt}
\end{kframe}
\end{knitrout}
\pause
{\bf
Are they orthogonal?}
\pause
\begin{columns}
  \begin{column}{0.5\textwidth}
\begin{knitrout}\scriptsize
\definecolor{shadecolor}{rgb}{0.878, 0.918, 0.933}\color{fgcolor}\begin{kframe}
\begin{alltt}
\hlkwd{sum}\hlstd{(ADvBC)}
\end{alltt}
\begin{verbatim}
## [1] 0
\end{verbatim}
\begin{alltt}
\hlkwd{sum}\hlstd{(AvD)}
\end{alltt}
\begin{verbatim}
## [1] 0
\end{verbatim}
\begin{alltt}
\hlkwd{sum}\hlstd{(BvC)}
\end{alltt}
\begin{verbatim}
## [1] 0
\end{verbatim}
\end{kframe}
\end{knitrout}
  \end{column}
  \pause
  \begin{column}{0.5\textwidth}
\begin{knitrout}\scriptsize
\definecolor{shadecolor}{rgb}{0.878, 0.918, 0.933}\color{fgcolor}\begin{kframe}
\begin{alltt}
\hlkwd{sum}\hlstd{(ADvBC} \hlopt{*} \hlstd{AvD)}
\end{alltt}
\begin{verbatim}
## [1] 0
\end{verbatim}
\begin{alltt}
\hlkwd{sum}\hlstd{(ADvBC} \hlopt{*} \hlstd{BvC)}
\end{alltt}
\begin{verbatim}
## [1] 0
\end{verbatim}
\begin{alltt}
\hlkwd{sum}\hlstd{(AvD} \hlopt{*} \hlstd{BvC)}
\end{alltt}
\begin{verbatim}
## [1] 0
\end{verbatim}
\end{kframe}
\end{knitrout}
  \end{column}
\end{columns}
\pause
\vfill
{%\centering
  \bf Yes, they are. \par}
\end{frame}




%% \begin{frame}[fragile]
%%   \frametitle{We need to put the contrasts into a matrix}
%%   The \verb+cbind+ function turns each vector into a row of a matrix
%% <<>>=
%% contrast.mat <- cbind(ADvBC, AvD, BvC)
%% contrast.mat
%% @
%% {\bf In \R, we need to use fractions.}
%% <<>>=
%% R.contrasts <- cbind(ADvBC, AvD/2, BvC/2)
%% R.contrasts
%% @
%% \end{frame}





\begin{frame}[fragile]
  \frametitle{We need to put the contrasts into a matrix}
  {\bf To use contrasts in {\tt R}, each set of coefficients must be formatted as a column in a matrix. \par }
  \pause
  \vfill
  {\bf We can use {\tt cbind} for this:}
\begin{knitrout}
\definecolor{shadecolor}{rgb}{0.878, 0.918, 0.933}\color{fgcolor}\begin{kframe}
\begin{alltt}
\hlstd{contrast.mat} \hlkwb{<-} \hlkwd{cbind}\hlstd{(ADvBC, AvD, BvC)}
\hlstd{contrast.mat}
\end{alltt}
\begin{verbatim}
##      ADvBC AvD BvC
## [1,]   0.5   1   0
## [2,]  -0.5   0   1
## [3,]  -0.5   0  -1
## [4,]   0.5  -1   0
\end{verbatim}
\end{kframe}
\end{knitrout}
\end{frame}




%% \begin{frame}[fragile]
%%   \frametitle{Format constrats for \R}
%%   {\bf \R~requires that the contrasts be converted to a form that it
%%   likes. The easiest way to do this is to use the {\tt
%%     make.contrasts} function in the {\tt gmodels} package}.
%% \begin{footnotesize}
%% <<>>=
%% ##install.packages("gmodels")
%% library(gmodels)
%% R.contrasts <- zapsmall(make.contrasts(contrast.mat))
%% R.contrasts
%% @
%% \end{footnotesize}
%% \end{frame}





\begin{frame}[fragile]
  \frametitle{Contrasts}
{\bf Fit the model with contrasts}
\begin{footnotesize}
\begin{knitrout}
\definecolor{shadecolor}{rgb}{0.878, 0.918, 0.933}\color{fgcolor}\begin{kframe}
\begin{alltt}
\hlstd{aov.out} \hlkwb{<-} \hlkwd{aov}\hlstd{(y} \hlopt{~} \hlstd{Brand,} \hlkwc{data}\hlstd{=sawData,}
               \hlkwc{contrasts}\hlstd{=}\hlkwd{list}\hlstd{(}\hlkwc{Brand}\hlstd{=contrast.mat))}
\end{alltt}
\end{kframe}
\end{knitrout}
\pause
\vfill
{\bf Now ``split'' apart the sum-of-squares}
\begin{knitrout}
\definecolor{shadecolor}{rgb}{0.878, 0.918, 0.933}\color{fgcolor}\begin{kframe}
\begin{alltt}
\hlkwd{summary}\hlstd{(aov.out,} \hlkwc{split} \hlstd{=} \hlkwd{list}\hlstd{(}\hlkwc{Brand} \hlstd{=}
                      \hlkwd{list}\hlstd{(}\hlstr{"ADvBC"}\hlstd{=}\hlnum{1}\hlstd{,} \hlstr{"AvD"}\hlstd{=}\hlnum{2}\hlstd{,} \hlstr{"BvC"}\hlstd{=}\hlnum{3}\hlstd{)))}
\end{alltt}
\begin{verbatim}
##                Df Sum Sq Mean Sq F value  Pr(>F)   
## Brand           3   1080   360.0   3.556 0.03823 * 
##   Brand: ADvBC  1    980   980.0   9.679 0.00672 **
##   Brand: AvD    1     10    10.0   0.099 0.75738   
##   Brand: BvC    1     90    90.0   0.889 0.35980   
## Residuals      16   1620   101.2                   
## ---
## Signif. codes:  0 '***' 0.001 '**' 0.01 '*' 0.05 '.' 0.1 ' ' 1
\end{verbatim}
\end{kframe}
\end{knitrout}
\end{footnotesize}
\end{frame}




\begin{frame}[fragile]
  \frametitle{Difference in means for each contrast}
  \footnotesize %\small
{\bf Group means}
\begin{knitrout}\scriptsize
\definecolor{shadecolor}{rgb}{0.878, 0.918, 0.933}\color{fgcolor}\begin{kframe}
\begin{alltt}
\hlstd{(group.means} \hlkwb{<-} \hlkwd{tapply}\hlstd{(sawData}\hlopt{$}\hlstd{y, sawData}\hlopt{$}\hlstd{Brand, mean))}
\end{alltt}
\begin{verbatim}
##  A  B  C  D 
## 33 43 49 31
\end{verbatim}
\end{kframe}
\end{knitrout}
\pause
\vfill
{\bf %\centering
  Difference in means for A vs D \par}
\begin{knitrout}\scriptsize
\definecolor{shadecolor}{rgb}{0.878, 0.918, 0.933}\color{fgcolor}\begin{kframe}
\begin{alltt}
\hlstd{group.means} \hlkwb{<-} \hlkwd{unname}\hlstd{(group.means)} \hlcom{# Drop names (optional)}
\hlstd{group.means[}\hlnum{1}\hlstd{]} \hlopt{-} \hlstd{group.means[}\hlnum{4}\hlstd{]}
\end{alltt}
\begin{verbatim}
## [1] 2
\end{verbatim}
\end{kframe}
\end{knitrout}
\pause
\vfill
{\bf %\centering
  Difference in means for B vs C \par}
\begin{knitrout}\scriptsize
\definecolor{shadecolor}{rgb}{0.878, 0.918, 0.933}\color{fgcolor}\begin{kframe}
\begin{alltt}
\hlstd{group.means[}\hlnum{2}\hlstd{]} \hlopt{-} \hlstd{group.means[}\hlnum{3}\hlstd{]}
\end{alltt}
\begin{verbatim}
## [1] -6
\end{verbatim}
\end{kframe}
\end{knitrout}
\pause
\vfill
{\bf %\centering
  Difference in means for AD vs BC \par}
%\scriptsize
\begin{knitrout}\scriptsize
\definecolor{shadecolor}{rgb}{0.878, 0.918, 0.933}\color{fgcolor}\begin{kframe}
\begin{alltt}
\hlkwd{mean}\hlstd{(group.means[}\hlkwd{c}\hlstd{(}\hlnum{1}\hlstd{,}\hlnum{4}\hlstd{)])} \hlopt{-} \hlkwd{mean}\hlstd{(group.means[}\hlnum{2}\hlopt{:}\hlnum{3}\hlstd{])}
\end{alltt}
\begin{verbatim}
## [1] -14
\end{verbatim}
\end{kframe}
\end{knitrout}

\end{frame}







\begin{frame}[fragile]
  \frametitle{Standard errors for each contrast}%{Contrasts}
  \small
%{\bf Compute the standard error of a contrast \par}
{\bf %\centering
  SE for A vs D \par}
\begin{knitrout}\footnotesize
\definecolor{shadecolor}{rgb}{0.878, 0.918, 0.933}\color{fgcolor}\begin{kframe}
\begin{alltt}
\hlkwd{se.contrast}\hlstd{(aov.out,} \hlkwd{list}\hlstd{(sawData}\hlopt{$}\hlstd{Brand}\hlopt{==}\hlstr{"A"}\hlstd{,}
                          \hlstd{sawData}\hlopt{$}\hlstd{Brand}\hlopt{==}\hlstr{"D"}\hlstd{))}
\end{alltt}
\begin{verbatim}
## [1] 6.363961
\end{verbatim}
\end{kframe}
\end{knitrout}
\pause
\vfill
{\bf %\centering
  SE for B vs C \par}
\begin{knitrout}\footnotesize
\definecolor{shadecolor}{rgb}{0.878, 0.918, 0.933}\color{fgcolor}\begin{kframe}
\begin{alltt}
\hlkwd{se.contrast}\hlstd{(aov.out,} \hlkwd{list}\hlstd{(sawData}\hlopt{$}\hlstd{Brand}\hlopt{==}\hlstr{"B"}\hlstd{,}
                           \hlstd{sawData}\hlopt{$}\hlstd{Brand}\hlopt{==}\hlstr{"C"}\hlstd{))}
\end{alltt}
\begin{verbatim}
## [1] 6.363961
\end{verbatim}
\end{kframe}
\end{knitrout}
\pause
\vfill
{\bf %\centering
  SE for AD vs BC \par}
%\scriptsize
\begin{knitrout}\footnotesize
\definecolor{shadecolor}{rgb}{0.878, 0.918, 0.933}\color{fgcolor}\begin{kframe}
\begin{alltt}
\hlkwd{se.contrast}\hlstd{(aov.out,} \hlkwd{list}\hlstd{(sawData}\hlopt{$}\hlstd{Brand}\hlopt{==}\hlstr{"A"} \hlopt{|}
                          \hlstd{sawData}\hlopt{$}\hlstd{Brand}\hlopt{==}\hlstr{"D"}\hlstd{,}
                          \hlstd{sawData}\hlopt{$}\hlstd{Brand}\hlopt{==}\hlstr{"B"} \hlopt{|}
                          \hlstd{sawData}\hlopt{$}\hlstd{Brand}\hlopt{==}\hlstr{"C"}\hlstd{))}
\end{alltt}
\begin{verbatim}
## [1] 4.5
\end{verbatim}
\end{kframe}
\end{knitrout}
% \begin{scriptsize}
%<<>>=
%aov.summary <- summary.lm(aov.out)
%aov.summary
%@
%<<>>=
%model.tables(aov.out, type="effects", se=TRUE)
%@
%\end{scriptsize}
\end{frame}



%% \begin{frame}[fragile]
%%   \frametitle{Contrasts}
%% {\bf Compute the 95\% confidence intervals}
%% <<>>=
%% aov.CI <- confint(aov.out, level=0.95)
%% aov.CI
%% @
%% \end{frame}






\section{Estimation}



\begin{frame}[plain]
  \frametitle{Today's Topics}
  \Large
  \tableofcontents[currentsection]
\end{frame}






\begin{frame}[fragile]
  \frametitle{Estimating confidence intervals}
  In an ANOVA context, confidence intervals can be
  constructed using the equation:
  \[
    \text{CI} = \text{Point estimate} \pm t_{\alpha/2,a(n-1)} \times \text{SE}
  \]
  As usual, the hard part is computing the SE\footnote{See page 300 of
    Dowdy et al. for SE formulas}
%  \pause
%\footnotesize
%<<>>=
%yA <- c(5, 8, 11, 7)
%yB <- c(2, 4, 6, 3)
%@
%\pause
%<<>>=
%n <- length(yA)
%#varP <- (var(yA) + var(yB)) / 2
%varP <- (var(yA)*(n-1) + var(yB)*(n-1))/(2*n-2)
%SE <- sqrt(varP/n + varP/n)
%ybardiff <- mean(yA) - mean(yB)
%alpha <- 0.05
%t <- qt(1-alpha/2, n+n-2)
%lower <- ybardiff - SE*t
%upper <- ybardiff + SE*t
%CI <- c(lower, upper)
%CI
%@
\end{frame}





%% \begin{frame}[fragile]
%%   \frametitle{Confidence interval from 2-sample $t$-test}
%%   \footnotesize
%% <<>>=
%% t.test(yA, yB, mu=0, paired=FALSE, var.equal=TRUE)
%% @
%% \end{frame}


\begin{frame}[fragile]
  \frametitle{Confidence intervals from one-way ANOVA}
  \small %\footnotesize
%<<echo=false>>=
%options("contrasts" = c("contr.treatment", "contr.poly"))
%@
%<<>>=
%dat <- data.frame(y = c(5, 8, 11, 7, 2, 4, 6, 3, 9, 12, 15, 16),
%                  group = rep(c("A","B","C"), each=4))
%aov1 <- aov(y ~ group, dat)
%@
%\pause
  {\bf SE's for the effect sizes ($\alpha$'s)}
\begin{knitrout}
\definecolor{shadecolor}{rgb}{0.878, 0.918, 0.933}\color{fgcolor}\begin{kframe}
\begin{alltt}
\hlstd{effects.SE} \hlkwb{<-} \hlkwd{model.tables}\hlstd{(aov.out,} \hlkwc{type}\hlstd{=}\hlstr{"effects"}\hlstd{,}
                           \hlkwc{se}\hlstd{=}\hlnum{TRUE}\hlstd{)}
\hlstd{effects.SE}
\end{alltt}
\begin{verbatim}
## Tables of effects
## 
##  Brand 
## Brand
##  A  B  C  D 
## -6  4 10 -8 
## 
## Standard errors of effects
##         Brand
##           4.5
## replic.     5
\end{verbatim}
\end{kframe}
\end{knitrout}
%  \pause
\end{frame}



\begin{frame}[fragile]
  \frametitle{Confidence intervals from one-way ANOVA}
  \small
  {\bf Extract the $\alpha$'s and the SEs}
\begin{knitrout}\footnotesize
\definecolor{shadecolor}{rgb}{0.878, 0.918, 0.933}\color{fgcolor}\begin{kframe}
\begin{alltt}
\hlcom{# str(effects.SE)}
\hlstd{alpha.i} \hlkwb{<-} \hlkwd{as.numeric}\hlstd{(effects.SE}\hlopt{$}\hlstd{tables}\hlopt{$}\hlstd{Brand)}
\hlstd{SE} \hlkwb{<-} \hlkwd{as.numeric}\hlstd{(effects.SE}\hlopt{$}\hlstd{se)}
\end{alltt}
\end{kframe}
\end{knitrout}
  \pause
  {\bf Compute confidence intervals}
\begin{knitrout}\footnotesize
\definecolor{shadecolor}{rgb}{0.878, 0.918, 0.933}\color{fgcolor}\begin{kframe}
\begin{alltt}
\hlstd{tc} \hlkwb{<-} \hlkwd{qt}\hlstd{(}\hlnum{0.975}\hlstd{,} \hlnum{4}\hlopt{*}\hlstd{(}\hlnum{5}\hlopt{-}\hlnum{1}\hlstd{))}
\hlstd{lowerCI} \hlkwb{<-} \hlstd{alpha.i} \hlopt{-} \hlstd{tc} \hlopt{*} \hlstd{SE}
\hlstd{upperCI} \hlkwb{<-} \hlstd{alpha.i} \hlopt{+} \hlstd{tc} \hlopt{*} \hlstd{SE}
\end{alltt}
\end{kframe}
\end{knitrout}
  \pause
  {\bf Put results into a {\tt data.frame}}
\begin{knitrout}\footnotesize
\definecolor{shadecolor}{rgb}{0.878, 0.918, 0.933}\color{fgcolor}\begin{kframe}
\begin{alltt}
\hlstd{CI} \hlkwb{<-} \hlkwd{data.frame}\hlstd{(}\hlkwc{effect.size}\hlstd{=alpha.i, SE,}
                 \hlstd{lowerCI, upperCI)}
\hlkwd{round}\hlstd{(CI,} \hlnum{2}\hlstd{)}
\end{alltt}
\begin{verbatim}
##   effect.size  SE lowerCI upperCI
## 1          -6 4.5  -15.54    3.54
## 2           4 4.5   -5.54   13.54
## 3          10 4.5    0.46   19.54
## 4          -8 4.5  -17.54    1.54
\end{verbatim}
\end{kframe}
\end{knitrout}
%  \scriptsize
%<<echo=false>>=
%options("contrasts" = c("contr.treatment", "contr.poly"))
%@
%<<>>=
%options("contrasts")
%@
%\pause
%<<>>=
%summary.lm(aov1)
%@
\end{frame}



\begin{frame}[fragile]
  \frametitle{Plot effects and CIs}
  \tiny %\scriptsize
\begin{knitrout}\tiny
\definecolor{shadecolor}{rgb}{0.878, 0.918, 0.933}\color{fgcolor}\begin{kframe}
\begin{alltt}
\hlkwd{plot}\hlstd{(}\hlnum{1}\hlopt{:}\hlnum{4}\hlstd{, CI}\hlopt{$}\hlstd{effect.size,} \hlkwc{xlim}\hlstd{=}\hlkwd{c}\hlstd{(}\hlnum{0.5}\hlstd{,} \hlnum{4.5}\hlstd{),} \hlkwc{ylim}\hlstd{=}\hlkwd{c}\hlstd{(}\hlopt{-}\hlnum{18}\hlstd{,} \hlnum{20}\hlstd{),} \hlkwc{xaxt}\hlstd{=}\hlstr{"n"}\hlstd{,}
     \hlkwc{xlab}\hlstd{=}\hlstr{"Brand"}\hlstd{,} \hlkwc{ylab}\hlstd{=}\hlstr{"Difference from grand mean"}\hlstd{,} \hlkwc{pch}\hlstd{=}\hlnum{16}\hlstd{,} \hlkwc{cex.lab}\hlstd{=}\hlnum{1.5}\hlstd{)}
\hlkwd{axis}\hlstd{(}\hlnum{1}\hlstd{,} \hlkwc{at}\hlstd{=}\hlnum{1}\hlopt{:}\hlnum{4}\hlstd{,} \hlkwc{labels}\hlstd{=}\hlkwd{c}\hlstd{(}\hlstr{"A"}\hlstd{,} \hlstr{"B"}\hlstd{,} \hlstr{"C"}\hlstd{,} \hlstr{"D"}\hlstd{))}
\hlkwd{abline}\hlstd{(}\hlkwc{h}\hlstd{=}\hlnum{0}\hlstd{,} \hlkwc{lty}\hlstd{=}\hlnum{3}\hlstd{)}
\hlkwd{arrows}\hlstd{(}\hlnum{1}\hlopt{:}\hlnum{4}\hlstd{, CI}\hlopt{$}\hlstd{lowerCI,} \hlnum{1}\hlopt{:}\hlnum{4}\hlstd{, CI}\hlopt{$}\hlstd{upperCI,} \hlkwc{code}\hlstd{=}\hlnum{3}\hlstd{,} \hlkwc{angle}\hlstd{=}\hlnum{90}\hlstd{,} \hlkwc{length}\hlstd{=}\hlnum{0.05}\hlstd{)}
\end{alltt}
\end{kframe}
\end{knitrout}
\centering
\includegraphics[width=0.85\textwidth]{figure/ciplot-1} \\
\end{frame}




%% \begin{comment}
%% \begin{frame}[fragile]
%%   \frametitle{Confidence intervals from one-way ANOVA}
%%   \scriptsize
%% <<>>=
%% options("contrasts" = c("contr.sum", "contr.poly"))
%% aov2 <- aov(y ~ group-1+offset(gm), dat)
%% @
%% \pause
%% <<>>=
%% summary.lm(aov2)
%% @
%% \end{frame}




%% \begin{frame}[fragile]
%%   \frametitle{Confidence intervals from one-way ANOVA}
%% %  \footnotesize
%% <<>>=
%% options("contrasts" = c("contr.sum", "contr.poly"))
%% @
%% \pause
%% <<>>=
%% confint(aov2)
%% @
%% \end{frame}
%% \end{comment}







\section{Power}


\begin{frame}[plain]
  \frametitle{Today's Topics}
  \Large
  \tableofcontents[currentsection]
\end{frame}



%% \begin{frame}
%%   \frametitle{Statistical Power}
%%   Reminder about Type I and Type II errors, and definition of power
%%   \begin{itemize}[<+->]
%%     \item $\alpha$ = Pr(rejecting a null hypothesis that is true)
%%     \item Power = $\beta$ = 1-Pr(failing to reject a null hypothesis that is false)
%%   \end{itemize}
%% %  \vspace{0.5cm}
%% %  \uncover<4>{
%% %    Prospective power analysis is a procedure for determining what
%% %    sample size is needed to obtain a user-specified power, for a
%% %    user-specified effect size
%% %  }
%% \end{frame}


\begin{frame}[fragile]
\frametitle{Power analysis for a 2-sample $t$-test}
\begin{knitrout}
\definecolor{shadecolor}{rgb}{0.878, 0.918, 0.933}\color{fgcolor}\begin{kframe}
\begin{alltt}
\hlkwd{power.t.test}\hlstd{(}\hlkwc{n}\hlstd{=}\hlkwa{NULL}\hlstd{,} \hlkwc{delta}\hlstd{=}\hlnum{3}\hlstd{,} \hlkwc{sd}\hlstd{=}\hlnum{2}\hlstd{,} \hlkwc{sig.level}\hlstd{=}\hlnum{0.05}\hlstd{,}
             \hlkwc{power}\hlstd{=}\hlnum{0.8}\hlstd{)}
\end{alltt}
\begin{verbatim}
## 
##      Two-sample t test power calculation 
## 
##               n = 8.06031
##           delta = 3
##              sd = 2
##       sig.level = 0.05
##           power = 0.8
##     alternative = two.sided
## 
## NOTE: n is number in *each* group
\end{verbatim}
\end{kframe}
\end{knitrout}
\end{frame}



\begin{frame}[fragile]
  \frametitle{Power analysis for one-way ANOVA}
\begin{small}
\begin{knitrout}
\definecolor{shadecolor}{rgb}{0.878, 0.918, 0.933}\color{fgcolor}\begin{kframe}
\begin{alltt}
\hlkwd{power.anova.test}\hlstd{(}\hlkwc{groups}\hlstd{=}\hlnum{4}\hlstd{,} \hlkwc{n}\hlstd{=}\hlnum{5}\hlstd{,} \hlkwc{between.var}\hlstd{=}\hlnum{360.0}\hlstd{,}
                 \hlkwc{within.var}\hlstd{=}\hlnum{101.2}\hlstd{,} \hlkwc{power}\hlstd{=}\hlkwa{NULL}\hlstd{)}
\end{alltt}
\begin{verbatim}
## 
##      Balanced one-way analysis of variance power calculation 
## 
##          groups = 4
##               n = 5
##     between.var = 360
##      within.var = 101.2
##       sig.level = 0.05
##           power = 0.9999359
## 
## NOTE: n is number in each group
\end{verbatim}
\end{kframe}
\end{knitrout}
\end{small}
\end{frame}










%% \begin{frame}[fragile]
%%   \frametitle{Assignment}
%%   \begin{enumerate}[\bf (1)]
%%     \item Estimate 95\% CIs for the effect sizes in the
%%       \inr{warblerWeight} data. Which effects are significant?
%%     \item What is the power of a two sample $t$-test with an effect
%%       size of 3, a pooled variance of 9, and $n=8$ at an alpha level
%%       of 0.05?
%%     \item What sample size would be necessary to achieve power=0.8
%%   \end{enumerate}
%%   \vfill
%%   {\bf Due by the end of lab today }
%%   \begin{itemize}
%%     \item Upload answers as a single R script to ELC
%%     \item Script should have comments answering the question in
%%       words along with the code necessary to product the results
%%   \end{itemize}
%% \end{frame}








\begin{frame}
  \frametitle{Assignment}
%  \scriptsize
  \footnotesize
  Researchers wish to know if food supplementation affects the growth
  of nestling Canada warblers. The treatment groups are: (A) No
  supplementation control, (B) low, (C) medium, (D) high, and (E) very
  high. The response variable is the weight of a 10 day old nestling.
  \begin{enumerate}[\bf (1)]
    \footnotesize
    \item<1-> The researchers are interested in the following
      contrasts. Are they orthogonal?
    \begin{itemize}
    \footnotesize
      \item Groups A,B vs C,D,E
      \item Groups A vs B
      \item Groups C vs D,E
      \item Groups D vs E
    \end{itemize}
    \item<2-> Using the {\tt warblerWeight} data, test the null
      hypothesis of each contrast by constructing an ANOVA table
       in \R.
%    \vspace{0.5cm}
    \item<3-> For each contrast:
      \begin{itemize}
        \footnotesize
        \item Compute the difference in the means
        \item The SE of the difference in means
        \item The 95\% CI for the difference in means
      \end{itemize}
%      compute the difference in the means and the
%      standard error of the difference
    \item<4-> Suppose you wanted to replicate the study with a
      smaller sample size of $n=2$ per treatment group? What would be
      your power?
  \end{enumerate}
  \uncover<5->{
%  \vspace{0.2cm}
%  {{\color{red} \bf Due:}
    Submit your answers in a
    self-contained script before the
    next lab. %The script should be self-contained.}
  }

\end{frame}
%\end{comment}






\end{document}


