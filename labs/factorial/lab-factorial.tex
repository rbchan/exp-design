\documentclass[color=usenames,dvipsnames]{beamer}\usepackage[]{graphicx}\usepackage[]{color}
%% maxwidth is the original width if it is less than linewidth
%% otherwise use linewidth (to make sure the graphics do not exceed the margin)
\makeatletter
\def\maxwidth{ %
  \ifdim\Gin@nat@width>\linewidth
    \linewidth
  \else
    \Gin@nat@width
  \fi
}
\makeatother

\definecolor{fgcolor}{rgb}{0, 0, 0}
\newcommand{\hlnum}[1]{\textcolor[rgb]{0.69,0.494,0}{#1}}%
\newcommand{\hlstr}[1]{\textcolor[rgb]{0.749,0.012,0.012}{#1}}%
\newcommand{\hlcom}[1]{\textcolor[rgb]{0.514,0.506,0.514}{\textit{#1}}}%
\newcommand{\hlopt}[1]{\textcolor[rgb]{0,0,0}{#1}}%
\newcommand{\hlstd}[1]{\textcolor[rgb]{0,0,0}{#1}}%
\newcommand{\hlkwa}[1]{\textcolor[rgb]{0,0,0}{\textbf{#1}}}%
\newcommand{\hlkwb}[1]{\textcolor[rgb]{0,0.341,0.682}{#1}}%
\newcommand{\hlkwc}[1]{\textcolor[rgb]{0,0,0}{\textbf{#1}}}%
\newcommand{\hlkwd}[1]{\textcolor[rgb]{0.004,0.004,0.506}{#1}}%
\let\hlipl\hlkwb

\usepackage{framed}
\makeatletter
\newenvironment{kframe}{%
 \def\at@end@of@kframe{}%
 \ifinner\ifhmode%
  \def\at@end@of@kframe{\end{minipage}}%
  \begin{minipage}{\columnwidth}%
 \fi\fi%
 \def\FrameCommand##1{\hskip\@totalleftmargin \hskip-\fboxsep
 \colorbox{shadecolor}{##1}\hskip-\fboxsep
     % There is no \\@totalrightmargin, so:
     \hskip-\linewidth \hskip-\@totalleftmargin \hskip\columnwidth}%
 \MakeFramed {\advance\hsize-\width
   \@totalleftmargin\z@ \linewidth\hsize
   \@setminipage}}%
 {\par\unskip\endMakeFramed%
 \at@end@of@kframe}
\makeatother

\definecolor{shadecolor}{rgb}{.97, .97, .97}
\definecolor{messagecolor}{rgb}{0, 0, 0}
\definecolor{warningcolor}{rgb}{1, 0, 1}
\definecolor{errorcolor}{rgb}{1, 0, 0}
\newenvironment{knitrout}{}{} % an empty environment to be redefined in TeX

\usepackage{alltt}
%\documentclass[color=usenames,dvipsnames,handout]{beamer}


%\usepackage[roman]{../../lab1}
\usepackage[sans]{../../lab1}

\hypersetup{pdftex,pdfstartview=FitV}









%% New command for inline code that isn't to be evaluated
\definecolor{inlinecolor}{rgb}{0.878, 0.918, 0.933}
\newcommand{\inr}[1]{\colorbox{inlinecolor}{\texttt{#1}}}
\IfFileExists{upquote.sty}{\usepackage{upquote}}{}
\begin{document}






\section{Intro}

\begin{frame}[plain]
  \LARGE
  \centering \par
  {\color{RoyalBlue}{Lab 7 -- $A \times B$ Factorial Designs}} \par
  \vspace{1cm}
  \Large
  October 9 \& 10, 2017
\end{frame}



\begin{comment}
\begin{frame}[fragile]
  \frametitle{Data}
\begin{knitrout}
\definecolor{shadecolor}{rgb}{0.878, 0.918, 0.933}\color{fgcolor}\begin{kframe}
\begin{alltt}
\hlstd{makeData} \hlkwb{<-} \hlkwd{data.frame}\hlstd{(}\hlkwc{precip}\hlstd{=}\hlkwd{rep}\hlstd{(}\hlkwd{c}\hlstd{(}\hlstr{"Wet"}\hlstd{,} \hlstr{"Dry"}\hlstd{),} \hlkwc{each}\hlstd{=}\hlnum{24}\hlstd{),}
                       \hlkwc{temp}\hlstd{=}\hlkwd{rep}\hlstd{(}\hlkwd{c}\hlstd{(}\hlstr{"Cold"}\hlstd{,} \hlstr{"Hot"}\hlstd{),} \hlkwc{times}\hlstd{=}\hlnum{24}\hlstd{))}
\hlstd{X} \hlkwb{<-} \hlkwd{model.matrix}\hlstd{(}\hlopt{~}\hlstd{precip}\hlopt{*}\hlstd{temp, makeData)} \hlcom{#~forest*season, makeData)}
\hlstd{E} \hlkwb{<-} \hlstd{X} \hlopt \hlkwd{c}\hlstd{(}\hlnum{20}\hlstd{,} \hlnum{3}\hlstd{,} \hlopt{-}\hlnum{3}\hlstd{,} \hlopt{-}\hlnum{10}\hlstd{)}
\hlkwd{set.seed}\hlstd{(}\hlnum{3440}\hlstd{)}
\hlstd{species} \hlkwb{<-} \hlkwd{round}\hlstd{(}\hlkwd{rnorm}\hlstd{(}\hlkwd{nrow}\hlstd{(X), E,} \hlkwc{sd}\hlstd{=}\hlnum{2}\hlstd{))}
\hlstd{species}
\end{alltt}
\begin{verbatim}
##  [1] 27  9 23 10 21 12 28  8 23 11 20 10 24 12 21 12 26  8 22 10 21 11 25
## [24]  9 17 18 25 15 15 16 21 22 21 17 20 17 19 18 21 17 19 20 21 18 19 15
## [47] 21 17
\end{verbatim}
\begin{alltt}
\hlstd{richness} \hlkwb{<-} \hlkwd{data.frame}\hlstd{(species, makeData)}
\hlkwd{summary}\hlstd{(}\hlkwd{aov}\hlstd{(species} \hlopt{~} \hlstd{precip}\hlopt{*}\hlstd{temp, richness))}
\end{alltt}
\begin{verbatim}
##             Df Sum Sq Mean Sq F value   Pr(>F)    
## precip       1   44.1    44.1   9.303  0.00387 ** 
## temp         1  736.3   736.3 155.389 4.92e-16 ***
## precip:temp  1  352.1   352.1  74.301 5.29e-11 ***
## Residuals   44  208.5     4.7                     
## ---
## Signif. codes:  0 '***' 0.001 '**' 0.01 '*' 0.05 '.' 0.1 ' ' 1
\end{verbatim}
\begin{alltt}
\hlkwd{summary}\hlstd{(}\hlkwd{aov}\hlstd{(species} \hlopt{~} \hlstd{precip}\hlopt{+}\hlstd{temp, richness))}
\end{alltt}
\begin{verbatim}
##             Df Sum Sq Mean Sq F value   Pr(>F)    
## precip       1   44.1    44.1   3.539   0.0664 .  
## temp         1  736.3   736.3  59.108 9.84e-10 ***
## Residuals   45  560.6    12.5                     
## ---
## Signif. codes:  0 '***' 0.001 '**' 0.01 '*' 0.05 '.' 0.1 ' ' 1
\end{verbatim}
\begin{alltt}
\hlkwd{write.csv}\hlstd{(richness,} \hlstr{"speciesRichness.csv"}\hlstd{,} \hlkwc{row.names}\hlstd{=}\hlnum{FALSE}\hlstd{)}
\end{alltt}
\end{kframe}
\end{knitrout}
\end{frame}
\end{comment}



\begin{frame}
  \frametitle{Situation}
  \Large
  \begin{itemize}[<+->]
    \item There are 2 factors thought to influence the response variable
    \item The effect of each factor might depend on the other factor
    \item We have replicates for each \alert{combination} of factors
  \end{itemize}
\end{frame}


%% \begin{frame}[fragile]
%%   \frametitle{Example: species richness by precip and temp}
%%   \scriptsize
%% <<>>=
%% richData <- read.csv("speciesRichness.csv")
%% head(richData, 4)
%% summary(richData)
%% @
%% \end{frame}



\begin{frame}[fragile]
  \frametitle{Example: effects of food and predators on voles}
  \scriptsize
\begin{knitrout}
\definecolor{shadecolor}{rgb}{0.878, 0.918, 0.933}\color{fgcolor}\begin{kframe}
\begin{alltt}
\hlstd{voleData} \hlkwb{<-} \hlkwd{read.csv}\hlstd{(}\hlstr{"microtus_data.csv"}\hlstd{)}
\hlkwd{head}\hlstd{(voleData,} \hlnum{7}\hlstd{)}
\end{alltt}
\begin{verbatim}
##   voles food predators
## 1    10    0   Present
## 2    12    0   Present
## 3     8    0   Present
## 4    14    0   Present
## 5    18    1   Present
## 6    20    1   Present
## 7    21    1   Present
\end{verbatim}
\begin{alltt}
\hlkwd{str}\hlstd{(voleData)}
\end{alltt}
\begin{verbatim}
## 'data.frame':	24 obs. of  3 variables:
##  $ voles    : int  10 12 8 14 18 20 21 24 20 18 ...
##  $ food     : int  0 0 0 0 1 1 1 1 2 2 ...
##  $ predators: Factor w/ 2 levels "Absent","Present": 2 2 2 2 2 2 2 2 2 2 ...
\end{verbatim}
\end{kframe}
\end{knitrout}
%\pause
%\vfill
%\centering
%\LARGE %\normalsize
%What is wrong with the format of the data? \par
\end{frame}



\begin{frame}[fragile]
  \frametitle{Must convert {\tt food} to a factor}
  \small
\begin{knitrout}
\definecolor{shadecolor}{rgb}{0.878, 0.918, 0.933}\color{fgcolor}\begin{kframe}
\begin{alltt}
\hlstd{voleData}\hlopt{$}\hlstd{food} \hlkwb{<-} \hlkwd{factor}\hlstd{(voleData}\hlopt{$}\hlstd{food)}
\hlkwd{str}\hlstd{(voleData)}
\end{alltt}
\begin{verbatim}
## 'data.frame':	24 obs. of  3 variables:
##  $ voles    : int  10 12 8 14 18 20 21 24 20 18 ...
##  $ food     : Factor w/ 3 levels "0","1","2": 1 1 1 1 2 2 2 2 3 3 ...
##  $ predators: Factor w/ 2 levels "Absent","Present": 2 2 2 2 2 2 2 2 2 2 ...
\end{verbatim}
\end{kframe}
\end{knitrout}
\end{frame}


\begin{frame}[fragile]
  \frametitle{See how many replicates you have}
%  The {\tt table} function tabulates the
\begin{knitrout}
\definecolor{shadecolor}{rgb}{0.878, 0.918, 0.933}\color{fgcolor}\begin{kframe}
\begin{alltt}
\hlkwd{table}\hlstd{(voleData}\hlopt{$}\hlstd{predators, voleData}\hlopt{$}\hlstd{food)}
\end{alltt}
\begin{verbatim}
##          
##           0 1 2
##   Absent  4 4 4
##   Present 4 4 4
\end{verbatim}
\end{kframe}
\end{knitrout}
\end{frame}



%% \begin{frame}[fragile]
%%   \frametitle{Boxplot with 2 factors}
%%   \vspace{-0.5cm}
%% %  \tiny
%% \begin{center}
%% <<fig=true,width=9,height=7>>=
%% boxplot(species ~ precip + temp, data=richData,
%%         ylab="Species richness", cex.lab=1.5)
%% @
%% \end{center}
%% \end{frame}


\begin{frame}[fragile]
  \frametitle{Boxplot with 2 factors}
%  \vspace{-0.5cm}
  \tiny
%\begin{center}

%\end{center}
\centering
\includegraphics[width=0.9\textwidth]{figure/box1-1} \par
\end{frame}


\section{\tt aov}


%% \begin{frame}[fragile]
%%   \frametitle{$A \times B$ interaction}
%%   \small
%% <<>>=
%% aov1 <- aov(species ~ precip * temp, data=richData)
%% summary(aov1)
%% @
%% \end{frame}


\begin{frame}[fragile]
  \frametitle{$A \times B$ interaction}
  \small
\begin{knitrout}
\definecolor{shadecolor}{rgb}{0.878, 0.918, 0.933}\color{fgcolor}\begin{kframe}
\begin{alltt}
\hlstd{aov1} \hlkwb{<-} \hlkwd{aov}\hlstd{(voles} \hlopt{~} \hlstd{food} \hlopt{*} \hlstd{predators,} \hlkwc{data}\hlstd{=voleData)}
\hlkwd{summary}\hlstd{(aov1)}
\end{alltt}
\begin{verbatim}
##                Df Sum Sq Mean Sq F value   Pr(>F)    
## food            2 1337.3   668.6   40.56 2.15e-07 ***
## predators       1  975.4   975.4   59.16 4.27e-07 ***
## food:predators  2  518.2   259.1   15.72 0.000112 ***
## Residuals      18  296.8    16.5                     
## ---
## Signif. codes:  0 '***' 0.001 '**' 0.01 '*' 0.05 '.' 0.1 ' ' 1
\end{verbatim}
\end{kframe}
\end{knitrout}
\end{frame}




%% \begin{frame}[fragile]
%%   \frametitle{No interaction}
%%   \small
%% <<>>=
%% aov2 <- aov(species ~ precip + temp, data=richData)
%% summary(aov2)
%% @
%% \end{frame}



\begin{frame}[fragile]
  \frametitle{No interaction}
  \small
\begin{knitrout}
\definecolor{shadecolor}{rgb}{0.878, 0.918, 0.933}\color{fgcolor}\begin{kframe}
\begin{alltt}
\hlstd{aov2} \hlkwb{<-} \hlkwd{aov}\hlstd{(voles} \hlopt{~} \hlstd{food} \hlopt{+} \hlstd{predators,} \hlkwc{data}\hlstd{=voleData)}
\hlkwd{summary}\hlstd{(aov2)}
\end{alltt}
\begin{verbatim}
##             Df Sum Sq Mean Sq F value   Pr(>F)    
## food         2 1337.3   668.6   16.41 6.06e-05 ***
## predators    1  975.4   975.4   23.94 8.81e-05 ***
## Residuals   20  815.0    40.8                     
## ---
## Signif. codes:  0 '***' 0.001 '**' 0.01 '*' 0.05 '.' 0.1 ' ' 1
\end{verbatim}
\end{kframe}
\end{knitrout}
\end{frame}


\section{Follow-up}


%% \begin{frame}[fragile]
%%   \frametitle{Follow-up}
%%   \footnotesize
%% <<>>=
%% summary(aov(species ~ precip, data=richData, subset=temp=="Hot"))
%% @
%% \pause
%% \vfill
%% <<>>=
%% summary(aov(species ~ precip, data=richData, subset=temp=="Cold"))
%% @
%% \end{frame}



\begin{frame}[fragile]
  \frametitle{Follow-up}
  \footnotesize
  {\centering \large \bf Assess effect of food for each level of predator
    factor \\}
\begin{knitrout}
\definecolor{shadecolor}{rgb}{0.878, 0.918, 0.933}\color{fgcolor}\begin{kframe}
\begin{alltt}
\hlkwd{summary}\hlstd{(}\hlkwd{aov}\hlstd{(voles} \hlopt{~} \hlstd{food,} \hlkwc{data}\hlstd{=voleData,} \hlkwc{subset}\hlstd{=predators}\hlopt{==}\hlstr{"Present"}\hlstd{))}
\end{alltt}
\begin{verbatim}
##             Df Sum Sq Mean Sq F value  Pr(>F)   
## food         2 193.50   96.75   14.82 0.00142 **
## Residuals    9  58.75    6.53                   
## ---
## Signif. codes:  0 '***' 0.001 '**' 0.01 '*' 0.05 '.' 0.1 ' ' 1
\end{verbatim}
\end{kframe}
\end{knitrout}
\pause
\vfill
\begin{knitrout}
\definecolor{shadecolor}{rgb}{0.878, 0.918, 0.933}\color{fgcolor}\begin{kframe}
\begin{alltt}
\hlkwd{summary}\hlstd{(}\hlkwd{aov}\hlstd{(voles} \hlopt{~} \hlstd{food,} \hlkwc{data}\hlstd{=voleData,} \hlkwc{subset}\hlstd{=predators}\hlopt{==}\hlstr{"Absent"}\hlstd{))}
\end{alltt}
\begin{verbatim}
##             Df Sum Sq Mean Sq F value   Pr(>F)    
## food         2   1662   831.0   31.42 8.71e-05 ***
## Residuals    9    238    26.4                     
## ---
## Signif. codes:  0 '***' 0.001 '**' 0.01 '*' 0.05 '.' 0.1 ' ' 1
\end{verbatim}
\end{kframe}
\end{knitrout}
\end{frame}




\begin{frame}[fragile]
  \frametitle{Tukey's HSD}
  \tiny %\scriptsize
\begin{knitrout}
\definecolor{shadecolor}{rgb}{0.878, 0.918, 0.933}\color{fgcolor}\begin{kframe}
\begin{alltt}
\hlkwd{TukeyHSD}\hlstd{(aov1)}
\end{alltt}
\begin{verbatim}
##   Tukey multiple comparisons of means
##     95% family-wise confidence level
## 
## Fit: aov(formula = voles ~ food * predators, data = voleData)
## 
## $food
##       diff       lwr       upr     p adj
## 1-0 13.875  8.693714 19.056286 0.0000061
## 2-0 17.250 12.068714 22.431286 0.0000003
## 2-1  3.375 -1.806286  8.556286 0.2464315
## 
## $predators
##                  diff       lwr       upr p adj
## Present-Absent -12.75 -16.23252 -9.267482 4e-07
## 
## $`food:predators`
##                       diff         lwr        upr     p adj
## 1:Absent-0:Absent    18.00   8.8756323  27.124368 0.0000827
## 2:Absent-0:Absent    28.50  19.3756323  37.624368 0.0000001
## 0:Present-0:Absent   -2.50 -11.6243677   6.624368 0.9487798
## 1:Present-0:Absent    7.25  -1.8743677  16.374368 0.1684043
## 2:Present-0:Absent    3.50  -5.6243677  12.624368 0.8221335
## 2:Absent-1:Absent    10.50   1.3756323  19.624368 0.0189039
## 0:Present-1:Absent  -20.50 -29.6243677 -11.375632 0.0000154
## 1:Present-1:Absent  -10.75 -19.8743677  -1.625632 0.0157740
## 2:Present-1:Absent  -14.50 -23.6243677  -5.375632 0.0010013
## 0:Present-2:Absent  -31.00 -40.1243677 -21.875632 0.0000000
## 1:Present-2:Absent  -21.25 -30.3743677 -12.125632 0.0000095
## 2:Present-2:Absent  -25.00 -34.1243677 -15.875632 0.0000010
## 1:Present-0:Present   9.75   0.6256323  18.874368 0.0323125
## 2:Present-0:Present   6.00  -3.1243677  15.124368 0.3351103
## 2:Present-1:Present  -3.75 -12.8743677   5.374368 0.7780829
\end{verbatim}
\end{kframe}
\end{knitrout}
\end{frame}



\section{Graphics}



\begin{frame}[fragile]
  \frametitle{Compute group means and SEs}
  \tiny
\begin{knitrout}
\definecolor{shadecolor}{rgb}{0.878, 0.918, 0.933}\color{fgcolor}\begin{kframe}
\begin{alltt}
\hlstd{ybar_ij.SE} \hlkwb{<-} \hlkwd{model.tables}\hlstd{(aov1,} \hlkwc{type}\hlstd{=}\hlstr{"means"}\hlstd{,} \hlkwc{se}\hlstd{=}\hlnum{TRUE}\hlstd{)}
\hlstd{ybar_ij.SE}
\end{alltt}
\begin{verbatim}
## Tables of means
## Grand mean
##        
## 22.625 
## 
##  food 
## food
##      0      1      2 
## 12.250 26.125 29.500 
## 
##  predators 
## predators
##  Absent Present 
##   29.00   16.25 
## 
##  food:predators 
##     predators
## food Absent Present
##    0 13.50  11.00  
##    1 31.50  20.75  
##    2 42.00  17.00  
## 
## Standard errors for differences of means
##          food predators food:predators
##         2.030     1.658          2.871
## replic.     8        12              4
\end{verbatim}
\end{kframe}
\end{knitrout}
\end{frame}




\begin{frame}[fragile]
  \frametitle{Extract group means and SEs}
  {\bf Group means}
\begin{knitrout}
\definecolor{shadecolor}{rgb}{0.878, 0.918, 0.933}\color{fgcolor}\begin{kframe}
\begin{alltt}
\hlstd{ybar_ij.} \hlkwb{<-} \hlstd{ybar_ij.SE}\hlopt{$}\hlstd{tables}\hlopt{$}\hlstr{"food:predators"}
\hlstd{ybar_ij.}
\end{alltt}
\begin{verbatim}
##     predators
## food Absent Present
##    0 13.50  11.00  
##    1 31.50  20.75  
##    2 42.00  17.00
\end{verbatim}
\end{kframe}
\end{knitrout}

\pause
\vfill

{\bf Standard error}
\begin{knitrout}
\definecolor{shadecolor}{rgb}{0.878, 0.918, 0.933}\color{fgcolor}\begin{kframe}
\begin{alltt}
\hlstd{SE_ij.} \hlkwb{<-} \hlkwd{as.numeric}\hlstd{(ybar_ij.SE}\hlopt{$}\hlstd{se}\hlopt{$}\hlstr{"food:predators"}\hlstd{)}
\hlstd{SE_ij.}
\end{alltt}
\begin{verbatim}
## [1] 2.871072
\end{verbatim}
\end{kframe}
\end{knitrout}
\end{frame}




%% \begin{frame}[fragile]
%%   \frametitle{Plot group means and SEs}
%% %  \tiny
%% <<ybarSE1,fig=TRUE,include=FALSE,width=8,height=6,echo=false>>=
%% plot(1:6, ybar_ij., xaxt="n", xlim=c(0.5, 6.5), ylim=c(0, 50), cex=1.5, cex.lab=1.5,
%%      pch=16, col=c("blue","blue","blue","black","black","black"),
%%      xlab="", ylab="Voles")
%% axis(1, 1:6, labels=c("Absent-0", "Absent-1", "Absent-2",
%%                       "Present-0", "Present-1", "Present-2"))
%% arrows(1:6, ybar_ij.-SE_ij., 1:6, ybar_ij.+SE_ij., code=3, angle=90, length=0.05, lwd=2)
%% @
%% %\begin{center}
%%   \includegraphics[width=0.95\textwidth]{lab07-factorial-ybarSE1}
%% %\end{center}
%% \end{frame}





\begin{frame}[fragile]
  \frametitle{Plot group means and SEs}
  \tiny

\begin{center}
  \includegraphics[width=0.7\textwidth]{figure/ybarSE2-1}
\end{center}
\end{frame}




\begin{frame}[fragile]
  \frametitle{Plot group means and SEs}
  \tiny

\begin{center}
  \includegraphics[width=0.8\textwidth]{figure/ybarSE3-1}
\end{center}
\end{frame}





\begin{frame}
%  \frametitle{Assignment - Due prior to lab next week}
  \frametitle{In-class exercise}
\scriptsize %\tiny

{\bf Fictitious Scenario \par}
Acid rain has lowered the pH of many
lakes in the northeastern United States, and as a result, fish
populations have declined. Managers have resorted to aerial
applications of lime (powdered calcium carbonate) in hopes of
increasing pH. To determine if lime applications result in increased
pH, they applied equal amounts of lime to 15 lakes, and as a control,
they applied the same amount of inert white powder to an additional 15
lakes. Researchers suspected that the effect of lime might depend upon
the buffering effects of the underlying bedrock. To assess this
hypothesis, the 30 lakes were chosen such that 10 had limestone
bedrock, 10 had granite bedrock, and 10 had shist bedrock. pH was
measured before and after each application, and the difference in pH
is recorded in the file ``acidityData.csv.'' \par
\pause
\vfill
{\bf Questions}
\begin{enumerate}[{\bf 1}]
  \item What are the null and alternative hypotheses?
  \item Test the null hypotheses using an $A \times B$ factorial ANOVA
    implemented with {\tt aov}. Create an ANOVA table using {\tt summary}.
  \item Does the effect of lime depend upon the bedrock type? If so,
    how? Answer this question by plotting the estimates of the effect of lime on
    pH change. Include 95\% confidence intervals.
\end{enumerate}
\pause
\vfill
Put all of your answers in an R script. Your script should be
self-contained, such that we can copy and paste your code into the R
console. Use comments (\#) to respond to questions and interpret results.

\end{frame}





\end{document}
