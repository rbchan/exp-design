\documentclass[color=usenames,dvipsnames]{beamer}\usepackage[]{graphicx}\usepackage[]{color}
%% maxwidth is the original width if it is less than linewidth
%% otherwise use linewidth (to make sure the graphics do not exceed the margin)
\makeatletter
\def\maxwidth{ %
  \ifdim\Gin@nat@width>\linewidth
    \linewidth
  \else
    \Gin@nat@width
  \fi
}
\makeatother

\definecolor{fgcolor}{rgb}{0, 0, 0}
\newcommand{\hlnum}[1]{\textcolor[rgb]{0.69,0.494,0}{#1}}%
\newcommand{\hlstr}[1]{\textcolor[rgb]{0.749,0.012,0.012}{#1}}%
\newcommand{\hlcom}[1]{\textcolor[rgb]{0.514,0.506,0.514}{\textit{#1}}}%
\newcommand{\hlopt}[1]{\textcolor[rgb]{0,0,0}{#1}}%
\newcommand{\hlstd}[1]{\textcolor[rgb]{0,0,0}{#1}}%
\newcommand{\hlkwa}[1]{\textcolor[rgb]{0,0,0}{\textbf{#1}}}%
\newcommand{\hlkwb}[1]{\textcolor[rgb]{0,0.341,0.682}{#1}}%
\newcommand{\hlkwc}[1]{\textcolor[rgb]{0,0,0}{\textbf{#1}}}%
\newcommand{\hlkwd}[1]{\textcolor[rgb]{0.004,0.004,0.506}{#1}}%
\let\hlipl\hlkwb

\usepackage{framed}
\makeatletter
\newenvironment{kframe}{%
 \def\at@end@of@kframe{}%
 \ifinner\ifhmode%
  \def\at@end@of@kframe{\end{minipage}}%
  \begin{minipage}{\columnwidth}%
 \fi\fi%
 \def\FrameCommand##1{\hskip\@totalleftmargin \hskip-\fboxsep
 \colorbox{shadecolor}{##1}\hskip-\fboxsep
     % There is no \\@totalrightmargin, so:
     \hskip-\linewidth \hskip-\@totalleftmargin \hskip\columnwidth}%
 \MakeFramed {\advance\hsize-\width
   \@totalleftmargin\z@ \linewidth\hsize
   \@setminipage}}%
 {\par\unskip\endMakeFramed%
 \at@end@of@kframe}
\makeatother

\definecolor{shadecolor}{rgb}{.97, .97, .97}
\definecolor{messagecolor}{rgb}{0, 0, 0}
\definecolor{warningcolor}{rgb}{1, 0, 1}
\definecolor{errorcolor}{rgb}{1, 0, 0}
\newenvironment{knitrout}{}{} % an empty environment to be redefined in TeX

\usepackage{alltt}
%\documentclass[color=usenames,dvipsnames,handout]{beamer}


%\usepackage[roman]{../../lab1}
\usepackage[sans]{../../lab1}

\hypersetup{pdftex,pdfstartview=FitV}









%% New command for inline code that isn't to be evaluated
\definecolor{inlinecolor}{rgb}{0.878, 0.918, 0.933}
\newcommand{\inr}[1]{\colorbox{inlinecolor}{\texttt{#1}}}
\IfFileExists{upquote.sty}{\usepackage{upquote}}{}
\begin{document}






\begin{frame}[plain]
%  \LARGE
  \huge
  \centering \par
  {\color{RoyalBlue}{Lab 9 -- Split-plot Designs}} \\
  \vspace{1cm}
  \Large
  October 15 \& 16, 2018 \\
  FANR 6750 \\
  \vfill
  \large
  Richard Chandler and Bob Cooper
\end{frame}





\section{Intro}

% \begin{frame}[plain]
%   \frametitle{Today's Topics}
%   \Large
%   \only<1>{\tableofcontents}%[hideallsubsections]}
% %  \only<2 | handout:0>{\tableofcontents[currentsection]}%,hideallsubsections]}
% \end{frame}




\begin{frame}
  \frametitle{Scenario}
  \large
%  \begin{itemize}[<+->]
%    \item
  We apply treatments to two experimental units: whole-units and
  sub-units
  \pause
  \vfill
%    \item
  Examples:
  \begin{itemize}
    \large
    \item Ag fields are sprayed with herbicides, and fertilizers
      are applied to plots within fields.
    \item Tenderizer is applied to roasts, and cooking times are
      applied to cores
  \end{itemize}
  \pause
  \vfill
%  \item
  We're interested in treatment effects at both levels
%  \end{itemize}
\end{frame}






\begin{frame}
  \frametitle{The additive model}
%  \Large
%\large
\[
y_{ijk} = \mu + \alpha_i + \beta_{j} + \alpha\beta_{ij} + \gamma_k +
\delta_{ik} + \varepsilon_{ijk}
\] %\par
The $\alpha$'s and $\beta$'s are fixed treatment effects. Note the interaction.
%\vspace{0.4cm}
\pause
\vfill
Because we want our inferences to apply to all roasts,
$\delta_{ik}$ is random. Specifically:
%\Large
\[
\delta_{ik} \sim \mbox{Normal}(0, \sigma^2_D)
\]
We might treat block effects as random too:
\[
\gamma_k \sim \mbox{Normal}(0, \sigma^2_C)
\]
\large
And as always,
%\Large
\[
\varepsilon_{ijk} \sim \mbox{Normal}(0, \sigma^2)
\]
\end{frame}





\section{aov}


\begin{frame}[fragile]
  \frametitle{The meat data}
%  {\bf
    It's important that you have variables for the whole unit (roast)
    and the block (carcass).
%  }
\small %\footnotesize
\begin{knitrout}\scriptsize
\definecolor{shadecolor}{rgb}{0.878, 0.918, 0.933}\color{fgcolor}\begin{kframe}
\begin{alltt}
\hlstd{meatData} \hlkwb{<-} \hlkwd{read.csv}\hlstd{(}\hlstr{"meatData.csv"}\hlstd{)}
\hlkwd{str}\hlstd{(meatData)}
\end{alltt}
\begin{verbatim}
## 'data.frame':	72 obs. of  5 variables:
##  $ Wbscore   : num  8.25 7.5 4.25 3.5 7.25 6.25 3.5 3.5 6.5 4.5 ...
##  $ tenderizer: Factor w/ 3 levels "C","P","V": 1 1 1 1 3 3 3 3 2 2 ...
##  $ time      : int  30 36 42 48 30 36 42 48 30 36 ...
##  $ carcass   : int  1 1 1 1 1 1 1 1 1 1 ...
##  $ roast     : int  1 1 1 1 2 2 2 2 3 3 ...
\end{verbatim}
\end{kframe}
\end{knitrout}
\pause
%{\bf
  Don't forget to convert to factors %}
\begin{knitrout}\footnotesize
\definecolor{shadecolor}{rgb}{0.878, 0.918, 0.933}\color{fgcolor}\begin{kframe}
\begin{alltt}
\hlstd{meatData}\hlopt{$}\hlstd{time} \hlkwb{<-} \hlkwd{factor}\hlstd{(meatData}\hlopt{$}\hlstd{time)}
\hlstd{meatData}\hlopt{$}\hlstd{carcass} \hlkwb{<-} \hlkwd{factor}\hlstd{(meatData}\hlopt{$}\hlstd{carcass)}
\hlstd{meatData}\hlopt{$}\hlstd{roast} \hlkwb{<-} \hlkwd{factor}\hlstd{(meatData}\hlopt{$}\hlstd{roast)}
\end{alltt}
\end{kframe}
\end{knitrout}
\end{frame}



%% \begin{frame}[fragile]
%%   \frametitle{The meat data}
%% %  \pause
%% %  \vfill
%% <<>>=
%% head(meatData, n=10)
%% @
%% \end{frame}






\begin{frame}[fragile]
  \frametitle{Carcass (block) effects as fixed}
\footnotesize
%{\bf Only 1 \verb+Error()+ term allowed in \verb+aov+
\begin{knitrout}\footnotesize
\definecolor{shadecolor}{rgb}{0.878, 0.918, 0.933}\color{fgcolor}\begin{kframe}
\begin{alltt}
\hlstd{aov.meat1} \hlkwb{<-} \hlkwd{aov}\hlstd{(Wbscore} \hlopt{~} \hlstd{tenderizer} \hlopt{*} \hlstd{time} \hlopt{+} \hlstd{carcass} \hlopt{+}
                 \hlkwd{Error}\hlstd{(roast),} \hlkwc{data}\hlstd{=meatData)}
\end{alltt}
\end{kframe}
\end{knitrout}
\pause
\begin{knitrout}\scriptsize
\definecolor{shadecolor}{rgb}{0.878, 0.918, 0.933}\color{fgcolor}\begin{kframe}
\begin{alltt}
\hlkwd{summary}\hlstd{(aov.meat1)}
\end{alltt}
\begin{verbatim}
## 
## Error: roast
##            Df Sum Sq Mean Sq F value   Pr(>F)    
## tenderizer  2 20.715  10.358  190.00 1.11e-08 ***
## carcass     5  3.903   0.781   14.32 0.000276 ***
## Residuals  10  0.545   0.055                     
## ---
## Signif. codes:  0 '***' 0.001 '**' 0.01 '*' 0.05 '.' 0.1 ' ' 1
## 
## Error: Within
##                 Df Sum Sq Mean Sq F value   Pr(>F)    
## time             3 170.08   56.69  656.62  < 2e-16 ***
## tenderizer:time  6   9.56    1.59   18.46 1.11e-10 ***
## Residuals       45   3.89    0.09                     
## ---
## Signif. codes:  0 '***' 0.001 '**' 0.01 '*' 0.05 '.' 0.1 ' ' 1
\end{verbatim}
\end{kframe}
\end{knitrout}
\end{frame}








%% \begin{frame}[fragile]
%%   \frametitle{Carcass (block) effects as random}
%%   \scriptsize
%% <<>>=
%% aov.meat2 <- aov(Wbscore ~ tenderizer * time +
%%                  Error(carcass/roast), data=meatData)
%% summary(aov.meat2)
%% @
%% \pause
%%   {\centering \bf Warning suggests that results may not be
%%     reliable. Better to use \inr{ lme} function, or just treat block
%%     effects as fixed in this case. \\}
%% \end{frame}






%% \begin{frame}[fragile]
%%   \frametitle{Effect sizes and standard errors}
%%   \scriptsize %\tiny %\footnotesize
%% <<>>=
%% model.tables(aov.meat2, type="effects", se=TRUE)
%% @
%% \end{frame}




%% \begin{frame}[fragile]
%%   \frametitle{The expected values (means)}
%%   \scriptsize %\tiny %\footnotesize
%% <<>>=
%% model.tables(aov.meat2, type="means") # Doesn't allow se=TRUE
%% @
%% \end{frame}




%% \begin{frame}[fragile]
%%   \frametitle{Multiple comparisons}
%% %  {\bf
%%   Because the interaction is significant, we need to control one
%%   factor when doing multiple comparisons of the other %}
%% %  Do a seperate ANOVA for each level of factor A (tenderizer in
%% %    this example)}
%%   \vfill
%% <<>>=
%% aov.meatP <- aov(Wbscore ~ time + carcass,
%%                  data=meatData,
%%                  subset = tenderizer=="P")
%% aov.meatV <- aov(Wbscore ~ time + carcass,
%%                  data=meatData,
%%                  subset = tenderizer=="V")
%% @
%% \pause
%% \vfill
%% <<fig=false,eval=false>>=
%% plot(TukeyHSD(aov.meatP, which="time"))
%% plot(TukeyHSD(aov.meatP, which="carcass"))
%% plot(TukeyHSD(aov.meatV, which="time"))
%% @
%% \end{frame}




\section{lme}




\begin{frame}[fragile]
  \frametitle{Using {\tt lme} instead of {\tt aov}}
\small
\begin{knitrout}\footnotesize
\definecolor{shadecolor}{rgb}{0.878, 0.918, 0.933}\color{fgcolor}\begin{kframe}
\begin{alltt}
\hlkwd{library}\hlstd{(nlme)}
\hlstd{lme.meat1} \hlkwb{<-} \hlkwd{lme}\hlstd{(Wbscore} \hlopt{~} \hlstd{tenderizer}\hlopt{*}\hlstd{time,} \hlkwc{data}\hlstd{=meatData,}
    \hlkwc{correlation}\hlstd{=}\hlkwd{corCompSymm}\hlstd{(),} \hlcom{# To make results same as aov()}
    \hlkwc{random} \hlstd{=} \hlopt{~}\hlnum{1}\hlopt{|}\hlstd{carcass}\hlopt{/}\hlstd{roast)}
\end{alltt}
\end{kframe}
\end{knitrout}
\pause
\begin{knitrout}\footnotesize
\definecolor{shadecolor}{rgb}{0.878, 0.918, 0.933}\color{fgcolor}\begin{kframe}
\begin{alltt}
\hlkwd{anova}\hlstd{(lme.meat1)}
\end{alltt}
\begin{verbatim}
##                 numDF denDF   F-value p-value
## (Intercept)         1    45 2105.6140  <.0001
## tenderizer          2    10  190.0001  <.0001
## time                3    45  656.6222  <.0001
## tenderizer:time     6    45   18.4585  <.0001
\end{verbatim}
\end{kframe}
\end{knitrout}
The interaction is significant. (You can Ignore the {\tt (Intercept)} term)
\end{frame}





\begin{frame}[fragile]
  \frametitle{Exploring the interaction}
  \small
%{\bf
  Is the time effect significant for each level of tenderizer?
%  \tiny %scriptsize
\begin{knitrout}\tiny
\definecolor{shadecolor}{rgb}{0.878, 0.918, 0.933}\color{fgcolor}\begin{kframe}
\begin{alltt}
\hlstd{lme.meatP} \hlkwb{<-} \hlkwd{lme}\hlstd{(Wbscore} \hlopt{~} \hlstd{time,} \hlkwc{data}\hlstd{=meatData,} \hlkwc{random} \hlstd{=} \hlopt{~}\hlnum{1}\hlopt{|}\hlstd{carcass}\hlopt{/}\hlstd{roast,}
                 \hlkwc{correlation}\hlstd{=}\hlkwd{corCompSymm}\hlstd{(),} \hlkwc{subset}\hlstd{=tenderizer}\hlopt{==}\hlstr{"P"}\hlstd{)}
\hlstd{lme.meatV} \hlkwb{<-} \hlkwd{lme}\hlstd{(Wbscore} \hlopt{~} \hlstd{time,} \hlkwc{data}\hlstd{=meatData,} \hlkwc{random} \hlstd{=} \hlopt{~}\hlnum{1}\hlopt{|}\hlstd{carcass}\hlopt{/}\hlstd{roast,}
                 \hlkwc{correlation}\hlstd{=}\hlkwd{corCompSymm}\hlstd{(),} \hlkwc{subset}\hlstd{=tenderizer}\hlopt{==}\hlstr{"V"}\hlstd{)}
\hlstd{lme.meatC} \hlkwb{<-} \hlkwd{lme}\hlstd{(Wbscore} \hlopt{~} \hlstd{time,} \hlkwc{data}\hlstd{=meatData,} \hlkwc{random} \hlstd{=} \hlopt{~}\hlnum{1}\hlopt{|}\hlstd{carcass}\hlopt{/}\hlstd{roast,}
                 \hlkwc{correlation}\hlstd{=}\hlkwd{corCompSymm}\hlstd{(),} \hlkwc{subset}\hlstd{=tenderizer}\hlopt{==}\hlstr{"C"}\hlstd{)}
\end{alltt}
\end{kframe}
\end{knitrout}
\pause
\vfill
\small %\scriptsize
Yes, it is
\begin{knitrout}\tiny
\definecolor{shadecolor}{rgb}{0.878, 0.918, 0.933}\color{fgcolor}\begin{kframe}
\begin{alltt}
\hlkwd{anova}\hlstd{(lme.meatP,} \hlkwc{Terms}\hlstd{=}\hlstr{"time"}\hlstd{)}
\end{alltt}
\begin{verbatim}
## F-test for: time 
##   numDF denDF  F-value p-value
## 1     3    15 126.6786  <.0001
\end{verbatim}
\begin{alltt}
\hlkwd{anova}\hlstd{(lme.meatV,} \hlkwc{Terms}\hlstd{=}\hlstr{"time"}\hlstd{)}
\end{alltt}
\begin{verbatim}
## F-test for: time 
##   numDF denDF F-value p-value
## 1     3    15 274.716  <.0001
\end{verbatim}
\begin{alltt}
\hlkwd{anova}\hlstd{(lme.meatC,} \hlkwc{Terms}\hlstd{=}\hlstr{"time"}\hlstd{)}
\end{alltt}
\begin{verbatim}
## F-test for: time 
##   numDF denDF  F-value p-value
## 1     3    15 305.6551  <.0001
\end{verbatim}
\end{kframe}
\end{knitrout}

\end{frame}



\begin{frame}[fragile]
  \frametitle{Multiple comparisons using {\tt glht}}
\scriptsize %\tiny
%{\bf Tukey's tests using \inr{ glht}}
\begin{knitrout}\tiny
\definecolor{shadecolor}{rgb}{0.878, 0.918, 0.933}\color{fgcolor}\begin{kframe}
\begin{alltt}
\hlcom{# install.packages("multcomp")}
\hlkwd{library}\hlstd{(multcomp)}
\hlstd{mcP} \hlkwb{<-} \hlkwd{glht}\hlstd{(lme.meatP,} \hlkwc{linfct}\hlstd{=}\hlkwd{mcp}\hlstd{(}\hlkwc{time}\hlstd{=}\hlstr{"Tukey"}\hlstd{))}
\hlkwd{summary}\hlstd{(mcP)}
\end{alltt}
\begin{verbatim}
## 
## 	 Simultaneous Tests for General Linear Hypotheses
## 
## Multiple Comparisons of Means: Tukey Contrasts
## 
## 
## Fit: lme.formula(fixed = Wbscore ~ time, data = meatData, random = ~1 | 
##     carcass/roast, correlation = corCompSymm(), subset = tenderizer == 
##     "P")
## 
## Linear Hypotheses:
##              Estimate Std. Error z value Pr(>|z|)    
## 36 - 30 == 0   -1.250      0.180  -6.944   <1e-05 ***
## 42 - 30 == 0   -2.333      0.180 -12.961   <1e-05 ***
## 48 - 30 == 0   -3.333      0.180 -18.516   <1e-05 ***
## 42 - 36 == 0   -1.083      0.180  -6.018   <1e-05 ***
## 48 - 36 == 0   -2.083      0.180 -11.573   <1e-05 ***
## 48 - 42 == 0   -1.000      0.180  -5.555   <1e-05 ***
## ---
## Signif. codes:  0 '***' 0.001 '**' 0.01 '*' 0.05 '.' 0.1 ' ' 1
## (Adjusted p values reported -- single-step method)
\end{verbatim}
\begin{alltt}
\hlcom{# confint(mcP)}
\hlcom{# plot(mcP)}
\end{alltt}
\end{kframe}
\end{knitrout}
\end{frame}



\begin{frame}[fragile]
  \frametitle{\small Differences between cooking times for ``P'' tenderizer}
  \centering \footnotesize %\small

\includegraphics[width=0.9\textwidth]{figure/mcp-1} \\
\end{frame}




\section{Assignment}




\begin{frame}[fragile]
  \frametitle{Nested and crossed assignment\footnote{Nested and crossed
      is same as split-plot, but without the block}}

%{%\centering %\large %\bf
%  \large
%  Nested and crossed is same as split-plot, but without the
%  block \\}
%\normalsize
%\pause
%\vfill
\small
{\bf Design}
\begin{itemize}%[<+- | visible@+->]
\small
  \item Sweet potato yield is studied in response to (a=3)
    types of herbicide.
  \item Each herbicide is applied to 5 fields
  \item Each field is divided into 4 plots. Each plot is treated with
    one of (b=4) fertilizers.
\end{itemize}
%\pause
%\vfill
{\bf Exercise}
\begin{enumerate}[(1)]%[<+- | visible@+->][\bf \color{PineGreen} (1)]
\small
  \item Import {\tt yieldData.csv} and conduct
    the appropriate ANOVA using \inr{aov} and \inr{lme}.
  \item Does the effect of herbicide depend on fertilizer?
  \item Use Tukey's test to determine which fertilizers differ
  \item State the null and alternative hypotheses, and summarize your
    results in 2-3 sentences. 
\end{enumerate}
%\vfill
%\centering
%\small
\footnotesize
\bf
Upload your self-contained script to ELC the day before your next
lab. \\
\vfill
\end{frame}






%% \begin{frame}
%%   \frametitle{Exercise II}
%%   \begin{itemize}
%%     \item Use \inr{ lme} to test for interaction between herbicide and
%%       treatment
%%     \item What is $\sigma^2_D$ (variance among fields) and $\sigma^2$
%%       (within field variance)?
%%     \item Fit model without interaction and use \inr{ glht} to do Tukey
%%       tests
%%   \end{itemize}
%% \end{frame}




\end{document}









%% \begin{frame}[fragile]
%%   \frametitle{Yield data}
%% <<eval=TRUE,echo=FALSE>>=
%% alpha <- c(-2, 0, 2)
%% beta <- c(-2, -2, 0, 4)
%% a <- 3     # number of treatments
%% b <- 5     # number of fields per treatment
%% c <- 4     # number of plots per field
%% alpha.long <- rep(alpha, each=b*c)
%% beta.long <- rep(beta, times=a*b)
%% sigmaSqB <- 1
%% delta <- rnorm(a*b)
%% delta.long <- rep(delta, each=c)
%% sigmaSq <- 0.5
%% epsilon <- rnorm(a*b*c, 0, sigmaSq)
%% y <- alpha.long + beta.long + delta.long + epsilon
%% herb <- factor(rep(c("1", "2", "3"), each=b*c))
%% fert <- factor(rep(c("A","B","C","D"), times=a*b))
%% field <- factor(rep(paste("field", 1:(a*b), sep=""), each=c))
%% yieldData <- data.frame(yield=y,
%%                         herbicide=herb,
%%                         fertilizer=fert,
%%                         field=field)

%% aov.yield1 <- aov(yield ~ herbicide * fertilizer +
%%                   Error(field),
%%                   data=yieldData)
%% summary(aov.yield1)
%% aov.yield2 <- aov(yield ~ herbicide + fertilizer +
%%                   Error(field),
%%                   data=yieldData)
%% summary(aov.yield2)
%% library(nlme)
%% lme.yield1 <- lme(yield ~ herbicide * fertilizer,
%%                   data=yieldData,
%%                   random=~1|field)
%% anova(lme.yield1)
%% lme.yield2 <- lme(yield ~ herbicide + fertilizer,
%%                   data=yieldData,
%%                   random=~1|field)
%% anova(lme.yield2)
%% library(multcomp)
%% mc <- glht(lme.yield2,
%%            linfct=mcp(herbicide="Tukey",
%%                       fertilizer="Tukey"))
%% summary(mc)
%% confint(mc)
%% plot(mc)

%% #allcomb <- expand.grid(herbicide=levels(yieldData$herbicide),
%% #                       fertilizer=levels(yieldData$fertilizer))
%% #X <- model.matrix(~herbicide*fertilizer+carcass, allcomb)
%% #glht(lme.yield1, linfct=X)
%% @
%% \end{frame}






%% \begin{frame}[fragile]
%%   \frametitle{Multiple comparisons}
%% <<>>=
%% tender <- meatData$tenderizer
%% tender.PorV <- tender %in% c("P", "V")
%% tender.C <- tender == "C"
%% mean(meatData$Wbscore[tender.PorV])
%% mean(meatData$Wbscore[tender.C])
%% se.contrast(aov2, list(tender.PorV, tender.C))
%% @
%% \end{frame}










%% \begin{frame}[fragile]
%%   \frametitle{Predict}
%% <<>>=
%% comb <- data.frame(tenderizer=factor(rep(c("P","V","C"), each=4)),
%%                    time=factor(rep(c("30", "36", "42", "48"))),
%%                    carcass="1")
%% comb$E <- predict(lme1, newdata=comb, level=0)
%% comb
%% @
%% \end{frame}





%% \begin{comment}
%% \begin{frame}%[fragile]
%%   \frametitle{Variance estimates}
%% {\bf Variance within roasts}
%% \[
%% \hat{\sigma^2} = 0.09
%% \]
%% \pause
%% {\bf Variance among roasts}
%% \[
%% \hat{\sigma^2_D} = ({MS_{WUR}} - \sigma^2)/b
%% \]
%% \pause
%% \[
%% \hat{\sigma^2_D} = (0.055 - 0.09)/4 = -0.00875
%% \]
%% {\bf Variance among carcasses}
%% \[
%% \hat{\sigma^2_C} = ({MS}_{C} - \sigma^2 - b\sigma^2_D)/(ab)
%% \]
%% \pause
%% \[
%% \hat{\sigma^2_C} = (0.781 - 0.09 - 4\times0)/12 = 0.057
%% \]
%% \end{frame}
%% \end{comment}

