\documentclass[color=usenames,dvipsnames]{beamer}\usepackage[]{graphicx}\usepackage[]{color}
%% maxwidth is the original width if it is less than linewidth
%% otherwise use linewidth (to make sure the graphics do not exceed the margin)
\makeatletter
\def\maxwidth{ %
  \ifdim\Gin@nat@width>\linewidth
    \linewidth
  \else
    \Gin@nat@width
  \fi
}
\makeatother

\definecolor{fgcolor}{rgb}{0, 0, 0}
\newcommand{\hlnum}[1]{\textcolor[rgb]{0.69,0.494,0}{#1}}%
\newcommand{\hlstr}[1]{\textcolor[rgb]{0.749,0.012,0.012}{#1}}%
\newcommand{\hlcom}[1]{\textcolor[rgb]{0.514,0.506,0.514}{\textit{#1}}}%
\newcommand{\hlopt}[1]{\textcolor[rgb]{0,0,0}{#1}}%
\newcommand{\hlstd}[1]{\textcolor[rgb]{0,0,0}{#1}}%
\newcommand{\hlkwa}[1]{\textcolor[rgb]{0,0,0}{\textbf{#1}}}%
\newcommand{\hlkwb}[1]{\textcolor[rgb]{0,0.341,0.682}{#1}}%
\newcommand{\hlkwc}[1]{\textcolor[rgb]{0,0,0}{\textbf{#1}}}%
\newcommand{\hlkwd}[1]{\textcolor[rgb]{0.004,0.004,0.506}{#1}}%
\let\hlipl\hlkwb

\usepackage{framed}
\makeatletter
\newenvironment{kframe}{%
 \def\at@end@of@kframe{}%
 \ifinner\ifhmode%
  \def\at@end@of@kframe{\end{minipage}}%
  \begin{minipage}{\columnwidth}%
 \fi\fi%
 \def\FrameCommand##1{\hskip\@totalleftmargin \hskip-\fboxsep
 \colorbox{shadecolor}{##1}\hskip-\fboxsep
     % There is no \\@totalrightmargin, so:
     \hskip-\linewidth \hskip-\@totalleftmargin \hskip\columnwidth}%
 \MakeFramed {\advance\hsize-\width
   \@totalleftmargin\z@ \linewidth\hsize
   \@setminipage}}%
 {\par\unskip\endMakeFramed%
 \at@end@of@kframe}
\makeatother

\definecolor{shadecolor}{rgb}{.97, .97, .97}
\definecolor{messagecolor}{rgb}{0, 0, 0}
\definecolor{warningcolor}{rgb}{1, 0, 1}
\definecolor{errorcolor}{rgb}{1, 0, 0}
\newenvironment{knitrout}{}{} % an empty environment to be redefined in TeX

\usepackage{alltt}
%\documentclass[color=usenames,dvipsnames,handout]{beamer}

%\usepackage[roman]{../../lab1}
\usepackage[sans]{../../lab1}


\hypersetup{pdftex,pdfstartview=FitV}








%% New command for inline code that isn't to be evaluated
\definecolor{inlinecolor}{rgb}{0.878, 0.918, 0.933}
\newcommand{\inr}[1]{\colorbox{inlinecolor}{\texttt{#1}}}
\IfFileExists{upquote.sty}{\usepackage{upquote}}{}
\begin{document}


% \setlength\fboxsep{0pt}



\section{Introduction}



\begin{frame}[plain,fragile]
  \LARGE
  \centering \par
%  {\bf Lab 2 -- Summary statistics, graphics, and the $t$-test \par}
  \textcolor{RoyalBlue}{\huge %\bf
    Lab 2 -- Summary statistics,
    graphics, and the $t$-test} \\
  \vspace{1cm}
  August 20 \& 21, 2018 \par
  FANR 6750 \par
  \vfill
  \large
  Richard Chandler and Bob Cooper \\
  University of Georgia \\
\end{frame}




\begin{frame}[plain]
  \frametitle{Recap}
  \Large
  {\bf Last week we covered:}
  \begin{itemize}
    \item Vectors
    \item Data frames
    \item Indexing
    \item Importing and exporting data
    \item Saving and loading workspaces
  \end{itemize}
  \note{Ask students if they have questions about last week's
    assignment or general questions about \R}
\end{frame}






\begin{frame}[plain]
  \frametitle{Today's Topics}
  \Large
  \only<1>{\tableofcontents}%[hideallsubsections]}
%  \only<2 | handout:0>{\tableofcontents[currentsection]}%,hideallsubsections]}
\end{frame}




\begin{frame}[fragile]
  \frametitle{Scenario}
  \small
%  \begin{itemize}%[<+->]
%    \item We have 2 samples of data
%    \item<2-> Questions: Do the samples come from the same %statistical
%      population, or do they come from populations with different
%      means?   %2 populations have the same mean?
%    \item<3-> Problem: We don't know the true population means ($\mu_1, \mu_1$)% -- we only
%      know the sample means ($\bar{y_1}, \bar{y_1}$) and sample standard deviations ($s_1.
    %% \item<4-> Under the assumption that the variances of the two
    %%   populations are equal, the relevant hypotheses are:
    %%   \begin{itemize}
    %%     \item $H_0: \mu_1 = \mu_2$
    %%     \item $H_A: \mu_1 \neq \mu_2$
    %%   \end{itemize}
%  \end{itemize}
  We have 2 samples of data and we want to know if they came from
  population. \\
  \pause
  \vfill

  The problem is that the true population means ($\mu_1, \mu_1$) are
  unknown. \\ %-- we only know the sample means ($\bar{y_1}, \bar{y_1}$)
%  and sample standard deviations ($s_1, s_2$). \\
  \pause
  \vfill

Under the assumption that the variances of the two populations are
equal, the relevant hypotheses are:
\begin{itemize}
  \footnotesize
  \item $H_0: \mu_1 = \mu_2$
  \item $H_A: \mu_1 \neq \mu_2$
\end{itemize}
  \normalsize
  \pause
  \vfill
%  \uncover<4->{
%\begin{center}
\begin{knitrout}
\definecolor{shadecolor}{rgb}{0.878, 0.918, 0.933}\color{fgcolor}
\includegraphics[width=\maxwidth]{figure/pop-1} 

\end{knitrout}
%\end{center}
%}
\end{frame}




\begin{frame}[fragile]
%  \frametitle{Reminder about the $t$-distribution}
%  \frametitle{$t$-distribution with df=18}
  \frametitle{Key points}
%\begin{center}

%\end{center}
\small
%\vspace{-1cm}
%{\bf Key points}
%\begin{itemize}%[<+->]
%  \item<1->
If the two sample means ($\bar{y_1}, \bar{y_2}$) are very
different and the standard error of the difference in means is
small, the $t$ statistic will be far from zero. \\
\pause
\vfill

%  \item<2->
If the $t$ statistic is more extreme than the critical
    values, you reject the null hypothesis ($H_0$). \\
\pause
    \vfill

%\end{itemize}
%\uncover<3>{
%\vspace{-0.1cm}
%\begin{center}
    \centering
  \includegraphics[width=0.6\textwidth]{figure/tdist-1} \\
%\end{center}
%}
\end{frame}


%\section{Summary Statistics}




\begin{frame}[fragile]
  \frametitle{Exercise I}
%  \large
  \begin{enumerate}[\bf (1)]
    \item<1-> Open {\bf R} and set the working directory to a convenient
      location on your computer. Do this using {\tt File > Change dir ...}, or the \inr{setwd} function.
    \item[]
    \item<2-> Put the file {\tt treedata.csv} into your working.
      directory
    \item[]
    \item<3-> Create a new \R~script and import {\tt treedata.csv}. Name your object {\tt treedata}.
    \item[]
    \item<4-> Using the indexing methods we covered last time to
      create 2 objects: \inr{yL} is the tree density data for the
      first 10 experimental units (low elevation), and \inr{yH} is
      the tree density data for the last 10 units (high elevation)
    \item[]
    \item<5-> Compute the mean, variance, and standard deviation of the 2 samples
  \end{enumerate}
\end{frame}









\section{Graphics}



\begin{frame}[plain]
  \frametitle{Today's Topics}
  \Large
  \tableofcontents[currentsection,hideallsubsections]
\end{frame}






\begin{frame}[fragile]
  \frametitle{Boxplots}
  \begin{center}
  \footnotesize

\vspace{-3mm}
\includegraphics[width=3in,height=3in]{figure/boxplot-1}
  \end{center}
\end{frame}





\begin{frame}[fragile]
  \frametitle{Histograms}
  \footnotesize

\begin{knitrout}
\definecolor{shadecolor}{rgb}{0.878, 0.918, 0.933}\color{fgcolor}\begin{kframe}
\begin{alltt}
\hlkwd{hist}\hlstd{(yL,} \hlkwc{xlab}\hlstd{=}\hlstr{"Tree density (low elevation)"}\hlstd{,} \hlkwc{col}\hlstd{=}\hlstr{"lightblue"}\hlstd{,}
     \hlkwc{xlim}\hlstd{=}\hlkwd{c}\hlstd{(}\hlnum{0}\hlstd{,} \hlnum{40}\hlstd{))}
\hlkwd{hist}\hlstd{(yH,} \hlkwc{xlab}\hlstd{=}\hlstr{"Tree density (high elevation)"}\hlstd{,} \hlkwc{col}\hlstd{=}\hlstr{"purple"}\hlstd{,}
     \hlkwc{xlim}\hlstd{=}\hlkwd{c}\hlstd{(}\hlnum{0}\hlstd{,} \hlnum{40}\hlstd{))}
\end{alltt}
\end{kframe}
\end{knitrout}
\includegraphics[width=\textwidth]{figure/histogram-1}
\end{frame}





%% \begin{frame}[fragile]
%%   \frametitle{Mean and error bars}
%%   \tiny %\scriptsize
%% <<meanerror,fig=TRUE,include=FALSE,width=6,fig.height=6>>=
%% y.bar <- c(mean(yL), mean(yH))
%% y.SE <- c(sd(yL)/sqrt(length(yL)), sd(yH)/sqrt(length(yH)))
%% plot(1:2, y.bar, ylim=c(0, 25), xlim=c(0.5, 2.5), xaxt="n", pch=16,
%%      xlab="", ylab="Tree density", cex=1.5)
%% axis(1, 1:2, c("Low elevation", "High elevation"))
%% arrows(1:2, y.bar-y.SE, 1:2, y.bar+y.SE, code=3, angle=90, length=0.05)
%% @
%% \vspace{-1cm}
%% \begin{center}
%%   \includegraphics[width=0.7\textwidth]{lab02-t-test-meanerror}
%% \end{center}
%% \end{frame}




\begin{frame}[fragile]
  \frametitle{Exercise II}
  \centering
  \Large
%  \begin{enumerate}[\bf (1)]
%    \item
      Create the same boxplots and histograms as before, but
      change the colors of the boxplots and the number of break points in
      the histograms. \par
%  \end{enumerate}
\end{frame}



\section{$t$ tests}




\begin{frame}[plain]
  \frametitle{Today's Topics}
  \Large
  \tableofcontents[currentsection,hideallsubsections]
\end{frame}



\subsection{Two-sample $t$ test}



\begin{comment}
\begin{frame}
  \frametitle{Conceptual Example}
  \begin{itemize}
    \item The (unknown) population means are:
      \begin{itemize}
        \item $\mu_A = 5$
        \item $\mu_B = 7$
      \end{itemize}
    \item Both populations have variance $\sigma^2 = 4$.
    \item We collect 2 samples, each with $n=15$
    \item The sample means are:
      \begin{itemize}
        \item $\bar{y_1}$
      \end{itemize}
  \end{itemize}
\end{frame}
\end{comment}



\begin{comment}
\begin{frame}[fragile]
  \frametitle{Example}
%\centering \par
  \vspace{-0.5cm}
\begin{center}
\begin{knitrout}
\definecolor{shadecolor}{rgb}{0.878, 0.918, 0.933}\color{fgcolor}
\includegraphics[width=\maxwidth]{figure/popsamp-1} 

\end{knitrout}
\end{center}
\end{frame}
\end{comment}




\begin{frame}[fragile]
  \frametitle{Two-sample $t$-test with equal variances}
  {\bf Step 1:} Compute the $t$ statistic\footnote{\scriptsize Remember, $H_0$ states that $\mu_L-\mu_H =0$.}:
  \[
  t = \frac{(\bar{y_L} - \bar{y_H}) - (\mu_L - \mu_H)}{
    \sqrt{s^2_p/n_L + s^2_p/n_H}}
%  t = \frac{(\bar{y_L} - \bar{y_H})}{
%    \sqrt{(s_L/\sqrt{n_L})^2 + (s_H/\sqrt{n_H})^2}}
  \]
%  where $s_L$ and $s_H$ are the standard deviations of the low and
%  high groups.
  where $s^2_p$ is the pooled variance:
  \[
  s^2_p = \frac{(n_L-1)s^2_L + (n_H-1)s^2_H}{n_L + n_H - 2}
%  s_L = \sqrt{\frac{1}{n_L-1} \sum_{i=1}^{n_L} (y_{L i} - \bar{y_L})^2}
  \]
  \pause
  {\bf Step 2:} Compare $t$ statistic to critical values
  \[
     \mbox{Critical value for 1-tailed test}\; t_{\alpha=0.05,18}= -1.73\, \mathrm{or}\, 1.73
  \]
  \[
     \mbox{Critical values for 2-tailed test}\; t_{\alpha=0.05,18}= -2.10\, \mathrm{and}\, 2.10
  \]
\end{frame}





%% \begin{frame}[fragile]
%%   \frametitle{Exercise III}
%%   \large
%%   \begin{enumerate}[\bf (1)]
%%     \item Test the null hypothesis $H_0: \mu_L = \mu_H$, with $H_a:
%%       \mu_L \neq \mu_H$
%%     \item[]
%%     \item Show steps for computing the $t$-statistic
%%     \item[]
%%     \item Use only the functions \verb+mean+, \verb+var+, and possibly
%%       \verb+length+
%%   \end{enumerate}
%% \end{frame}




\begin{frame}[fragile]
  \frametitle{Do it by hand in \R}
  {\bf Step 1:} Compute the $t$ statistic:
  \footnotesize
%  \note{Make them do this before you show the answer}
\begin{knitrout}\scriptsize
\definecolor{shadecolor}{rgb}{0.878, 0.918, 0.933}\color{fgcolor}\begin{kframe}
\begin{alltt}
\hlstd{mean.L} \hlkwb{<-} \hlkwd{mean}\hlstd{(yL)}
\hlstd{mean.H} \hlkwb{<-} \hlkwd{mean}\hlstd{(yH)}
\hlstd{s2.L} \hlkwb{<-} \hlkwd{var}\hlstd{(yL)}
\hlstd{s2.H} \hlkwb{<-} \hlkwd{var}\hlstd{(yH)}
\hlstd{n.L} \hlkwb{<-} \hlkwd{length}\hlstd{(yL)} \hlcom{# length returns the number of elements in a vector}
\hlstd{n.H} \hlkwb{<-} \hlkwd{length}\hlstd{(yH)}
\hlstd{s2.p} \hlkwb{<-} \hlstd{((n.L}\hlopt{-}\hlnum{1}\hlstd{)}\hlopt{*}\hlstd{s2.L} \hlopt{+} \hlstd{(n.H}\hlopt{-}\hlnum{1}\hlstd{)}\hlopt{*}\hlstd{s2.H)}\hlopt{/}\hlstd{(n.L}\hlopt{+}\hlstd{n.H}\hlopt{-}\hlnum{2}\hlstd{)}
\hlstd{SE} \hlkwb{<-} \hlkwd{sqrt}\hlstd{(s2.p}\hlopt{/}\hlstd{n.L} \hlopt{+} \hlstd{s2.p}\hlopt{/}\hlstd{n.H)}
\hlstd{t.stat} \hlkwb{<-} \hlstd{(mean.L} \hlopt{-} \hlstd{mean.H)} \hlopt{/} \hlstd{SE}
\hlstd{t.stat}
\end{alltt}
\begin{verbatim}
## [1] 5.404896
\end{verbatim}
\end{kframe}
\end{knitrout}
\pause
%  \normalsize
  {\bf Step 2:} Compare $t$ statistic to critical values (two-tailed)
  \footnotesize
\begin{knitrout}\tiny
\definecolor{shadecolor}{rgb}{0.878, 0.918, 0.933}\color{fgcolor}\begin{kframe}
\begin{alltt}
\hlstd{alpha} \hlkwb{<-} \hlnum{0.05}
\hlcom{## NOTE: qt returns critical values. No need to use "t tables"}
\hlstd{critical.vals} \hlkwb{<-} \hlkwd{qt}\hlstd{(}\hlkwd{c}\hlstd{(alpha}\hlopt{/}\hlnum{2}\hlstd{,} \hlnum{1}\hlopt{-}\hlstd{alpha}\hlopt{/}\hlnum{2}\hlstd{),} \hlkwc{df}\hlstd{=n.L}\hlopt{+}\hlstd{n.H}\hlopt{-}\hlnum{2}\hlstd{)}
\hlstd{critical.vals}
\end{alltt}
\begin{verbatim}
## [1] -2.100922  2.100922
\end{verbatim}
\end{kframe}
\end{knitrout}
\scriptsize
  {\bf Conclusion:} Reject $H_0$ because %5.4048957
  5.4 is more
  extreme than the critical values.
\end{frame}




\begin{frame}[fragile]
  \frametitle{Let \R~do all the work -- Option 1}
  \footnotesize
\begin{knitrout}
\definecolor{shadecolor}{rgb}{0.878, 0.918, 0.933}\color{fgcolor}\begin{kframe}
\begin{alltt}
\hlkwd{t.test}\hlstd{(yH, yL,} \hlkwc{var.equal}\hlstd{=}\hlnum{TRUE}\hlstd{,}
       \hlkwc{paired}\hlstd{=}\hlnum{FALSE}\hlstd{,} \hlkwc{alternative}\hlstd{=}\hlstr{"two.sided"}\hlstd{)}
\end{alltt}
\begin{verbatim}
## 
## 	Two Sample t-test
## 
## data:  yH and yL
## t = -5.4049, df = 18, p-value = 3.898e-05
## alternative hypothesis: true difference in means is not equal to 0
## 95 percent confidence interval:
##  -19.580772  -8.619228
## sample estimates:
## mean of x mean of y 
##       6.1      20.2
\end{verbatim}
\end{kframe}
\end{knitrout}
\vfill
{\centering %\bf
  Make sure you set \inr{var.equal=TRUE}. Otherwise, \R~will assume
  that the variances of the two populations are unequal. \\
}
\end{frame}



\begin{frame}[fragile]
  \frametitle{Let \R~do all the work -- Option 2}
  \footnotesize
\begin{knitrout}
\definecolor{shadecolor}{rgb}{0.878, 0.918, 0.933}\color{fgcolor}\begin{kframe}
\begin{alltt}
\hlkwd{t.test}\hlstd{(treeDensity} \hlopt{~} \hlstd{Elevation,} \hlkwc{data}\hlstd{=treedata,} \hlkwc{var.equal}\hlstd{=}\hlnum{TRUE}\hlstd{,}
       \hlkwc{paired}\hlstd{=}\hlnum{FALSE}\hlstd{,} \hlkwc{alternative}\hlstd{=}\hlstr{"two.sided"}\hlstd{)}
\end{alltt}
\begin{verbatim}
## 
## 	Two Sample t-test
## 
## data:  treeDensity by Elevation
## t = -5.4049, df = 18, p-value = 3.898e-05
## alternative hypothesis: true difference in means is not equal to 0
## 95 percent confidence interval:
##  -19.580772  -8.619228
## sample estimates:
## mean in group High  mean in group Low 
##                6.1               20.2
\end{verbatim}
\end{kframe}
\end{knitrout}
\vfill
{\centering %\bf
  This second option returns identical results, but it is preferred
  because the notation is much more similar to the notation used to
  fit ANOVA models. \\
}
\end{frame}





%\subsection{Two sample $t$-test, unequal variances}




\subsection{Equality of variance test}



\begin{comment}
\begin{frame}[fragile]
  \frametitle{Test equality of variances by hand}
  {\bf Step 1:} Compute ratio of variances $s^2_1/s^2_2$ (the $F$-statistic)
  \footnotesize
\begin{knitrout}
\definecolor{shadecolor}{rgb}{0.878, 0.918, 0.933}\color{fgcolor}\begin{kframe}
\begin{alltt}
\hlstd{var.ratio} \hlkwb{<-} \hlkwd{var}\hlstd{(yL)}\hlopt{/}\hlkwd{var}\hlstd{(yH)} \hlcom{# F-statistic}
\hlstd{var.ratio}
\end{alltt}
\begin{verbatim}
## [1] 1.149877
\end{verbatim}
\end{kframe}
\end{knitrout}
  \pause
  \normalsize
  {\bf Step 2:} Compare to critical value
  \footnotesize
\begin{knitrout}
\definecolor{shadecolor}{rgb}{0.878, 0.918, 0.933}\color{fgcolor}\begin{kframe}
\begin{alltt}
\hlstd{alpha} \hlkwb{<-} \hlnum{0.05}
\hlstd{critical.vals} \hlkwb{<-} \hlkwd{qf}\hlstd{(}\hlkwd{c}\hlstd{(alpha}\hlopt{/}\hlnum{2}\hlstd{,} \hlnum{1}\hlopt{-}\hlstd{alpha}\hlopt{/}\hlnum{2}\hlstd{),} \hlkwc{df1}\hlstd{=n.L}\hlopt{-}\hlnum{1}\hlstd{,} \hlkwc{df2}\hlstd{=n.H}\hlopt{-}\hlnum{1}\hlstd{)}
\hlstd{critical.vals}
\end{alltt}
\begin{verbatim}
## [1] 0.2483859 4.0259942
\end{verbatim}
\end{kframe}
\end{knitrout}
\pause
{\bf \normalsize Do we reject the null hypothesis?}
%\pause
%<<>>=
%var.ratio > critical.vals # Fail to reject null hypothesis
@
\end{frame}
\end{comment}





\begin{frame}[fragile]
  \frametitle{Test equality of variances using {\tt var.test}
  }
  The standard 2 sample $t$-test assumes that the variances are
  equal. Here's how you can test this assumption:
%  \footnotesize
\begin{knitrout}\small
\definecolor{shadecolor}{rgb}{0.878, 0.918, 0.933}\color{fgcolor}\begin{kframe}
\begin{alltt}
\hlkwd{var.test}\hlstd{(yL, yH)}
\end{alltt}
\begin{verbatim}
## 
## 	F test to compare two variances
## 
## data:  yL and yH
## F = 1.1499, num df = 9, denom df = 9, p-value = 0.8386
## alternative hypothesis: true ratio of variances is not equal to 1
## 95 percent confidence interval:
##  0.2856132 4.6293987
## sample estimates:
## ratio of variances 
##           1.149877
\end{verbatim}
\end{kframe}
\end{knitrout}
\end{frame}


\begin{comment}
\begin{frame}
  \frametitle{$F$-distribution}

\end{frame}
\end{comment}



\subsection{Paired $t$-test}



\begin{frame}[fragile]
  \frametitle{Suppose the samples are paired}
  {%\bf
    The Caterpillar Data from class}
\begin{knitrout}
\definecolor{shadecolor}{rgb}{0.878, 0.918, 0.933}\color{fgcolor}\begin{kframe}
\begin{alltt}
\hlstd{location} \hlkwb{<-} \hlnum{1}\hlopt{:}\hlnum{12}
\hlstd{untreated} \hlkwb{<-} \hlkwd{c}\hlstd{(}\hlnum{23}\hlstd{,}\hlnum{18}\hlstd{,}\hlnum{29}\hlstd{,}\hlnum{22}\hlstd{,}\hlnum{33}\hlstd{,}\hlnum{20}\hlstd{,}\hlnum{17}\hlstd{,}\hlnum{25}\hlstd{,}\hlnum{27}\hlstd{,}\hlnum{30}\hlstd{,}\hlnum{25}\hlstd{,}\hlnum{27}\hlstd{)}
\hlstd{treated} \hlkwb{<-} \hlkwd{c}\hlstd{(}\hlnum{19}\hlstd{,}\hlnum{18}\hlstd{,}\hlnum{24}\hlstd{,}\hlnum{23}\hlstd{,}\hlnum{31}\hlstd{,}\hlnum{22}\hlstd{,}\hlnum{16}\hlstd{,}\hlnum{23}\hlstd{,}\hlnum{24}\hlstd{,}\hlnum{26}\hlstd{,}\hlnum{24}\hlstd{,}\hlnum{28}\hlstd{)}
\end{alltt}
\end{kframe}
\end{knitrout}
  \pause
  \vfill
  {%\bf
    For paired $t$-tests, we want to know if the \alert{mean of the
    differences} differs from zero}
  \vfill
  \pause
\begin{knitrout}
\definecolor{shadecolor}{rgb}{0.878, 0.918, 0.933}\color{fgcolor}\begin{kframe}
\begin{alltt}
\hlstd{diff} \hlkwb{<-} \hlstd{untreated}\hlopt{-}\hlstd{treated}
\hlstd{diff}
\end{alltt}
\begin{verbatim}
##  [1]  4  0  5 -1  2 -2  1  2  3  4  1 -1
\end{verbatim}
\begin{alltt}
\hlkwd{mean}\hlstd{(diff)} \hlcom{## Estimate of the mean of the differences}
\end{alltt}
\begin{verbatim}
## [1] 1.5
\end{verbatim}
\end{kframe}
\end{knitrout}

\end{frame}



%##boxplot(treated, untreated, names=c("Treated", "Untreated"),
%##        col=c("lightgreen", "purple"),


\begin{frame}[fragile]
  \frametitle{Is the mean of the differences $>0$?}
  \begin{center}
  \footnotesize

\vspace{-.2cm}
\includegraphics[width=3in,height=3in]{figure/box2-1}
  \end{center}
\end{frame}



\begin{comment}
\begin{frame}[fragile]
  \frametitle{Is the mean of the differences $>0$?}
  \begin{center}
  \footnotesize
%<<pplot,fig=TRUE,include=FALSE,fig.width=8,fig.height=6>>=
%#par(mai=c(0.9,0.9,0.1,0.1))
%plot(1:12, treated, col="lightgreen", pch=16, cex=2,
%     xlab="Location", ylab="Caterpillars", ylim=c(10,40))
%points(1:12, untreated, col="purple", pch=16, cex=2)
%legend(9, 40, c("Treated", "Untreated"),
%       col=c("lightgreen", "purple"), pch=16, pt.cex=2)
%@
%<<>>=
%hist(diff, breaks=4)
%@
\includegraphics[width=2.933in,height=2.2in]{figure/pplot-1}
  \end{center}
\end{frame}
\end{comment}



\begin{comment}
\begin{frame}[fragile]
  \frametitle{The differences}
\begin{center}
  \footnotesize

\includegraphics[width=2.933in,height=2.2in]{figure/diff-1}
\end{center}
\end{frame}
\end{comment}






\begin{frame}[fragile]
  \frametitle{Paired $t$-test}
  {\bf Recall:} Paired $t$-test is the same as a one-sample $t$-test on the
      differences. The hypothesis \emph{in the Caterpillar example} is one-tailed:
  \begin{itemize}
    \item $H_0: \mu_d \le 0$
    \item $H_A: \mu_d > 0$
  \end{itemize}
  \pause
  \vfill
  \normalsize
  {\bf Step 1}. Calculate the standard deviation of the differences.
  \[
%    \mbox{SEM} = \frac{s}{\sqrt{n}} = \frac{\sqrt{\frac{1}{n-1} \sum_{i=1}^n
%        (y_i - \bar{y})^2}}{\sqrt{n}}
    s_d = \sqrt{\frac{1}{n-1} \sum_{i=1}^n
        (y_i - \bar{y})^2}%}{\sqrt{n}}
  \]
  \pause
  \vfill
  {\bf Step 2}. Calculate the test statistic.
  \[
%     t = \frac{\bar{y} - 0}{\mbox{SEM}}
     t = \frac{\bar{y} - 0}{s_d/\sqrt{n}}
  \]
  {\bf Step 3}. Compare to critical value.
%  \[
%  \]
\end{frame}




%% \begin{frame}
%%   \frametitle{Exercise IV}
%%   {\bf \Large Figure out 2 ways of doing paired $t$-test using the
%%     function {\tt t.test}}.
%% \end{frame}



%% \begin{frame}[fragile]
%%   \frametitle{Paired $t$-test in \R}
%%   \tiny
%% <<>>=
%% diff <- untreated-treated
%% diff
%% t.test(untreated, treated, var.equal=TRUE, paired=TRUE)
%% t.test(diff, var.equal=TRUE, paired=FALSE) # Same thing
%% @
%% \end{frame}




\begin{frame}
  \frametitle{Assignment}
  \footnotesize
  {\bf Create a script to do the following:}
  \begin{enumerate}[\bf (1)]
    \item Do a paired $t$ test on the caterpillar data without using the \inr{t.test}
      function. Use only the functions \inr{mean}, \inr{sd}, and
      possibly \inr{length}.
    \item Do the paired $t$ test again, but this time using the \inr{
        t.test} function.
      \begin{itemize}
        \scriptsize
        \item You will need to use the \texttt{``paired''} argument when using the \inr{t.test} function
        \item Assume variances are equal
      \end{itemize}
    \item Do a standard ({\it unpaired}) two-sample $t$ test using
      the \inr{t.test} function.
%    \item Assume equal variances for all tests
    \item Add a comment to the end of your script explain your
      results. Also include the null and alternative hypotheses for
      each test. % why the
%      results differ between the paired and unpaired analysis.
  \end{enumerate}
  \vfill
  {\bf Upload your script\footnote{\scriptsize Or upload an Rmarkdown
      (.Rmd) file} to ELC before next week's lab.}
  \begin{itemize}
    \item The script must be self-contained.
    \item In other words, you should be able to copy and paste the
      entire thing into the {\bf R} console, and it should return the
      correct answers to the questions.
  \end{itemize}
  \vfill
   {\bf Read pp. 127--131 in ``Introductory Statistics with \R''}
\end{frame}




\end{document}



%% \section{Bonus material}



%% \begin{frame}[fragile]
%% \frametitle{Understanding t-distribution}
%% \footnotesize
%% <<sim-t,fig=TRUE,include=FALSE,eps=FALSE>>=
%% nReplicates <- 500
%% nSamples <- 50
%% mean.diff <- tval <- rep(NA, length=nReplicates)
%% t.test.results <- list()
%% for(i in 1:nReplicates) {
%%   x1 <- rnorm(nSamples, mean=20, sd=2)
%% #  x2 <- rnorm(nSamples, mean=20, sd=2)
%%   mean.diff[i] <- mean(x1) - 0 #mean(x2)
%%   tval[i] <- mean.diff[i]/(sd(x1)/sqrt(nSamples))
%%   t.test.results[[i]] <- t.test(x1) #, x2)
%%   }
%% head(cbind(tval, sapply(t.test.results, "[[", "statistic")))
%% hist(mean.diff, freq=FALSE)
%% hist(sapply(t.test.results, "[[", "statistic"), freq=FALSE)
%% curve(dt(x, df=nSamples*2-2), add=TRUE)
%% @
%% \end{frame}





%% \begin{frame}
%%   \includegraphics[width=\textwidth]{lab02-t-test-sim-t}
%% \end{frame}



%% \end{document}





















%% \section{Visualizing data}


%% \begin{frame}[fragile]
%% \frametitle{dog}
%% <<>>=
%% goats <- c(3, 8, 7, 12, 20, 5, 3, 13)
%% treatment <- factor(rep(c("Treatment", "Control"), each=4))
%% temp <- c(2, 8, 8, 11, 22, 3, 5, 10)
%% goatdata <- data.frame(goats, treatment, temp)
%% goatdata

%% @
%% \end{frame}


%% \begin{frame}[fragile]
%%   \frametitle{Boxplots}
%% <<>>=
%% #boxplot(goats ~ treatment, data=goatdata)
%% # boxplot(treatment, goats, )
%% @
%% \end{frame}

















%% \begin{comment}
%% \section{One sample $t$-tests}
%% \begin{frame}
%%   \frametitle{Scenario}
%%   We have a sample of n observations from a population with mean
%%   $\mu$.

%%   The question is, does $\mu$ equal some value, say $X$?

%%   More precisely, the null hypothesis is $H_0: \mu = X$ and the
%%   alternative hypothesis is $H_a: \mu \neq X$.
%% \end{frame}
%% \end{comment}

