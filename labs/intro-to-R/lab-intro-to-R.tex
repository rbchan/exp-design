\documentclass[color=usenames,dvipsnames]{beamer}\usepackage[]{graphicx}\usepackage[]{color}
% maxwidth is the original width if it is less than linewidth
% otherwise use linewidth (to make sure the graphics do not exceed the margin)
\makeatletter
\def\maxwidth{ %
  \ifdim\Gin@nat@width>\linewidth
    \linewidth
  \else
    \Gin@nat@width
  \fi
}
\makeatother

\definecolor{fgcolor}{rgb}{0, 0, 0}
\newcommand{\hlnum}[1]{\textcolor[rgb]{0.69,0.494,0}{#1}}%
\newcommand{\hlstr}[1]{\textcolor[rgb]{0.749,0.012,0.012}{#1}}%
\newcommand{\hlcom}[1]{\textcolor[rgb]{0.514,0.506,0.514}{\textit{#1}}}%
\newcommand{\hlopt}[1]{\textcolor[rgb]{0,0,0}{#1}}%
\newcommand{\hlstd}[1]{\textcolor[rgb]{0,0,0}{#1}}%
\newcommand{\hlkwa}[1]{\textcolor[rgb]{0,0,0}{\textbf{#1}}}%
\newcommand{\hlkwb}[1]{\textcolor[rgb]{0,0.341,0.682}{#1}}%
\newcommand{\hlkwc}[1]{\textcolor[rgb]{0,0,0}{\textbf{#1}}}%
\newcommand{\hlkwd}[1]{\textcolor[rgb]{0.004,0.004,0.506}{#1}}%
\let\hlipl\hlkwb

\usepackage{framed}
\makeatletter
\newenvironment{kframe}{%
 \def\at@end@of@kframe{}%
 \ifinner\ifhmode%
  \def\at@end@of@kframe{\end{minipage}}%
  \begin{minipage}{\columnwidth}%
 \fi\fi%
 \def\FrameCommand##1{\hskip\@totalleftmargin \hskip-\fboxsep
 \colorbox{shadecolor}{##1}\hskip-\fboxsep
     % There is no \\@totalrightmargin, so:
     \hskip-\linewidth \hskip-\@totalleftmargin \hskip\columnwidth}%
 \MakeFramed {\advance\hsize-\width
   \@totalleftmargin\z@ \linewidth\hsize
   \@setminipage}}%
 {\par\unskip\endMakeFramed%
 \at@end@of@kframe}
\makeatother

\definecolor{shadecolor}{rgb}{.97, .97, .97}
\definecolor{messagecolor}{rgb}{0, 0, 0}
\definecolor{warningcolor}{rgb}{1, 0, 1}
\definecolor{errorcolor}{rgb}{1, 0, 0}
\newenvironment{knitrout}{}{} % an empty environment to be redefined in TeX

\usepackage{alltt}
%\documentclass[color=usenames,dvipsnames,handout]{beamer}

%\usepackage[roman]{../../lab1}
\usepackage[sans]{../../lab1}
% \usepackage{Sweave}



\hypersetup{pdfpagemode=UseNone,pdfstartview=FitH}



\title{Lab 1 -- Introduction to {\bf R}}
\author{Richard Chandler and Bob Cooper}
%\date{August 13 \& 14, 2018}

%\newcommand{\R}{{\bf R}}


%% Switching from Sweave to knitr
%\DefineVerbatimEnvironment{Sinput}{Verbatim}{fontshape=sl,formatcom=\color{red}}
%\DefineVerbatimEnvironment{Soutput}{Verbatim}{formatcom=\color{MidnightBlue}}
%\DefineVerbatimEnvironment{Scode}{Verbatim}{fontshape=sl}



% <<knitr-setup, include=FALSE>>=
% opts_hooks$set(comment=function(x) return("comment"=NA))
% ##knit_hooks$set(inline = function(x) {
% ##    if(!require(highr)) {
% ##        install.packages("highr")
% ##    }
% ##    if (is.numeric(x)) return(knitr:::format_sci(x, 'latex'))
% ##    highr:::hi_latex(x)
% ##})
% @













%% New command for inline code that isn't to be evaluated
\definecolor{inlinecolor}{rgb}{0.878, 0.918, 0.933}
\newcommand{\inr}[1]{\colorbox{inlinecolor}{\texttt{#1}}}
\IfFileExists{upquote.sty}{\usepackage{upquote}}{}
\begin{document}

% This would affect all code boxes. Not a good idea.
% \setlength\fboxsep{0pt}



\begin{frame}[plain]
  \LARGE
%  \maketitle
  {\centering
%  \textcolor{RoyalBlue}{\huge \bf Lab 1 -- Introduction to \R} \\
  {\huge \bf Lab 1 -- Introduction to \R} \\
  \vspace{0.9cm}
  \includegraphics[width=0.4\textwidth]{figs/Rlogo} \\
  \vspace{0.5cm}
%  August 13 \& 14, 2018 \\
  FANR 6750 \par
  \vfill
  \large
  Richard Chandler and Bob Cooper \\
  University of Georgia \\
  }
\end{frame}




\section{Why Use \R?}


\begin{frame}[plain]
  \frametitle{Today's Topics}
  \Large
  \only<1>{\tableofcontents}%[hideallsubsections]}
  \only<2 | handout:0>{\tableofcontents[currentsection]}%,hideallsubsections]}
\end{frame}



\begin{frame}
  \frametitle{Good and Not So Good Things About \R}
%  {\Large \textcolor{bb}{Good}}
  {\Large Good}
  \large
  \begin{itemize}%[<+->]
    \item<1-> Powerful platform for statistical analysis %Flexible %If you're going to use a stats program, might as well use one that does everything!
%    \item<1-> Many \inr{packages} \colorbox{BurntOrange}{written} for ecologists
    \item<1-> Many packages written for ecologists
    \item<1-> It's free
    \item<1-> Scripts save time
    \item<1-> \R~teaches you statistics
  \end{itemize}
  \vspace{0.5cm}
  \uncover<2->{
%  \pause
  {\Large Not so good??}}
  \begin{itemize}
    \item<2-> Steep learning curve
    \item<2-> Help pages written for people familiar with \R
    \item<2-> Developed by statisticians for statisticians
    \item<2-> Not as fast as some languages
  \end{itemize}
%  }
\end{frame}




%\begin{frame}
%  \frametitle{Not So Good Things About \R}
%\end{frame}




\section{Installing \R}





\begin{frame}
  \frametitle{Downloading \R}
  \begin{columns}
    \begin{column}{0.4\textwidth}
      \small
      \begin{itemize}
      \item Go to \url{www.r-project.org}
      \item Click on ``CRAN''
      \item Choose a mirror near you (there is one in TN)
      \end{itemize}
    \end{column}
    \begin{column}{0.6\textwidth}
      \fbox{\includegraphics[width=\textwidth]{figs/download-R-win}} \par
      \fbox{\includegraphics[width=\textwidth]{figs/download-R}}
    \end{column}
  \end{columns}
\end{frame}



\begin{frame}
  \frametitle{Using the \R~GUI}
  \R's ``Graphical User Interface'' is operating system specific. Here
  is how it looks under Windows.
  \begin{center}
    \includegraphics[width=\textwidth]{figs/R-GUI-win}
  \end{center}
\end{frame}



\begin{frame}
  \frametitle{Alternatives to the \R~Gui}
  \begin{columns}
    \small
    \begin{column}{0.6\textwidth}
      \centering
      Emacs and ESS \\ \tiny
      \url{http://vgoulet.act.ulaval.ca/en/emacs/} \par
      \includegraphics[width=\textwidth]{figs/emacs}
    \end{column}
    \begin{column}{0.4\textwidth}
      \centering
      RStudio \\
      \url{http://www.rstudio.com/} \par
      \includegraphics[width=\textwidth]{figs/Rstudio}
    \end{column}
  \end{columns}
%  \pause
  \small
  \vspace{0.5cm}
  %% You are encouraged to learn and use these programs, but we will not
  %% use them for instruction because we want to focus on \R~itself, not
  %% the interface.
  You are encouraged to use these programs to run \R, instead of the default GUI. However, instruction will focus on \R~itself, not on the details or Emacs or RStudio. 
\end{frame}


\section{Basic Usage}




%\begin{frame}
%  \frametitle{Today's Topics}
%  \Large
%  \tableofcontents[currentsection]%,hideallsubsections]
%\end{frame}

\subsection{Basic calculations}




\begin{frame}[fragile]
  \frametitle{How to read the code in the lab slides}
%  \small
  Anything in a shaded box like this one is \R~code, and you can copy
  and paste it into the Console:  
\begin{knitrout}
\definecolor{shadecolor}{rgb}{0.878, 0.918, 0.933}\color{fgcolor}\begin{kframe}
\begin{alltt}
\hlnum{2}\hlopt{+}\hlnum{2}
\end{alltt}
\begin{verbatim}
## [1] 4
\end{verbatim}
\end{kframe}
\end{knitrout}
\pause %\vfill
  The line \inr{2+2} is \R~code (\alert{input}). Note that the command
  prompt (\verb+>+) is not shown.
\pause
\vfill
  The line \inr{\#\# [1] 4} is \alert{output}. Output is
  always indicated by two hash signs (\texttt{\#\#}). Anything after
  \texttt{\#} is ignored by \R~at the command line. The \inr{[1]}
  simply indicates that this is the first value in the output.
%% \pause \vfill
%%   You can copy and paste the entire code box directly into your
%%   console, but it might be easier to work with the \R~script
%%   that accompanies the PDF: %. In this case, the script is called
%%   \texttt{lab-intro-to-R.R}.
\end{frame}






\begin{frame}[fragile]
  \frametitle{Overgrown calculator}
%  \includegraphics[width=\textwidth]{figs/math-ops}
  \small
%  Simple addition
%<<add, size='small'>>=
%2+2
%@
%\pause \vfill
  Square-root of 3
\begin{knitrout}\small
\definecolor{shadecolor}{rgb}{0.878, 0.918, 0.933}\color{fgcolor}\begin{kframe}
\begin{alltt}
\hlkwd{sqrt}\hlstd{(}\hlnum{3}\hlstd{)}
\end{alltt}
\begin{verbatim}
## [1] 1.732051
\end{verbatim}
\end{kframe}
\end{knitrout}
\pause \vfill
  6 squared
\begin{knitrout}\small
\definecolor{shadecolor}{rgb}{0.878, 0.918, 0.933}\color{fgcolor}\begin{kframe}
\begin{alltt}
\hlnum{6}\hlopt{^}\hlnum{2}
\end{alltt}
\begin{verbatim}
## [1] 36
\end{verbatim}
\end{kframe}
\end{knitrout}
\pause \vfill
  cosine of $\pi$
\begin{knitrout}\small
\definecolor{shadecolor}{rgb}{0.878, 0.918, 0.933}\color{fgcolor}\begin{kframe}
\begin{alltt}
\hlkwd{cos}\hlstd{(pi)}
\end{alltt}
\begin{verbatim}
## [1] -1
\end{verbatim}
\end{kframe}
\end{knitrout}
%<<pi>>=
%pi
%@
\end{frame}





%\section{Data frames}





\begin{frame}[fragile]
  \frametitle{Objects and assignment arrow}
      Everything in \R~is an object. We can create objects using the
      \inr{<-} assignment arrow.
      \pause \vfill
      In this example, we assign the value 2 to the object \inr{y}:
\begin{knitrout}
\definecolor{shadecolor}{rgb}{0.878, 0.918, 0.933}\color{fgcolor}\begin{kframe}
\begin{alltt}
\hlstd{y} \hlkwb{<-} \hlnum{2}
\end{alltt}
\end{kframe}
\end{knitrout}
      \pause \vfill
      You cannot have a space between \inr{<} and \inr{-}
      \pause \vfill
      You can use \inr{=} instead of \inr{<-} but this can cause confusion
      \pause \vfill
      Typing the name of an object returns its value:
\begin{knitrout}
\definecolor{shadecolor}{rgb}{0.878, 0.918, 0.933}\color{fgcolor}\begin{kframe}
\begin{alltt}
\hlstd{y}
\end{alltt}
\begin{verbatim}
## [1] 2
\end{verbatim}
\end{kframe}
\end{knitrout}
\end{frame}


\subsection{Vectors}


\begin{frame}[fragile]
  \frametitle{Vectors}
We store data in objects so that they can be easily manipulated:
\begin{knitrout}\small
\definecolor{shadecolor}{rgb}{0.878, 0.918, 0.933}\color{fgcolor}\begin{kframe}
\begin{alltt}
\hlstd{y}\hlopt{*}\hlnum{2}\hlopt{+}\hlnum{1}
\end{alltt}
\begin{verbatim}
## [1] 5
\end{verbatim}
\end{kframe}
\end{knitrout}
\pause \vfill
Usually, we want more than one number in an object. In statistics, a
vector is simply a set of numbers that can be thought of as a row or
column of a matrix \\
\pause \vfill
The easiest way to create a vector is to use the \inr{c}
function to ``combine'' numbers:
\begin{knitrout}\small
\definecolor{shadecolor}{rgb}{0.878, 0.918, 0.933}\color{fgcolor}\begin{kframe}
\begin{alltt}
\hlstd{z} \hlkwb{<-} \hlkwd{c}\hlstd{(}\hlopt{-}\hlnum{1}\hlstd{,} \hlnum{9}\hlstd{,} \hlnum{33}\hlstd{,} \hlopt{-}\hlnum{4}\hlstd{)}
\hlstd{z}
\end{alltt}
\begin{verbatim}
## [1] -1  9 33 -4
\end{verbatim}
\end{kframe}
\end{knitrout}
\end{frame}



%% \begin{frame}[fragile]
%%   \frametitle{Vectors}
%% %  \begin{block}{Definition}
%% %  \Large
%%   \begin{itemize}
%% %    \item In statistics, a vector is a set of numbers
%%     \item The easiest way to create a vector is to use the \verb+c+
%%       function to ``combine'' numbers
%%   \end{itemize}
%% <<>>=
%% x <- c(1, 2, 3, 9, -100)
%% x
%% @
%% %  \end{block}
%% %  This code creates a vector of continuous numbers:

%% \end{frame}


%\begin{frame}
%  \frametitle{More about the assignment arrow}
%
%\end{frame}


\begin{frame}[fragile]
  \frametitle{Other useful ways of creating vectors}
A sequence of numbers
\begin{knitrout}\small
\definecolor{shadecolor}{rgb}{0.878, 0.918, 0.933}\color{fgcolor}\begin{kframe}
\begin{alltt}
\hlstd{x1} \hlkwb{<-} \hlnum{1}\hlopt{:}\hlnum{3} \hlcom{# Same as: x1 <- c(1, 2, 3)}
          \hlcom{# Note: anything after "#" is a comment}
\hlstd{x1}
\end{alltt}
\begin{verbatim}
## [1] 1 2 3
\end{verbatim}
\end{kframe}
\end{knitrout}
\pause \vfill
\inr{seq} is more general
\begin{knitrout}\small
\definecolor{shadecolor}{rgb}{0.878, 0.918, 0.933}\color{fgcolor}\begin{kframe}
\begin{alltt}
\hlstd{x2} \hlkwb{<-} \hlkwd{seq}\hlstd{(}\hlkwc{from}\hlstd{=}\hlnum{1}\hlstd{,} \hlkwc{to}\hlstd{=}\hlnum{7}\hlstd{,} \hlkwc{by}\hlstd{=}\hlnum{2}\hlstd{)}
\hlstd{x2}
\end{alltt}
\begin{verbatim}
## [1] 1 3 5 7
\end{verbatim}
\end{kframe}
\end{knitrout}
\pause \vfill
Use \inr{rep} to repeat elements of a vector
\begin{knitrout}\small
\definecolor{shadecolor}{rgb}{0.878, 0.918, 0.933}\color{fgcolor}\begin{kframe}
\begin{alltt}
\hlkwd{rep}\hlstd{(x2,} \hlkwc{times}\hlstd{=}\hlnum{2}\hlstd{)}
\end{alltt}
\begin{verbatim}
## [1] 1 3 5 7 1 3 5 7
\end{verbatim}
\end{kframe}
\end{knitrout}
\end{frame}



\begin{frame}[fragile]
  \frametitle{Help with a function}
\LARGE
These do the same thing
\begin{knitrout}
\definecolor{shadecolor}{rgb}{0.878, 0.918, 0.933}\color{fgcolor}\begin{kframe}
\begin{alltt}
\hlopt{?}\hlstd{rep}
\hlkwd{help}\hlstd{(rep)}
\end{alltt}
\end{kframe}
\end{knitrout}
\end{frame}




\begin{frame}[fragile]
  \frametitle{Types of vectors}
%  \begin{itemize}[<+->]
%  \item
  Numeric vectors are used for continuous variables
\begin{knitrout}\small
\definecolor{shadecolor}{rgb}{0.878, 0.918, 0.933}\color{fgcolor}\begin{kframe}
\begin{alltt}
\hlstd{y1} \hlkwb{<-} \hlkwd{c}\hlstd{(}\hlnum{2.1}\hlstd{,} \hlnum{3.5}\hlstd{,} \hlnum{99.0}\hlstd{)}
\hlkwd{class}\hlstd{(y1)}
\end{alltt}
\begin{verbatim}
## [1] "numeric"
\end{verbatim}
\end{kframe}
\end{knitrout}
%  \item
\pause \vfill
%  Names can be stored as character vectors
%<<y2-class, size='small'>>=
%y2 <- c("Treatment", "Control")
%class(y2)
%@
%  \item
%\pause \vfill
Factors can be used to store categorical variables: %character strings
\begin{knitrout}\small
\definecolor{shadecolor}{rgb}{0.878, 0.918, 0.933}\color{fgcolor}\begin{kframe}
\begin{alltt}
\hlstd{y2} \hlkwb{<-} \hlkwd{factor}\hlstd{(}\hlkwd{c}\hlstd{(}\hlstr{"Treatment"}\hlstd{,} \hlstr{"Control"}\hlstd{,} \hlstr{"Treatment"}\hlstd{))}
\hlstd{y2}
\end{alltt}
\begin{verbatim}
## [1] Treatment Control   Treatment
## Levels: Control Treatment
\end{verbatim}
\end{kframe}
\end{knitrout}
%  \end{itemize}
\end{frame}



\begin{frame}[fragile]
  \frametitle{Vectorized arithmetic}
  How could we calculate the body mass index (BMI = $\text{weight}/\text{height}^2$) from the following data:
  \begin{center}
    \begin{tabular}{ccccccc}
      \hline
      & \multicolumn{6}{c}{Individual} \\
      \cline{2-7}
      & 1 & 2 & 3 & 4 & 5 & 6 \\
      \hline
      Weight & 60 & 72 & 57 & 90 & 95 & 72 \\
      Height & 1.8 & 1.8 & 1.7 & 1.9 & 1.7 & 1.9 \\
      \hline
    \end{tabular}
  \end{center}
  \pause
  First, create the vectors:
\begin{knitrout}
\definecolor{shadecolor}{rgb}{0.878, 0.918, 0.933}\color{fgcolor}\begin{kframe}
\begin{alltt}
\hlstd{weight} \hlkwb{<-} \hlkwd{c}\hlstd{(}\hlnum{60}\hlstd{,} \hlnum{72}\hlstd{,} \hlnum{57}\hlstd{,} \hlnum{90}\hlstd{,} \hlnum{95}\hlstd{,} \hlnum{72}\hlstd{)}
\hlstd{height} \hlkwb{<-} \hlkwd{c}\hlstd{(}\hlnum{1.8}\hlstd{,} \hlnum{1.8}\hlstd{,} \hlnum{1.7}\hlstd{,} \hlnum{1.9}\hlstd{,} \hlnum{1.7}\hlstd{,} \hlnum{1.9}\hlstd{)}
\end{alltt}
\end{kframe}
\end{knitrout}
  \pause
  Then, evaluate the equation in just one line:
\begin{knitrout}\footnotesize
\definecolor{shadecolor}{rgb}{0.878, 0.918, 0.933}\color{fgcolor}\begin{kframe}
\begin{alltt}
\hlstd{BMI} \hlkwb{<-} \hlstd{weight}\hlopt{/}\hlstd{height}\hlopt{^}\hlnum{2}
\hlstd{BMI}
\end{alltt}
\begin{verbatim}
## [1] 18.51852 22.22222 19.72318 24.93075 32.87197 19.94460
\end{verbatim}
\end{kframe}
\end{knitrout}
\end{frame}



\begin{frame}[fragile]
  \frametitle{In-class assignment}
  {Calculate the circumference and area of circles with radii: 3,5,6,11} \\
  \pause
  \vspace{0.5cm}
  \begin{enumerate}[\bf (1)]
    \item Create a new script called ``lab1-prob1.R''. You can do this by:
      \begin{enumerate}[i]
        \item Clicking on the Console
        \item Choosing \verb_"File > New File > R Script"_ from the drop-down menu
        \item Clicking on \verb_"File > Save as..."_
      \end{enumerate}
    \item Create a vector containing the radii
    \item Store the computed circumferences and areas in 2 new vectors
  \end{enumerate}
  \vspace{1cm}
  \centering
  {\bf You should be able to do this in just 3 lines in your
    script. Write all of your code in the script, not in console. \\}
\end{frame}




%% \begin{frame}[fragile]
%%   \frametitle{A common beginner's problem}
%%   If you accidentally hit ``return'' or fail to complete a command, you
%%   will see the cursor on a new line beginning with \verb_+_ instead of
%%   \verb+>+. \\
%%   \pause
%%   \begin{center}
%%     \includegraphics[width=0.7\textwidth]{figs/plus-prompt}
%%   \end{center}
%%   \pause
%%   Just hit the ``Esc'' key or complete the command.
%% \end{frame}




% NOTE: Make them do this by hand
\begin{frame}[fragile]
  \frametitle{Summarizing vectors}
  \small
  \begin{center}
    \begin{tabular}{ccccc}
      \hline
%      & \multicolumn{5}{c}{Number of chiggers on 5 different people} \\
      \multicolumn{5}{c}{Number of ticks on 5 dogs} \\
      \hline %\cline{1-5}
      $y_1$ & $y_2$ & $y_3$ & $y_4$ & $y_5$ \\
      \hline
      4 & 7 & 2 & 3 & 150 \\
      \hline
    \end{tabular}
  \end{center}
%\begin{itemize}
%\item
What is the sum of this vector? $\sum_{i=1}^5 y_i$ = ???
  \pause
\begin{knitrout}\footnotesize
\definecolor{shadecolor}{rgb}{0.878, 0.918, 0.933}\color{fgcolor}\begin{kframe}
\begin{alltt}
\hlstd{y} \hlkwb{<-} \hlkwd{c}\hlstd{(}\hlnum{4}\hlstd{,}\hlnum{7}\hlstd{,}\hlnum{2}\hlstd{,}\hlnum{3}\hlstd{,}\hlnum{150}\hlstd{)}
\hlkwd{sum}\hlstd{(y)}
\end{alltt}
\begin{verbatim}
## [1] 166
\end{verbatim}
\end{kframe}
\end{knitrout}
\pause
What is the mean? $\frac{\sum_{i=1}^5 y_i}{5}$ = ???
  \pause
\begin{knitrout}\footnotesize
\definecolor{shadecolor}{rgb}{0.878, 0.918, 0.933}\color{fgcolor}\begin{kframe}
\begin{alltt}
\hlkwd{mean}\hlstd{(y)}
\end{alltt}
\begin{verbatim}
## [1] 33.2
\end{verbatim}
\end{kframe}
\end{knitrout}
  \pause
  And the variance? $\frac{\sum_{i=1}^5 (y_i - \bar{y})^2}{5-1}$ = ???
\begin{knitrout}\footnotesize
\definecolor{shadecolor}{rgb}{0.878, 0.918, 0.933}\color{fgcolor}\begin{kframe}
\begin{alltt}
\hlkwd{var}\hlstd{(y)}
\end{alltt}
\begin{verbatim}
## [1] 4266.7
\end{verbatim}
\end{kframe}
\end{knitrout}
\end{frame}


%\end{document}


%\subsection{Indexing}


\begin{frame}[fragile]
  \frametitle{Indexing vectors}
  \small
  Extract the first and third elements of a vector
\begin{knitrout}\small
\definecolor{shadecolor}{rgb}{0.878, 0.918, 0.933}\color{fgcolor}\begin{kframe}
\begin{alltt}
\hlstd{y} \hlkwb{<-} \hlkwd{c}\hlstd{(}\hlnum{2}\hlstd{,} \hlnum{4}\hlstd{,} \hlnum{8}\hlstd{,} \hlnum{4}\hlstd{,} \hlnum{25}\hlstd{)}
\hlstd{y.sub1} \hlkwb{<-} \hlstd{y[}\hlkwd{c}\hlstd{(}\hlnum{1}\hlstd{,}\hlnum{3}\hlstd{)]}
\hlstd{y.sub1}
\end{alltt}
\begin{verbatim}
## [1] 2 8
\end{verbatim}
\end{kframe}
\end{knitrout}
%\item
\pause \vfill
Remove the second element
\begin{knitrout}\small
\definecolor{shadecolor}{rgb}{0.878, 0.918, 0.933}\color{fgcolor}\begin{kframe}
\begin{alltt}
\hlstd{y.sub2} \hlkwb{<-} \hlstd{y[}\hlopt{-}\hlnum{2}\hlstd{]}
\hlstd{y.sub2}
\end{alltt}
\begin{verbatim}
## [1]  2  8  4 25
\end{verbatim}
\end{kframe}
\end{knitrout}
\pause \vfill
%\item
Rearrange the order of the vector
\begin{knitrout}\small
\definecolor{shadecolor}{rgb}{0.878, 0.918, 0.933}\color{fgcolor}\begin{kframe}
\begin{alltt}
\hlstd{y.re} \hlkwb{<-} \hlstd{y[}\hlkwd{c}\hlstd{(}\hlnum{5}\hlstd{,}\hlnum{4}\hlstd{,}\hlnum{3}\hlstd{,}\hlnum{2}\hlstd{,}\hlnum{1}\hlstd{)]}
\hlstd{y.re}
\end{alltt}
\begin{verbatim}
## [1] 25  4  8  4  2
\end{verbatim}
\end{kframe}
\end{knitrout}
%\end{itemize}
\end{frame}


\begin{frame}[fragile]
  \frametitle{Indexing vectors}
  \small
  Which elements of the vector are greater than 4? (logical test)
\begin{knitrout}\small
\definecolor{shadecolor}{rgb}{0.878, 0.918, 0.933}\color{fgcolor}\begin{kframe}
\begin{alltt}
\hlstd{y} \hlkwb{<-} \hlkwd{c}\hlstd{(}\hlnum{2}\hlstd{,} \hlnum{4}\hlstd{,} \hlnum{6}\hlstd{,} \hlnum{4}\hlstd{,} \hlnum{25}\hlstd{)}
\hlstd{y}\hlopt{>}\hlnum{4}
\end{alltt}
\begin{verbatim}
## [1] FALSE FALSE  TRUE FALSE  TRUE
\end{verbatim}
\end{kframe}
\end{knitrout}
\pause \vfill
  Extract the elements greater than 4 (logical indexing)
\begin{knitrout}\small
\definecolor{shadecolor}{rgb}{0.878, 0.918, 0.933}\color{fgcolor}\begin{kframe}
\begin{alltt}
\hlstd{y.sub4} \hlkwb{<-} \hlstd{y[y}\hlopt{>}\hlnum{4}\hlstd{]}
\hlstd{y.sub4}
\end{alltt}
\begin{verbatim}
## [1]  6 25
\end{verbatim}
\end{kframe}
\end{knitrout}
\end{frame}


\subsection{Data frames}


\begin{frame}[fragile]
  \frametitle{Data frames}
  \large
  Most basic datasets are stored as \verb+data.frame+s
  \pause \vfill
  They are like a matrix, except each column can be a
      different type of vector (numeric, factor, etc...)
  \pause \vfill
   They have attributes for row names (e.g. the names of
      the experimental units) and column names (e.g. the names of
      the response and predictor variables)
\end{frame}



\begin{frame}[fragile]
  \frametitle{Creating a data frame}
  {Simple example with 3 variables measured at 4 sites}
\begin{knitrout}\small
\definecolor{shadecolor}{rgb}{0.878, 0.918, 0.933}\color{fgcolor}\begin{kframe}
\begin{alltt}
\hlstd{y} \hlkwb{<-} \hlkwd{c}\hlstd{(}\hlnum{3}\hlstd{,} \hlnum{9}\hlstd{,} \hlnum{7}\hlstd{,} \hlnum{4}\hlstd{)}
\hlstd{x1} \hlkwb{<-} \hlkwd{factor}\hlstd{(}\hlkwd{c}\hlstd{(}\hlstr{'High'}\hlstd{,} \hlstr{'High'}\hlstd{,} \hlstr{'Low'}\hlstd{,} \hlstr{'Low'}\hlstd{))}
\hlstd{x2} \hlkwb{<-} \hlkwd{c}\hlstd{(}\hlnum{2.2}\hlstd{,} \hlnum{3.4}\hlstd{,} \hlnum{4.4}\hlstd{,} \hlnum{3.9}\hlstd{)}
\hlstd{mydata} \hlkwb{<-} \hlkwd{data.frame}\hlstd{(}\hlkwc{Goats}\hlstd{=y,} \hlkwc{Elev}\hlstd{=x1,} \hlkwc{Temp}\hlstd{=x2)}
\hlkwd{rownames}\hlstd{(mydata)} \hlkwb{<-} \hlkwd{c}\hlstd{(}\hlstr{'Site1'}\hlstd{,} \hlstr{'Site2'}\hlstd{,} \hlstr{'Site3'}\hlstd{,} \hlstr{'Site4'}\hlstd{)}
\hlstd{mydata}
\end{alltt}
\begin{verbatim}
##       Goats Elev Temp
## Site1     3 High  2.2
## Site2     9 High  3.4
## Site3     7  Low  4.4
## Site4     4  Low  3.9
\end{verbatim}
\end{kframe}
\end{knitrout}
\end{frame}


\begin{frame}[fragile]
  \frametitle{Indexing data frames}
  \small
  Bracket method. Extract data from row 1, columns 1 and 3
\begin{knitrout}\footnotesize
\definecolor{shadecolor}{rgb}{0.878, 0.918, 0.933}\color{fgcolor}\begin{kframe}
\begin{alltt}
\hlstd{mydata[}\hlnum{1}\hlstd{,}\hlkwd{c}\hlstd{(}\hlnum{1}\hlstd{,}\hlnum{3}\hlstd{)]}
\end{alltt}
\begin{verbatim}
##       Goats Temp
## Site1     3  2.2
\end{verbatim}
\end{kframe}
\end{knitrout}
\pause \vfill
Bracket method. Extract from rows 2 and 3, columns 1 and 3
\begin{knitrout}\footnotesize
\definecolor{shadecolor}{rgb}{0.878, 0.918, 0.933}\color{fgcolor}\begin{kframe}
\begin{alltt}
\hlstd{mydata[}\hlkwd{c}\hlstd{(}\hlstr{'Site2'}\hlstd{,} \hlstr{'Site3'}\hlstd{),} \hlkwd{c}\hlstd{(}\hlstr{'Goats'}\hlstd{,} \hlstr{'Temp'}\hlstd{)]}
\end{alltt}
\begin{verbatim}
##       Goats Temp
## Site2     9  3.4
## Site3     7  4.4
\end{verbatim}
\end{kframe}
\end{knitrout}
\pause \vfill
Dollar sign method. Extract data from column 1
\begin{knitrout}\footnotesize
\definecolor{shadecolor}{rgb}{0.878, 0.918, 0.933}\color{fgcolor}\begin{kframe}
\begin{alltt}
\hlstd{mydata}\hlopt{$}\hlstd{Elev}
\end{alltt}
\begin{verbatim}
## [1] High High Low  Low 
## Levels: High Low
\end{verbatim}
\end{kframe}
\end{knitrout}
\end{frame}


\begin{frame}[fragile]
  \frametitle{Summarizing data frames}
  \small
  View the data as a `spreadsheet'
\begin{knitrout}\footnotesize
\definecolor{shadecolor}{rgb}{0.878, 0.918, 0.933}\color{fgcolor}\begin{kframe}
\begin{alltt}
\hlkwd{View}\hlstd{(mydata)}
\end{alltt}
\end{kframe}
\end{knitrout}
\pause \vfill
Compute some summary statistics
\begin{knitrout}\scriptsize
\definecolor{shadecolor}{rgb}{0.878, 0.918, 0.933}\color{fgcolor}\begin{kframe}
\begin{alltt}
\hlkwd{summary}\hlstd{(mydata)}
\end{alltt}
\begin{verbatim}
##      Goats        Elev        Temp      
##  Min.   :3.00   High:2   Min.   :2.200  
##  1st Qu.:3.75   Low :2   1st Qu.:3.100  
##  Median :5.50            Median :3.650  
##  Mean   :5.75            Mean   :3.475  
##  3rd Qu.:7.50            3rd Qu.:4.025  
##  Max.   :9.00            Max.   :4.400
\end{verbatim}
\end{kframe}
\end{knitrout}
\pause \vfill
A very compact summary
\begin{knitrout}\scriptsize
\definecolor{shadecolor}{rgb}{0.878, 0.918, 0.933}\color{fgcolor}\begin{kframe}
\begin{alltt}
\hlkwd{str}\hlstd{(mydata)}
\end{alltt}
\begin{verbatim}
## 'data.frame':	4 obs. of  3 variables:
##  $ Goats: num  3 9 7 4
##  $ Elev : Factor w/ 2 levels "High","Low": 1 1 2 2
##  $ Temp : num  2.2 3.4 4.4 3.9
\end{verbatim}
\end{kframe}
\end{knitrout}
\end{frame}



\subsection{Importing and Exporting Data}

\begin{frame}[fragile]
  \frametitle{Importing and exporting data}
  Data can be imported using \inr{read.csv} and exported using \inr{write.csv}. \\
  \small
  \pause \vfill
  Export the \verb+data.frame+ we created earlier
\begin{knitrout}\small
\definecolor{shadecolor}{rgb}{0.878, 0.918, 0.933}\color{fgcolor}\begin{kframe}
\begin{alltt}
\hlkwd{write.csv}\hlstd{(mydata,} \hlkwc{file}\hlstd{=}\hlstr{"mydata.csv"}\hlstd{)}
\end{alltt}
\end{kframe}
\end{knitrout}
  \pause \vfill
The data will be exported to your working directory:
\begin{knitrout}\scriptsize
\definecolor{shadecolor}{rgb}{0.878, 0.918, 0.933}\color{fgcolor}\begin{kframe}
\begin{alltt}
\hlkwd{getwd}\hlstd{()} \hlcom{# Go to this location and look for 'mydata.csv'}
\end{alltt}
\end{kframe}
\end{knitrout}
  \pause \vfill
Import it back in:
\begin{knitrout}\small
\definecolor{shadecolor}{rgb}{0.878, 0.918, 0.933}\color{fgcolor}\begin{kframe}
\begin{alltt}
\hlstd{mydata2} \hlkwb{<-} \hlkwd{read.csv}\hlstd{(}\hlstr{"mydata.csv"}\hlstd{)}
\end{alltt}
\end{kframe}
\end{knitrout}
\end{frame}







\begin{frame}[fragile]
  \frametitle{The working directory}
The working directory is the location on your computer where \R~will
look for files and export files. \\
\pause \vfill
You can check your working directory like this:
\begin{knitrout}\small
\definecolor{shadecolor}{rgb}{0.878, 0.918, 0.933}\color{fgcolor}\begin{kframe}
\begin{alltt}
\hlkwd{getwd}\hlstd{()}
\end{alltt}
\begin{verbatim}
## [1] "/home/rchandler/courses/exp-design/labs/intro-to-R"
\end{verbatim}
\end{kframe}
\end{knitrout}
\pause \vfill
Change your working directory to another location:
\begin{knitrout}\small
\definecolor{shadecolor}{rgb}{0.878, 0.918, 0.933}\color{fgcolor}\begin{kframe}
\begin{alltt}
\hlcom{## Note the forward slashes, which could be replaced by "\textbackslash{}\textbackslash{}"}
\hlkwd{setwd}\hlstd{(}\hlstr{"C:/work/courses/"}\hlstd{)}
\end{alltt}
\end{kframe}
\end{knitrout}
\pause \vfill
\centering
{\bf At the beginning of every \R~session, you should use
  \inr{setwd} to set your working directory \\}
\end{frame}








\subsection{Removing objects and saving workspaces}

\begin{frame}[fragile]
  \frametitle{The workspace}
  View the objects in your workspace
\begin{knitrout}\scriptsize
\definecolor{shadecolor}{rgb}{0.878, 0.918, 0.933}\color{fgcolor}\begin{kframe}
\begin{alltt}
\hlkwd{ls}\hlstd{()}
\end{alltt}
\begin{verbatim}
##  [1] "BMI"     "height"  "mydata"  "mydata2" "rnw2pdf" "weight"  "x1"     
##  [8] "x2"      "y"       "y.re"    "y.sub1"  "y.sub2"  "y.sub4"  "y1"     
## [15] "y2"      "z"
\end{verbatim}
\end{kframe}
\end{knitrout}
\pause \vfill
Remove (delete) some objects
\begin{knitrout}\scriptsize
\definecolor{shadecolor}{rgb}{0.878, 0.918, 0.933}\color{fgcolor}\begin{kframe}
\begin{alltt}
\hlkwd{rm}\hlstd{(x1, x2, mydata2, y, y1, y2, y.re,}
   \hlstd{y.sub1, y.sub2, y.sub4, height, weight, BMI, z)}
\hlkwd{ls}\hlstd{()}
\end{alltt}
\begin{verbatim}
## [1] "mydata"  "rnw2pdf"
\end{verbatim}
\end{kframe}
\end{knitrout}
\end{frame}





\begin{frame}[fragile]
  \frametitle{Saving and restoring the workspace}
Save all the objects in your workspace to a file and then remove them:
\begin{knitrout}
\definecolor{shadecolor}{rgb}{0.878, 0.918, 0.933}\color{fgcolor}\begin{kframe}
\begin{alltt}
\hlkwd{save.image}\hlstd{(}\hlstr{"myimage.RData"}\hlstd{)}  \hlcom{## Save workspace}
\hlkwd{rm}\hlstd{(}\hlkwc{list}\hlstd{=}\hlkwd{ls}\hlstd{())}                \hlcom{## Remove all objects}
\hlkwd{ls}\hlstd{()}                         \hlcom{## All are gone}
\end{alltt}
\begin{verbatim}
## character(0)
\end{verbatim}
\end{kframe}
\end{knitrout}
\pause
\vfill
Load the objects back into the workspace:
\begin{knitrout}\small
\definecolor{shadecolor}{rgb}{0.878, 0.918, 0.933}\color{fgcolor}\begin{kframe}
\begin{alltt}
\hlkwd{load}\hlstd{(}\hlstr{"myimage.RData"}\hlstd{)}        \hlcom{## Load the saved workspace}
\hlkwd{ls}\hlstd{()}                         \hlcom{## Check that the objects are back}
\end{alltt}
\begin{verbatim}
## [1] "mydata"  "rnw2pdf"
\end{verbatim}
\end{kframe}
\end{knitrout}
\end{frame}




\section{Getting help}



\begin{frame}[fragile]
  \frametitle{Additional Resources}
  \large
  `Official' manuals
\begin{knitrout}
\definecolor{shadecolor}{rgb}{0.878, 0.918, 0.933}\color{fgcolor}\begin{kframe}
\begin{alltt}
\hlkwd{help.start}\hlstd{()}
\end{alltt}
\end{kframe}
\end{knitrout}
\vfill
Useful books
\begin{itemize}
  \large
\item Venables, W.N. and B.D. Ripley. 2002. Modern Applied Statistics with
  S, 4th ed. Springer.
\item Crawley, M.J.. 2013. The \R~Book, 2nd ed. Wiley
\end{itemize}
%  \item[]
%  \item
\vfill
Online
    \begin{itemize}
      \large
      \item Always Google your error messages
      \item \url{http://stackoverflow.com/}
      \item \url{https://preludeinr.com/}
    \end{itemize}
%\end{itemize}
\end{frame}






\begin{frame}[fragile]
  \frametitle{Assignment -- Part I}
  \footnotesize
  \begin{enumerate}[\bf (1)]
    \item Create an \R~script named something like \verb+Chandler-R_lab1.R+
    \item In your \R~script, write code to create a \verb+data.frame+
      that contains the information shown in the table below
    \item Add code to export the \verb+data.frame+ as a .csv file and
      then import it back into \R.
  \end{enumerate}
  \begin{center}
    \scriptsize
    \begin{tabular}{lccc}
      \hline
      Individual & Mass & Weight & Treatment \\
      \hline
      1 & 3 & 4 & Control \\
      2 & 3 & 5 & Control \\
      3 & 2 & 4 & Control \\
      4 & 4 & 6 & Treatment \\
      5 & 5 & 5 & Treatment \\
      6 & 5 & 7 & Treatment \\
      \hline
    \end{tabular}
  \end{center}
  Upload your \R~script\footnote{\scriptsize \noindent If you
    are familiar with RMarkdown, you can submit a \texttt{.Rmd} file
    instead of a \texttt{.R} file} to ELC before your next lab. The script should
  be self-contained, meaning that it will run correctly when we copy
  and paste it in the console.
\end{frame}



\begin{frame}[fragile]
  \frametitle{Assignment -- Part II}
  \begin{columns}
    \begin{column}{0.3\textwidth}
      \includegraphics[width=1\textwidth]{figs/introStatsR} \\
%      \tiny
    \end{column}
    \begin{column}{0.6\textwidth}
%      \begin{enumerate}[\bf (1)]
%        \item
      Read chapters 1, 4, \& 5 before next lab
%        \item[]
%        \item Create an R script to import a .csv file (you can use
%          any file you want). Upload the script and the .csv file to
%          ELC.
%      \end{enumerate}
    \end{column}
  \end{columns}
  \vfill
  \footnotesize
  Dalgaard, P. 2008. Introductory Statistics with R. 2nd
  edition. Springer. \\
%  \vspace{0.3cm}
  Available for free through the UGA library: \\ \tiny
  \url{http://preproxy.galib.uga.edu/login?url=http://dx.doi.org/10.1007/978-0-387-79054-1}
\end{frame}



\end{document}
