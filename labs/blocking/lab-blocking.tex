\documentclass[color=usenames,dvipsnames]{beamer}\usepackage[]{graphicx}\usepackage[]{color}
%% maxwidth is the original width if it is less than linewidth
%% otherwise use linewidth (to make sure the graphics do not exceed the margin)
\makeatletter
\def\maxwidth{ %
  \ifdim\Gin@nat@width>\linewidth
    \linewidth
  \else
    \Gin@nat@width
  \fi
}
\makeatother

\definecolor{fgcolor}{rgb}{0, 0, 0}
\newcommand{\hlnum}[1]{\textcolor[rgb]{0.69,0.494,0}{#1}}%
\newcommand{\hlstr}[1]{\textcolor[rgb]{0.749,0.012,0.012}{#1}}%
\newcommand{\hlcom}[1]{\textcolor[rgb]{0.514,0.506,0.514}{\textit{#1}}}%
\newcommand{\hlopt}[1]{\textcolor[rgb]{0,0,0}{#1}}%
\newcommand{\hlstd}[1]{\textcolor[rgb]{0,0,0}{#1}}%
\newcommand{\hlkwa}[1]{\textcolor[rgb]{0,0,0}{\textbf{#1}}}%
\newcommand{\hlkwb}[1]{\textcolor[rgb]{0,0.341,0.682}{#1}}%
\newcommand{\hlkwc}[1]{\textcolor[rgb]{0,0,0}{\textbf{#1}}}%
\newcommand{\hlkwd}[1]{\textcolor[rgb]{0.004,0.004,0.506}{#1}}%
\let\hlipl\hlkwb

\usepackage{framed}
\makeatletter
\newenvironment{kframe}{%
 \def\at@end@of@kframe{}%
 \ifinner\ifhmode%
  \def\at@end@of@kframe{\end{minipage}}%
  \begin{minipage}{\columnwidth}%
 \fi\fi%
 \def\FrameCommand##1{\hskip\@totalleftmargin \hskip-\fboxsep
 \colorbox{shadecolor}{##1}\hskip-\fboxsep
     % There is no \\@totalrightmargin, so:
     \hskip-\linewidth \hskip-\@totalleftmargin \hskip\columnwidth}%
 \MakeFramed {\advance\hsize-\width
   \@totalleftmargin\z@ \linewidth\hsize
   \@setminipage}}%
 {\par\unskip\endMakeFramed%
 \at@end@of@kframe}
\makeatother

\definecolor{shadecolor}{rgb}{.97, .97, .97}
\definecolor{messagecolor}{rgb}{0, 0, 0}
\definecolor{warningcolor}{rgb}{1, 0, 1}
\definecolor{errorcolor}{rgb}{1, 0, 0}
\newenvironment{knitrout}{}{} % an empty environment to be redefined in TeX

\usepackage{alltt}
%\documentclass[color=usenames,dvipsnames,handout]{beamer}


%\usepackage[roman]{../../lab1}
\usepackage[sans]{../../lab1}

\hypersetup{pdftex,pdfstartview=FitV}









%% New command for inline code that isn't to be evaluated
\definecolor{inlinecolor}{rgb}{0.878, 0.918, 0.933}
\newcommand{\inr}[1]{\colorbox{inlinecolor}{\texttt{#1}}}
\IfFileExists{upquote.sty}{\usepackage{upquote}}{}
\begin{document}


%\setlength\fboxsep{0pt}



%\section{Intro}

\begin{frame}[plain]
%  \maketitle
  \LARGE
  \centering \par
  {\color{RoyalBlue}{Lab 6 -- Randomized Complete Block Design \par}}
  \vspace{1cm}
  \large
%  October 1 \& 2, 2018 \\
  September 24 \& 25, 2018 \\
  FANR 6750 \\
  \vfill
  \large
  Richard Chandler and Bob Cooper
\end{frame}


\section{Recap}

\begin{frame}
  \frametitle{Recap}
  \large
%  \begin{itemize}%[<+->]
%    \item
  Like one-way ANOVA but experimental units are
  organized into blocks to account for extraneous sources of
  variation \\
      \pause
      \vfill
%    \item
      Blocks could be regions, time periods, individual subjects,
      etc$\ldots$ \\
      \pause
      \vfill
%    \item
      Blocking must occur during the design phase of the study \\
      \pause
      \vfill
%  \end{itemize}
      Additive model:
  \[
    y_{ij} = \mu + \alpha_i + \beta_j + \varepsilon_{ij}
  \]
\end{frame}


\section{Blocked ANOVA ``By Hand''}


\begin{comment}
\begin{frame}[fragile]
  \frametitle{Gypsy Moth Data}
\begin{knitrout}
\definecolor{shadecolor}{rgb}{0.878, 0.918, 0.933}\color{fgcolor}\begin{kframe}
\begin{alltt}
\hlstd{a} \hlkwb{<-} \hlnum{3} \hlcom{# Number of treatments}
\hlstd{b} \hlkwb{<-} \hlnum{4} \hlcom{# Number of blocks}
\hlstd{larvae} \hlkwb{<-} \hlkwd{c}\hlstd{(}\hlnum{16}\hlstd{,} \hlnum{3}\hlstd{,} \hlnum{10}\hlstd{,} \hlnum{18}\hlstd{,} \hlnum{25}\hlstd{,} \hlnum{10}\hlstd{,} \hlnum{15}\hlstd{,} \hlnum{32}\hlstd{,} \hlnum{14}\hlstd{,} \hlnum{2}\hlstd{,} \hlnum{16}\hlstd{,} \hlnum{12}\hlstd{)}
\hlstd{treatment} \hlkwb{<-} \hlkwd{factor}\hlstd{(}\hlkwd{rep}\hlstd{(}\hlkwd{c}\hlstd{(}\hlstr{"Bt"}\hlstd{,} \hlstr{"Control"}\hlstd{,} \hlstr{"Dimilin"}\hlstd{),} \hlkwc{each}\hlstd{=b))}
\hlstd{block} \hlkwb{<-} \hlkwd{factor}\hlstd{(}\hlkwd{rep}\hlstd{(}\hlnum{1}\hlopt{:}\hlnum{4}\hlstd{,} \hlkwc{times}\hlstd{=a))}
\hlstd{gypsyData} \hlkwb{<-} \hlkwd{data.frame}\hlstd{(}\hlkwc{caterpillars}\hlstd{=larvae,} \hlkwc{pesticide}\hlstd{=treatment,} \hlkwc{region}\hlstd{=block)}
\hlstd{gypsyData}
\end{alltt}
\begin{verbatim}
##    caterpillars pesticide region
## 1            16        Bt      1
## 2             3        Bt      2
## 3            10        Bt      3
## 4            18        Bt      4
## 5            25   Control      1
## 6            10   Control      2
## 7            15   Control      3
## 8            32   Control      4
## 9            14   Dimilin      1
## 10            2   Dimilin      2
## 11           16   Dimilin      3
## 12           12   Dimilin      4
\end{verbatim}
\begin{alltt}
\hlkwd{write.csv}\hlstd{(gypsyData,} \hlstr{"gypsyData.csv"}\hlstd{,} \hlkwc{row.names}\hlstd{=}\hlnum{FALSE}\hlstd{)}
\end{alltt}
\end{kframe}
\end{knitrout}
\end{frame}
\end{comment}



\begin{frame}[fragile]
  \frametitle{Gypsy Moth Data}
%  \small
  \footnotesize
\begin{knitrout}\small
\definecolor{shadecolor}{rgb}{0.878, 0.918, 0.933}\color{fgcolor}\begin{kframe}
\begin{alltt}
\hlstd{gypsyData} \hlkwb{<-} \hlkwd{read.csv}\hlstd{(}\hlstr{"gypsyData.csv"}\hlstd{)}
\hlstd{gypsyData}\hlopt{$}\hlstd{region} \hlkwb{<-} \hlkwd{factor}\hlstd{(gypsyData}\hlopt{$}\hlstd{region)} \hlcom{# Convert to factor}
\hlstd{gypsyData}
\end{alltt}
\begin{verbatim}
##    caterpillars pesticide region
## 1            16        Bt      1
## 2             3        Bt      2
## 3            10        Bt      3
## 4            18        Bt      4
## 5            25   Control      1
## 6            10   Control      2
## 7            15   Control      3
## 8            32   Control      4
## 9            14   Dimilin      1
## 10            2   Dimilin      2
## 11           16   Dimilin      3
## 12           12   Dimilin      4
\end{verbatim}
\end{kframe}
\end{knitrout}
\vfill
\alert{Note:} Numeric grouping variables must be coded as factors.
\end{frame}




\begin{frame}[fragile]
  \frametitle{Compute the means}
  \small
{Grand mean ($\bar{y}_.$)}
\begin{knitrout}\footnotesize
\definecolor{shadecolor}{rgb}{0.878, 0.918, 0.933}\color{fgcolor}\begin{kframe}
\begin{alltt}
\hlstd{caterpillars} \hlkwb{<-} \hlstd{gypsyData}\hlopt{$}\hlstd{caterpillars}
\hlstd{(grand.mean} \hlkwb{<-} \hlkwd{mean}\hlstd{(caterpillars))}
\end{alltt}
\begin{verbatim}
## [1] 14.41667
\end{verbatim}
\end{kframe}
\end{knitrout}
\pause
%\vspace{0.3cm}
{ Treatment means ($\bar{y}_i$)}
\begin{knitrout}\footnotesize
\definecolor{shadecolor}{rgb}{0.878, 0.918, 0.933}\color{fgcolor}\begin{kframe}
\begin{alltt}
\hlstd{pesticide} \hlkwb{<-} \hlstd{gypsyData}\hlopt{$}\hlstd{pesticide}
\hlstd{(treatment.means} \hlkwb{<-} \hlkwd{tapply}\hlstd{(caterpillars, pesticide, mean))}
\end{alltt}
\begin{verbatim}
##      Bt Control Dimilin 
##   11.75   20.50   11.00
\end{verbatim}
\end{kframe}
\end{knitrout}
\pause
%\vspace{0.3cm}
{ Block means ($\bar{y}_j$)}
\begin{knitrout}\footnotesize
\definecolor{shadecolor}{rgb}{0.878, 0.918, 0.933}\color{fgcolor}\begin{kframe}
\begin{alltt}
\hlstd{region} \hlkwb{<-} \hlstd{gypsyData}\hlopt{$}\hlstd{region}
\hlstd{(block.means} \hlkwb{<-} \hlkwd{tapply}\hlstd{(caterpillars, region, mean))}
\end{alltt}
\begin{verbatim}
##        1        2        3        4 
## 18.33333  5.00000 13.66667 20.66667
\end{verbatim}
\end{kframe}
\end{knitrout}
\end{frame}


\begin{frame}[fragile]
  \frametitle{Treatment sums-of-squares}
  {\Large
  \[
  b \times \sum_{i=1}^a (\bar{y}_i - \bar{y}_.)^2
  \]
  }
\begin{knitrout}
\definecolor{shadecolor}{rgb}{0.878, 0.918, 0.933}\color{fgcolor}\begin{kframe}
\begin{alltt}
\hlstd{b} \hlkwb{<-} \hlnum{4}
\hlstd{b} \hlkwb{<-} \hlkwd{nlevels}\hlstd{(region)}
\hlstd{SS.treat} \hlkwb{<-} \hlstd{b}\hlopt{*}\hlkwd{sum}\hlstd{((treatment.means} \hlopt{-} \hlstd{grand.mean)}\hlopt{^}\hlnum{2}\hlstd{)}
\hlstd{SS.treat}
\end{alltt}
\begin{verbatim}
## [1] 223.1667
\end{verbatim}
\end{kframe}
\end{knitrout}
\end{frame}



\begin{frame}[fragile]
  \frametitle{Block sums-of-squares}
  {\Large
  \[
  a \times \sum_{j=1}^b (\bar{y}_j - \bar{y}_.)^2
  \]
  }
\begin{knitrout}
\definecolor{shadecolor}{rgb}{0.878, 0.918, 0.933}\color{fgcolor}\begin{kframe}
\begin{alltt}
\hlstd{a} \hlkwb{<-} \hlkwd{nlevels}\hlstd{(pesticide)}
\hlstd{SS.block} \hlkwb{<-} \hlstd{a}\hlopt{*}\hlkwd{sum}\hlstd{((block.means} \hlopt{-} \hlstd{grand.mean)}\hlopt{^}\hlnum{2}\hlstd{)}
\hlstd{SS.block}
\end{alltt}
\begin{verbatim}
## [1] 430.9167
\end{verbatim}
\end{kframe}
\end{knitrout}
\end{frame}



\begin{frame}[fragile]
  \frametitle{Within groups sums-of-squares}
  {\Large
    \[
    \sum_{i=1}^a \sum_{j=1}^b (y_{ij} - \bar{y}_i - \bar{y}_j + \bar{y}_.)^2
    \]
  }
%  \pause
\begin{knitrout}\small
\definecolor{shadecolor}{rgb}{0.878, 0.918, 0.933}\color{fgcolor}\begin{kframe}
\begin{alltt}
\hlstd{treatment.means.long} \hlkwb{<-} \hlkwd{rep}\hlstd{(treatment.means,} \hlkwc{each}\hlstd{=b)}
\hlstd{block.means.long} \hlkwb{<-} \hlkwd{rep}\hlstd{(block.means,} \hlkwc{times}\hlstd{=a)}
\hlstd{SS.within} \hlkwb{<-} \hlkwd{sum}\hlstd{((caterpillars} \hlopt{-} \hlstd{treatment.means.long} \hlopt{-}
                  \hlstd{block.means.long} \hlopt{+} \hlstd{grand.mean)}\hlopt{^}\hlnum{2}\hlstd{)}
\hlstd{SS.within}
\end{alltt}
\begin{verbatim}
## [1] 114.8333
\end{verbatim}
\end{kframe}
\end{knitrout}
\pause
\vfill
{\bf NOTE:} For this to work, \inr{treatment.means} and
\inr{block.means} must be in the same order as in the original
data. \\
\end{frame}






\begin{frame}[fragile]
  \frametitle{Create ANOVA table}
\begin{knitrout}\small
\definecolor{shadecolor}{rgb}{0.878, 0.918, 0.933}\color{fgcolor}\begin{kframe}
\begin{alltt}
\hlstd{df.treat} \hlkwb{<-} \hlstd{a}\hlopt{-}\hlnum{1}
\hlstd{df.block} \hlkwb{<-} \hlstd{b}\hlopt{-}\hlnum{1}
\hlstd{df.within} \hlkwb{<-} \hlstd{df.treat}\hlopt{*}\hlstd{df.block}
\hlstd{ANOVAtable} \hlkwb{<-} \hlkwd{data.frame}\hlstd{(}
    \hlkwc{df} \hlstd{=} \hlkwd{c}\hlstd{(df.treat, df.block, df.within),}
    \hlkwc{SS} \hlstd{=} \hlkwd{c}\hlstd{(SS.treat, SS.block, SS.within))}
\hlkwd{rownames}\hlstd{(ANOVAtable)} \hlkwb{<-} \hlkwd{c}\hlstd{(}\hlstr{"Treatment"}\hlstd{,} \hlstr{"Block"}\hlstd{,} \hlstr{"Within"}\hlstd{)}
\hlstd{ANOVAtable}
\end{alltt}
\begin{verbatim}
##           df       SS
## Treatment  2 223.1667
## Block      3 430.9167
## Within     6 114.8333
\end{verbatim}
\end{kframe}
\end{knitrout}
\end{frame}



\begin{frame}[fragile]
  \frametitle{Create ANOVA table, continued$\dots$}
  \small
  { Mean squares}
\begin{knitrout}
\definecolor{shadecolor}{rgb}{0.878, 0.918, 0.933}\color{fgcolor}\begin{kframe}
\begin{alltt}
\hlstd{MSE} \hlkwb{<-} \hlstd{ANOVAtable}\hlopt{$}\hlstd{SS} \hlopt{/} \hlstd{ANOVAtable}\hlopt{$}\hlstd{df}
\hlstd{ANOVAtable}\hlopt{$}\hlstd{MSE} \hlkwb{<-} \hlstd{MSE}
\end{alltt}
\end{kframe}
\end{knitrout}
\pause
%\vspace{0.3cm}
{ $F$ values}
\begin{knitrout}
\definecolor{shadecolor}{rgb}{0.878, 0.918, 0.933}\color{fgcolor}\begin{kframe}
\begin{alltt}
\hlstd{F} \hlkwb{<-} \hlkwd{c}\hlstd{(MSE[}\hlnum{1}\hlstd{]}\hlopt{/}\hlstd{MSE[}\hlnum{3}\hlstd{], MSE[}\hlnum{2}\hlstd{]}\hlopt{/}\hlstd{MSE[}\hlnum{3}\hlstd{],} \hlnum{NA}\hlstd{)}
\hlstd{ANOVAtable}\hlopt{$}\hlstd{F} \hlkwb{<-} \hlstd{F}
\end{alltt}
\end{kframe}
\end{knitrout}
\pause
%\vspace{0.3cm}
{ $P$-values}
\begin{knitrout}
\definecolor{shadecolor}{rgb}{0.878, 0.918, 0.933}\color{fgcolor}\begin{kframe}
\begin{alltt}
\hlstd{P} \hlkwb{<-} \hlkwd{c}\hlstd{(}\hlnum{1} \hlopt{-} \hlkwd{pf}\hlstd{(F[}\hlnum{1}\hlstd{],} \hlnum{2}\hlstd{,} \hlnum{6}\hlstd{),} \hlnum{1} \hlopt{-} \hlkwd{pf}\hlstd{(F[}\hlnum{2}\hlstd{],} \hlnum{3}\hlstd{,} \hlnum{6}\hlstd{),} \hlnum{NA}\hlstd{)}
\hlstd{ANOVAtable}\hlopt{$}\hlstd{P} \hlkwb{<-} \hlstd{P}
\hlkwd{round}\hlstd{(ANOVAtable,} \hlnum{3}\hlstd{)}
\end{alltt}
\begin{verbatim}
##           df      SS     MSE     F     P
## Treatment  2 223.167 111.583 5.830 0.039
## Block      3 430.917 143.639 7.505 0.019
## Within     6 114.833  19.139    NA    NA
\end{verbatim}
\end{kframe}
\end{knitrout}
\end{frame}


\begin{frame}[fragile]
  \frametitle{Reminder about $P$-values}
  \footnotesize

Critical value
\begin{knitrout}\scriptsize
\definecolor{shadecolor}{rgb}{0.878, 0.918, 0.933}\color{fgcolor}\begin{kframe}
\begin{alltt}
\hlkwd{qf}\hlstd{(}\hlnum{0.95}\hlstd{,} \hlkwc{df1}\hlstd{=}\hlnum{2}\hlstd{,} \hlkwc{df2}\hlstd{=}\hlnum{6}\hlstd{)} \hlcom{# 95% of the distribution is before this value of F}
\end{alltt}
\begin{verbatim}
## [1] 5.143253
\end{verbatim}
\end{kframe}
\end{knitrout}
$P$-value
\begin{knitrout}\scriptsize
\definecolor{shadecolor}{rgb}{0.878, 0.918, 0.933}\color{fgcolor}\begin{kframe}
\begin{alltt}
\hlnum{1}\hlopt{-}\hlkwd{pf}\hlstd{(F[}\hlnum{1}\hlstd{],} \hlkwc{df1}\hlstd{=}\hlnum{2}\hlstd{,} \hlkwc{df2}\hlstd{=}\hlnum{6}\hlstd{)} \hlcom{# Proportion of the distribution beyond this F value}
\end{alltt}
\begin{verbatim}
## [1] 0.03921514
\end{verbatim}
\end{kframe}
\end{knitrout}
\begin{center}
  \includegraphics[width=9cm]{figure/Fdist-1}
\end{center}
\end{frame}



\section{Blocked ANOVA Using {\tt aov}}


\begin{frame}[fragile]
  \frametitle{Using {\tt aov}}
  \small
\begin{knitrout}
\definecolor{shadecolor}{rgb}{0.878, 0.918, 0.933}\color{fgcolor}\begin{kframe}
\begin{alltt}
\hlstd{aov1} \hlkwb{<-} \hlkwd{aov}\hlstd{(caterpillars} \hlopt{~} \hlstd{pesticide} \hlopt{+} \hlstd{region, gypsyData)}
\hlkwd{summary}\hlstd{(aov1)}
\end{alltt}
\begin{verbatim}
##             Df Sum Sq Mean Sq F value Pr(>F)  
## pesticide    2  223.2  111.58   5.830 0.0392 *
## region       3  430.9  143.64   7.505 0.0187 *
## Residuals    6  114.8   19.14                 
## ---
## Signif. codes:  0 '***' 0.001 '**' 0.01 '*' 0.05 '.' 0.1 ' ' 1
\end{verbatim}
\end{kframe}
\end{knitrout}
\pause
\begin{knitrout}
\definecolor{shadecolor}{rgb}{0.878, 0.918, 0.933}\color{fgcolor}\begin{kframe}
\begin{alltt}
\hlkwd{round}\hlstd{(ANOVAtable,} \hlnum{3}\hlstd{)}
\end{alltt}
\begin{verbatim}
##           df      SS     MSE     F     P
## Treatment  2 223.167 111.583 5.830 0.039
## Block      3 430.917 143.639 7.505 0.019
## Within     6 114.833  19.139    NA    NA
\end{verbatim}
\end{kframe}
\end{knitrout}
\end{frame}


\begin{frame}[fragile]
  \frametitle{Using {\tt aov}}
  { Look what happens if we ignore the blocking variable \par}
  \small
\begin{knitrout}
\definecolor{shadecolor}{rgb}{0.878, 0.918, 0.933}\color{fgcolor}\begin{kframe}
\begin{alltt}
\hlstd{aov2} \hlkwb{<-} \hlkwd{aov}\hlstd{(caterpillars} \hlopt{~} \hlstd{pesticide, gypsyData)}
\hlkwd{summary}\hlstd{(aov2)}
\end{alltt}
\begin{verbatim}
##             Df Sum Sq Mean Sq F value Pr(>F)
## pesticide    2  223.2  111.58    1.84  0.214
## Residuals    9  545.8   60.64
\end{verbatim}
\end{kframe}
\end{knitrout}
\pause
\vfill
%\centering
  { Why is the effect of pesticide no longer significant? \\}
\end{frame}



\begin{frame}[fragile]
  \frametitle{Treating block effects as random effects}
%  \small
  \footnotesize
\begin{knitrout}
\definecolor{shadecolor}{rgb}{0.878, 0.918, 0.933}\color{fgcolor}\begin{kframe}
\begin{alltt}
\hlstd{aov3} \hlkwb{<-} \hlkwd{aov}\hlstd{(caterpillars} \hlopt{~} \hlstd{pesticide} \hlopt{+} \hlkwd{Error}\hlstd{(region), gypsyData)}
\hlkwd{summary}\hlstd{(aov3)}
\end{alltt}
\begin{verbatim}
## 
## Error: region
##           Df Sum Sq Mean Sq F value Pr(>F)
## Residuals  3  430.9   143.6               
## 
## Error: Within
##           Df Sum Sq Mean Sq F value Pr(>F)  
## pesticide  2  223.2  111.58    5.83 0.0392 *
## Residuals  6  114.8   19.14                 
## ---
## Signif. codes:  0 '***' 0.001 '**' 0.01 '*' 0.05 '.' 0.1 ' ' 1
\end{verbatim}
\end{kframe}
\end{knitrout}
\pause
\vfill
The values of the ANOVA table are the same as before, and there is
no reason to use random effects here if interest only lies in
testing the null hypothesis concerning pesticides. Later, we will see
cases where it is important to use random and fixed effects.
\end{frame}




\begin{frame}
  \frametitle{Assignment: Due before lab next week}
  \tiny
%  \scriptsize
  Plantations of {\it Pinus caribaea} were established at
  four locations on Puerto Rico.  %As part of a spacing study,
  Four spacings were used at each location to determine the effect of
  stocking density on tree height. Twenty years after the
  plantations were established, the following tree heights were
  recorded: \\ %par
  \vfill
  {  \centering
%  \begin{columns}
  %   \begin{center}
%  \begin{column}{0.5\textwidth}
    \tiny %\scriptsize
    \begin{tabular}{lcc}
      \hline
      Location  & Spacing (ft) & Height (ft) \\
      \hline
      Caracoles & 5            & 72          \\
                & 7            & 80          \\
                & 10           & 85          \\
                & 14           & 91          \\
      Utado     & 5            & 75          \\
                & 7            & 90          \\
                & 10           & 94          \\
                & 14           & 112         \\
      Guzman    & 5            & 88          \\
                & 7            & 95          \\
                & 10           & 94          \\
                & 14           & 91          \\
      Lares     & 5            & 79          \\
                & 7            & 94          \\
                & 10           & 104         \\
                & 14           & 106         \\
      \hline
    \end{tabular} \\
  }
    Create an \R~script to the address the following:
    \begin{enumerate}[(1)]
      \item What are the null and alternative hypotheses?.
      \item Test for effects of location and spacing on plant height
        using the \inr{aov} function. Do the ANOVA again but without
        \inr{aov}. Treat the block effects as fixed, not random. HINTS:
        \begin{itemize}
          \tiny
          \item Spacing must be treated as a factor.
          \item You must put the group means and block means in the
            correct order when computing the sums-of-squares.
        \end{itemize}
      \item Perform a Tukey test to determine which spacings differ.
      \item Summarize the main results in 2-3 sentences. Upload the
        script to ELC the day before your next lab.
    \end{enumerate}
\end{frame}








\end{document}
