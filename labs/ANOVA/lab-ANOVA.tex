\documentclass[color=usenames,dvipsnames]{beamer}\usepackage[]{graphicx}\usepackage[]{color}
% maxwidth is the original width if it is less than linewidth
% otherwise use linewidth (to make sure the graphics do not exceed the margin)
\makeatletter
\def\maxwidth{ %
  \ifdim\Gin@nat@width>\linewidth
    \linewidth
  \else
    \Gin@nat@width
  \fi
}
\makeatother

\definecolor{fgcolor}{rgb}{0, 0, 0}
\newcommand{\hlnum}[1]{\textcolor[rgb]{0.69,0.494,0}{#1}}%
\newcommand{\hlstr}[1]{\textcolor[rgb]{0.749,0.012,0.012}{#1}}%
\newcommand{\hlcom}[1]{\textcolor[rgb]{0.514,0.506,0.514}{\textit{#1}}}%
\newcommand{\hlopt}[1]{\textcolor[rgb]{0,0,0}{#1}}%
\newcommand{\hlstd}[1]{\textcolor[rgb]{0,0,0}{#1}}%
\newcommand{\hlkwa}[1]{\textcolor[rgb]{0,0,0}{\textbf{#1}}}%
\newcommand{\hlkwb}[1]{\textcolor[rgb]{0,0.341,0.682}{#1}}%
\newcommand{\hlkwc}[1]{\textcolor[rgb]{0,0,0}{\textbf{#1}}}%
\newcommand{\hlkwd}[1]{\textcolor[rgb]{0.004,0.004,0.506}{#1}}%
\let\hlipl\hlkwb

\usepackage{framed}
\makeatletter
\newenvironment{kframe}{%
 \def\at@end@of@kframe{}%
 \ifinner\ifhmode%
  \def\at@end@of@kframe{\end{minipage}}%
  \begin{minipage}{\columnwidth}%
 \fi\fi%
 \def\FrameCommand##1{\hskip\@totalleftmargin \hskip-\fboxsep
 \colorbox{shadecolor}{##1}\hskip-\fboxsep
     % There is no \\@totalrightmargin, so:
     \hskip-\linewidth \hskip-\@totalleftmargin \hskip\columnwidth}%
 \MakeFramed {\advance\hsize-\width
   \@totalleftmargin\z@ \linewidth\hsize
   \@setminipage}}%
 {\par\unskip\endMakeFramed%
 \at@end@of@kframe}
\makeatother

\definecolor{shadecolor}{rgb}{.97, .97, .97}
\definecolor{messagecolor}{rgb}{0, 0, 0}
\definecolor{warningcolor}{rgb}{1, 0, 1}
\definecolor{errorcolor}{rgb}{1, 0, 0}
\newenvironment{knitrout}{}{} % an empty environment to be redefined in TeX

\usepackage{alltt}
%\documentclass[color=usenames,dvipsnames,handout]{beamer}

\usepackage[sans]{../../lab1}

\hypersetup{pdftex,pdfstartview=FitV}








%% New command for inline code that isn't to be evaluated
\definecolor{inlinecolor}{rgb}{0.878, 0.918, 0.933}
\newcommand{\inr}[1]{\colorbox{inlinecolor}{\texttt{#1}}}
\IfFileExists{upquote.sty}{\usepackage{upquote}}{}
\begin{document}


%\setlength\fboxsep{0pt}


%\section{Intro}

\begin{frame}[plain]
%  \maketitle
  \LARGE
  \centering \par
  \textcolor{RoyalBlue}{Lab 3 -- Completely Randomized ANOVA \\
  (a.k.a. One-way ANOVA)} \par
  \vspace{1cm}
%  \large
%  Aug 27 \& 28, 2018 \\
  FANR 6750 \\
  \vfill
  \large
  Richard Chandler and Bob Cooper
\end{frame}





%% \begin{frame}[plain]
%%   \frametitle{Recap}
%%   \Large
%%   Last week we covered:
%%   \begin{itemize}
%%     \item $t$-tests
%%   \end{itemize}
%%   \note{Ask students if they have questions about last week or general
%%     questions about \R}
%% \end{frame}



\begin{frame}[plain]
  \frametitle{Today's Topics}
  \LARGE
  \only<1>{\tableofcontents}%[hideallsubsections]}
%  \only<2 | handout:0>{\tableofcontents[currentsection]}%,hideallsubsections]}
\end{frame}



\section{Brief Overview of ANOVA}



\begin{frame}
  \frametitle{One-way ANOVA}
  {\Large Scenario}
  \begin{itemize}
    \item We have independent samples from $a>2$ groups
    \item We assume the residuals are normally distributed with a mean of zero and a common variance
  \end{itemize}
%  \vspace{0.5cm}
  \pause
  \vfill
  {\Large Questions}
  \begin{itemize}
    \item Do the means differ?
    \item By how much? (What are the effect sizes?)
  \end{itemize}
%  \vspace{0.5cm}
  \pause
  \vfill
  {\Large Null hypothesis}
  \begin{itemize}
    \item $H_0: \mu_1 = \mu_2 = \mu_3 = \ldots = \mu_a$
    \item Or:
    \item $H_0: \alpha_1 = \alpha_2 = \alpha_3 = \ldots = \alpha_a = 0$
  \end{itemize}
\end{frame}





\begin{frame}
  \frametitle{Additive model}
  \Large
  \[
    y_{ij} = \mu + \alpha_i + \varepsilon_{ij}
  \]
  \large
  \vfill
  where:
  \begin{itemize}
    \item Residuals: $\varepsilon_{ij} \sim \mathrm{Norm}(0, \sigma^2)$
    \item Group means: $\mu_i = \mu + \alpha_i$
  \end{itemize}
\end{frame}




%% \begin{frame}[fragile]
%%   \frametitle{One-way ANOVA}
%%   \begin{center}
%% <<popdists,include=FALSE,echo=FALSE>>=
%% op <- par(mai=c(0.9,0.1,0.9,0.1))
%% curve(dnorm(x, 5, 2), 0, 20, xlab="Some variable (y)", ylim=c(0,0.21),
%%       yaxt="n", ylab="", frame=FALSE, lwd=2, cex.lab=1.3)
%% curve(dnorm(x, 6, 2), 0, 20, add=TRUE, col="purple", lwd=2)
%% curve(dnorm(x, 10, 2), 0, 20, add=TRUE, col="blue", lwd=2)
%% segments(5, 0, 5, dnorm(5, 5, 2), lty=2)
%% segments(6, 0, 6, dnorm(6, 6, 2), col="purple", lty=2)
%% segments(10, 0, 10, dnorm(10, 10, 2), col="blue", lty=2)
%% text(5, dnorm(5,5,2), expression(mu[1]), pos=3)
%% text(6, dnorm(6,6,2), expression(mu[2]), pos=3, col="purple")
%% text(10, dnorm(10,10,2), expression(mu[3]), pos=3, col="blue")
%% legend(13, 0.2, c("Group 1", "Group 2", "Group 3"),
%%        lwd=2, col=c("black", "purple", "blue"))
%% par(op)
%% @
%% \includegraphics[height=8cm,width=8cm]{lab03-ANOVA1-popdists}
%% \end{center}
%% \end{frame}



\section{ANOVA in \R}





%% <<>>=
%% set.seed(32245)
%% n <- 8 # number of samples per group
%% a <- 3 # number of groups
%% sqShelterwood <- rnorm(n, mean=5, sd=2)
%% sqSingletree <- rnorm(n, mean=6, sd=2)
%% sqControl <- rnorm(n, mean=10, sd=2)
%% sqdata <- data.frame(
%%     sqDensity=c(sqShelterwood, sqSingletree, sqControl),
%%     Treatment=rep(c("Shelterwood", "Single-tree", "Control"), each=n))
%% rownames(sqdata) <- paste("Site", 1:(n*a), sep="")
%% write.csv(sqdata, "sqdata.csv")
%% @




\begin{frame}[fragile]
  \frametitle{Chain Saw Data}
{%\bf
  The data as 4 vectors}
\begin{knitrout}
\definecolor{shadecolor}{rgb}{0.878, 0.918, 0.933}\color{fgcolor}\begin{kframe}
\begin{alltt}
\hlstd{kick.angle.brandA} \hlkwb{<-} \hlkwd{c}\hlstd{(}\hlnum{42}\hlstd{,}\hlnum{17}\hlstd{,}\hlnum{24}\hlstd{,}\hlnum{39}\hlstd{,}\hlnum{43}\hlstd{)}
\hlstd{kick.angle.brandB} \hlkwb{<-} \hlkwd{c}\hlstd{(}\hlnum{28}\hlstd{,}\hlnum{50}\hlstd{,}\hlnum{44}\hlstd{,}\hlnum{32}\hlstd{,}\hlnum{61}\hlstd{)}
\hlstd{kick.angle.brandC} \hlkwb{<-} \hlkwd{c}\hlstd{(}\hlnum{57}\hlstd{,}\hlnum{45}\hlstd{,}\hlnum{48}\hlstd{,}\hlnum{41}\hlstd{,}\hlnum{54}\hlstd{)}
\hlstd{kick.angle.brandD} \hlkwb{<-} \hlkwd{c}\hlstd{(}\hlnum{29}\hlstd{,}\hlnum{40}\hlstd{,}\hlnum{22}\hlstd{,}\hlnum{34}\hlstd{,}\hlnum{30}\hlstd{)}
\end{alltt}
\end{kframe}
\end{knitrout}
\pause
\vspace{0.5cm}
{%\bf
  Format as a \inr{data.frame}}
\begin{knitrout}
\definecolor{shadecolor}{rgb}{0.878, 0.918, 0.933}\color{fgcolor}\begin{kframe}
\begin{alltt}
\hlstd{n} \hlkwb{<-} \hlkwd{length}\hlstd{(kick.angle.brandA)}
\hlstd{a} \hlkwb{<-} \hlnum{4}
\hlstd{sawData} \hlkwb{<-} \hlkwd{data.frame}\hlstd{(}
    \hlkwc{Kick.angle}\hlstd{=}\hlkwd{c}\hlstd{(kick.angle.brandA, kick.angle.brandB,}
        \hlstd{kick.angle.brandC, kick.angle.brandD),}
    \hlkwc{Brand}\hlstd{=}\hlkwd{rep}\hlstd{(}\hlkwd{c}\hlstd{(}\hlstr{"A"}\hlstd{,}\hlstr{"B"}\hlstd{,}\hlstr{"C"}\hlstd{,}\hlstr{"D"}\hlstd{),} \hlkwc{each}\hlstd{=n))}
\end{alltt}
\end{kframe}
\end{knitrout}
\end{frame}






\begin{frame}[fragile]
  \frametitle{Vizualize the data}
\begin{knitrout}\scriptsize
\definecolor{shadecolor}{rgb}{0.878, 0.918, 0.933}\color{fgcolor}\begin{kframe}
\begin{alltt}
\hlkwd{boxplot}\hlstd{(Kick.angle} \hlopt{~} \hlstd{Brand,} \hlkwc{data}\hlstd{=sawData,} \hlkwc{xlab}\hlstd{=}\hlstr{"Brand"}\hlstd{,}
        \hlkwc{ylab}\hlstd{=}\hlstr{"Kickback angle"}\hlstd{,} \hlkwc{cex.lab}\hlstd{=}\hlnum{1.3}\hlstd{,} \hlkwc{col}\hlstd{=}\hlstr{"darkcyan"}\hlstd{)}
\end{alltt}
\end{kframe}
\end{knitrout}
%\vfill{-2cm}
\begin{center}
%\centering
  \includegraphics[width=0.75\textwidth]{figure/boxplot1-1} %\\
\end{center}
\end{frame}








\begin{frame}[fragile]
  \frametitle{Two ANOVA functions: {\tt aov} and {\tt lm}}
  {\large
%  \begin{itemize}
%    \item
    \R~has 2 common functions for doing ANOVA: \inr{aov} and \inr{lm} \\
    \vfill
%    \item
    We will primarily use \inr{aov} in this class
%  \end{itemize}
  }
%  \pause
%  \vspace{1cm}
  \vfill
  {%\bf
    Crude characterization}
  \begin{center}
    \small
  \begin{tabular}[H]{lcc}
    \hline
      & {\tt aov} & {\tt lm} \\
    \hline
    Emphasis & ANOVA tables & Linear models \\
    Typical use & Designed experiments & Regression analysis \\
    Multiple error strata? & Yes & No \\
    \hline
  \end{tabular}
  \end{center}
\end{frame}




\begin{frame}[fragile]
  \frametitle{Using {\tt aov}}
{%\bf
  Do the analysis}
\begin{knitrout}
\definecolor{shadecolor}{rgb}{0.878, 0.918, 0.933}\color{fgcolor}\begin{kframe}
\begin{alltt}
\hlstd{aov.out1} \hlkwb{<-} \hlkwd{aov}\hlstd{(Kick.angle} \hlopt{~} \hlstd{Brand,} \hlkwc{data}\hlstd{=sawData)}
\end{alltt}
\end{kframe}
\end{knitrout}
\pause
%\vspace{0.5cm}
\vfill
{%\bf
  View the ANOVA table}
%\small
\begin{knitrout}\footnotesize
\definecolor{shadecolor}{rgb}{0.878, 0.918, 0.933}\color{fgcolor}\begin{kframe}
\begin{alltt}
\hlkwd{summary}\hlstd{(aov.out1)}
\end{alltt}
\begin{verbatim}
##             Df Sum Sq Mean Sq F value Pr(>F)  
## Brand        3   1080   360.0   3.556 0.0382 *
## Residuals   16   1620   101.2                 
## ---
## Signif. codes:  0 '***' 0.001 '**' 0.01 '*' 0.05 '.' 0.1 ' ' 1
\end{verbatim}
\end{kframe}
\end{knitrout}
\end{frame}



\begin{frame}[fragile]
  \frametitle{Estimates of means ($\mu$'s) and SE}
%{\bf \large Estimates of means and SE}
\begin{knitrout}
\definecolor{shadecolor}{rgb}{0.878, 0.918, 0.933}\color{fgcolor}\begin{kframe}
\begin{alltt}
\hlkwd{model.tables}\hlstd{(aov.out1,} \hlkwc{type}\hlstd{=}\hlstr{"means"}\hlstd{,} \hlkwc{se}\hlstd{=}\hlnum{TRUE}\hlstd{)}
\end{alltt}
\begin{verbatim}
## Tables of means
## Grand mean
##    
## 39 
## 
##  Brand 
## Brand
##  A  B  C  D 
## 33 43 49 31 
## 
## Standard errors for differences of means
##         Brand
##         6.364
## replic.     5
\end{verbatim}
\end{kframe}
\end{knitrout}
\end{frame}




\begin{frame}[fragile]
  \frametitle{Estimates of effect sizes ($\alpha$'s) and SE}
%{\bf \large Estimates of means and SE}
\begin{knitrout}
\definecolor{shadecolor}{rgb}{0.878, 0.918, 0.933}\color{fgcolor}\begin{kframe}
\begin{alltt}
\hlkwd{model.tables}\hlstd{(aov.out1,} \hlkwc{type}\hlstd{=}\hlstr{"effects"}\hlstd{,} \hlkwc{se}\hlstd{=}\hlnum{TRUE}\hlstd{)}
\end{alltt}
\begin{verbatim}
## Tables of effects
## 
##  Brand 
## Brand
##  A  B  C  D 
## -6  4 10 -8 
## 
## Standard errors of effects
##         Brand
##           4.5
## replic.     5
\end{verbatim}
\end{kframe}
\end{knitrout}
\end{frame}




\begin{frame}[fragile]
  \frametitle{Create ANOVA table by hand}
{%\bf
  Grand mean}
%  \small
\begin{knitrout}\footnotesize
\definecolor{shadecolor}{rgb}{0.878, 0.918, 0.933}\color{fgcolor}\begin{kframe}
\begin{alltt}
\hlstd{ybar.} \hlkwb{<-} \hlkwd{mean}\hlstd{(sawData}\hlopt{$}\hlstd{Kick.angle)}
\hlstd{ybar.}
\end{alltt}
\begin{verbatim}
## [1] 39
\end{verbatim}
\end{kframe}
\end{knitrout}
\pause
\normalsize
{%\bf
  Find the group means, the hard way}
%  \small
\begin{knitrout}\footnotesize
\definecolor{shadecolor}{rgb}{0.878, 0.918, 0.933}\color{fgcolor}\begin{kframe}
\begin{alltt}
\hlstd{ybar.i} \hlkwb{<-} \hlkwd{c}\hlstd{(}\hlkwc{A}\hlstd{=}\hlkwd{mean}\hlstd{(kick.angle.brandA),} \hlkwc{B}\hlstd{=}\hlkwd{mean}\hlstd{(kick.angle.brandB),}
            \hlkwc{D}\hlstd{=}\hlkwd{mean}\hlstd{(kick.angle.brandC),} \hlkwc{C}\hlstd{=}\hlkwd{mean}\hlstd{(kick.angle.brandD))}
\hlstd{ybar.i}
\end{alltt}
\begin{verbatim}
##  A  B  D  C 
## 33 43 49 31
\end{verbatim}
\end{kframe}
\end{knitrout}
\pause
\normalsize
{%\bf
  Find the group means, the easier way}
%  \small
\begin{knitrout}\footnotesize
\definecolor{shadecolor}{rgb}{0.878, 0.918, 0.933}\color{fgcolor}\begin{kframe}
\begin{alltt}
\hlstd{ybar.i} \hlkwb{<-} \hlkwd{tapply}\hlstd{(sawData}\hlopt{$}\hlstd{Kick.angle, sawData}\hlopt{$}\hlstd{Brand, mean)}
\hlstd{ybar.i}
\end{alltt}
\begin{verbatim}
##  A  B  C  D 
## 33 43 49 31
\end{verbatim}
\end{kframe}
\end{knitrout}


\end{frame}



\begin{frame}[fragile]
  \frametitle{Sums of squares}
{%\bf
  Sum of squares among}
\begin{knitrout}
\definecolor{shadecolor}{rgb}{0.878, 0.918, 0.933}\color{fgcolor}\begin{kframe}
\begin{alltt}
\hlstd{SSa} \hlkwb{<-} \hlstd{n}\hlopt{*}\hlkwd{sum}\hlstd{((ybar.i} \hlopt{-} \hlstd{ybar.)}\hlopt{^}\hlnum{2}\hlstd{)}
\hlstd{SSa}
\end{alltt}
\begin{verbatim}
## [1] 1080
\end{verbatim}
\end{kframe}
\end{knitrout}
\pause
\vfill
{%\bf
  Sum of squares within}
\begin{knitrout}\small
\definecolor{shadecolor}{rgb}{0.878, 0.918, 0.933}\color{fgcolor}\begin{kframe}
\begin{alltt}
\hlcom{## Extract the response variable}
\hlstd{y.ij} \hlkwb{<-} \hlstd{sawData}\hlopt{$}\hlstd{Kick.angle}
\hlcom{## Expand the group means and put them in the correct order}
\hlcom{## This will only work if 'ybar.i' has names}
\hlstd{ybar.ij} \hlkwb{<-} \hlstd{ybar.i[}\hlkwd{as.character}\hlstd{(sawData}\hlopt{$}\hlstd{Brand)]}
\hlstd{SSw} \hlkwb{<-} \hlkwd{sum}\hlstd{((y.ij} \hlopt{-} \hlstd{ybar.ij)}\hlopt{^}\hlnum{2}\hlstd{)}
\hlstd{SSw}
\end{alltt}
\begin{verbatim}
## [1] 1620
\end{verbatim}
\end{kframe}
\end{knitrout}
\end{frame}



\begin{frame}[fragile]
  \frametitle{Means squares and $F$ statistic}
  {%\bf
    Mean squares among}
\begin{knitrout}\footnotesize
\definecolor{shadecolor}{rgb}{0.878, 0.918, 0.933}\color{fgcolor}\begin{kframe}
\begin{alltt}
\hlstd{df1} \hlkwb{<-} \hlstd{a}\hlopt{-}\hlnum{1}
\hlstd{MSa} \hlkwb{<-} \hlstd{SSa} \hlopt{/} \hlstd{df1}
\hlstd{MSa}
\end{alltt}
\begin{verbatim}
## [1] 360
\end{verbatim}
\end{kframe}
\end{knitrout}
\pause
\vfill
  {%\bf
    Mean squares within}
\begin{knitrout}\footnotesize
\definecolor{shadecolor}{rgb}{0.878, 0.918, 0.933}\color{fgcolor}\begin{kframe}
\begin{alltt}
\hlstd{df2} \hlkwb{<-} \hlstd{a}\hlopt{*}\hlstd{(n}\hlopt{-}\hlnum{1}\hlstd{)}
\hlstd{MSw} \hlkwb{<-} \hlstd{SSw} \hlopt{/} \hlstd{df2}
\hlstd{MSw}
\end{alltt}
\begin{verbatim}
## [1] 101.25
\end{verbatim}
\end{kframe}
\end{knitrout}
\pause
\vfill
  {%\bf
    $F$ statistic}
\begin{knitrout}\footnotesize
\definecolor{shadecolor}{rgb}{0.878, 0.918, 0.933}\color{fgcolor}\begin{kframe}
\begin{alltt}
\hlstd{F.stat} \hlkwb{<-} \hlstd{MSa} \hlopt{/} \hlstd{MSw}
\hlstd{F.stat}
\end{alltt}
\begin{verbatim}
## [1] 3.555556
\end{verbatim}
\end{kframe}
\end{knitrout}
\end{frame}



\begin{frame}[fragile]
  \frametitle{Critical values and $p$-values}
  {%\bf
    Critical value}
\begin{knitrout}
\definecolor{shadecolor}{rgb}{0.878, 0.918, 0.933}\color{fgcolor}\begin{kframe}
\begin{alltt}
\hlstd{F.crit} \hlkwb{<-} \hlkwd{qf}\hlstd{(}\hlnum{0.95}\hlstd{, df1, df2)}
\hlstd{F.crit}
\end{alltt}
\begin{verbatim}
## [1] 3.238872
\end{verbatim}
\end{kframe}
\end{knitrout}
\pause
\vfill
  {%\bf
    $p$-value}
\begin{knitrout}
\definecolor{shadecolor}{rgb}{0.878, 0.918, 0.933}\color{fgcolor}\begin{kframe}
\begin{alltt}
\hlstd{p.value} \hlkwb{<-} \hlnum{1} \hlopt{-} \hlkwd{pf}\hlstd{(F.stat, df1, df2)}
\hlstd{p.value}
\end{alltt}
\begin{verbatim}
## [1] 0.03823275
\end{verbatim}
\end{kframe}
\end{knitrout}
\pause
\vfill
Conclusion: Reject the null
\end{frame}





\begin{comment}
\begin{frame}[fragile]
  \frametitle{A fake dataset}
%  \begin{tiny}
%## <<echo=FALSE>>=
%## set.seed(342)
%## n <- 20
%## pop1 <- rnorm(n, 8, sd=1.9)
%## pop2 <- rnorm(n, 8, sd=1.9)
%## pop3 <- rnorm(n, 10, sd=1.9)
%## pop4 <- rnorm(n, 8, sd=1.9)
%## pop5 <- rnorm(n, 10, sd=1.9)
%## warblerWeight <- data.frame(weight=c(pop1, pop2, pop3, pop4, pop5),
%##                             population=rep(LETTERS[1:5], each=n))
%## write.csv(warblerWeight, "warblerWeight.csv")
%## @

  \end{tiny}
\end{frame}
\end{comment}


%% \begin{frame}
%%   \frametitle{Exercise I}
%%   \large
%%   {\bf Create a script to do the following:}
%%   \begin{enumerate}[\bf (1)]
%%     \item Import the data file: {\tt warblerWeight.csv}
%%     \item Do an ANOVA using {\tt aov}
%%     \item Do an ANOVA without using {\tt aov} (or {\tt lm}).
%%     \item Answer the question: Do we reject the null hypothesis?
%%   \end{enumerate}
%%   \pause
%%   \vspace{0.5cm}
%%   {\bf Save the script. You will need it for subsequent exercises.}
%% \end{frame}



\section{Multiple comparisons}





\begin{frame}[plain]
  \frametitle{Today's Topics}
  \LARGE
  \tableofcontents[currentsection]
\end{frame}




%% \begin{frame}[fragile]
%%   \frametitle{Vizualize the data using barplot}
%% {\bf Simple usage}
%% <<barplot1,include=FALSE>>=
%% barplot(group.means, ylim=c(0, 60), xlab="Brand",
%%         ylab="Chainsaw kickback angle")
%% @
%% \begin{center}
%%   \includegraphics[width=6cm]{lab03-ANOVA1-barplot1}
%% \end{center}
%% \end{frame}






\begin{frame}[fragile]
  \frametitle{Group means $+1$ SE}
%{\bf With error bars ($+1$ SE)}
\begin{knitrout}\scriptsize
\definecolor{shadecolor}{rgb}{0.878, 0.918, 0.933}\color{fgcolor}\begin{kframe}
\begin{alltt}
\hlstd{xx} \hlkwb{<-} \hlkwd{barplot}\hlstd{(ybar.i,} \hlkwc{ylim}\hlstd{=}\hlkwd{c}\hlstd{(}\hlnum{0}\hlstd{,} \hlnum{60}\hlstd{),} \hlkwc{xlab}\hlstd{=}\hlstr{"Brand"}\hlstd{,} \hlkwc{cex.lab}\hlstd{=}\hlnum{1.5}\hlstd{,}
              \hlkwc{ylab}\hlstd{=}\hlstr{"Chainsaw kickback angle"}\hlstd{)}
\hlstd{mean.SE} \hlkwb{<-} \hlnum{6.364} \hlcom{# from model.tables(). See slide 9.}
\hlkwd{arrows}\hlstd{(xx, ybar.i, xx, ybar.i}\hlopt{+}\hlstd{mean.SE,} \hlkwc{angle}\hlstd{=}\hlnum{90}\hlstd{,} \hlkwc{length}\hlstd{=}\hlnum{0.05}\hlstd{)}
\end{alltt}
\end{kframe}
\end{knitrout}
\vspace{-.1cm}
\begin{center}
%\centering
  \includegraphics[width=0.75\textwidth]{figure/barplot2-1} %\\
\end{center}
\end{frame}







%% \begin{frame}[fragile]
%%   \frametitle{Vizualize the data using boxplot}
%% <<stripchart1,include=FALSE>>=
%% stripchart(y ~ Brand, data=sawData, xlab="Brand",
%%            ylab="Saw kickback angle", cex.lab=1.3,
%%            vertical=TRUE, pch=16)
%% @
%% \begin{center}
%%   \includegraphics[width=6cm]{lab03-ANOVA1-stripchart1}
%% \end{center}
%% \end{frame}





%% \begin{frame}[fragile]
%%   \frametitle{Testing pairwise differences in means}
%% {\bf Option 1: Many pairwise $t$-tests}
%% \small
%% <<>>=
%% pairwise.t.test(sawData$y, sawData$Brand,
%%                 p.adjust.method="bonferroni")
%% @ %def
%% \end{frame}




%% \begin{frame}[fragile]
%%   \frametitle{Testing pairwise differences in means}
%%   {\bf Option 2: Fisher's LSD test}
%% {\tiny
%% <<>>=
%% ##install.packages("agricolae") # download the agricolae package
%% ##library(agricolae) # load the package
%% print(LSD.test(aov.out1, "Brand"))
%% @
%% }
%% \end{frame}





\begin{frame}[fragile]
%  \frametitle{Testing pairwise differences in means}
  \frametitle{Tukey's Honestly Significant Difference test}
%  {\bf Option 2: Tukey's Honestly Significant Difference test}
\begin{knitrout}\small
\definecolor{shadecolor}{rgb}{0.878, 0.918, 0.933}\color{fgcolor}\begin{kframe}
\begin{alltt}
\hlkwd{TukeyHSD}\hlstd{(aov.out1)}
\end{alltt}
\begin{verbatim}
##   Tukey multiple comparisons of means
##     95% family-wise confidence level
## 
## Fit: aov(formula = Kick.angle ~ Brand, data = sawData)
## 
## $Brand
##     diff        lwr        upr     p adj
## B-A   10  -8.207419 28.2074187 0.4213711
## C-A   16  -2.207419 34.2074187 0.0955690
## D-A   -2 -20.207419 16.2074187 0.9888365
## C-B    6 -12.207419 24.2074187 0.7826478
## D-B  -12 -30.207419  6.2074187 0.2726522
## D-C  -18 -36.207419  0.2074187 0.0532168
\end{verbatim}
\end{kframe}
\end{knitrout}
\end{frame}




\begin{frame}[fragile]
  \frametitle{Plot Tukey's confidence intervals}
%  {\bf Option 3: Tukey's Honestly Significant Difference test}
  \tiny
\begin{knitrout}\footnotesize
\definecolor{shadecolor}{rgb}{0.878, 0.918, 0.933}\color{fgcolor}\begin{kframe}
\begin{alltt}
\hlkwd{plot}\hlstd{(}\hlkwd{TukeyHSD}\hlstd{(aov.out1))}
\end{alltt}
\end{kframe}
\end{knitrout}
\begin{center}
  \includegraphics[width=0.65\textwidth]{figure/tukey-plot-1}
\end{center}
\end{frame}




%% \begin{frame}
%%   \frametitle{Exercise II}
%%   {\bf Extend your script to do the following:}
%%   \begin{enumerate}[\bf (1)]
%%     \item Make a barplot of group means $+1$ SE using warbler data
%%     \item Use Tukey's HSD test to determine if any of the means are
%%       significantly different at $\alpha=0.05$
%%   \end{enumerate}
%% %  {\bf Save your script.}
%% \end{frame}




\begin{frame}
  \frametitle{Assignment}
  \small
%  {\centering \large Do Assignment 1 (it's on ELC). \par}
  A biologist wants to compare the growth of four different tree
  species she is considering for use in reforestation efforts.  All 32
  seedlings of the four species are planted at the same time in a
  large plot. Heights in meters are recorded after several years. The data
  are in the file {\tt treeHt.csv}: \\
  \vfill
  Create an \R~script to do the following:
  \begin{enumerate}[\bf (1)]
  \item Create an ANOVA table using the \inr{aov} and \inr{summary}
    functions.
  \item Create an ANOVA table (degrees of freedom,
    sums-of-squares, mean-squared error, and F-value) without using
    \inr{aov}. Compute either the critical value of F or
    the $p$-value.
  \item Add a comment to the script indicating what the null
    hypothesis is, and whether or not it can be rejected at the
    $\alpha = 0.05$ level.
  \item Use Tukey's HSD test to determine which pairs of means differ
    at the $\alpha = 0.05$ level.  Add a comment, indicating which
    pairs are different.
  \item Create a barplot showing the means and SEs.
  \end{enumerate}

\end{frame}







\end{document}



































%% \begin{comment}
%% \section{Contrasts}





%% \begin{frame}[fragile]
%%   \frametitle{Contrasts}
%%   {\bf Suppose we want to make 3 {\it a priori} comparisons:}
%%   \begin{enumerate}
%%     \item[(1)] Groups A\&D vs B\&C
%%     \item[(2)] Groups A vs D
%%     \item[(3)] Groups B vs C
%%   \end{enumerate}
%%   \pause
%%   \vspace{0.5cm}
%%   \begin{center}
%%     \begin{tabular}{llr}
%%       \hline
%%         & Comparision & Null hypothesis \\
%%       \hline
%%       1 & AD vs BC & $\frac{\mu_A + \mu_D}{2} - \frac{\mu_B + \mu_C}{2} = 0$ \\
%%       2 & A vs D & $\mu_A - \mu_D = 0$ \\
%%       3 & B vs C & $\mu_B - \mu_C = 0$ \\
%%       \hline
%%     \end{tabular}
%%   \end{center}
%% \end{frame}


%% \begin{frame}[fragile]
%%   \frametitle{Constructing contrasts in \R}
%% <<>>=
%% ADvBC <- c(1/2, -1/2, -1/2, 1/2)
%% AvD <- c(1, 0, 0, -1)
%% BvC <- c(0, 1, -1, 0)
%% @
%% \pause
%% {\bf
%% Are they orthogonal?}
%% \pause
%% {\bf Yes!}
%% <<>>=
%% sum(ADvBC * AvD)
%% sum(ADvBC * BvC)
%% sum(AvD * BvC)
%% @
%% \end{frame}




%% %% \begin{frame}[fragile]
%% %%   \frametitle{We need to put the contrasts into a matrix}
%% %%   The \verb+cbind+ function turns each vector into a row of a matrix
%% %% <<>>=
%% %% contrast.mat <- cbind(ADvBC, AvD, BvC)
%% %% contrast.mat
%% %% @
%% %% {\bf In \R, we need to use fractions.}
%% %% <<>>=
%% %% R.contrasts <- cbind(ADvBC, AvD/2, BvC/2)
%% %% R.contrasts
%% %% @
%% %% \end{frame}





%% \begin{frame}[fragile]
%%   \frametitle{We need to put the contrasts into a matrix}
%%   {\bf The {\tt rbind} function turns each vector into a row of a matrix}
%% \begin{footnotesize}
%% <<>>=
%% contrast.mat <- rbind(ADvBC, AvD, BvC)
%% contrast.mat
%% @
%% \end{footnotesize}
%% \end{frame}


%% \begin{frame}[fragile]
%%   \frametitle{Format constrats for \R}
%%   {\bf \R~requires that the contrasts be converted to a form that it
%%   likes. The easiest way to do this is to use the {\tt
%%     make.contrasts} function in the {\tt gmodels} package}.
%% \begin{footnotesize}
%% <<>>=
%% install.packages("gmodels")
%% library(gmodels)
%% R.contrasts <- zapsmall(make.contrasts(contrast.mat))
%% R.contrasts
%% @
%% \end{footnotesize}
%% \end{frame}





%% \begin{frame}[fragile]
%%   \frametitle{Contrasts}
%% {\bf Rerun the model with new contrasts and ``split'' apart the sum-of-squares}
%% \begin{footnotesize}
%% <<>>=
%% aov.out2 <- aov(y ~ Brand, data=sawData,
%%                 contrasts=list(Brand=R.contrasts))
%% summary(aov.out2, split = list(Brand =
%%                       list("ADvBC"=1, "AvD"=2, "BvC"=3)))
%% @
%% \end{footnotesize}
%% \end{frame}


%% \begin{frame}[fragile]
%%   \frametitle{Contrasts}
%% {\bf Compute the standard error of a contrast \par}
%% SE for A vs D
%% <<>>=
%% se.contrast(aov.out2, list(sawData$Brand=="A",
%%                            sawData$Brand=="D"))
%% @
%% \pause
%% SE for B vs C
%% <<>>=
%% se.contrast(aov.out2, list(sawData$Brand=="B",
%%                            sawData$Brand=="C"))
%% @
%% \pause
%% SE for AD vs BC
%% \scriptsize
%% <<>>=
%% se.contrast(aov.out2, list(sawData$Brand=="A" | sawData$Brand=="D",
%%                            sawData$Brand=="B" | sawData$Brand=="C"))
%% @
%% % \begin{scriptsize}
%% %<<>>=
%% %aov.summary <- summary.lm(aov.out2)
%% %aov.summary
%% %@
%% %<<>>=
%% %model.tables(aov.out2, type="effects", se=TRUE)
%% %@
%% %\end{scriptsize}
%% \end{frame}



%% %% \begin{frame}[fragile]
%% %%   \frametitle{Contrasts}
%% %% {\bf Compute the 95\% confidence intervals}
%% %% <<>>=
%% %% aov.CI <- confint(aov.out2, level=0.95)
%% %% aov.CI
%% %% @
%% %% \end{frame}




%%   \begin{enumerate}[\bf (1)]
%%     \item<1-> Using the {\tt warblerWeight} data, construct an $F$-table
%%        in \R~with the following contrasts:
%%     \begin{itemize}
%%       \item Groups A,B vs C,D,E
%%       \item Groups A vs B
%%       \item Groups C vs D,E
%%       \item Groups D vs E
%%     \end{itemize}
%% %    \vspace{0.5cm}
%%     \item<2-> For each contrast, compute the difference in the means and the
%%       standard error of the difference
%%   \end{enumerate}
%%   \uncover<3->{
%%   \vspace{0.5cm}
%%   {\large {\color{red} \bf Due:} Upload the \R~script to ELC before the
%%     next lab. The script should be self-contained.}
%%   }


%% \end{comment}

