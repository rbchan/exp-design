\documentclass[color=usenames,dvipsnames]{beamer}\usepackage[]{graphicx}\usepackage[]{color}
%% maxwidth is the original width if it is less than linewidth
%% otherwise use linewidth (to make sure the graphics do not exceed the margin)
\makeatletter
\def\maxwidth{ %
  \ifdim\Gin@nat@width>\linewidth
    \linewidth
  \else
    \Gin@nat@width
  \fi
}
\makeatother

\definecolor{fgcolor}{rgb}{0, 0, 0}
\newcommand{\hlnum}[1]{\textcolor[rgb]{0.69,0.494,0}{#1}}%
\newcommand{\hlstr}[1]{\textcolor[rgb]{0.749,0.012,0.012}{#1}}%
\newcommand{\hlcom}[1]{\textcolor[rgb]{0.514,0.506,0.514}{\textit{#1}}}%
\newcommand{\hlopt}[1]{\textcolor[rgb]{0,0,0}{#1}}%
\newcommand{\hlstd}[1]{\textcolor[rgb]{0,0,0}{#1}}%
\newcommand{\hlkwa}[1]{\textcolor[rgb]{0,0,0}{\textbf{#1}}}%
\newcommand{\hlkwb}[1]{\textcolor[rgb]{0,0.341,0.682}{#1}}%
\newcommand{\hlkwc}[1]{\textcolor[rgb]{0,0,0}{\textbf{#1}}}%
\newcommand{\hlkwd}[1]{\textcolor[rgb]{0.004,0.004,0.506}{#1}}%
\let\hlipl\hlkwb

\usepackage{framed}
\makeatletter
\newenvironment{kframe}{%
 \def\at@end@of@kframe{}%
 \ifinner\ifhmode%
  \def\at@end@of@kframe{\end{minipage}}%
  \begin{minipage}{\columnwidth}%
 \fi\fi%
 \def\FrameCommand##1{\hskip\@totalleftmargin \hskip-\fboxsep
 \colorbox{shadecolor}{##1}\hskip-\fboxsep
     % There is no \\@totalrightmargin, so:
     \hskip-\linewidth \hskip-\@totalleftmargin \hskip\columnwidth}%
 \MakeFramed {\advance\hsize-\width
   \@totalleftmargin\z@ \linewidth\hsize
   \@setminipage}}%
 {\par\unskip\endMakeFramed%
 \at@end@of@kframe}
\makeatother

\definecolor{shadecolor}{rgb}{.97, .97, .97}
\definecolor{messagecolor}{rgb}{0, 0, 0}
\definecolor{warningcolor}{rgb}{1, 0, 1}
\definecolor{errorcolor}{rgb}{1, 0, 0}
\newenvironment{knitrout}{}{} % an empty environment to be redefined in TeX

\usepackage{alltt}
%\documentclass[color=usenames,dvipsnames,handout]{beamer}


\usepackage[sans]{../../lab1}
\usepackage{bm}


\hypersetup{pdftex,pdfstartview=FitV}









%% New command for inline code that isn't to be evaluated
\definecolor{inlinecolor}{rgb}{0.878, 0.918, 0.933}
\newcommand{\inr}[1]{\colorbox{inlinecolor}{\texttt{#1}}}
\IfFileExists{upquote.sty}{\usepackage{upquote}}{}
\begin{document}


% \setlength\fboxsep{0pt}




\begin{frame}[plain]
  \huge
  \centering \par
  {\color{RoyalBlue}{Lab 13 -- Generalized Linear Models}} \\
  \vspace{1cm}
  \LARGE
  November 12 \& 13, 2018 \\
  FANR 6750 \\
  \vfill
  \large
  Richard Chandler and Bob Cooper
\end{frame}





\section{Logistic regression}


\begin{frame}
  \frametitle{Logistic Regression}
  \large
    \begin{gather*}
      \mathrm{logit}(p_i) = \beta_0 + \beta_1 x_{i1} + \beta_2 x_{i2} + \cdots \\
      y_i \sim \mathrm{Binomial}(N, p_i)
  \end{gather*}
  \pause
  {\bf where: \\}
  $N$ is the number of ``trials'' (e.g. coin flips) \\
  $p_i$ is the probability of a success for sample unit $i$
\end{frame}






\begin{frame}[fragile]
  \frametitle{Presence-absence and abundance}

\begin{center}
\tiny
\begin{knitrout}\tiny
\definecolor{shadecolor}{rgb}{0.878, 0.918, 0.933}\color{fgcolor}\begin{kframe}
\begin{alltt}
\hlstd{frogData[}\hlnum{1}\hlopt{:}\hlnum{25}\hlstd{,]} \hlcom{# First 25 rows}
\end{alltt}
\begin{verbatim}
##    presence abundance elevation habitat
## 1         0         0        58     Oak
## 2         1         7       191     Oak
## 3         0         0        43     Oak
## 4         1        11       374     Oak
## 5         1        11       337     Oak
## 6         1         1        64     Oak
## 7         1         4       195     Oak
## 8         1         6       263     Oak
## 9         0         0       181     Oak
## 10        1         1        59     Oak
## 11        1        50       489   Maple
## 12        1         5       317   Maple
## 13        0         0        12   Maple
## 14        1         4       245   Maple
## 15        1        47       474   Maple
## 16        1         1        83   Maple
## 17        1        46       467   Maple
## 18        1        51       485   Maple
## 19        1        23       335   Maple
## 20        0         0        20   Maple
## 21        1        27       430    Pine
## 22        1         2       223    Pine
## 23        0         0        68    Pine
## 24        1        50       483    Pine
## 25        0         0        78    Pine
\end{verbatim}
\end{kframe}
\end{knitrout}
\end{center}
\end{frame}



%% \begin{frame}[fragile]
%%   \frametitle{Raw data}
%%   \tiny
%% %  \vspace{-0.5cm}
%% \begin{center}
%% <<raw-elev,fig=true,include=false>>=
%% plot(presence ~ elevation, frogData,
%%      xlab="Elevation", ylab="Frog Occurrence")
%% @
%%   \includegraphics[width=0.7\textwidth]{lab13-GLMs-raw-elev}
%% \end{center}
%% \end{frame}




\begin{frame}[fragile]
  \frametitle{Logistic regression using {\tt glm}}
\begin{knitrout}\tiny
\definecolor{shadecolor}{rgb}{0.878, 0.918, 0.933}\color{fgcolor}\begin{kframe}
\begin{alltt}
\hlstd{fm1} \hlkwb{<-} \hlkwd{glm}\hlstd{(presence} \hlopt{~} \hlstd{habitat} \hlopt{+} \hlstd{elevation,}
           \hlkwc{family}\hlstd{=}\hlkwd{binomial}\hlstd{(}\hlkwc{link}\hlstd{=}\hlstr{"logit"}\hlstd{),} \hlkwc{data}\hlstd{=frogData)}
\end{alltt}
\end{kframe}
\end{knitrout}
\pause
\begin{knitrout}\tiny
\definecolor{shadecolor}{rgb}{0.878, 0.918, 0.933}\color{fgcolor}\begin{kframe}
\begin{alltt}
\hlkwd{summary}\hlstd{(fm1)}
\end{alltt}
\begin{verbatim}
## 
## Call:
## glm(formula = presence ~ habitat + elevation, family = binomial(link = "logit"), 
##     data = frogData)
## 
## Deviance Residuals: 
##     Min       1Q   Median       3Q      Max  
## -1.8843  -0.6169   0.1674   0.6050   1.3775  
## 
## Coefficients:
##               Estimate Std. Error z value Pr(>|z|)  
## (Intercept)  -1.092759   1.055124  -1.036   0.3004  
## habitatMaple  0.096781   1.367518   0.071   0.9436  
## habitatPine  -0.240443   1.154650  -0.208   0.8350  
## elevation     0.013658   0.006011   2.272   0.0231 *
## ---
## Signif. codes:  0 '***' 0.001 '**' 0.01 '*' 0.05 '.' 0.1 ' ' 1
## 
## (Dispersion parameter for binomial family taken to be 1)
## 
##     Null deviance: 34.795  on 29  degrees of freedom
## Residual deviance: 23.132  on 26  degrees of freedom
## AIC: 31.132
## 
## Number of Fisher Scoring iterations: 6
\end{verbatim}
\end{kframe}
\end{knitrout}
\end{frame}


%% \begin{frame}[fragile]
%%   \frametitle{Test main effects}
%%   \footnotesize
%% <<>>=
%% anova(fm1, test="Chisq")
%% @
%% \end{frame}


\begin{frame}[fragile]
  \frametitle{Occurrence probability and elevation}
  \footnotesize
\begin{knitrout}\scriptsize
\definecolor{shadecolor}{rgb}{0.878, 0.918, 0.933}\color{fgcolor}\begin{kframe}
\begin{alltt}
\hlstd{predData.elev} \hlkwb{<-} \hlkwd{data.frame}\hlstd{(}\hlkwc{elevation}\hlstd{=}\hlkwd{seq}\hlstd{(}\hlnum{12}\hlstd{,} \hlnum{489}\hlstd{,} \hlkwc{length}\hlstd{=}\hlnum{50}\hlstd{),}
                       \hlkwc{habitat}\hlstd{=}\hlstr{"Oak"}\hlstd{)}
\hlkwd{head}\hlstd{(predData.elev)}
\end{alltt}
\begin{verbatim}
##   elevation habitat
## 1  12.00000     Oak
## 2  21.73469     Oak
## 3  31.46939     Oak
## 4  41.20408     Oak
## 5  50.93878     Oak
## 6  60.67347     Oak
\end{verbatim}
\end{kframe}
\end{knitrout}
\pause
{\bf To get confidence intervals on (0,1) scale, predict on logit
  (link) scale and then backtransform using inverse-link}
\pause
%\scriptsize
\begin{knitrout}\scriptsize
\definecolor{shadecolor}{rgb}{0.878, 0.918, 0.933}\color{fgcolor}\begin{kframe}
\begin{alltt}
\hlstd{pred.link} \hlkwb{<-} \hlkwd{predict}\hlstd{(fm1,} \hlkwc{newdata}\hlstd{=predData.elev,} \hlkwc{se.fit}\hlstd{=}\hlnum{TRUE}\hlstd{,} \hlkwc{type}\hlstd{=}\hlstr{"link"}\hlstd{)}
\hlstd{predData.elev}\hlopt{$}\hlstd{p} \hlkwb{<-} \hlkwd{plogis}\hlstd{(pred.link}\hlopt{$}\hlstd{fit)}
\hlstd{predData.elev}\hlopt{$}\hlstd{lower} \hlkwb{<-} \hlkwd{plogis}\hlstd{(pred.link}\hlopt{$}\hlstd{fit} \hlopt{-} \hlnum{1.96}\hlopt{*}\hlstd{pred.link}\hlopt{$}\hlstd{se.fit)}
\hlstd{predData.elev}\hlopt{$}\hlstd{upper} \hlkwb{<-} \hlkwd{plogis}\hlstd{(pred.link}\hlopt{$}\hlstd{fit} \hlopt{+} \hlnum{1.96}\hlopt{*}\hlstd{pred.link}\hlopt{$}\hlstd{se.fit)}
\end{alltt}
\end{kframe}
\end{knitrout}
\end{frame}



\begin{frame}[fragile]
  \frametitle{Occurrence probability and elevation}
\begin{knitrout}\tiny
\definecolor{shadecolor}{rgb}{0.878, 0.918, 0.933}\color{fgcolor}\begin{kframe}
\begin{alltt}
\hlkwd{plot}\hlstd{(p} \hlopt{~} \hlstd{elevation,} \hlkwc{data}\hlstd{=predData.elev,} \hlkwc{type}\hlstd{=}\hlstr{"l"}\hlstd{,} \hlkwc{ylim}\hlstd{=}\hlkwd{c}\hlstd{(}\hlnum{0}\hlstd{,}\hlnum{1}\hlstd{),}
     \hlkwc{xlab}\hlstd{=}\hlstr{"Elevation"}\hlstd{,} \hlkwc{ylab}\hlstd{=}\hlstr{"Probability of occurrence"}\hlstd{)}
\hlkwd{points}\hlstd{(presence} \hlopt{~} \hlstd{elevation, frogData)}
\hlkwd{lines}\hlstd{(lower} \hlopt{~} \hlstd{elevation,} \hlkwc{data}\hlstd{=predData.elev,} \hlkwc{lty}\hlstd{=}\hlnum{2}\hlstd{)}
\hlkwd{lines}\hlstd{(upper} \hlopt{~} \hlstd{elevation,} \hlkwc{data}\hlstd{=predData.elev,} \hlkwc{lty}\hlstd{=}\hlnum{2}\hlstd{)}
\end{alltt}
\end{kframe}
\end{knitrout}
\centering
\includegraphics[width=0.75\textwidth]{figure/pred1-1} \\
\end{frame}




\begin{frame}[fragile]
  \frametitle{Occurrence probability and habitat}
\begin{knitrout}\tiny
\definecolor{shadecolor}{rgb}{0.878, 0.918, 0.933}\color{fgcolor}\begin{kframe}
\begin{alltt}
\hlstd{predData.hab} \hlkwb{<-} \hlkwd{data.frame}\hlstd{(}\hlkwc{habitat}\hlstd{=}\hlkwd{c}\hlstd{(}\hlstr{"Oak"}\hlstd{,} \hlstr{"Maple"}\hlstd{,} \hlstr{"Pine"}\hlstd{),} \hlkwc{elevation}\hlstd{=}\hlnum{250}\hlstd{)}
\hlstd{pred} \hlkwb{<-} \hlkwd{predict}\hlstd{(fm1,} \hlkwc{newdata}\hlstd{=predData.hab,} \hlkwc{se.fit}\hlstd{=}\hlnum{TRUE}\hlstd{,} \hlkwc{type}\hlstd{=}\hlstr{"link"}\hlstd{)}
\hlstd{bp} \hlkwb{<-} \hlkwd{barplot}\hlstd{(}\hlkwd{plogis}\hlstd{(pred}\hlopt{$}\hlstd{fit),} \hlkwc{ylab}\hlstd{=}\hlstr{"Probability of occurrence"}\hlstd{,} \hlkwc{cex.lab}\hlstd{=}\hlnum{1.5}\hlstd{,}
              \hlkwc{names}\hlstd{=}\hlkwd{c}\hlstd{(}\hlstr{"Oak"}\hlstd{,} \hlstr{"Maple"}\hlstd{,} \hlstr{"Pine"}\hlstd{),} \hlkwc{col}\hlstd{=}\hlstr{"lightblue"}\hlstd{,} \hlkwc{ylim}\hlstd{=}\hlkwd{c}\hlstd{(}\hlnum{0}\hlstd{,} \hlnum{1.1}\hlstd{))}
\hlkwd{arrows}\hlstd{(bp,} \hlkwd{plogis}\hlstd{(pred}\hlopt{$}\hlstd{fit), bp,} \hlkwd{plogis}\hlstd{(pred}\hlopt{$}\hlstd{fit} \hlopt{+} \hlstd{pred}\hlopt{$}\hlstd{se.fit),}
       \hlkwc{angle}\hlstd{=}\hlnum{90}\hlstd{,} \hlkwc{code}\hlstd{=}\hlnum{3}\hlstd{,} \hlkwc{length}\hlstd{=}\hlnum{0.1}\hlstd{)}
\end{alltt}
\end{kframe}
\end{knitrout}
\centering
\includegraphics[width=0.6\textwidth]{figure/habitat-1} \\
\end{frame}




\section{Poisson regression}



\begin{frame}
  \frametitle{Poisson Regression}
    \begin{gather*}
      \mathrm{log}(\lambda_i) = \beta_0 + \beta_1 x_{i1} + \beta_2 x_{i2} + \cdots \\
      y_i \sim \mathrm{Poisson}(\lambda_i)
  \end{gather*}
  \pause
  {\bf where: \\}
  $\lambda_i$ is the expected value of $y_i$
\end{frame}







\begin{frame}[fragile]
  \frametitle{Poisson regression using {\tt glm}}
\begin{knitrout}\tiny
\definecolor{shadecolor}{rgb}{0.878, 0.918, 0.933}\color{fgcolor}\begin{kframe}
\begin{alltt}
\hlstd{fm2} \hlkwb{<-} \hlkwd{glm}\hlstd{(abundance} \hlopt{~} \hlstd{habitat} \hlopt{+} \hlstd{elevation,}
           \hlkwc{family}\hlstd{=}\hlkwd{poisson}\hlstd{(}\hlkwc{link}\hlstd{=}\hlstr{"log"}\hlstd{),} \hlkwc{data}\hlstd{=frogData)}
\end{alltt}
\end{kframe}
\end{knitrout}
\pause
\begin{knitrout}\tiny
\definecolor{shadecolor}{rgb}{0.878, 0.918, 0.933}\color{fgcolor}\begin{kframe}
\begin{alltt}
\hlkwd{summary}\hlstd{(fm2)}
\end{alltt}
\begin{verbatim}
## 
## Call:
## glm(formula = abundance ~ habitat + elevation, family = poisson(link = "log"), 
##     data = frogData)
## 
## Deviance Residuals: 
##     Min       1Q   Median       3Q      Max  
## -2.6308  -1.0810  -0.1067   0.3353   2.7935  
## 
## Coefficients:
##                Estimate Std. Error z value Pr(>|z|)    
## (Intercept)  -0.9403442  0.2430730  -3.869 0.000109 ***
## habitatMaple  0.1533915  0.1971009   0.778 0.436428    
## habitatPine   0.0881110  0.1994981   0.442 0.658733    
## elevation     0.0097836  0.0006291  15.551  < 2e-16 ***
## ---
## Signif. codes:  0 '***' 0.001 '**' 0.01 '*' 0.05 '.' 0.1 ' ' 1
## 
## (Dispersion parameter for poisson family taken to be 1)
## 
##     Null deviance: 700.762  on 29  degrees of freedom
## Residual deviance:  44.891  on 26  degrees of freedom
## AIC: 140.13
## 
## Number of Fisher Scoring iterations: 5
\end{verbatim}
\end{kframe}
\end{knitrout}
\end{frame}



\begin{frame}[fragile]
  \frametitle{Abundance and elevation}
%<<newdat2,size='scriptsize'>>=
%newdat <- data.frame(elevation=seq(12, 489, length=50),
%                     habitat="Oak")
%head(newdat)
%@
\footnotesize
%\pause
{\bf To get confidence intervals on (0,$\infty$) scale, predict on log
  (link) scale and then backtransform using inverse-link}
\pause
%\scriptsize
\begin{knitrout}\scriptsize
\definecolor{shadecolor}{rgb}{0.878, 0.918, 0.933}\color{fgcolor}\begin{kframe}
\begin{alltt}
\hlstd{pred.link} \hlkwb{<-} \hlkwd{predict}\hlstd{(fm2,} \hlkwc{newdata}\hlstd{=predData.elev,} \hlkwc{se.fit}\hlstd{=}\hlnum{TRUE}\hlstd{,} \hlkwc{type}\hlstd{=}\hlstr{"link"}\hlstd{)}
\hlstd{predData.elev}\hlopt{$}\hlstd{lambda} \hlkwb{<-} \hlkwd{exp}\hlstd{(pred.link}\hlopt{$}\hlstd{fit)} \hlcom{# exp is the inverse-link function}
\hlstd{predData.elev}\hlopt{$}\hlstd{lower} \hlkwb{<-} \hlkwd{exp}\hlstd{(pred.link}\hlopt{$}\hlstd{fit} \hlopt{-} \hlnum{1.96}\hlopt{*}\hlstd{pred.link}\hlopt{$}\hlstd{se.fit)}
\hlstd{predData.elev}\hlopt{$}\hlstd{upper} \hlkwb{<-} \hlkwd{exp}\hlstd{(pred.link}\hlopt{$}\hlstd{fit} \hlopt{+} \hlnum{1.96}\hlopt{*}\hlstd{pred.link}\hlopt{$}\hlstd{se.fit)}
\end{alltt}
\end{kframe}
\end{knitrout}
\end{frame}



\begin{frame}[fragile]
  \frametitle{Abundance and elevation}
  \tiny
\begin{knitrout}\tiny
\definecolor{shadecolor}{rgb}{0.878, 0.918, 0.933}\color{fgcolor}\begin{kframe}
\begin{alltt}
\hlkwd{plot}\hlstd{(lambda} \hlopt{~} \hlstd{elevation, predData.elev,} \hlkwc{type}\hlstd{=}\hlstr{"l"}\hlstd{,} \hlkwc{ylim}\hlstd{=}\hlkwd{c}\hlstd{(}\hlnum{0}\hlstd{,}\hlnum{60}\hlstd{),}
     \hlkwc{xlab}\hlstd{=}\hlstr{"Elevation"}\hlstd{,} \hlkwc{ylab}\hlstd{=}\hlstr{"Expected abundance"}\hlstd{)}
\hlkwd{points}\hlstd{(abundance} \hlopt{~} \hlstd{elevation, frogData)}
\hlkwd{lines}\hlstd{(lower} \hlopt{~} \hlstd{elevation, predData.elev,} \hlkwc{lty}\hlstd{=}\hlnum{2}\hlstd{)}
\hlkwd{lines}\hlstd{(upper} \hlopt{~} \hlstd{elevation, predData.elev,} \hlkwc{lty}\hlstd{=}\hlnum{2}\hlstd{)}
\end{alltt}
\end{kframe}
\end{knitrout}
\centering
\includegraphics[width=0.75\textwidth]{figure/pred2-1} \\
\end{frame}







%% \begin{frame}[fragile]
%%   \frametitle{Abundance, elevation, and habitat}
%%   \tiny
%% \begin{center}
%% <<habitat2,fig=true,include=false>>=
%% nd <- data.frame(habitat=rep(c("Oak", "Maple", "Pine"), each=20),
%%                  elevation=rep(seq(12, 489, length=20), times=3))
%% pred <- predict(fm2, newdata=nd, se.fit=TRUE, type="link")
%% nd$lambda <- exp(pred$fit)
%% plot(lambda ~ elevation, nd, type="l", ylim=c(0,55), subset=habitat=="Oak",
%%      xlab="Elevation", ylab="Expected abundance")
%% lines(lambda ~ elevation, nd, subset=habitat=="Maple", lty=2)
%% lines(lambda ~ elevation, nd, subset=habitat=="Pine", lty=3)
%% @
%%   \includegraphics[width=0.5\textwidth]{lab13-GLMs-habitat2}
%% \end{center}
%% \end{frame}



%\begin{frame}[fragile]

%\end{frame}


\begin{frame}
  \frametitle{Assignment}
  Researchers want to know how latitude and landscape type influence the
  probability that American Crows are infected by West Nile Virus. One
  hundred crows are captured and tested for West Nile Virus in urban
  and rural landscapes spanning a latitude gradient.
  \begin{enumerate}[\bf (1)]
    \item Fit a logistic regression model to the {\tt crowData.csv}
      dataset to assess the effects of latitude and landscape type
    \item Interpret the parameter estimates
    \item Plot the relationship between infection probability and
      latitude, for rural and urban landscapes, on the same graph
    \item Include the data points (color coded by landscape) and a legend in
      the graph
    \item Include confidence intervals
  \end{enumerate}
\end{frame}







\end{document}






% \section{Simulating data}



% \begin{frame}
%   \frametitle{Simulating data}
%   \large
%   {\bf R} has many functions for generating random variables
%   \begin{center}
%     \begin{tabular}{lr}
%       \hline
%       Distribution & {\bf R} function \\
%       \hline
%       Uniform & {\tt runif} \\
%       Normal & {\tt rnorm} \\
%       Binomial & {\tt rbinom} \\
%       Poisson & {\tt rpois} \\
%       Negative binomial & {\tt rnbinom} \\
%       \hline
%     \end{tabular}
%   \end{center}
% \end{frame}



% \begin{frame}[fragile]
%   \frametitle{Random Poisson variables}
%   {\bf Ten Poisson variates with $\lambda=0.5$}
% <<rpois1>>=
% rpois(n=10, 0.5)
% @
% \vfill
% \pause
%   {\bf Ten Poisson variates with $\lambda=5$}
% <<rpois2>>=
% rpois(n=10, 5)
% @
% \vfill
% \pause
%   {\bf Ten Poisson variates with $\lambda=15$}
% <<rpois3>>=
% rpois(n=10, 15)
% @
% \end{frame}




% \begin{frame}[fragile]
%   \frametitle{Simulate Poisson variables}
%   {\bf 1000 Poisson variabes with $\lambda=3$}
%   \small
% <<p1000,fig.show='hide'>>=
% x <- rpois(n=1000, 3)
% hist(x, col="purple")
% @
% \centering
% \includegraphics[width=0.6\textwidth]{figure/p1000-1} \\
% \end{frame}




% \begin{frame}[fragile]
%   \frametitle{Simulate Poisson regression data}
%   {\bf Create a design matrix}
% <<X>>=
% X <- model.matrix(~habitat + elevation, frogData)
% @
% \pause
% \vfill
% {\bf Choose values of $\bm \beta$ parameters
% <<beta>>=
% beta <- c(0.5, 1, -3, 0.01)
% @
% \pause
% \vfill
% {\bf Calculate $\mu$, the expected value of $y$}
% <<lambda>>=
% lambda <- exp(X %*% beta)
% @
% \pause
% \vfill
% {\bf Simulate $y$, the response variable}
% <<sampleSize>>=
% sampleSize <- nrow(X)
% y <- rpois(sampleSize, lambda)
% @
% \end{frame}








% \begin{frame}[fragile]
%   \frametitle{Fit model to simulated data}
%   \small
% <<fm3,size='small'>>=
% fm3 <- glm(y ~ habitat + elevation, family=poisson, frogData)
% @
% \scriptsize
% <<summary-fm3,size='scriptsize'>>=
% summary(fm3)
% @
% \end{frame}




% \begin{frame}[fragile]
%   \frametitle{Check confidence intervals}
% <<CI>>=
% CI <- confint(fm3)
% cbind(CI, beta)
% @
% \end{frame}






% \begin{frame}[fragile]
% <<echo=FALSE>>=
% set.seed(340)
% vegData <- data.frame(veght=rnorm(30, 5, 1))
% write.csv(vegData, "vegData.csv", row.names=FALSE)
% @
%   \frametitle{Assignment}
%   \begin{enumerate}[\bf (1)]
%     \item Import the data {\tt vegData.csv}
%     \item Simulate a Poisson response variable assuming the model:
%       $\lambda_i = \exp(\beta0 + \beta1\mbox{VegHt}_i)$, where $\beta_0=-1$ and
%       $\beta_1 = 0.5$
%     \item Fit the Poisson regression model to the data
%     \item Do the confidence intervals include the data generating
%       $\bm \beta$ values?
%     \item Plot the estimated relationship between abundance and
%       vegetation height. Include 95\% confidence intervals.
%   \end{enumerate}
% \end{frame}












%% \begin{frame}
%%   \frametitle{Goodness-of-fit}
%%   Parametric bootstrap
%%   \begin{enumerate}[<+->]
%%     \item Fit a model
%%     \item Pick a fit statistic
%%     \item Calculate the fit statistic for the
%%   \end{enumerate}
%% \end{frame}





%% \begin{frame}[fragile]
%%   \frametitle{Goodness-of-fit}
%% <<>>=
%% fitStat <- function(fm) sum(resid(fm)^2)
%% boot(fm2$data, fitStat, sim="parametric", ran.gen=simulate, R=100)
%% @
%% \end{frame}





%% \begin{frame}[fragile]
%%   \frametitle{Goodness-of-fit}
%% <<>>=
%% nSims <- 1000
%% fitStats <- rep(NA, nSim)
%% for(i in 1:nSims) {
%%     frogData$abundance.star <- as.numeric(simulate(fm2))
%%     fit.star <- glm(abundance.star ~ habitat + elevation,
%%                  family=poisson(link="log"), data=frogData)
%%     fitStats[i] <- SSE
%% }
%% @
%% \end{frame}



