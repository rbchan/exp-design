\documentclass[color=usenames,dvipsnames]{beamer}\usepackage[]{graphicx}\usepackage[]{color}
%% maxwidth is the original width if it is less than linewidth
%% otherwise use linewidth (to make sure the graphics do not exceed the margin)
\makeatletter
\def\maxwidth{ %
  \ifdim\Gin@nat@width>\linewidth
    \linewidth
  \else
    \Gin@nat@width
  \fi
}
\makeatother

\definecolor{fgcolor}{rgb}{0, 0, 0}
\newcommand{\hlnum}[1]{\textcolor[rgb]{0.69,0.494,0}{#1}}%
\newcommand{\hlstr}[1]{\textcolor[rgb]{0.749,0.012,0.012}{#1}}%
\newcommand{\hlcom}[1]{\textcolor[rgb]{0.514,0.506,0.514}{\textit{#1}}}%
\newcommand{\hlopt}[1]{\textcolor[rgb]{0,0,0}{#1}}%
\newcommand{\hlstd}[1]{\textcolor[rgb]{0,0,0}{#1}}%
\newcommand{\hlkwa}[1]{\textcolor[rgb]{0,0,0}{\textbf{#1}}}%
\newcommand{\hlkwb}[1]{\textcolor[rgb]{0,0.341,0.682}{#1}}%
\newcommand{\hlkwc}[1]{\textcolor[rgb]{0,0,0}{\textbf{#1}}}%
\newcommand{\hlkwd}[1]{\textcolor[rgb]{0.004,0.004,0.506}{#1}}%
\let\hlipl\hlkwb

\usepackage{framed}
\makeatletter
\newenvironment{kframe}{%
 \def\at@end@of@kframe{}%
 \ifinner\ifhmode%
  \def\at@end@of@kframe{\end{minipage}}%
  \begin{minipage}{\columnwidth}%
 \fi\fi%
 \def\FrameCommand##1{\hskip\@totalleftmargin \hskip-\fboxsep
 \colorbox{shadecolor}{##1}\hskip-\fboxsep
     % There is no \\@totalrightmargin, so:
     \hskip-\linewidth \hskip-\@totalleftmargin \hskip\columnwidth}%
 \MakeFramed {\advance\hsize-\width
   \@totalleftmargin\z@ \linewidth\hsize
   \@setminipage}}%
 {\par\unskip\endMakeFramed%
 \at@end@of@kframe}
\makeatother

\definecolor{shadecolor}{rgb}{.97, .97, .97}
\definecolor{messagecolor}{rgb}{0, 0, 0}
\definecolor{warningcolor}{rgb}{1, 0, 1}
\definecolor{errorcolor}{rgb}{1, 0, 0}
\newenvironment{knitrout}{}{} % an empty environment to be redefined in TeX

\usepackage{alltt}
%\documentclass[color=usenames,dvipsnames,handout]{beamer}



\usepackage[sans]{../../lab1}
\usepackage{bm}


\hypersetup{pdftex,pdfstartview=FitV}









%% New command for inline code that isn't to be evaluated
\definecolor{inlinecolor}{rgb}{0.878, 0.918, 0.933}
\newcommand{\inr}[1]{\colorbox{inlinecolor}{\texttt{#1}}}
\IfFileExists{upquote.sty}{\usepackage{upquote}}{}
\begin{document}


%\setlength\fboxsep{0pt}



\begin{frame}[plain]
  \centering \huge
%  \color{MidnightBlue} %\bf
  {\color{RoyalBlue}{Lab 14 -- Model Selection and Multimodel Inference}} \\
  \vspace{1cm}
  \LARGE
  November 26 \& 27, 2018 \\
  FANR 6750 \\
  \vfill
  \large
  Richard Chandler and Bob Cooper
\end{frame}




\section{Model Fitting}



\begin{frame}[plain]
  \frametitle{Today's Topics}
  \LARGE
  \only<1>{\tableofcontents}%[hideallsubsections]}
  \only<2 | handout:0>{\tableofcontents[currentsection]}%,hideallsubsections]}
\end{frame}




\begin{frame}[fragile]
  \frametitle{Swiss Data}
\begin{knitrout}\small
\definecolor{shadecolor}{rgb}{0.878, 0.918, 0.933}\color{fgcolor}\begin{kframe}
\begin{alltt}
\hlstd{swissData} \hlkwb{<-} \hlkwd{read.csv}\hlstd{(}\hlstr{"swissData.csv"}\hlstd{)}
\hlkwd{head}\hlstd{(swissData,} \hlkwc{n}\hlstd{=}\hlnum{11}\hlstd{)}
\end{alltt}
\begin{verbatim}
##    elevation forest water sppRichness
## 1        450      3    No          35
## 2        450     21    No          51
## 3       1050     32    No          46
## 4        950      9   Yes          31
## 5       1150     35   Yes          50
## 6        550      2    No          43
## 7        750      6    No          37
## 8        650     60   Yes          47
## 9        550      5   Yes          37
## 10       550     13    No          43
## 11      1150     50    No          52
\end{verbatim}
\end{kframe}
\end{knitrout}
\end{frame}


\begin{frame}[fragile]
  \frametitle{Four linear models}
\begin{knitrout}
\definecolor{shadecolor}{rgb}{0.878, 0.918, 0.933}\color{fgcolor}\begin{kframe}
\begin{alltt}
\hlstd{fm1} \hlkwb{<-} \hlkwd{lm}\hlstd{(sppRichness} \hlopt{~} \hlstd{forest,} \hlkwc{data}\hlstd{=swissData)}
\hlstd{fm2} \hlkwb{<-} \hlkwd{lm}\hlstd{(sppRichness} \hlopt{~} \hlstd{elevation,} \hlkwc{data}\hlstd{=swissData)}
\hlstd{fm3} \hlkwb{<-} \hlkwd{lm}\hlstd{(sppRichness} \hlopt{~} \hlstd{forest} \hlopt{+} \hlstd{elevation} \hlopt{+}
          \hlstd{water,} \hlkwc{data}\hlstd{=swissData)}
\hlstd{fm4} \hlkwb{<-} \hlkwd{lm}\hlstd{(sppRichness} \hlopt{~} \hlstd{forest} \hlopt{+} \hlstd{elevation} \hlopt{+}
          \hlkwd{I}\hlstd{(elevation}\hlopt{^}\hlnum{2}\hlstd{)} \hlopt{+} \hlstd{water,} \hlkwc{data}\hlstd{=swissData)}
\end{alltt}
\end{kframe}
\end{knitrout}
\end{frame}



\begin{frame}[fragile]
  \frametitle{Model 4 -- Estimates}
  \vspace{-5pt}
\begin{knitrout}\scriptsize
\definecolor{shadecolor}{rgb}{0.878, 0.918, 0.933}\color{fgcolor}\begin{kframe}
\begin{alltt}
\hlkwd{summary}\hlstd{(fm4)}
\end{alltt}
\begin{verbatim}
## 
## Call:
## lm(formula = sppRichness ~ forest + elevation + I(elevation^2) + 
##     water, data = swissData)
## 
## Residuals:
##     Min      1Q  Median      3Q     Max 
## -11.314  -3.205  -0.377   3.334  15.082 
## 
## Coefficients:
##                  Estimate Std. Error t value Pr(>|t|)    
## (Intercept)     4.518e+01  1.286e+00  35.137  < 2e-16 ***
## forest          2.311e-01  1.276e-02  18.111  < 2e-16 ***
## elevation      -1.016e-02  2.572e-03  -3.951   0.0001 ***
## I(elevation^2)  6.103e-08  9.661e-07   0.063   0.9497    
## waterYes       -3.013e+00  6.821e-01  -4.418 1.46e-05 ***
## ---
## Signif. codes:  0 '***' 0.001 '**' 0.01 '*' 0.05 '.' 0.1 ' ' 1
## 
## Residual standard error: 4.954 on 262 degrees of freedom
## Multiple R-squared:  0.7929,	Adjusted R-squared:  0.7897 
## F-statistic: 250.8 on 4 and 262 DF,  p-value: < 2.2e-16
\end{verbatim}
\end{kframe}
\end{knitrout}
\end{frame}





\begin{frame}[fragile]
  \frametitle{Model 4 -- ANOVA table}
\begin{knitrout}\scriptsize
\definecolor{shadecolor}{rgb}{0.878, 0.918, 0.933}\color{fgcolor}\begin{kframe}
\begin{alltt}
\hlkwd{summary.aov}\hlstd{(fm4)}
\end{alltt}
\begin{verbatim}
##                 Df Sum Sq Mean Sq F value   Pr(>F)    
## forest           1  13311   13311  542.40  < 2e-16 ***
## elevation        1  10820   10820  440.89  < 2e-16 ***
## I(elevation^2)   1      7       7    0.27    0.604    
## water            1    479     479   19.52 1.46e-05 ***
## Residuals      262   6430      25                     
## ---
## Signif. codes:  0 '***' 0.001 '**' 0.01 '*' 0.05 '.' 0.1 ' ' 1
\end{verbatim}
\end{kframe}
\end{knitrout}
\small
%\vfill
We could compute AIC using the equation
$\mathrm{AIC}=n\log(\mathrm{RSS}/n)+2K$, where RSS is the residual
sum-of-squares.
\pause
\vfill
However, we will use the more general formula: $\mathrm{AIC} =
-2\mathcal{L}(\hat{\theta} ; {\bf y}) + 2K$.
\end{frame}







\section{Model Selection}






\begin{frame}[plain]
  \frametitle{Outline}
  \huge
  \tableofcontents[currentsection]
\end{frame}





\begin{frame}[fragile]
  \frametitle{Compute AIC for each model}
  {Sample size}
  \small
\begin{knitrout}\footnotesize
\definecolor{shadecolor}{rgb}{0.878, 0.918, 0.933}\color{fgcolor}\begin{kframe}
\begin{alltt}
\hlstd{n} \hlkwb{<-} \hlkwd{nrow}\hlstd{(swissData)}
\end{alltt}
\end{kframe}
\end{knitrout}
\pause
\vfill
%{Residual sums-of-squares (from ANOVA tables)}
{log-likelihood for each model}
\begin{knitrout}\footnotesize
\definecolor{shadecolor}{rgb}{0.878, 0.918, 0.933}\color{fgcolor}\begin{kframe}
\begin{alltt}
\hlstd{logL} \hlkwb{<-} \hlkwd{c}\hlstd{(}\hlkwd{logLik}\hlstd{(fm1),} \hlkwd{logLik}\hlstd{(fm2),} \hlkwd{logLik}\hlstd{(fm3),} \hlkwd{logLik}\hlstd{(fm4))}
\end{alltt}
\end{kframe}
\end{knitrout}
\pause
\vfill
%{\bf Number of parameters}
{Number of parameters}
\begin{knitrout}\footnotesize
\definecolor{shadecolor}{rgb}{0.878, 0.918, 0.933}\color{fgcolor}\begin{kframe}
\begin{alltt}
\hlstd{K} \hlkwb{<-} \hlkwd{c}\hlstd{(}\hlnum{3}\hlstd{,} \hlnum{3}\hlstd{,} \hlnum{5}\hlstd{,} \hlnum{6}\hlstd{)}
\end{alltt}
\end{kframe}
\end{knitrout}
\pause
\vfill
%{\bf AIC}
{AIC}
\begin{knitrout}\footnotesize
\definecolor{shadecolor}{rgb}{0.878, 0.918, 0.933}\color{fgcolor}\begin{kframe}
\begin{alltt}
\hlstd{AIC} \hlkwb{<-} \hlopt{-}\hlnum{2}\hlopt{*}\hlstd{logL} \hlopt{+} \hlnum{2}\hlopt{*}\hlstd{K}
\end{alltt}
\end{kframe}
\end{knitrout}
\pause
\vfill
%  {\bf $\Delta$AIC}
  {$\Delta$AIC}
\begin{knitrout}\footnotesize
\definecolor{shadecolor}{rgb}{0.878, 0.918, 0.933}\color{fgcolor}\begin{kframe}
\begin{alltt}
\hlstd{delta} \hlkwb{<-} \hlstd{AIC} \hlopt{-} \hlkwd{min}\hlstd{(AIC)}
\end{alltt}
\end{kframe}
\end{knitrout}
\pause
\vfill
%  {\bf AIC Weights}
  {AIC Weights}
\begin{knitrout}\footnotesize
\definecolor{shadecolor}{rgb}{0.878, 0.918, 0.933}\color{fgcolor}\begin{kframe}
\begin{alltt}
\hlstd{w} \hlkwb{<-} \hlkwd{exp}\hlstd{(}\hlopt{-}\hlnum{0.5}\hlopt{*}\hlstd{delta)}\hlopt{/}\hlkwd{sum}\hlstd{(}\hlkwd{exp}\hlstd{(}\hlopt{-}\hlnum{0.5}\hlopt{*}\hlstd{delta))}
\end{alltt}
\end{kframe}
\end{knitrout}
\end{frame}




\begin{frame}[fragile]
  \frametitle{AIC table}
  \small
%  {\bf Put vectors in data.frame}
  {Put vectors in data.frame}
\begin{knitrout}\footnotesize
\definecolor{shadecolor}{rgb}{0.878, 0.918, 0.933}\color{fgcolor}\begin{kframe}
\begin{alltt}
\hlstd{ms} \hlkwb{<-} \hlkwd{data.frame}\hlstd{(logL, K, AIC, delta, w)}
\hlkwd{rownames}\hlstd{(ms)} \hlkwb{<-} \hlkwd{c}\hlstd{(}\hlstr{"fm1"}\hlstd{,} \hlstr{"fm2"}\hlstd{,} \hlstr{"fm3"}\hlstd{,} \hlstr{"fm4"}\hlstd{)}
\hlkwd{round}\hlstd{(ms,} \hlkwc{digits}\hlstd{=}\hlnum{2}\hlstd{)}
\end{alltt}
\begin{verbatim}
##        logL K     AIC  delta    w
## fm1 -939.03 3 1884.06 266.90 0.00
## fm2 -934.07 3 1874.15 256.99 0.00
## fm3 -803.58 5 1617.16   0.00 0.73
## fm4 -803.58 6 1619.15   2.00 0.27
\end{verbatim}
\end{kframe}
\end{knitrout}
\pause
%  {\bf Sort data.frame based on AIC values}
  {Sort data.frame based on AIC values}
\begin{knitrout}\footnotesize
\definecolor{shadecolor}{rgb}{0.878, 0.918, 0.933}\color{fgcolor}\begin{kframe}
\begin{alltt}
\hlstd{ms} \hlkwb{<-} \hlstd{ms[}\hlkwd{order}\hlstd{(ms}\hlopt{$}\hlstd{AIC),]}
\hlkwd{round}\hlstd{(ms,} \hlkwc{digits}\hlstd{=}\hlnum{2}\hlstd{)}
\end{alltt}
\begin{verbatim}
##        logL K     AIC  delta    w
## fm3 -803.58 5 1617.16   0.00 0.73
## fm4 -803.58 6 1619.15   2.00 0.27
## fm2 -934.07 3 1874.15 256.99 0.00
## fm1 -939.03 3 1884.06 266.90 0.00
\end{verbatim}
\end{kframe}
\end{knitrout}
\end{frame}





\begin{frame}[fragile]
  \frametitle{Similar process using R's {\tt AIC} function}
\begin{knitrout}
\definecolor{shadecolor}{rgb}{0.878, 0.918, 0.933}\color{fgcolor}\begin{kframe}
\begin{alltt}
\hlkwd{AIC}\hlstd{(fm1, fm2, fm3, fm4)}
\end{alltt}
\begin{verbatim}
##     df      AIC
## fm1  3 1884.057
## fm2  3 1874.146
## fm3  5 1617.157
## fm4  6 1619.153
\end{verbatim}
\end{kframe}
\end{knitrout}
\pause
\vfill
%{\bf Notes}
{Notes}
\begin{itemize}[<+->]
%  \item {\bf R} uses {\tt logLik(fm)} instead of {\tt n*log(RSS/n)}
  \item If we had used the residual sums-of-squares instead of the
    log-likelihoods, the AIC values would have been different, but the
    $\Delta$AIC values would have been the same
  \item Either approach is fine with linear models, but log-likelihoods
    must be used with GLMs and other models fit using maximum likelihood
\end{itemize}
\end{frame}





\section{Multi-model Inference}




\begin{frame}[plain]
  \frametitle{Outline}
  \huge
  \tableofcontents[currentsection]
\end{frame}






\begin{frame}[fragile]
  \frametitle{Model-specific predictions}
  \footnotesize
%  {\bf Predict number of species at site 1000m high with 25\% forest
%    cover, and no water, for \alert{each} model}
  {Expected number of species at 1000m elevation, 25\% forest
    cover, and no water, \alert{for each model}}
\begin{knitrout}\scriptsize
\definecolor{shadecolor}{rgb}{0.878, 0.918, 0.933}\color{fgcolor}\begin{kframe}
\begin{alltt}
\hlstd{predData1} \hlkwb{<-} \hlkwd{data.frame}\hlstd{(}\hlkwc{elevation}\hlstd{=}\hlnum{1000}\hlstd{,} \hlkwc{forest}\hlstd{=}\hlnum{25}\hlstd{,} \hlkwc{water}\hlstd{=}\hlstr{"No"}\hlstd{)}
\end{alltt}
\end{kframe}
\end{knitrout}
\pause
%\vfill
\vspace{-7pt}
\begin{knitrout}\scriptsize
\definecolor{shadecolor}{rgb}{0.878, 0.918, 0.933}\color{fgcolor}\begin{kframe}
\begin{alltt}
\hlstd{E1} \hlkwb{<-} \hlkwd{predict}\hlstd{(fm1,} \hlkwc{newdata}\hlstd{=predData1,} \hlkwc{type}\hlstd{=}\hlstr{"response"}\hlstd{)}
\hlkwd{as.numeric}\hlstd{(E1)} \hlcom{# remove names (optional)}
\end{alltt}
\begin{verbatim}
## [1] 37.90222
\end{verbatim}
\end{kframe}
\end{knitrout}
\pause
%\vfill
\vspace{-7pt}
\begin{knitrout}\scriptsize
\definecolor{shadecolor}{rgb}{0.878, 0.918, 0.933}\color{fgcolor}\begin{kframe}
\begin{alltt}
\hlstd{E2} \hlkwb{<-} \hlkwd{predict}\hlstd{(fm2,} \hlkwc{newdata}\hlstd{=predData1,} \hlkwc{type}\hlstd{=}\hlstr{"response"}\hlstd{)}
\hlkwd{as.numeric}\hlstd{(E2)}
\end{alltt}
\begin{verbatim}
## [1] 42.53368
\end{verbatim}
\end{kframe}
\end{knitrout}
\pause
%\vfill
\vspace{-7pt}
\begin{knitrout}\scriptsize
\definecolor{shadecolor}{rgb}{0.878, 0.918, 0.933}\color{fgcolor}\begin{kframe}
\begin{alltt}
\hlstd{E3} \hlkwb{<-} \hlkwd{predict}\hlstd{(fm3,} \hlkwc{newdata}\hlstd{=predData1,} \hlkwc{type}\hlstd{=}\hlstr{"response"}\hlstd{)}
\hlkwd{as.numeric}\hlstd{(E3)}
\end{alltt}
\begin{verbatim}
## [1] 40.88604
\end{verbatim}
\end{kframe}
\end{knitrout}
\pause
%\vfill
\vspace{-7pt}
\begin{knitrout}\scriptsize
\definecolor{shadecolor}{rgb}{0.878, 0.918, 0.933}\color{fgcolor}\begin{kframe}
\begin{alltt}
\hlstd{E4} \hlkwb{<-} \hlkwd{predict}\hlstd{(fm4,} \hlkwc{newdata}\hlstd{=predData1,} \hlkwc{type}\hlstd{=}\hlstr{"response"}\hlstd{)}
\hlkwd{as.numeric}\hlstd{(E4)}
\end{alltt}
\begin{verbatim}
## [1] 40.86092
\end{verbatim}
\end{kframe}
\end{knitrout}
\end{frame}




\begin{frame}[fragile]
  \frametitle{Model-averaged prediction}
%  {\bf Expected number of species at 1000m, 25\% forest cover, and no
%    water, averaged over \alert{all} 4 models}
  {Expected number of species at 1000m, 25\% forest cover, and no
    water, \alert{averaged over all 4 models}}
  \pause
  \vspace{1pt}
\begin{knitrout}
\definecolor{shadecolor}{rgb}{0.878, 0.918, 0.933}\color{fgcolor}\begin{kframe}
\begin{alltt}
\hlstd{E1}\hlopt{*}\hlstd{w[}\hlnum{1}\hlstd{]} \hlopt{+} \hlstd{E2}\hlopt{*}\hlstd{w[}\hlnum{2}\hlstd{]} \hlopt{+} \hlstd{E3}\hlopt{*}\hlstd{w[}\hlnum{3}\hlstd{]} \hlopt{+} \hlstd{E4}\hlopt{*}\hlstd{w[}\hlnum{4}\hlstd{]}
\end{alltt}
\begin{verbatim}
##        1 
## 40.87927
\end{verbatim}
\end{kframe}
\end{knitrout}
\end{frame}





\begin{frame}[fragile]
  \frametitle{Model-averaged regression lines}
%  {\bf Predict species richness over range of forest cover, for each model}
  {Predict species richness over range of forest cover, for each model}
\begin{knitrout}
\definecolor{shadecolor}{rgb}{0.878, 0.918, 0.933}\color{fgcolor}\begin{kframe}
\begin{alltt}
\hlstd{predData2} \hlkwb{<-} \hlkwd{data.frame}\hlstd{(}\hlkwc{forest}\hlstd{=}\hlkwd{seq}\hlstd{(}\hlnum{0}\hlstd{,} \hlnum{100}\hlstd{,} \hlkwc{length}\hlstd{=}\hlnum{50}\hlstd{),}
                        \hlkwc{elevation}\hlstd{=}\hlnum{1000}\hlstd{,} \hlkwc{water}\hlstd{=}\hlstr{"No"}\hlstd{)}
\hlstd{E1} \hlkwb{<-} \hlkwd{predict}\hlstd{(fm1,} \hlkwc{newdata}\hlstd{=predData2)}
\hlstd{E2} \hlkwb{<-} \hlkwd{predict}\hlstd{(fm2,} \hlkwc{newdata}\hlstd{=predData2)}
\hlstd{E3} \hlkwb{<-} \hlkwd{predict}\hlstd{(fm3,} \hlkwc{newdata}\hlstd{=predData2)}
\hlstd{E4} \hlkwb{<-} \hlkwd{predict}\hlstd{(fm4,} \hlkwc{newdata}\hlstd{=predData2)}
\hlstd{Emat} \hlkwb{<-} \hlkwd{cbind}\hlstd{(E1, E2, E3, E4)}
\end{alltt}
\end{kframe}
\end{knitrout}
\pause
\vfill
%{\bf How do we model-average these vectors?}
{How do we model-average these vectors?}
\pause
\begin{knitrout}
\definecolor{shadecolor}{rgb}{0.878, 0.918, 0.933}\color{fgcolor}\begin{kframe}
\begin{alltt}
\hlstd{Evec} \hlkwb{<-} \hlstd{Emat} \hlopt \hlstd{w}
\end{alltt}
\end{kframe}
\end{knitrout}
\end{frame}




\begin{frame}[fragile]
  \frametitle{Model-averaged regression line}
\begin{knitrout}\tiny
\definecolor{shadecolor}{rgb}{0.878, 0.918, 0.933}\color{fgcolor}\begin{kframe}
\begin{alltt}
\hlkwd{plot}\hlstd{(sppRichness}\hlopt{~}\hlstd{forest,} \hlkwc{data}\hlstd{=swissData,} \hlkwc{xlab}\hlstd{=}\hlstr{"Forest cover"}\hlstd{,} \hlkwc{ylab}\hlstd{=}\hlstr{"Species richness"}\hlstd{,} \hlkwc{cex.lab}\hlstd{=}\hlnum{1.5}\hlstd{)}
\hlkwd{lines}\hlstd{(E1} \hlopt{~} \hlstd{forest, predData2,} \hlkwc{col}\hlstd{=}\hlstr{"lightgreen"}\hlstd{,} \hlkwc{lwd}\hlstd{=}\hlnum{4}\hlstd{)}
\hlkwd{lines}\hlstd{(E2} \hlopt{~} \hlstd{forest, predData2,} \hlkwc{col}\hlstd{=}\hlstr{"orange"}\hlstd{,} \hlkwc{lwd}\hlstd{=}\hlnum{3}\hlstd{)}
\hlkwd{lines}\hlstd{(E3} \hlopt{~} \hlstd{forest, predData2,} \hlkwc{col}\hlstd{=}\hlstr{"purple"}\hlstd{,} \hlkwc{lwd}\hlstd{=}\hlnum{2}\hlstd{)}
\hlkwd{lines}\hlstd{(E4} \hlopt{~} \hlstd{forest, predData2,} \hlkwc{col}\hlstd{=}\hlstr{"red"}\hlstd{,} \hlkwc{lwd}\hlstd{=}\hlnum{1}\hlstd{)}
\hlkwd{lines}\hlstd{(Evec} \hlopt{~} \hlstd{forest, predData2,} \hlkwc{col}\hlstd{=}\hlkwd{rgb}\hlstd{(}\hlnum{0}\hlstd{,}\hlnum{0}\hlstd{,}\hlnum{1}\hlstd{,}\hlnum{0.2}\hlstd{),} \hlkwc{lwd}\hlstd{=}\hlnum{10}\hlstd{)}
\hlkwd{legend}\hlstd{(}\hlnum{60}\hlstd{,} \hlnum{30}\hlstd{,} \hlkwd{c}\hlstd{(}\hlstr{"Model 1"}\hlstd{,}\hlstr{"Model 2"}\hlstd{,}\hlstr{"Model 3"}\hlstd{,}\hlstr{"Model 4"}\hlstd{,}\hlstr{"Model averaged"}\hlstd{),} \hlkwc{lty}\hlstd{=}\hlnum{1}\hlstd{,} \hlkwc{cex}\hlstd{=}\hlnum{1.2}\hlstd{,}
       \hlkwc{lwd}\hlstd{=}\hlkwd{c}\hlstd{(}\hlnum{4}\hlstd{,}\hlnum{3}\hlstd{,}\hlnum{2}\hlstd{,}\hlnum{1}\hlstd{,}\hlnum{10}\hlstd{),} \hlkwc{col}\hlstd{=}\hlkwd{c}\hlstd{(}\hlstr{"lightgreen"}\hlstd{,} \hlstr{"orange"}\hlstd{,} \hlstr{"purple"}\hlstd{,} \hlstr{"red"}\hlstd{,} \hlkwd{rgb}\hlstd{(}\hlnum{0}\hlstd{,}\hlnum{0}\hlstd{,}\hlnum{1}\hlstd{,}\hlnum{0.2}\hlstd{)))}
\end{alltt}
\end{kframe}
\end{knitrout}
%\begin{center}
\centering
  \includegraphics[width=0.75\textwidth]{figure/reglines-1} \\
%\end{center}
\end{frame}






% \begin{frame}
%   \frametitle{Assignment}

%   NOTES FOR 2016: Elev model gets 100\% of the weight. Try to fix that somehow.

%   {\bf Use the {\tt jayData} from lab 12 to do the following:}
%   \begin{enumerate}[\bf (1)]
%     \item Fit four linear models of jay abundance. Include at least
%       one interaction, and one quadratic term for elevation.
%     \item Create AIC table by hand, not using \R's {\tt AIC} function
% %    \item Model-average predictions of jay abundance at 25\% forest
% %      cover (i.e., forest=0.25)
%     \item Model-average regression lines of jay abundance and
%       elevation. Plot the averaged regression line along with the
%       regression lines from each model.
%     \item Create a map showing the model-averaged estimates of jay
%       abundance on Santa Cruz Island. Hint: use {\tt predict} with
%       {\tt cruzData} supplied as the {\tt newdata} argument.
%   \end{enumerate}
% \end{frame}




\end{document}
