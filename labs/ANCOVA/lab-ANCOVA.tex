\documentclass[color=usenames,dvipsnames]{beamer}\usepackage[]{graphicx}\usepackage[]{color}
%% maxwidth is the original width if it is less than linewidth
%% otherwise use linewidth (to make sure the graphics do not exceed the margin)
\makeatletter
\def\maxwidth{ %
  \ifdim\Gin@nat@width>\linewidth
    \linewidth
  \else
    \Gin@nat@width
  \fi
}
\makeatother

\definecolor{fgcolor}{rgb}{0, 0, 0}
\newcommand{\hlnum}[1]{\textcolor[rgb]{0.69,0.494,0}{#1}}%
\newcommand{\hlstr}[1]{\textcolor[rgb]{0.749,0.012,0.012}{#1}}%
\newcommand{\hlcom}[1]{\textcolor[rgb]{0.514,0.506,0.514}{\textit{#1}}}%
\newcommand{\hlopt}[1]{\textcolor[rgb]{0,0,0}{#1}}%
\newcommand{\hlstd}[1]{\textcolor[rgb]{0,0,0}{#1}}%
\newcommand{\hlkwa}[1]{\textcolor[rgb]{0,0,0}{\textbf{#1}}}%
\newcommand{\hlkwb}[1]{\textcolor[rgb]{0,0.341,0.682}{#1}}%
\newcommand{\hlkwc}[1]{\textcolor[rgb]{0,0,0}{\textbf{#1}}}%
\newcommand{\hlkwd}[1]{\textcolor[rgb]{0.004,0.004,0.506}{#1}}%
\let\hlipl\hlkwb

\usepackage{framed}
\makeatletter
\newenvironment{kframe}{%
 \def\at@end@of@kframe{}%
 \ifinner\ifhmode%
  \def\at@end@of@kframe{\end{minipage}}%
  \begin{minipage}{\columnwidth}%
 \fi\fi%
 \def\FrameCommand##1{\hskip\@totalleftmargin \hskip-\fboxsep
 \colorbox{shadecolor}{##1}\hskip-\fboxsep
     % There is no \\@totalrightmargin, so:
     \hskip-\linewidth \hskip-\@totalleftmargin \hskip\columnwidth}%
 \MakeFramed {\advance\hsize-\width
   \@totalleftmargin\z@ \linewidth\hsize
   \@setminipage}}%
 {\par\unskip\endMakeFramed%
 \at@end@of@kframe}
\makeatother

\definecolor{shadecolor}{rgb}{.97, .97, .97}
\definecolor{messagecolor}{rgb}{0, 0, 0}
\definecolor{warningcolor}{rgb}{1, 0, 1}
\definecolor{errorcolor}{rgb}{1, 0, 0}
\newenvironment{knitrout}{}{} % an empty environment to be redefined in TeX

\usepackage{alltt}
%\documentclass[color=usenames,dvipsnames,handout]{beamer}



%\usepackage[roman]{../../lab1}
\usepackage[sans]{../../lab1}

\hypersetup{pdftex,pdfstartview=FitV}









%% New command for inline code that isn't to be evaluated
\definecolor{inlinecolor}{rgb}{0.878, 0.918, 0.933}
\newcommand{\inr}[1]{\colorbox{inlinecolor}{\texttt{#1}}}










\IfFileExists{upquote.sty}{\usepackage{upquote}}{}
\begin{document}


%\setlength\fboxsep{0pt}



\begin{frame}[plain]
  \huge
  \centering \par
  {\color{RoyalBlue}{Lab 11 -- ANCOVA}} \\
  \vspace{1cm}
  \Large
  October 29 \& 30, 2018 \\
  FANR 6750 \\
  \vfill
  \large
  Richard Chandler and Bob Cooper

  \note{
  FIXME for 2016: Change variable ``length'' to something like
  ``size''. And center the covariate so that mu is the grand mean
  }
\end{frame}



\section{Regression}




\begin{frame}
  \frametitle{ANCOVA overview}
  {\bf Scenario}
  \begin{itemize}
    \item We are interested in doing a one-way ANOVA
    \item However, we need to account for variation associated with a
      continuous predictor variable
  \end{itemize}
  \pause
  \vfill
%  \centering
  \bf
  ANCOVA can be thought of as a hybrid between ANOVA and regression \\
  \pause
  \vfill
%  \centering
  \bf
  ANOVA, regression, and ANCOVA are linear models \\
\end{frame}



\begin{frame}[fragile]
  \frametitle{The Diet Data}
%  \small
  Import the data and view the levels of the factor
\begin{knitrout}\small
\definecolor{shadecolor}{rgb}{0.878, 0.918, 0.933}\color{fgcolor}\begin{kframe}
\begin{alltt}
\hlstd{dietData} \hlkwb{<-} \hlkwd{read.csv}\hlstd{(}\hlstr{"dietData.csv"}\hlstd{)}
\hlkwd{levels}\hlstd{(dietData}\hlopt{$}\hlstd{diet)}
\end{alltt}
\begin{verbatim}
## [1] "Control" "High"    "Low"     "Med"
\end{verbatim}
\end{kframe}
\end{knitrout}
\pause
\vspace{0.5cm}
{\large Reorder the levels of the factor, just for convenience}
\begin{knitrout}\small
\definecolor{shadecolor}{rgb}{0.878, 0.918, 0.933}\color{fgcolor}\begin{kframe}
\begin{alltt}
\hlkwd{levels}\hlstd{(dietData}\hlopt{$}\hlstd{diet)} \hlkwb{<-} \hlkwd{list}\hlstd{(}\hlkwc{Control}\hlstd{=}\hlstr{"Control"}\hlstd{,} \hlkwc{Low}\hlstd{=}\hlstr{"Low"}\hlstd{,}
                              \hlkwc{Med}\hlstd{=}\hlstr{"Med"}\hlstd{,} \hlkwc{High}\hlstd{=}\hlstr{"High"}\hlstd{)}
\hlkwd{levels}\hlstd{(dietData}\hlopt{$}\hlstd{diet)}
\end{alltt}
\begin{verbatim}
## [1] "Control" "Low"     "Med"     "High"
\end{verbatim}
\end{kframe}
\end{knitrout}
\end{frame}




\begin{frame}[fragile]
  \frametitle{The diet data}
%<<scatter,echo=FALSE,fig.show="hide">>=
%plot(weight ~ age, dietData)
%@
%<<bp,echo=FALSE,fig.show=FALSE>>=
%boxplot(weight ~ diet, dietData, ylab="Weight")
%@
  \begin{columns}
    \begin{column}{0.5\textwidth}
      \tiny %\scriptsize
\begin{knitrout}
\definecolor{shadecolor}{rgb}{0.878, 0.918, 0.933}\color{fgcolor}\begin{kframe}
\begin{alltt}
\hlkwd{plot}\hlstd{(weight} \hlopt{~} \hlstd{age, dietData)}
\end{alltt}
\end{kframe}
\end{knitrout}
    \includegraphics[width=\textwidth]{figure/plot-wt-1}
    \end{column}
    \begin{column}{0.5\textwidth}
      \tiny %\scriptsize
\begin{knitrout}
\definecolor{shadecolor}{rgb}{0.878, 0.918, 0.933}\color{fgcolor}\begin{kframe}
\begin{alltt}
\hlkwd{boxplot}\hlstd{(weight} \hlopt{~} \hlstd{diet, dietData,} \hlkwc{ylab}\hlstd{=}\hlstr{"Weight"}\hlstd{)}
\end{alltt}
\end{kframe}
\end{knitrout}
    \includegraphics[width=\textwidth]{figure/bp-1}
    \end{column}
  \end{columns}
\end{frame}




\begin{frame}[fragile]
  \frametitle{Simple linear regression using {\tt lm}}
%  \scriptsize

\begin{knitrout}\scriptsize
\definecolor{shadecolor}{rgb}{0.878, 0.918, 0.933}\color{fgcolor}\begin{kframe}
\begin{alltt}
\hlstd{fm1} \hlkwb{<-} \hlkwd{lm}\hlstd{(weight} \hlopt{~} \hlstd{age, dietData)}
\hlkwd{summary}\hlstd{(fm1)}
\end{alltt}
\begin{verbatim}
## 
## Call:
## lm(formula = weight ~ age, data = dietData)
## 
## Residuals:
##     Min      1Q  Median      3Q     Max 
## -5.6906 -1.2625  0.0522  1.0233  6.1680 
## 
## Coefficients:
##             Estimate Std. Error t value Pr(>|t|)    
## (Intercept) 21.32523    0.80685  26.430  < 2e-16 ***
## age          0.51807    0.06742   7.685 2.07e-10 ***
## ---
## Signif. codes:  0 '***' 0.001 '**' 0.01 '*' 0.05 '.' 0.1 ' ' 1
## 
## Residual standard error: 2.171 on 58 degrees of freedom
## Multiple R-squared:  0.5045,	Adjusted R-squared:  0.496 
## F-statistic: 59.05 on 1 and 58 DF,  p-value: 2.072e-10
\end{verbatim}
\end{kframe}
\end{knitrout}
\pause
\footnotesize
{The two estimates correspond to the intercept and slope parameters}
\end{frame}



\begin{frame}[fragile]
  \frametitle{Regression line and confidence interval}
  {%\bf %\footnotesize
    Regression lines and CIs can be created using
    \inr{predict} \\}
  \begin{enumerate}[\bf (1)]
    \item Create a new {\tt data.frame} containing a sequence of
        values of the predictor variable {\it age}
    \item Predict {\it weight} using these values of {\it age}
    \item Put predictions and data together for easy plotting
  \end{enumerate}
  \pause
  \vfill
  \small
\begin{knitrout}\footnotesize
\definecolor{shadecolor}{rgb}{0.878, 0.918, 0.933}\color{fgcolor}\begin{kframe}
\begin{alltt}
\hlstd{age} \hlkwb{<-} \hlstd{dietData}\hlopt{$}\hlstd{age}
\hlstd{nd} \hlkwb{<-} \hlkwd{data.frame}\hlstd{(}\hlkwc{age}\hlstd{=}\hlkwd{seq}\hlstd{(}\hlkwd{min}\hlstd{(age),} \hlkwd{max}\hlstd{(age),} \hlkwc{length}\hlstd{=}\hlnum{50}\hlstd{))}
\hlstd{E1} \hlkwb{<-} \hlkwd{predict}\hlstd{(fm1,} \hlkwc{newdata}\hlstd{=nd,} \hlkwc{se.fit}\hlstd{=}\hlnum{TRUE}\hlstd{,}
              \hlkwc{interval}\hlstd{=}\hlstr{"confidence"}\hlstd{)}
\hlstd{predictionData} \hlkwb{<-} \hlkwd{data.frame}\hlstd{(E1}\hlopt{$}\hlstd{fit, nd)}
\end{alltt}
\end{kframe}
\end{knitrout}
%\pause
%\vfill
%{\bf Plot raw data and the regression results}
\end{frame}




\begin{frame}[fragile]
  \frametitle{Regression line and confidence interval}
  \scriptsize
\begin{knitrout}\scriptsize
\definecolor{shadecolor}{rgb}{0.878, 0.918, 0.933}\color{fgcolor}\begin{kframe}
\begin{alltt}
\hlkwd{plot}\hlstd{(weight} \hlopt{~} \hlstd{age,} \hlkwc{data}\hlstd{=dietData)}                   \hlcom{# raw data}
\hlkwd{lines}\hlstd{(fit} \hlopt{~} \hlstd{age,} \hlkwc{data}\hlstd{=predictionData,} \hlkwc{lwd}\hlstd{=}\hlnum{2}\hlstd{)}        \hlcom{# fitted line}
\hlkwd{lines}\hlstd{(lwr} \hlopt{~} \hlstd{age,} \hlkwc{data}\hlstd{=predictionData,} \hlkwc{lwd}\hlstd{=}\hlnum{2}\hlstd{,} \hlkwc{lty}\hlstd{=}\hlnum{3}\hlstd{)} \hlcom{# lower CI}
\hlkwd{lines}\hlstd{(upr} \hlopt{~} \hlstd{age,} \hlkwc{data}\hlstd{=predictionData,} \hlkwc{lwd}\hlstd{=}\hlnum{2}\hlstd{,} \hlkwc{lty}\hlstd{=}\hlnum{3}\hlstd{)} \hlcom{# upper CI}
\end{alltt}
\end{kframe}
\end{knitrout}
  \vspace{-9mm}
  \begin{center}
    \includegraphics[width=0.65\textwidth]{figure/plot-fm1-1}
  \end{center}
\end{frame}



\section{One-way ANOVA}



\begin{frame}[fragile]
  \frametitle{One-way ANOVA using {\tt lm}}
{%\bf
  Change the \inr{contrasts} option so that the estimates will
  correspond to the additive model, and then fit the ANOVA
}
\begin{knitrout}
\definecolor{shadecolor}{rgb}{0.878, 0.918, 0.933}\color{fgcolor}\begin{kframe}
\begin{alltt}
\hlkwd{options}\hlstd{(}\hlkwc{contrasts}\hlstd{=}\hlkwd{c}\hlstd{(}\hlstr{"contr.sum"}\hlstd{,} \hlstr{"contr.poly"}\hlstd{))}
\hlstd{fm2} \hlkwb{<-} \hlkwd{lm}\hlstd{(weight} \hlopt{~} \hlstd{diet, dietData)}
\hlkwd{summary.aov}\hlstd{(fm2)}
\end{alltt}
\begin{verbatim}
##             Df Sum Sq Mean Sq F value Pr(>F)
## diet         3   54.6  18.216   2.053  0.117
## Residuals   56  496.9   8.873
\end{verbatim}
\end{kframe}
\end{knitrout}
\pause
\vfill
{%\bf
  The \inr{aov} function gives identical results}
\begin{knitrout}
\definecolor{shadecolor}{rgb}{0.878, 0.918, 0.933}\color{fgcolor}\begin{kframe}
\begin{alltt}
\hlkwd{summary}\hlstd{(}\hlkwd{aov}\hlstd{(weight} \hlopt{~} \hlstd{diet, dietData))}
\end{alltt}
\begin{verbatim}
##             Df Sum Sq Mean Sq F value Pr(>F)
## diet         3   54.6  18.216   2.053  0.117
## Residuals   56  496.9   8.873
\end{verbatim}
\end{kframe}
\end{knitrout}
\end{frame}



\begin{frame}[fragile]
  \frametitle{An alternative summary}
  \scriptsize
\begin{knitrout}\tiny
\definecolor{shadecolor}{rgb}{0.878, 0.918, 0.933}\color{fgcolor}\begin{kframe}
\begin{alltt}
\hlkwd{summary}\hlstd{(fm2)}
\end{alltt}
\begin{verbatim}
## 
## Call:
## lm(formula = weight ~ diet, data = dietData)
## 
## Residuals:
##     Min      1Q  Median      3Q     Max 
## -7.6371 -1.9253 -0.0366  1.9770  5.4576 
## 
## Coefficients:
##             Estimate Std. Error t value Pr(>|t|)    
## (Intercept) 27.13962    0.38456  70.573   <2e-16 ***
## diet1       -1.02179    0.66608  -1.534    0.131    
## diet2       -0.56593    0.66608  -0.850    0.399    
## diet3        0.08027    0.66608   0.121    0.905    
## ---
## Signif. codes:  0 '***' 0.001 '**' 0.01 '*' 0.05 '.' 0.1 ' ' 1
## 
## Residual standard error: 2.979 on 56 degrees of freedom
## Multiple R-squared:  0.09908,	Adjusted R-squared:  0.05082 
## F-statistic: 2.053 on 3 and 56 DF,  p-value: 0.1169
\end{verbatim}
\end{kframe}
\end{knitrout}
 \pause
 {%\bf
   Because we changed the \inr{contrast} option to {\tt
     contr.sum}, the intercept is the grand mean ($\mu$) and the other
   estimates are the effect sizes ($\alpha_i$) \\}
%     contr.sum}, The intercept is the mean weight for the reference level
%   (Control) when age=0. The other estimates indicate the
%   difference from the reference level. \\}
\end{frame}



%\begin{comment}
\begin{frame}[fragile]
  \frametitle{One-way ANOVA}
  {%\centering
    \small %\bf
    The \inr{predict} function can also be used to obtain
    group means, SEs, and CIs from a one-way ANOVA \\}
  \footnotesize %\scriptsize
\begin{knitrout}
\definecolor{shadecolor}{rgb}{0.878, 0.918, 0.933}\color{fgcolor}\begin{kframe}
\begin{alltt}
\hlstd{nd2} \hlkwb{<-} \hlkwd{data.frame}\hlstd{(}\hlkwc{diet}\hlstd{=}\hlkwd{levels}\hlstd{(dietData}\hlopt{$}\hlstd{diet))}
\hlstd{E2} \hlkwb{<-} \hlkwd{predict}\hlstd{(fm2,} \hlkwc{newdata}\hlstd{=nd2,}
              \hlkwc{se.fit}\hlstd{=}\hlnum{TRUE}\hlstd{,} \hlkwc{interval}\hlstd{=}\hlstr{"confidence"}\hlstd{)}
\end{alltt}
\end{kframe}
\end{knitrout}
\pause
\vfill
\begin{knitrout}
\definecolor{shadecolor}{rgb}{0.878, 0.918, 0.933}\color{fgcolor}\begin{kframe}
\begin{alltt}
\hlstd{predData2} \hlkwb{<-} \hlkwd{data.frame}\hlstd{(E2}\hlopt{$}\hlstd{fit, nd2)}
\hlstd{predData2}
\end{alltt}
\begin{verbatim}
##        fit      lwr      upr    diet
## 1 26.11783 24.57710 27.65856 Control
## 2 26.57368 25.03295 28.11442     Low
## 3 27.21988 25.67915 28.76062     Med
## 4 28.64707 27.10634 30.18780    High
\end{verbatim}
\end{kframe}
\end{knitrout}

\end{frame}


\begin{frame}
  \frametitle{One-way ANOVA}
  \vspace{-0.5cm}
  \begin{center}
    \includegraphics[width=0.8\textwidth]{figure/barplot1-1}
  \end{center}
\end{frame}
%\end{comment}



\section{ANCOVA}



\begin{frame}[fragile]
  \frametitle{ANCOVA preliminaries}
  \small
  { Additive model}
  \[
    y_{ij} = \mu + \alpha_i + \beta(x_{ij} - \bar{x}) + \varepsilon_{ij}
  \]
  \pause
  \vfill
  {Make sure the {\tt contrasts} are set as before}
\begin{knitrout}
\definecolor{shadecolor}{rgb}{0.878, 0.918, 0.933}\color{fgcolor}\begin{kframe}
\begin{alltt}
\hlkwd{options}\hlstd{(}\hlkwc{contrasts}\hlstd{=}\hlkwd{c}\hlstd{(}\hlstr{"contr.sum"}\hlstd{,} \hlstr{"contr.poly"}\hlstd{))}
\end{alltt}
\end{kframe}
\end{knitrout}
  \pause
  \vfill
  {Centering the covariate isn't required, but doing so allow the
    intercept to be interpretted as the grand mean}
%  \small
  \footnotesize
\begin{knitrout}
\definecolor{shadecolor}{rgb}{0.878, 0.918, 0.933}\color{fgcolor}\begin{kframe}
\begin{alltt}
\hlstd{dietData}\hlopt{$}\hlstd{ageCentered} \hlkwb{<-} \hlstd{dietData}\hlopt{$}\hlstd{age} \hlopt{-} \hlkwd{mean}\hlstd{(dietData}\hlopt{$}\hlstd{age)}
\end{alltt}
\end{kframe}
\end{knitrout}
\end{frame}



\begin{frame}[fragile]
  \frametitle{ANCOVA}
%  \pause
{%\scriptsize
  \footnotesize
  %\bf
  Put the covariate before the treatment
  variable in the formula. \\}
%\scriptsize %\tiny
\begin{knitrout}\tiny
\definecolor{shadecolor}{rgb}{0.878, 0.918, 0.933}\color{fgcolor}\begin{kframe}
\begin{alltt}
\hlstd{fm3} \hlkwb{<-} \hlkwd{lm}\hlstd{(weight} \hlopt{~} \hlstd{ageCentered} \hlopt{+} \hlstd{diet, dietData)}
\end{alltt}
\end{kframe}
\end{knitrout}
\pause
%\vfill
\begin{knitrout}\tiny
\definecolor{shadecolor}{rgb}{0.878, 0.918, 0.933}\color{fgcolor}\begin{kframe}
\begin{alltt}
\hlkwd{summary}\hlstd{(fm3)}
\end{alltt}
\begin{verbatim}
## 
## Call:
## lm(formula = weight ~ ageCentered + diet, data = dietData)
## 
## Residuals:
##     Min      1Q  Median      3Q     Max 
## -3.8214 -1.2213 -0.2519  1.2161  4.9185 
## 
## Coefficients:
##             Estimate Std. Error t value Pr(>|t|)    
## (Intercept)  27.1396     0.2406 112.787  < 2e-16 ***
## ageCentered   0.5573     0.0594   9.382  5.2e-13 ***
## diet1        -1.7446     0.4238  -4.116  0.00013 ***
## diet2        -0.3758     0.4173  -0.901  0.37171    
## diet3         0.7819     0.4234   1.847  0.07020 .  
## ---
## Signif. codes:  0 '***' 0.001 '**' 0.01 '*' 0.05 '.' 0.1 ' ' 1
## 
## Residual standard error: 1.864 on 55 degrees of freedom
## Multiple R-squared:  0.6536,	Adjusted R-squared:  0.6284 
## F-statistic: 25.94 on 4 and 55 DF,  p-value: 4.147e-12
\end{verbatim}
\end{kframe}
\end{knitrout}
\end{frame}




\begin{frame}[fragile]
  \frametitle{The ANOVA table}
  {%\bf
    The null hypothesis of no diet
    effect is rejected, even though it was not rejected before.}
  \small
\begin{knitrout}
\definecolor{shadecolor}{rgb}{0.878, 0.918, 0.933}\color{fgcolor}\begin{kframe}
\begin{alltt}
\hlkwd{summary.aov}\hlstd{(fm3)}
\end{alltt}
\begin{verbatim}
##             Df Sum Sq Mean Sq F value   Pr(>F)    
## ageCentered  1 278.25  278.25  80.095 2.54e-12 ***
## diet         3  82.22   27.41   7.889 0.000182 ***
## Residuals   55 191.07    3.47                     
## ---
## Signif. codes:  0 '***' 0.001 '**' 0.01 '*' 0.05 '.' 0.1 ' ' 1
\end{verbatim}
\end{kframe}
\end{knitrout}
\end{frame}






\begin{frame}[fragile]
  \frametitle{Predict weight}
  {%\bf
    Create predictions of {\tt weight} over a sequences of {\tt
      ages}, for every level of {\tt diet}}
  \small
\begin{knitrout}\small
\definecolor{shadecolor}{rgb}{0.878, 0.918, 0.933}\color{fgcolor}\begin{kframe}
\begin{alltt}
\hlstd{ageC} \hlkwb{<-} \hlstd{dietData}\hlopt{$}\hlstd{ageCentered}
\hlstd{nd3} \hlkwb{<-} \hlkwd{data.frame}\hlstd{(}
    \hlkwc{diet}\hlstd{=}\hlkwd{rep}\hlstd{(}\hlkwd{c}\hlstd{(}\hlstr{"Control"}\hlstd{,} \hlstr{"Low"}\hlstd{,} \hlstr{"Med"}\hlstd{,} \hlstr{"High"}\hlstd{),} \hlkwc{each}\hlstd{=}\hlnum{20}\hlstd{),}
    \hlkwc{ageCentered}\hlstd{=}\hlkwd{rep}\hlstd{(}\hlkwd{seq}\hlstd{(}\hlkwd{min}\hlstd{(ageC),} \hlkwd{max}\hlstd{(ageC),}
                          \hlkwc{length}\hlstd{=}\hlnum{20}\hlstd{),}
                      \hlkwc{times}\hlstd{=}\hlnum{4}\hlstd{))}
\hlstd{E3} \hlkwb{<-} \hlkwd{predict}\hlstd{(fm3,} \hlkwc{newdata}\hlstd{=nd3,} \hlkwc{se.fit}\hlstd{=}\hlnum{TRUE}\hlstd{,}
              \hlkwc{interval}\hlstd{=}\hlstr{"confidence"}\hlstd{)}
\hlstd{predData3} \hlkwb{<-} \hlkwd{data.frame}\hlstd{(E3}\hlopt{$}\hlstd{fit, nd3)}
\end{alltt}
\end{kframe}
\end{knitrout}
\end{frame}




\begin{frame}[fragile]
  \frametitle{Plot the regression lines}
\footnotesize %\small

\begin{knitrout}\footnotesize
\definecolor{shadecolor}{rgb}{0.878, 0.918, 0.933}\color{fgcolor}\begin{kframe}
\begin{alltt}
\hlkwd{plot}\hlstd{(weight} \hlopt{~} \hlstd{ageCentered, dietData,} \hlkwc{pch}\hlstd{=}\hlnum{16}\hlstd{,} \hlkwc{cex}\hlstd{=}\hlnum{1.5}\hlstd{,}
     \hlkwc{col}\hlstd{=}\hlkwd{rep}\hlstd{(colrs,} \hlkwc{each}\hlstd{=}\hlnum{15}\hlstd{))}
\hlkwd{lines}\hlstd{(fit} \hlopt{~} \hlstd{ageCentered, predData3,} \hlkwc{subset}\hlstd{=diet}\hlopt{==}\hlstr{"Control"}\hlstd{,}
      \hlkwc{col}\hlstd{=colrs[}\hlnum{1}\hlstd{],} \hlkwc{lwd}\hlstd{=}\hlnum{2}\hlstd{)}
\hlkwd{lines}\hlstd{(fit} \hlopt{~} \hlstd{ageCentered, predData3,} \hlkwc{subset}\hlstd{=diet}\hlopt{==}\hlstr{"Low"}\hlstd{,} \hlkwc{lty}\hlstd{=}\hlnum{1}\hlstd{,}
      \hlkwc{col}\hlstd{=colrs[}\hlnum{2}\hlstd{],} \hlkwc{lwd}\hlstd{=}\hlnum{2}\hlstd{)}
\hlkwd{lines}\hlstd{(fit} \hlopt{~} \hlstd{ageCentered, predData3,} \hlkwc{subset}\hlstd{=diet}\hlopt{==}\hlstr{"Med"}\hlstd{,} \hlkwc{lty}\hlstd{=}\hlnum{1}\hlstd{,}
      \hlkwc{col}\hlstd{=colrs[}\hlnum{3}\hlstd{],} \hlkwc{lwd}\hlstd{=}\hlnum{2}\hlstd{)}
\hlkwd{lines}\hlstd{(fit} \hlopt{~} \hlstd{ageCentered, predData3,} \hlkwc{subset}\hlstd{=diet}\hlopt{==}\hlstr{"High"}\hlstd{,} \hlkwc{lty}\hlstd{=}\hlnum{1}\hlstd{,}
      \hlkwc{col}\hlstd{=colrs[}\hlnum{4}\hlstd{],} \hlkwc{lwd}\hlstd{=}\hlnum{2}\hlstd{)}
\hlkwd{legend}\hlstd{(}\hlnum{4}\hlstd{,} \hlnum{23}\hlstd{,} \hlkwd{c}\hlstd{(}\hlstr{"High"}\hlstd{,} \hlstr{"Med"}\hlstd{,} \hlstr{"Low"}\hlstd{,} \hlstr{"Control"}\hlstd{),} \hlkwc{pch}\hlstd{=}\hlnum{16}\hlstd{,}
       \hlkwc{title}\hlstd{=}\hlstr{"Diet"}\hlstd{,} \hlkwc{lwd}\hlstd{=}\hlnum{2}\hlstd{,} \hlkwc{col}\hlstd{=}\hlkwd{rev}\hlstd{(colrs))}
\end{alltt}
\end{kframe}
\end{knitrout}
\end{frame}






\begin{frame}
  \frametitle{Plot the regression lines}
  \vspace{-0.5cm}
  \begin{center}
    \only<1 | handout:0>{\includegraphics[width=0.8\textwidth]{figure/scatplot0-1}}
    \only<2>{\includegraphics[width=0.8\textwidth]{figure/scatplot1-1}}
  \end{center}
\end{frame}







\begin{frame}[fragile]
  \frametitle{Multiple comparisons}
%{\bf Use {\tt multcomp} package}
\scriptsize
\begin{knitrout}
\definecolor{shadecolor}{rgb}{0.878, 0.918, 0.933}\color{fgcolor}\begin{kframe}
\begin{alltt}
\hlcom{## install.packages("multcomp")}
\hlkwd{library}\hlstd{(multcomp)}
\hlkwd{summary}\hlstd{(}\hlkwd{glht}\hlstd{(fm3,} \hlkwc{linfct}\hlstd{=}\hlkwd{mcp}\hlstd{(}\hlkwc{diet}\hlstd{=}\hlstr{"Tukey"}\hlstd{)))}
\end{alltt}
\begin{verbatim}
## 
## 	 Simultaneous Tests for General Linear Hypotheses
## 
## Multiple Comparisons of Means: Tukey Contrasts
## 
## 
## Fit: lm(formula = weight ~ ageCentered + diet, data = dietData)
## 
## Linear Hypotheses:
##                     Estimate Std. Error t value Pr(>|t|)    
## Low - Control == 0    1.3688     0.6875   1.991  0.20389    
## Med - Control == 0    2.5265     0.6973   3.623  0.00336 ** 
## High - Control == 0   3.0830     0.6832   4.513  < 0.001 ***
## Med - Low == 0        1.1577     0.6828   1.696  0.33583    
## High - Low == 0       1.7143     0.6817   2.515  0.06861 .  
## High - Med == 0       0.5566     0.6869   0.810  0.84931    
## ---
## Signif. codes:  0 '***' 0.001 '**' 0.01 '*' 0.05 '.' 0.1 ' ' 1
## (Adjusted p values reported -- single-step method)
\end{verbatim}
\end{kframe}
\end{knitrout}
\end{frame}







\begin{frame}[fragile]
  \frametitle{Assignment}
  {\bf \large Complete the following and upload your \R~script to ELC
    before lab next week \\}
%  \pause
  \vfill
  \begin{enumerate}[\bf \color{PineGreen} (1)]
%    \item Run a new analysis using {\tt lm} to test for an interaction
%      of diet and age
%    \item Is the interaction significant? Report the $F$-value and
%      $P$-value used for this test.
    \item Fit an ANCOVA model to the data in {\tt treeData.csv}, which
      represent the height of trees following a fertilizer
      experiment. The covariate is pH.
    \item Use: {\tt options(contrasts=c("contr.sum", "contr.poly"))}
      so that your estimates correspond to the additive model from the
      lecture notes
    \item Interpret each of the estimates from {\tt lm}. What is the
      null hypothesis associated with each $p$-value?
    \item Plot the data and the regression lines. Use different colors
      or symbols to distinguish the treatment groups.
    \item Which fertilizer treatments are significantly different?
  \end{enumerate}

\end{frame}



\end{document}

