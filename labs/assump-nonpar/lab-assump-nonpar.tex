\documentclass[color=usenames,dvipsnames]{beamer}\usepackage[]{graphicx}\usepackage[]{color}
% maxwidth is the original width if it is less than linewidth
% otherwise use linewidth (to make sure the graphics do not exceed the margin)
\makeatletter
\def\maxwidth{ %
  \ifdim\Gin@nat@width>\linewidth
    \linewidth
  \else
    \Gin@nat@width
  \fi
}
\makeatother

\definecolor{fgcolor}{rgb}{0, 0, 0}
\newcommand{\hlnum}[1]{\textcolor[rgb]{0.69,0.494,0}{#1}}%
\newcommand{\hlstr}[1]{\textcolor[rgb]{0.749,0.012,0.012}{#1}}%
\newcommand{\hlcom}[1]{\textcolor[rgb]{0.514,0.506,0.514}{\textit{#1}}}%
\newcommand{\hlopt}[1]{\textcolor[rgb]{0,0,0}{#1}}%
\newcommand{\hlstd}[1]{\textcolor[rgb]{0,0,0}{#1}}%
\newcommand{\hlkwa}[1]{\textcolor[rgb]{0,0,0}{\textbf{#1}}}%
\newcommand{\hlkwb}[1]{\textcolor[rgb]{0,0.341,0.682}{#1}}%
\newcommand{\hlkwc}[1]{\textcolor[rgb]{0,0,0}{\textbf{#1}}}%
\newcommand{\hlkwd}[1]{\textcolor[rgb]{0.004,0.004,0.506}{#1}}%
\let\hlipl\hlkwb

\usepackage{framed}
\makeatletter
\newenvironment{kframe}{%
 \def\at@end@of@kframe{}%
 \ifinner\ifhmode%
  \def\at@end@of@kframe{\end{minipage}}%
  \begin{minipage}{\columnwidth}%
 \fi\fi%
 \def\FrameCommand##1{\hskip\@totalleftmargin \hskip-\fboxsep
 \colorbox{shadecolor}{##1}\hskip-\fboxsep
     % There is no \\@totalrightmargin, so:
     \hskip-\linewidth \hskip-\@totalleftmargin \hskip\columnwidth}%
 \MakeFramed {\advance\hsize-\width
   \@totalleftmargin\z@ \linewidth\hsize
   \@setminipage}}%
 {\par\unskip\endMakeFramed%
 \at@end@of@kframe}
\makeatother

\definecolor{shadecolor}{rgb}{.97, .97, .97}
\definecolor{messagecolor}{rgb}{0, 0, 0}
\definecolor{warningcolor}{rgb}{1, 0, 1}
\definecolor{errorcolor}{rgb}{1, 0, 0}
\newenvironment{knitrout}{}{} % an empty environment to be redefined in TeX

\usepackage{alltt}
%\documentclass[color=usenames,dvipsnames,handout]{beamer}

%\usepackage[roman]{../../lab1}
\usepackage[sans]{../../lab1}

\hypersetup{pdftex,pdfstartview=FitV}









%% New command for inline code that isn't to be evaluated
\definecolor{inlinecolor}{rgb}{0.878, 0.918, 0.933}
\newcommand{\inr}[1]{\colorbox{inlinecolor}{\texttt{#1}}}
\IfFileExists{upquote.sty}{\usepackage{upquote}}{}
\begin{document}





%\section{Intro}

\begin{frame}[plain]
%  \maketitle
  \LARGE
  \centering \par
  {\bf \color{RoyalBlue}{Lab 5 -- Assumptions of ANOVA}} \par
  \vspace{1cm}
  \Large
%  September 17 \& 18, 2018 \\
  FANR 6750 \\
  \vfill
  \large
  Richard Chandler and Bob Cooper
\end{frame}





\begin{frame}[plain]
  \frametitle{Today's Topics}
  \Large
  \only<1>{\tableofcontents}%[hideallsubsections]}
%  \only<2 | handout:0>{\tableofcontents[currentsection]}%,hideallsubsections]}
\end{frame}






\section{Assumptions of ANOVA}







%% \begin{frame}[fragile]
%%   \frametitle{Is it normally distributed?}
%%   \small
%% <<boxfor0,fig=true,include=false>>=
%% boxplot(forestCover$pcForest, col="lightgreen",
%%         ylab="Percent forest cover", main="", cex.lab=1.5)
%% @
%% \vspace{-0.5cm}
%% \centering
%% \includegraphics[width=0.7\textwidth]{lab05-assump-nonpar-boxfor0} \\
%% \end{frame}





%% \begin{frame}[fragile]
%%   \frametitle{Is it normally distributed?}
%%   \small
%% <<histfor0,fig=true,include=false>>=
%% hist(forestCover$pcForest, col="lightgreen", cex.lab=1.5,
%%      xlab="Percent forest cover", main="")
%% @
%% \vspace{-0.5cm}
%% \centering
%% \includegraphics[width=0.7\textwidth]{lab05-assump-nonpar-histfor0} \\
%% \end{frame}



%% \begin{frame}[fragile]
%%   \frametitle{Is it normally distributed?}
%% <<>>=
%% shapiro.test(forestCover$pcForest)
%% @
%% \end{frame}




\begin{frame}
  \frametitle{Assumptions of ANOVA}
  {%\bf
    A common misconception is that the response variable must be
    normally distributed when conducting an ANOVA.}
  \pause
  \vfill
  {%\bf
    This is incorrect because the normality assumptions pertain to
    the {\it residuals}, \alert{not} the response variable.

    \pause
    \vfill
    The key assumption of
    ANOVA is that the residuals are independent and come from a normal
    distribution with mean 0 and variance $\sigma^2$.}
  \pause
  \large
\[
  y_{ij} = \mu + \alpha_i + \varepsilon_{ij}
\]
\[
  \varepsilon_{ij} \sim \text{Normal}(0, \sigma^2)
\]
\pause
%\vfill
\normalsize
  {%\bf
    We can assess this assumption by looking at the residuals
    themselves or by looking at the response variable for each treatment group. }
\end{frame}





\begin{frame}[fragile]
  \frametitle{A fake dataset}

\small
%  {%\bf %\large
    Consider the data: %}
%  \scriptsize %\small
\begin{knitrout}\tiny
\definecolor{shadecolor}{rgb}{0.878, 0.918, 0.933}\color{fgcolor}\begin{kframe}
\begin{alltt}
\hlstd{infectionRates} \hlkwb{<-} \hlkwd{read.csv}\hlstd{(}\hlstr{"infectionRates.csv"}\hlstd{)}
\hlkwd{str}\hlstd{(infectionRates)}
\end{alltt}
\begin{verbatim}
## 'data.frame':	90 obs. of  2 variables:
##  $ percentInfected: num  0.21 0.25 0.17 0.26 0.21 0.21 0.22 0.27 0.23 0.14 ...
##  $ landscape      : Factor w/ 3 levels "Park","Suburban",..: 1 1 1 1 1 1 1 1 1 1 ...
\end{verbatim}
\begin{alltt}
\hlkwd{summary}\hlstd{(infectionRates)}
\end{alltt}
\begin{verbatim}
##  percentInfected    landscape 
##  Min.   :0.010   Park    :30  
##  1st Qu.:0.040   Suburban:30  
##  Median :0.090   Urban   :30  
##  Mean   :0.121                
##  3rd Qu.:0.210                
##  Max.   :0.330
\end{verbatim}
\end{kframe}
\end{knitrout}
%  \normalsize %\bf
\vfill
These data are made-up, but imagine they come from a study in which
100 crows are placed in $n=30$ enclosures in each of 3 landscapes. The
response variable is the proportion of crows infected with West Nile
virus at the end of the study. \\
\end{frame}







\begin{frame}[fragile]
  \frametitle{One-way ANOVA}
\begin{knitrout}\footnotesize
\definecolor{shadecolor}{rgb}{0.878, 0.918, 0.933}\color{fgcolor}\begin{kframe}
\begin{alltt}
\hlstd{anova1} \hlkwb{<-} \hlkwd{aov}\hlstd{(percentInfected} \hlopt{~} \hlstd{landscape,}
              \hlkwc{data}\hlstd{=infectionRates)}
\hlkwd{summary}\hlstd{(anova1)}
\end{alltt}
\begin{verbatim}
##             Df Sum Sq Mean Sq F value Pr(>F)    
## landscape    2 0.6384  0.3192     306 <2e-16 ***
## Residuals   87 0.0908  0.0010                   
## ---
## Signif. codes:  0 '***' 0.001 '**' 0.01 '*' 0.05 '.' 0.1 ' ' 1
\end{verbatim}
\end{kframe}
\end{knitrout}
\pause
\vfill
%\bf
Significant, but did we meet the assumptions?
\end{frame}



\begin{frame}[fragile]
  \frametitle{Boxplots for the three landscape types}
\begin{knitrout}\footnotesize
\definecolor{shadecolor}{rgb}{0.878, 0.918, 0.933}\color{fgcolor}\begin{kframe}
\begin{alltt}
\hlkwd{boxplot}\hlstd{(percentInfected}\hlopt{~}\hlstd{landscape, infectionRates,}
        \hlkwc{col}\hlstd{=}\hlstr{"lightgreen"}\hlstd{,} \hlkwc{cex.lab}\hlstd{=}\hlnum{1.5}\hlstd{,} \hlkwc{cex.axis}\hlstd{=}\hlnum{1.3}\hlstd{,}
        \hlkwc{ylab}\hlstd{=}\hlstr{"Percent forest cover"}\hlstd{)}
\end{alltt}
\end{kframe}
\end{knitrout}
\vspace{-0.5cm}
\centering
\includegraphics[width=0.6\textwidth]{figure/boxfor1-1} \\
\vfill
Notice that the variances don't look equal among groups.
\end{frame}



% \begin{frame}[fragile]
%   \frametitle{Are group variances equal?}
%   \small
% <<bartlett,size='footnotesize'>>=
% bartlett.test(percentInfected~landscape, data=infectionRates)
% @
% \vfill
% {We reject the null hypothesis that the group variances are equal,
%   suggesting that we failed to meet the assumption of constant variance.}
% \end{frame}






\begin{frame}[fragile]
  \frametitle{Histogram of residuals}
\scriptsize
%\begin{center}
\begin{knitrout}
\definecolor{shadecolor}{rgb}{0.878, 0.918, 0.933}\color{fgcolor}\begin{kframe}
\begin{alltt}
\hlstd{resids} \hlkwb{<-} \hlkwd{resid}\hlstd{(anova1)}
\hlkwd{hist}\hlstd{(resids,} \hlkwc{col}\hlstd{=}\hlstr{"turquoise"}\hlstd{,} \hlkwc{breaks}\hlstd{=}\hlnum{10}\hlstd{,} \hlkwc{xlab}\hlstd{=}\hlstr{"Residuals"}\hlstd{)}
\end{alltt}
\end{kframe}
\end{knitrout}
%\end{center}
\centering
\includegraphics[width=0.6\textwidth]{figure/histresid0-1} \\
\vfill
The histogram doesn't look bad, but we need to test the normality
assumption.
\end{frame}





\begin{frame}[fragile]
  \frametitle{Normality test on residuals}
\begin{knitrout}
\definecolor{shadecolor}{rgb}{0.878, 0.918, 0.933}\color{fgcolor}\begin{kframe}
\begin{alltt}
\hlkwd{shapiro.test}\hlstd{(resids)}
\end{alltt}
\begin{verbatim}
## 
## 	Shapiro-Wilk normality test
## 
## data:  resids
## W = 0.95528, p-value = 0.003596
\end{verbatim}
\end{kframe}
\end{knitrout}
%\pause
\vspace{0.6cm}
%% {\bf Remember, this is not the correct test:}
%% <<>>=
%% shapiro.test(infectionRates$pcForest)
%% @
{We reject the null hypothesis that the residuals come from a normal
  distribution.
  \pause
  \vfill 
  Since we failed to meet the key assumption of ANOVA,
  we should consider transformations and/or nonparametric tests.
}  
\end{frame}









%\begin{comment}
\section{Transformations}




\begin{frame}[plain]
 \frametitle{Outline}
 \Large
 \tableofcontents[currentsection]
\end{frame}




\begin{frame}[fragile]
  \frametitle{Logarithmic Transformation}
  % \LARGE
  \small
  $y$ is the transformed response variable. $u$ is original response variable.
  \[
  y = \log(u + C)
  \]
  % \large
  \vspace{-5mm}
  \begin{itemize}%[<+->]
    \small %\normalsize
    \item The constant $C$ is often 1 if there are no zeros in the data ($u$)
    \item Useful when group variances are proportional to the means
  \end{itemize}
  \pause
%  <<log,fig.width=6,fig.height=6,fig.align='center',out.width="35%",size='scriptsize'>>=
\begin{knitrout}\scriptsize
\definecolor{shadecolor}{rgb}{0.878, 0.918, 0.933}\color{fgcolor}\begin{kframe}
\begin{alltt}
\hlkwd{boxplot}\hlstd{(}\hlkwd{log}\hlstd{(percentInfected)}\hlopt{~}\hlstd{landscape, infectionRates,} \hlkwc{col}\hlstd{=}\hlstr{"green"}\hlstd{,}
      \hlkwc{cex.lab}\hlstd{=}\hlnum{1.5}\hlstd{,} \hlkwc{cex.axis}\hlstd{=}\hlnum{1.3}\hlstd{,} \hlkwc{ylab}\hlstd{=}\hlstr{"log(percent forest cover)"}\hlstd{)}
\end{alltt}
\end{kframe}
\end{knitrout}
\includegraphics[width=0.45\textwidth]{figure/boxfor1-1} \hfill
\includegraphics[width=0.45\textwidth]{figure/log-1} \\
\end{frame}



%% \begin{frame}[fragile]
%%   \frametitle{Logarithmic Transformation}
%% <<loguy,fig=true,include=FALSE, width=12, height=6>>=
%% par(mfrow=c(1,2), mai=c(0.9, 0.9, 0.1, 0.1))
%% hist(u2, main="")
%% hist(log(u2), main="")
%% @
%% \includegraphics[width=\textwidth]{lab05-assump-nonpar-loguy}
%% \end{frame}



\begin{frame}[fragile]
  \frametitle{Square Root Transformation}
%  \LARGE
  \small
  \[
  y = \sqrt{u + C}
  \]
%  \large
  \vspace{-5mm}
  \begin{itemize}%[<+->]
    \small
    \item $C$ is often 0.5 or some other small number
    \item Useful when group variances are proportional to the means
%      (count data)
  \end{itemize}
  \pause
\begin{knitrout}\scriptsize
\definecolor{shadecolor}{rgb}{0.878, 0.918, 0.933}\color{fgcolor}\begin{kframe}
\begin{alltt}
\hlkwd{boxplot}\hlstd{(}\hlkwd{sqrt}\hlstd{(percentInfected)}\hlopt{~}\hlstd{landscape, infectionRates,} \hlkwc{col}\hlstd{=}\hlstr{"yellow"}\hlstd{,}
        \hlkwc{cex.lab}\hlstd{=}\hlnum{1.5}\hlstd{,} \hlkwc{cex.axis}\hlstd{=}\hlnum{1.3}\hlstd{,} \hlkwc{ylab}\hlstd{=}\hlstr{"sqrt(percent forest cover)"}\hlstd{)}
\end{alltt}
\end{kframe}
\end{knitrout}
\includegraphics[width=0.45\textwidth]{figure/boxfor1-1} \hfill
\includegraphics[width=0.45\textwidth]{figure/sqrt-1} \\
\end{frame}



%% \begin{frame}[fragile]
%%   \frametitle{Square Root Transformation}
%% <<sqrtuy,fig=true,include=FALSE, width=12, height=6>>=
%% par(mfrow=c(1,2), mai=c(0.9, 0.9, 0.1, 0.1))
%% hist(u2, main="")
%% hist(sqrt(u2), main="")
%% @
%% \includegraphics[width=\textwidth]{lab05-assump-nonpar-sqrtuy}
%% \end{frame}



\begin{frame}[fragile]
  \frametitle{Arcsine-square root Transformation}
%  \LARGE
  \small
  \[
  y = \mathrm{arcsin}(\sqrt{u})
  \]
%  \large
  \vspace{-5mm}
  \begin{itemize}%[<+->]
    \small
    \item Used on proportions.
    \item logit transformation is an alternative: $y = \log(\frac{u}{1-u})$
%    \item Binomial (logistic) regression is an alternative for proportions.
  \end{itemize}
  \pause
\begin{knitrout}\scriptsize
\definecolor{shadecolor}{rgb}{0.878, 0.918, 0.933}\color{fgcolor}\begin{kframe}
\begin{alltt}
\hlkwd{boxplot}\hlstd{(}\hlkwd{asin}\hlstd{(}\hlkwd{sqrt}\hlstd{(percentInfected))}\hlopt{~}\hlstd{landscape, infectionRates,} \hlkwc{col}\hlstd{=}\hlstr{"orange"}\hlstd{,}
        \hlkwc{cex.lab}\hlstd{=}\hlnum{1.5}\hlstd{,} \hlkwc{cex.axis}\hlstd{=}\hlnum{1.3}\hlstd{,} \hlkwc{ylab}\hlstd{=}\hlstr{"asin(sqrt(percent forest cover))"}\hlstd{)}
\end{alltt}
\end{kframe}
\end{knitrout}
\includegraphics[width=0.45\textwidth]{figure/boxfor1-1} \hfill
\includegraphics[width=0.45\textwidth]{figure/asin-1} \\
\end{frame}




\begin{frame}[fragile]
  \frametitle{Reciprocal Transformation}
%  \LARGE
  \small
  \[
  y = \frac{1}{u + C}
  \]
%  \large
  \vspace{-5mm}
  \begin{itemize}%[<+->]
    \small
    \item $C$ is often 1 but could be 0 if there are no zeros in $u$
    \item Useful when group SDs are proportional to the squared group means
  \end{itemize}
  \pause
\begin{knitrout}\scriptsize
\definecolor{shadecolor}{rgb}{0.878, 0.918, 0.933}\color{fgcolor}\begin{kframe}
\begin{alltt}
\hlkwd{boxplot}\hlstd{(}\hlnum{1}\hlopt{/}\hlstd{percentInfected}\hlopt{~}\hlstd{landscape, infectionRates,} \hlkwc{col}\hlstd{=}\hlstr{"pink"}\hlstd{,}
        \hlkwc{cex.lab}\hlstd{=}\hlnum{1.5}\hlstd{,} \hlkwc{cex.axis}\hlstd{=}\hlnum{1.3}\hlstd{,} \hlkwc{ylab}\hlstd{=}\hlstr{"1/percent forest cover"}\hlstd{)}
\end{alltt}
\end{kframe}
\end{knitrout}
\includegraphics[width=0.45\textwidth]{figure/boxfor1-1} \hfill
\includegraphics[width=0.45\textwidth]{figure/recip-1} \\
\end{frame}



\begin{frame}[fragile]
  \frametitle{ANOVA on transformed data}
  \small
  {Tranformation can be done in the \inr{aov} formula}
\begin{knitrout}\footnotesize
\definecolor{shadecolor}{rgb}{0.878, 0.918, 0.933}\color{fgcolor}\begin{kframe}
\begin{alltt}
\hlstd{anova2} \hlkwb{<-} \hlkwd{aov}\hlstd{(}\hlkwd{log}\hlstd{(percentInfected)}\hlopt{~}\hlstd{landscape,}
              \hlkwc{data}\hlstd{=infectionRates)}
\hlkwd{summary}\hlstd{(anova2)}
\end{alltt}
\begin{verbatim}
##             Df Sum Sq Mean Sq F value Pr(>F)    
## landscape    2  60.93   30.46   303.5 <2e-16 ***
## Residuals   87   8.73    0.10                   
## ---
## Signif. codes:  0 '***' 0.001 '**' 0.01 '*' 0.05 '.' 0.1 ' ' 1
\end{verbatim}
\end{kframe}
\end{knitrout}
  \pause
  \vfill
  {The log transformation didn't help much: We still reject the
    normality assumption.} 
\begin{knitrout}\footnotesize
\definecolor{shadecolor}{rgb}{0.878, 0.918, 0.933}\color{fgcolor}\begin{kframe}
\begin{alltt}
\hlkwd{shapiro.test}\hlstd{(}\hlkwd{resid}\hlstd{(anova2))}
\end{alltt}
\begin{verbatim}
## 
## 	Shapiro-Wilk normality test
## 
## data:  resid(anova2)
## W = 0.97092, p-value = 0.04106
\end{verbatim}
\end{kframe}
\end{knitrout}
\end{frame}




\section{Non-parametrics}




\begin{frame}[plain]
 \frametitle{Outline}
 \Large
 \tableofcontents[currentsection]
\end{frame}




\begin{frame}[fragile]
  \frametitle{Non-parametric Tests}
%  {\bf Mann-Whitney $U$-test}
  \large
  {Wilcoxan rank sum test}
  \begin{itemize}
    \item For 2 group comparisons
    \item a.k.a. the Mann-Whitney $U$ test
    \item \inr{wilcox.test}
  \end{itemize}
  \pause
  \vspace{0.5cm}
  {Kruskal-Wallis One-Way ANOVA}
  \begin{itemize}
    \item For testing differences in $>2$ groups
    \item \inr{kruskal.test}
  \end{itemize}
\pause
\vfill
\centering
These two functions can be used in almost the exact same way as
\inr{t.test} and \inr{aov}, respectively. \\
\end{frame}







%% \begin{frame}
%%   \frametitle{In-class assignment}
%%   \begin{itemize}
%%     \item[\bf (1)] Import the \inr{forestCover} data
%%     \item[\bf (2)] Decide which transformation do you think would be best?
%%     \item[\bf (3)] Conduct an ANOVA on the untransformed and transformed
%%       data. Use at least two of the following tranformations:
%%       \begin{itemize}
%%         \item log
%%         \item square-root
%%         \item acrsine square-root
%%         \item reciprocal
%%       \end{itemize}
%%     \item[\bf (4)] Determine which (if any) transformation is best for \inr{pcForest}
%%     \item[\bf (5)] Does transformation alter the main conclusion?
%%   \end{itemize}
%% \end{frame}




\begin{frame}
  \frametitle{Assignment}
  \small
%  \footnotesize
  \begin{enumerate}
    \item[\bf (1)] Decide which transformation is best for the
      \inr{infectionRates} data by conducting ANOVAs using the log,
      square-root, arcsine square-root, and reciprocal
      transformations. Use boxplots, historgrams, and Shapiro's tests
      to determine the best transformation.
      % following tranformations:
      % \begin{itemize}
      %   \item log
      %   \item square-root
      %   \item arcsine square-root
      %   \item reciprocal
      % \end{itemize}
    \item[\bf (2)] Does transformation alter the conclusion about the
      null hypothesis of no difference in means? If not, were the
      transformations necessary?
    \item[\bf (3)] Test the hypothesis that infection rates are equal
      between suburban and urban landscapes using a Wilcoxan rank sum
      test. What is the conclusion?
    \item[\bf (4)] Conduct a Kruskal-Wallis test on the data. What
      is the conclusion? % (in terms of the null hypothesis)?
  \end{enumerate}
%  \centering
  \vfill
  Use comments in your \R~script (or .Rmd file) to explain your
  answers. Upload your results to ELC at least one day before your
  next lab.\\ 
\end{frame}









\end{document}
