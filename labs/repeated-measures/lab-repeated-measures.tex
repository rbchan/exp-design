\documentclass[color=usenames,dvipsnames]{beamer}\usepackage[]{graphicx}\usepackage[]{color}
%% maxwidth is the original width if it is less than linewidth
%% otherwise use linewidth (to make sure the graphics do not exceed the margin)
\makeatletter
\def\maxwidth{ %
  \ifdim\Gin@nat@width>\linewidth
    \linewidth
  \else
    \Gin@nat@width
  \fi
}
\makeatother

\definecolor{fgcolor}{rgb}{0, 0, 0}
\newcommand{\hlnum}[1]{\textcolor[rgb]{0.69,0.494,0}{#1}}%
\newcommand{\hlstr}[1]{\textcolor[rgb]{0.749,0.012,0.012}{#1}}%
\newcommand{\hlcom}[1]{\textcolor[rgb]{0.514,0.506,0.514}{\textit{#1}}}%
\newcommand{\hlopt}[1]{\textcolor[rgb]{0,0,0}{#1}}%
\newcommand{\hlstd}[1]{\textcolor[rgb]{0,0,0}{#1}}%
\newcommand{\hlkwa}[1]{\textcolor[rgb]{0,0,0}{\textbf{#1}}}%
\newcommand{\hlkwb}[1]{\textcolor[rgb]{0,0.341,0.682}{#1}}%
\newcommand{\hlkwc}[1]{\textcolor[rgb]{0,0,0}{\textbf{#1}}}%
\newcommand{\hlkwd}[1]{\textcolor[rgb]{0.004,0.004,0.506}{#1}}%
\let\hlipl\hlkwb

\usepackage{framed}
\makeatletter
\newenvironment{kframe}{%
 \def\at@end@of@kframe{}%
 \ifinner\ifhmode%
  \def\at@end@of@kframe{\end{minipage}}%
  \begin{minipage}{\columnwidth}%
 \fi\fi%
 \def\FrameCommand##1{\hskip\@totalleftmargin \hskip-\fboxsep
 \colorbox{shadecolor}{##1}\hskip-\fboxsep
     % There is no \\@totalrightmargin, so:
     \hskip-\linewidth \hskip-\@totalleftmargin \hskip\columnwidth}%
 \MakeFramed {\advance\hsize-\width
   \@totalleftmargin\z@ \linewidth\hsize
   \@setminipage}}%
 {\par\unskip\endMakeFramed%
 \at@end@of@kframe}
\makeatother

\definecolor{shadecolor}{rgb}{.97, .97, .97}
\definecolor{messagecolor}{rgb}{0, 0, 0}
\definecolor{warningcolor}{rgb}{1, 0, 1}
\definecolor{errorcolor}{rgb}{1, 0, 0}
\newenvironment{knitrout}{}{} % an empty environment to be redefined in TeX

\usepackage{alltt}
%\documentclass[color=usenames,dvipsnames,handout]{beamer}



%\usepackage[roman]{../../lab1}
\usepackage[sans]{../../lab1}

\hypersetup{pdftex,pdfstartview=FitV}









%% New command for inline code that isn't to be evaluated
\definecolor{inlinecolor}{rgb}{0.878, 0.918, 0.933}
\newcommand{\inr}[1]{\colorbox{inlinecolor}{\texttt{#1}}}
\IfFileExists{upquote.sty}{\usepackage{upquote}}{}
\begin{document}


%\setlength\fboxsep{0pt}



\begin{frame}[plain]
  \huge %\LARGE
  \centering \par
  {\color{RoyalBlue}{Lab 10 -- Repeated Measures}} \\
  \vspace{1cm}
  \Large
  October 22 \& 23, 2018 \\
  FANR 6750 \\
  \vfill
  \large
  Richard Chandler and Bob Cooper
\end{frame}


\section{Intro}


\begin{frame}%[plain]
  \frametitle{Overview}
  \large
  {\bf Design}
  \begin{itemize}
    \item We randomly assign each ``subject'' to a treatment
    \item We record the response to the treatment over time
  \end{itemize}
  \pause
  \vspace{0.5cm}
  {\bf Sources of variation}
  \begin{itemize}
    \item Treatment
    \item Time
    \item Treatment-time interaction
    \item Random variation among subjects
    \item Random variation within subjects
  \end{itemize}
\end{frame}




\begin{frame}[fragile]
   \frametitle{Approaches}
   \Large
%   \only<1>{\tableofcontents}%[hideallsubsections]}
 %  \only<2 | handout:0>{\tableofcontents[currentsection]}%,hideallsubsections]}
%   \begin{itemize}
%     \item<1-> Univariate
%    \begin{itemize}
      \large
%      \item<1-> Univariate
      {\bf Univariate}
        \begin{itemize}
        \normalsize
          \item<1-> This is just a split-plot analysis with adjusted $p$-values
          \item<2-> Adjustments: Greenhouse-Geisser or Huynh-Feldt methods
          \item<3-> In \R, you must do a MANOVA to obtain these adjusted $P$-values
        \end{itemize}
%      \item[]
%     \item<6-> Multivariate
%       \begin{itemize}
%      \large
%      \item<4-> MANOVA
      \vfill
      \uncover<4->{{\bf MANOVA \\}}
        \begin{itemize}
          \large
          \item<5-> Testing based on Wilks' lambda or Pillai's test statistic
          \item<5-> This is usually followed by a profile analysis
        \end{itemize}
%      \item[]
%      \item<6-> Mixed effects model with (ARMA) correlation structure
      \vfill
      \uncover<6->{{\bf Mixed effects model with (ARMA) correlation
          structure \\}}
        \begin{itemize}
%        \large %\normalsize
        \item<7-> This can be done using \inr{lme}
        \item<7-> We might cover this later% in the course
      \end{itemize}
%    \end{itemize}
%   \end{itemize}
\end{frame}







\section{Univariate Split-plot Approach}



\begin{frame}[fragile]
  \frametitle{The plant data}
  \scriptsize
\begin{knitrout}\scriptsize
\definecolor{shadecolor}{rgb}{0.878, 0.918, 0.933}\color{fgcolor}\begin{kframe}
\begin{alltt}
\hlstd{plantData} \hlkwb{<-} \hlkwd{read.csv}\hlstd{(}\hlstr{"plantData.csv"}\hlstd{)}
\hlstd{plantData}\hlopt{$}\hlstd{plant} \hlkwb{<-} \hlkwd{factor}\hlstd{(plantData}\hlopt{$}\hlstd{plant)}
\hlstd{plantData}\hlopt{$}\hlstd{week} \hlkwb{<-} \hlkwd{factor}\hlstd{(plantData}\hlopt{$}\hlstd{week)}
\hlkwd{str}\hlstd{(plantData)}
\end{alltt}
\begin{verbatim}
## 'data.frame':	50 obs. of  4 variables:
##  $ plant     : Factor w/ 10 levels "1","2","3","4",..: 1 1 1 1 1 2 2 2 2 2 ...
##  $ fertilizer: Factor w/ 2 levels "H","L": 2 2 2 2 2 2 2 2 2 2 ...
##  $ week      : Factor w/ 5 levels "1","2","3","4",..: 1 2 3 4 5 1 2 3 4 5 ...
##  $ leaves    : int  4 5 6 8 10 3 4 6 6 9 ...
\end{verbatim}
\end{kframe}
\end{knitrout}
\pause
\begin{knitrout}\scriptsize
\definecolor{shadecolor}{rgb}{0.878, 0.918, 0.933}\color{fgcolor}\begin{kframe}
\begin{alltt}
\hlkwd{head}\hlstd{(plantData,} \hlkwc{n}\hlstd{=}\hlnum{8}\hlstd{)}
\end{alltt}
\begin{verbatim}
##   plant fertilizer week leaves
## 1     1          L    1      4
## 2     1          L    2      5
## 3     1          L    3      6
## 4     1          L    4      8
## 5     1          L    5     10
## 6     2          L    1      3
## 7     2          L    2      4
## 8     2          L    3      6
\end{verbatim}
\end{kframe}
\end{knitrout}
\end{frame}




\begin{frame}[fragile]
  \frametitle{Leaf Growth}
%  \vspace{-0.5cm}
  \begin{center}
\begin{knitrout}
\definecolor{shadecolor}{rgb}{0.878, 0.918, 0.933}\color{fgcolor}
\includegraphics[width=\maxwidth]{figure/plotDataLines-1} 

\end{knitrout}
\end{center}
\end{frame}


%\subsection{Split-plot}



\begin{frame}[fragile]
  \frametitle{Univariate split-plot approach}
  \scriptsize
\begin{knitrout}\scriptsize
\definecolor{shadecolor}{rgb}{0.878, 0.918, 0.933}\color{fgcolor}\begin{kframe}
\begin{alltt}
\hlstd{aov1} \hlkwb{<-} \hlkwd{aov}\hlstd{(leaves} \hlopt{~} \hlstd{fertilizer}\hlopt{*}\hlstd{week} \hlopt{+} \hlkwd{Error}\hlstd{(plant),}
            \hlkwc{data}\hlstd{=plantData)}
\end{alltt}
\end{kframe}
\end{knitrout}
%{\bf Actually, this is a nested and crossed example because there is no block}
\pause
\begin{knitrout}\scriptsize
\definecolor{shadecolor}{rgb}{0.878, 0.918, 0.933}\color{fgcolor}\begin{kframe}
\begin{alltt}
\hlkwd{summary}\hlstd{(aov1)}
\end{alltt}
\begin{verbatim}
## 
## Error: plant
##            Df Sum Sq Mean Sq F value Pr(>F)
## fertilizer  1  16.82   16.82   2.604  0.145
## Residuals   8  51.68    6.46               
## 
## Error: Within
##                 Df Sum Sq Mean Sq F value Pr(>F)    
## week             4 267.40   66.85 158.225 <2e-16 ***
## fertilizer:week  4   5.08    1.27   3.006 0.0326 *  
## Residuals       32  13.52    0.42                   
## ---
## Signif. codes:  0 '***' 0.001 '**' 0.01 '*' 0.05 '.' 0.1 ' ' 1
\end{verbatim}
\end{kframe}
\end{knitrout}
\pause
\vfill
\footnotesize
{%\centering
  %\bf %\large
%  In general, and especially if spherecity cannot be assumed,
  We need to adjust the $p$-values for the time and
  interaction effects \par
  \pause
  \vfill
  In {\bf R}, this requires reformatting the data and running a MANOVA \par}
\end{frame}











\begin{frame}[fragile]
  \frametitle{Format data for MANOVA}
%  {\bf \small To obtain corrected $P$-values, we must conduct a MANOVA}
  \footnotesize
\begin{knitrout}\footnotesize
\definecolor{shadecolor}{rgb}{0.878, 0.918, 0.933}\color{fgcolor}\begin{kframe}
\begin{alltt}
\hlstd{plantData2} \hlkwb{<-} \hlkwd{reshape}\hlstd{(plantData,} \hlkwc{idvar}\hlstd{=}\hlstr{"plant"}\hlstd{,}
                      \hlkwc{timevar}\hlstd{=}\hlstr{"week"}\hlstd{,} \hlkwc{v.names}\hlstd{=}\hlstr{"leaves"}\hlstd{,}
                      \hlkwc{direction}\hlstd{=}\hlstr{"wide"}\hlstd{)}
\end{alltt}
\end{kframe}
\end{knitrout}
\pause
\begin{knitrout}\scriptsize
\definecolor{shadecolor}{rgb}{0.878, 0.918, 0.933}\color{fgcolor}\begin{kframe}
\begin{alltt}
\hlstd{plantData2}
\end{alltt}
\begin{verbatim}
##    plant fertilizer leaves.1 leaves.2 leaves.3 leaves.4 leaves.5
## 1      1          L        4        5        6        8       10
## 6      2          L        3        4        6        6        9
## 11     3          L        6        7        9       10       12
## 16     4          L        5        7        8       10       12
## 21     5          L        5        6        7        8       10
## 26     6          H        4        6        9        9       11
## 31     7          H        3        5        7       10       12
## 36     8          H        6        8       11       10       14
## 41     9          H        5        7        9       10       12
## 46    10          H        5        8        9       11       11
\end{verbatim}
\end{kframe}
\end{knitrout}
\end{frame}







\begin{frame}[fragile]
  \frametitle{MANOVA and adjusted $P$-values}
  \scriptsize %\small
\begin{knitrout}\scriptsize
\definecolor{shadecolor}{rgb}{0.878, 0.918, 0.933}\color{fgcolor}\begin{kframe}
\begin{alltt}
\hlstd{manova1} \hlkwb{<-} \hlkwd{manova}\hlstd{(}\hlkwd{cbind}\hlstd{(leaves.1, leaves.2, leaves.3,}
                        \hlstd{leaves.4, leaves.5)} \hlopt{~} \hlstd{fertilizer,}
                  \hlkwc{data}\hlstd{=plantData2)}
\end{alltt}
\end{kframe}
\end{knitrout}
\pause
\begin{knitrout}\scriptsize
\definecolor{shadecolor}{rgb}{0.878, 0.918, 0.933}\color{fgcolor}\begin{kframe}
\begin{alltt}
\hlkwd{anova}\hlstd{(manova1,} \hlkwc{X}\hlstd{=}\hlopt{~}\hlnum{1}\hlstd{,} \hlkwc{test}\hlstd{=}\hlstr{"Spherical"}\hlstd{)}
\end{alltt}
\begin{verbatim}
## Analysis of Variance Table
## 
## 
## Contrasts orthogonal to
## ~1
## 
## Greenhouse-Geisser epsilon: 0.5882
## Huynh-Feldt epsilon:        0.8490
## 
##             Df        F num Df den Df   Pr(>F)   G-G Pr  H-F Pr
## (Intercept)  1 158.2249      4     32 0.000000 0.000000 0.00000
## fertilizer   1   3.0059      4     32 0.032613 0.066622 0.04224
## Residuals    8
\end{verbatim}
\end{kframe}
\end{knitrout}
\pause
\footnotesize
{%\bf
  The last 3 columns are $p$-values corresponding to the effects of time
  (Intercept) and interaction (fertilizer).
%  I suggest using the H-F $p$-values.
  No adjustment is necessary
  for the main effect so you can use the $p$-value from \inr{aov}. 
  \par}
\end{frame}






\begin{frame}[fragile]
  \frametitle{An aside -- sphericity}
  \small
  {Technically, adjusted $p$-values and MANOVA aren't necessary if the
    assumption of sphericity holds. \alert{However}, we recommend
    doing the adjustments (or the MANOVA) anyway because the test of
    sphericity has low power. \\}
  \pause
  \vfill
  Sphericity is the multivariate analogue of the homogeneity of variance assumption of ANOVA.
  \pause
  \vfill
  Here is how you test the assumption:
  \footnotesize
\begin{knitrout}\footnotesize
\definecolor{shadecolor}{rgb}{0.878, 0.918, 0.933}\color{fgcolor}\begin{kframe}
\begin{alltt}
\hlkwd{mauchly.test}\hlstd{(manova1,} \hlkwc{X}\hlstd{=}\hlopt{~}\hlnum{1}\hlstd{)}
\end{alltt}
\begin{verbatim}
## 
## 	Mauchly's test of sphericity
## 	Contrasts orthogonal to
## 	~1
## 
## 
## data:  SSD matrix from manova(cbind(leaves.1, leaves.2, leaves.3, leaves.4, leaves.5) ~  SSD matrix from     fertilizer, data = plantData2)
## W = 0.099297, p-value = 0.1062
\end{verbatim}
\end{kframe}
\end{knitrout}
\pause
\vfill
\small %\normalsize
{\centering We fail to reject the null hypothesis, so sphericity
  can be assumed. \\}
%  This tells us that the adjusted
%  $p$-values are legitimate. \par }
\end{frame}








\begin{frame}[fragile]
  \frametitle{Compare {\tt aov} and {\tt manova} results}
  \tiny
\begin{knitrout}\tiny
\definecolor{shadecolor}{rgb}{0.878, 0.918, 0.933}\color{fgcolor}\begin{kframe}
\begin{alltt}
\hlkwd{summary}\hlstd{(aov1)}
\end{alltt}
\begin{verbatim}
## 
## Error: plant
##            Df Sum Sq Mean Sq F value Pr(>F)
## fertilizer  1  16.82   16.82   2.604  0.145
## Residuals   8  51.68    6.46               
## 
## Error: Within
##                 Df Sum Sq Mean Sq F value Pr(>F)    
## week             4 267.40   66.85 158.225 <2e-16 ***
## fertilizer:week  4   5.08    1.27   3.006 0.0326 *  
## Residuals       32  13.52    0.42                   
## ---
## Signif. codes:  0 '***' 0.001 '**' 0.01 '*' 0.05 '.' 0.1 ' ' 1
\end{verbatim}
\begin{alltt}
\hlkwd{anova}\hlstd{(manova1,} \hlkwc{X}\hlstd{=}\hlopt{~}\hlnum{1}\hlstd{,} \hlkwc{test}\hlstd{=}\hlstr{"Spherical"}\hlstd{)}
\end{alltt}
\begin{verbatim}
## Analysis of Variance Table
## 
## 
## Contrasts orthogonal to
## ~1
## 
## Greenhouse-Geisser epsilon: 0.5882
## Huynh-Feldt epsilon:        0.8490
## 
##             Df        F num Df den Df   Pr(>F)   G-G Pr  H-F Pr
## (Intercept)  1 158.2249      4     32 0.000000 0.000000 0.00000
## fertilizer   1   3.0059      4     32 0.032613 0.066622 0.04224
## Residuals    8
\end{verbatim}
\end{kframe}
\end{knitrout}
\end{frame}











\section{Multivariate}




\begin{frame}[fragile]
  \frametitle{Multivariate tests}
  \small
  {%\bf
    An alternative to the adjusted
    $p$-value approach is to do a multivariate analysis relaxing the
    assumptions about the structure of the variance-covariance
    matrix. We're already most of the way there. \par}
  \pause
  \vspace{0.5cm}
  {%\bf
    Wilks' lambda}
  \footnotesize
\begin{knitrout}\footnotesize
\definecolor{shadecolor}{rgb}{0.878, 0.918, 0.933}\color{fgcolor}\begin{kframe}
\begin{alltt}
\hlkwd{anova}\hlstd{(manova1,} \hlkwc{X}\hlstd{=}\hlopt{~}\hlnum{1}\hlstd{,} \hlkwc{test}\hlstd{=}\hlstr{"Wilks"}\hlstd{)}
\end{alltt}
\begin{verbatim}
## Analysis of Variance Table
## 
## 
## Contrasts orthogonal to
## ~1
## 
##             Df    Wilks approx F num Df den Df    Pr(>F)    
## (Intercept)  1 0.008487  146.042      4      5 2.308e-05 ***
## fertilizer   1 0.144772    7.384      4      5   0.02503 *  
## Residuals    8                                              
## ---
## Signif. codes:  0 '***' 0.001 '**' 0.01 '*' 0.05 '.' 0.1 ' ' 1
\end{verbatim}
\end{kframe}
\end{knitrout}
\pause
{%\bf
  As before, we conclude that the effect of fertilizer changes
  over time} \\. This test is less
\\  powerful than ANOVA, as evidenced by the $p$-value $>0.05$.}
\end{frame}




\begin{frame}[fragile]
  \frametitle{Multivariate tests}
  \small
  {%\bf
    Pillai's trace is an alternative to Wilks' lambda. In this
    case, it returns the same $p$-values as Wilks' test.}
  \vfill
  \footnotesize
\begin{knitrout}\footnotesize
\definecolor{shadecolor}{rgb}{0.878, 0.918, 0.933}\color{fgcolor}\begin{kframe}
\begin{alltt}
\hlkwd{anova}\hlstd{(manova1,} \hlkwc{X}\hlstd{=}\hlopt{~}\hlnum{1}\hlstd{,} \hlkwc{test}\hlstd{=}\hlstr{"Pillai"}\hlstd{)}
\end{alltt}
\begin{verbatim}
## Analysis of Variance Table
## 
## 
## Contrasts orthogonal to
## ~1
## 
##             Df  Pillai approx F num Df den Df    Pr(>F)    
## (Intercept)  1 0.99151  146.042      4      5 2.308e-05 ***
## fertilizer   1 0.85523    7.384      4      5   0.02503 *  
## Residuals    8                                             
## ---
## Signif. codes:  0 '***' 0.001 '**' 0.01 '*' 0.05 '.' 0.1 ' ' 1
\end{verbatim}
\end{kframe}
\end{knitrout}

\end{frame}








%\section{Profile analysis}




\begin{frame}[fragile]
  \frametitle{Profile analysis}
  {%\bf
    Profile analysis requires calculating the differences (ie, the
    number of leaves grown each week).}
  \vspace{1cm}
  \footnotesize
\begin{knitrout}\footnotesize
\definecolor{shadecolor}{rgb}{0.878, 0.918, 0.933}\color{fgcolor}\begin{kframe}
\begin{alltt}
\hlstd{manova2} \hlkwb{<-} \hlkwd{manova}\hlstd{(}
    \hlkwd{cbind}\hlstd{(leaves.2}\hlopt{-}\hlstd{leaves.1, leaves.3}\hlopt{-}\hlstd{leaves.2,}
          \hlstd{leaves.4}\hlopt{-}\hlstd{leaves.3, leaves.5}\hlopt{-}\hlstd{leaves.4)} \hlopt{~}
    \hlstd{fertilizer,} \hlkwc{data}\hlstd{=plantData2)}
\end{alltt}
\end{kframe}
\end{knitrout}
\end{frame}






\begin{frame}[fragile]
  \frametitle{Profile Analysis}
  \footnotesize
  {%\bf
    During which intervals do the growth rates differ?}
  \pause
  \scriptsize %\tiny
\begin{knitrout}\scriptsize
\definecolor{shadecolor}{rgb}{0.878, 0.918, 0.933}\color{fgcolor}\begin{kframe}
\begin{alltt}
\hlkwd{summary.aov}\hlstd{(manova2)}
\end{alltt}
\begin{verbatim}
##  Response 1 :
##             Df Sum Sq Mean Sq F value  Pr(>F)   
## fertilizer   1    2.5     2.5    12.5 0.00767 **
## Residuals    8    1.6     0.2                   
## ---
## Signif. codes:  0 '***' 0.001 '**' 0.01 '*' 0.05 '.' 0.1 ' ' 1
## 
##  Response 2 :
##             Df Sum Sq Mean Sq F value Pr(>F)
## fertilizer   1    1.6     1.6     3.2 0.1114
## Residuals    8    4.0     0.5               
## 
##  Response 3 :
##             Df Sum Sq Mean Sq F value Pr(>F)
## fertilizer   1    0.1     0.1  0.0625 0.8089
## Residuals    8   12.8     1.6               
## 
##  Response 4 :
##             Df Sum Sq Mean Sq F value Pr(>F)
## fertilizer   1    0.1     0.1  0.0909 0.7707
## Residuals    8    8.8     1.1
\end{verbatim}
\end{kframe}
\end{knitrout}
\end{frame}





\begin{frame}[fragile]
  \frametitle{Plot the growth rates}
  \small
  {%\bf
    Calculate mean growth rate for each time interval}%, for each fertilizer}
  \pause
\begin{knitrout}\small
\definecolor{shadecolor}{rgb}{0.878, 0.918, 0.933}\color{fgcolor}\begin{kframe}
\begin{alltt}
\hlstd{leavesMat} \hlkwb{<-} \hlstd{plantData2[,}\hlnum{3}\hlopt{:}\hlnum{7}\hlstd{]}
\hlstd{growthMat} \hlkwb{<-} \hlstd{leavesMat[,}\hlnum{2}\hlopt{:}\hlnum{5}\hlstd{]} \hlopt{-} \hlstd{leavesMat[,}\hlnum{1}\hlopt{:}\hlnum{4}\hlstd{]}
\hlkwd{colnames}\hlstd{(growthMat)} \hlkwb{<-} \hlkwd{paste}\hlstd{(}\hlstr{"interval"}\hlstd{,} \hlnum{1}\hlopt{:}\hlnum{4}\hlstd{,} \hlkwc{sep}\hlstd{=}\hlstr{"."}\hlstd{)}
\hlstd{(lowFertilizer} \hlkwb{<-} \hlkwd{colMeans}\hlstd{(growthMat[}\hlnum{1}\hlopt{:}\hlnum{5}\hlstd{,]))}
\end{alltt}
\begin{verbatim}
## interval.1 interval.2 interval.3 interval.4 
##        1.2        1.4        1.2        2.2
\end{verbatim}
\begin{alltt}
\hlstd{(highFertilizer} \hlkwb{<-} \hlkwd{colMeans}\hlstd{(growthMat[}\hlnum{6}\hlopt{:}\hlnum{10}\hlstd{,]))}
\end{alltt}
\begin{verbatim}
## interval.1 interval.2 interval.3 interval.4 
##        2.2        2.2        1.0        2.0
\end{verbatim}
\end{kframe}
\end{knitrout}
\pause
{\bf Calculate the standard errors for these growth rates}
\footnotesize
\begin{knitrout}\footnotesize
\definecolor{shadecolor}{rgb}{0.878, 0.918, 0.933}\color{fgcolor}\begin{kframe}
\begin{alltt}
\hlstd{SE} \hlkwb{<-} \hlkwd{sqrt}\hlstd{(}\hlkwd{diag}\hlstd{(}\hlkwd{vcov}\hlstd{(manova2)))}
\hlstd{SE} \hlkwb{<-} \hlstd{SE[}\hlkwd{names}\hlstd{(SE)}\hlopt{==}\hlstr{":(Intercept)"}\hlstd{]} \hlcom{# Only use "intercept" SEs}
\hlkwd{unname}\hlstd{(SE)} \hlcom{## Ignore the names}
\end{alltt}
\begin{verbatim}
## [1] 0.2000000 0.3162278 0.5656854 0.4690416
\end{verbatim}
\end{kframe}
\end{knitrout}
\end{frame}


\begin{frame}[fragile]
  \frametitle{Plot the growth rates}
\begin{knitrout}\small
\definecolor{shadecolor}{rgb}{0.878, 0.918, 0.933}\color{fgcolor}\begin{kframe}
\begin{alltt}
\hlkwd{plot}\hlstd{(}\hlnum{1}\hlopt{:}\hlnum{4}\hlopt{-}\hlnum{0.05}\hlstd{, lowFertilizer,} \hlkwc{type}\hlstd{=}\hlstr{"b"}\hlstd{,} \hlkwc{xlim}\hlstd{=}\hlkwd{c}\hlstd{(}\hlnum{0.9}\hlstd{,} \hlnum{4.1}\hlstd{),}
     \hlkwc{ylim}\hlstd{=}\hlkwd{c}\hlstd{(}\hlopt{-}\hlnum{1}\hlstd{,} \hlnum{4}\hlstd{),} \hlkwc{xaxp}\hlstd{=}\hlkwd{c}\hlstd{(}\hlnum{1}\hlstd{,}\hlnum{4}\hlstd{,}\hlnum{3}\hlstd{),} \hlkwc{cex.lab}\hlstd{=}\hlnum{1.5}\hlstd{,}
     \hlkwc{xlab}\hlstd{=}\hlstr{"Time interval"}\hlstd{,} \hlkwc{ylab}\hlstd{=}\hlstr{"Growth rate (leaves/week)"}\hlstd{)}
\hlkwd{abline}\hlstd{(}\hlkwc{h}\hlstd{=}\hlnum{0}\hlstd{,} \hlkwc{lty}\hlstd{=}\hlnum{3}\hlstd{)}
\hlkwd{arrows}\hlstd{(}\hlnum{1}\hlopt{:}\hlnum{4}\hlopt{-}\hlnum{.05}\hlstd{, lowFertilizer}\hlopt{-}\hlstd{SE,} \hlnum{1}\hlopt{:}\hlnum{4}\hlopt{-}\hlnum{.05}\hlstd{, lowFertilizer}\hlopt{+}\hlstd{SE,}
       \hlkwc{angle}\hlstd{=}\hlnum{90}\hlstd{,} \hlkwc{code}\hlstd{=}\hlnum{3}\hlstd{,} \hlkwc{length}\hlstd{=}\hlnum{0.05}\hlstd{)}
\hlkwd{lines}\hlstd{(}\hlnum{1}\hlopt{:}\hlnum{4}\hlopt{+}\hlnum{0.05}\hlstd{, highFertilizer,} \hlkwc{type}\hlstd{=}\hlstr{"b"}\hlstd{,} \hlkwc{pch}\hlstd{=}\hlnum{17}\hlstd{,} \hlkwc{col}\hlstd{=}\hlnum{4}\hlstd{)}
\hlkwd{arrows}\hlstd{(}\hlnum{1}\hlopt{:}\hlnum{4}\hlopt{+}\hlnum{0.05}\hlstd{, highFertilizer}\hlopt{-}\hlstd{SE,} \hlnum{1}\hlopt{:}\hlnum{4}\hlopt{+}\hlnum{0.05}\hlstd{,}
       \hlstd{highFertilizer}\hlopt{+}\hlstd{SE,} \hlkwc{angle}\hlstd{=}\hlnum{90}\hlstd{,} \hlkwc{code}\hlstd{=}\hlnum{3}\hlstd{,} \hlkwc{length}\hlstd{=}\hlnum{0.05}\hlstd{)}
\hlkwd{legend}\hlstd{(}\hlnum{1}\hlstd{,} \hlnum{4}\hlstd{,} \hlkwd{c}\hlstd{(}\hlstr{"Low fertilizer"}\hlstd{,} \hlstr{"High fertilizer"}\hlstd{),}
       \hlkwc{col}\hlstd{=}\hlkwd{c}\hlstd{(}\hlstr{"black"}\hlstd{,} \hlstr{"blue"}\hlstd{),} \hlkwc{pch}\hlstd{=}\hlkwd{c}\hlstd{(}\hlnum{1}\hlstd{,}\hlnum{17}\hlstd{))}
\end{alltt}
\end{kframe}
\end{knitrout}
\end{frame}






\begin{frame}[fragile]
  \frametitle{Plot the growth rates}
  \tiny
  \centering
    \includegraphics[width=\textwidth]{figure/gr-1} \\
\end{frame}





\begin{frame}
  \frametitle{Assignment}
  \footnotesize
  A researcher wants to assess the effects of density on the growth
  of the dark toadfish ({\it Neophrynichthys latus}). One fish is
  placed in each of 15 fish tanks that vary with respect to
  density. The weight of one ``focal fish'' per tank  is recorded on 6
  consecutive weeks. Five tanks have low density (1 conspecific), 5
  tanks have medium density (5 conspecifics), and 5 tanks have high
  density (10 conspecifics). The data are in the file {\tt fish.csv}.
  \pause
  \vfill
  \begin{enumerate}[\bf (1)]
    \item Conduct a repeated measures ANOVA using \inr{aov}. Calculate the adjusted $p$-values using the Huynh-Feldt
      method. Does the effect of density change over time?
    \item Use the Wilks' lambda statistic to test if the effect of
      density changes over time. What is your conclusion?
    \item Conduct a profile analysis. In which time intervals is the
      effect of density on growth rate significant?
  \end{enumerate}
  %% Conduct a repeated measures ANOVA to determine if the difference in
  %% home range size of male and female hawks changes over time.
  %% \begin{itemize}
  %%   \item Conduct the univariate analysis use the adjusted P-values
  %%   \item Conduct the multivariate analysis and Wilks' statistic
  %%   \item
  %% \end{itemize}
\end{frame}






\end{document}







