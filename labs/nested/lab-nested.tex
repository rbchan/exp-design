\documentclass[color=usenames,dvipsnames]{beamer}\usepackage[]{graphicx}\usepackage[]{color}
%% maxwidth is the original width if it is less than linewidth
%% otherwise use linewidth (to make sure the graphics do not exceed the margin)
\makeatletter
\def\maxwidth{ %
  \ifdim\Gin@nat@width>\linewidth
    \linewidth
  \else
    \Gin@nat@width
  \fi
}
\makeatother

\definecolor{fgcolor}{rgb}{0, 0, 0}
\newcommand{\hlnum}[1]{\textcolor[rgb]{0.69,0.494,0}{#1}}%
\newcommand{\hlstr}[1]{\textcolor[rgb]{0.749,0.012,0.012}{#1}}%
\newcommand{\hlcom}[1]{\textcolor[rgb]{0.514,0.506,0.514}{\textit{#1}}}%
\newcommand{\hlopt}[1]{\textcolor[rgb]{0,0,0}{#1}}%
\newcommand{\hlstd}[1]{\textcolor[rgb]{0,0,0}{#1}}%
\newcommand{\hlkwa}[1]{\textcolor[rgb]{0,0,0}{\textbf{#1}}}%
\newcommand{\hlkwb}[1]{\textcolor[rgb]{0,0.341,0.682}{#1}}%
\newcommand{\hlkwc}[1]{\textcolor[rgb]{0,0,0}{\textbf{#1}}}%
\newcommand{\hlkwd}[1]{\textcolor[rgb]{0.004,0.004,0.506}{#1}}%
\let\hlipl\hlkwb

\usepackage{framed}
\makeatletter
\newenvironment{kframe}{%
 \def\at@end@of@kframe{}%
 \ifinner\ifhmode%
  \def\at@end@of@kframe{\end{minipage}}%
  \begin{minipage}{\columnwidth}%
 \fi\fi%
 \def\FrameCommand##1{\hskip\@totalleftmargin \hskip-\fboxsep
 \colorbox{shadecolor}{##1}\hskip-\fboxsep
     % There is no \\@totalrightmargin, so:
     \hskip-\linewidth \hskip-\@totalleftmargin \hskip\columnwidth}%
 \MakeFramed {\advance\hsize-\width
   \@totalleftmargin\z@ \linewidth\hsize
   \@setminipage}}%
 {\par\unskip\endMakeFramed%
 \at@end@of@kframe}
\makeatother

\definecolor{shadecolor}{rgb}{.97, .97, .97}
\definecolor{messagecolor}{rgb}{0, 0, 0}
\definecolor{warningcolor}{rgb}{1, 0, 1}
\definecolor{errorcolor}{rgb}{1, 0, 0}
\newenvironment{knitrout}{}{} % an empty environment to be redefined in TeX

\usepackage{alltt}
%\documentclass[color=usenames,dvipsnames,handout]{beamer}



%\usepackage[roman]{../../lab1}
\usepackage[sans]{../../lab1}

\hypersetup{pdftex,pdfstartview=FitV}









%% New command for inline code that isn't to be evaluated
\definecolor{inlinecolor}{rgb}{0.878, 0.918, 0.933}
\newcommand{\inr}[1]{\colorbox{inlinecolor}{\texttt{#1}}}
\IfFileExists{upquote.sty}{\usepackage{upquote}}{}
\begin{document}


%\setlength\fboxsep{0pt}



\begin{frame}[plain]
  \LARGE
  \centering
  {\color{RoyalBlue}{\bf Lab 8 -- Nested ANOVA} \\}
  \vspace{1cm}
  \Large
  October 8 \& 9, 2017 \\
  FANR 6750 \\
  \vfill
  \large
  Richard Chandler and Bob Cooper
\end{frame}







\section{Overview}



\begin{frame}
  \frametitle{Outline}
  \Large
  \only<1>{\tableofcontents}%[hideallsubsections]}
%  \only<2 | handout:0>{\tableofcontents[currentsection]}%,hideallsubsections]}
\end{frame}




\begin{frame}
  \frametitle{Scenario}
  \large
  We subsample each experimental unit \\
  \pause
  \vfill
  For example
      \begin{itemize}
        \large
        \item We count larvae at multiple subplots within a plot
        \item We weigh multiple chicks in a brood
      \end{itemize}
  \pause
  \vfill
  We're interested in treatment effects at the experimental (whole) unit
  level, not the subunit level
\end{frame}




\begin{frame}
  \frametitle{The additive model}
  \large
\[
y_{ijk} = \mu + \alpha_i + \beta_{ij} + \varepsilon_{ijk}
\] \\
\vspace{1cm}
\pause
\large
Because we want our inferences to apply to all experimental units, not
just the ones in our sample, $\beta_{ij}$ is random.
\pause
\vfill
Specifically:
\[
\beta_{ij} \sim \mbox{Normal}(0, \sigma^2_B)
\]

\pause
\large
And as always,
\[
\varepsilon_{ijk} \sim \mbox{Normal}(0, \sigma^2)
\]
\end{frame}


\begin{frame}[fragile]
  \frametitle{Hypotheses}
{\bf Treatment effects \\}
$H_0: \alpha_1 = \cdots = \alpha_a = 0$ \\
$H_a:$ at least one inequality \\
\pause
\vspace{0.5cm}
{\bf Random variation among experimental units \\}
$H_0: \sigma^2_B = 0$ \\
$H_a: \sigma^2_B > 0$
\end{frame}



\begin{frame}[fragile]
  \frametitle{Example data}

\small
{\bf Import data}
\begin{knitrout}\small
\definecolor{shadecolor}{rgb}{0.878, 0.918, 0.933}\color{fgcolor}\begin{kframe}
\begin{alltt}
\hlstd{gypsyData} \hlkwb{<-} \hlkwd{read.csv}\hlstd{(}\hlstr{"gypsyData.csv"}\hlstd{)}
\hlkwd{str}\hlstd{(gypsyData)}
\end{alltt}
\begin{verbatim}
## 'data.frame':	36 obs. of  3 variables:
##  $ larvae   : num  16 16 15.8 14.2 13.9 14.2 13.5 13.4 14 13.1 ...
##  $ Treatment: Factor w/ 3 levels "Bt","Control",..: 1 1 1 1 1 1 1 1 1 1 ...
##  $ Plot     : int  1 1 1 1 2 2 2 2 3 3 ...
\end{verbatim}
\end{kframe}
\end{knitrout}
\pause
\vfill
{\bf Convert {\tt Plot} to a factor and then cross-tabulate}
\begin{knitrout}\small
\definecolor{shadecolor}{rgb}{0.878, 0.918, 0.933}\color{fgcolor}\begin{kframe}
\begin{alltt}
\hlstd{gypsyData}\hlopt{$}\hlstd{Plot} \hlkwb{<-} \hlkwd{factor}\hlstd{(gypsyData}\hlopt{$}\hlstd{Plot)}
\hlkwd{table}\hlstd{(gypsyData}\hlopt{$}\hlstd{Treatment, gypsyData}\hlopt{$}\hlstd{Plot)}
\end{alltt}
\begin{verbatim}
##          
##           1 2 3 4 5 6 7 8 9
##   Bt      4 4 4 0 0 0 0 0 0
##   Control 0 0 0 4 4 4 0 0 0
##   Dimilin 0 0 0 0 0 0 4 4 4
\end{verbatim}
\end{kframe}
\end{knitrout}
\end{frame}



\section{Using {\tt aov}}



\begin{frame}[fragile]
  \frametitle{Incorrect analysis}
\begin{knitrout}\small
\definecolor{shadecolor}{rgb}{0.878, 0.918, 0.933}\color{fgcolor}\begin{kframe}
\begin{alltt}
\hlstd{aov.wrong} \hlkwb{<-} \hlkwd{aov}\hlstd{(larvae} \hlopt{~} \hlstd{Treatment} \hlopt{+} \hlstd{Plot,}
                 \hlkwc{data}\hlstd{=gypsyData)}
\end{alltt}
\end{kframe}
\end{knitrout}
\pause
\begin{knitrout}\small
\definecolor{shadecolor}{rgb}{0.878, 0.918, 0.933}\color{fgcolor}\begin{kframe}
\begin{alltt}
\hlkwd{summary}\hlstd{(aov.wrong)}
\end{alltt}
\begin{verbatim}
##             Df Sum Sq Mean Sq F value Pr(>F)    
## Treatment    2 215.39  107.69  208.89 <2e-16 ***
## Plot         6  11.17    1.86    3.61 0.0093 ** 
## Residuals   27  13.92    0.52                   
## ---
## Signif. codes:  0 '***' 0.001 '**' 0.01 '*' 0.05 '.' 0.1 ' ' 1
\end{verbatim}
\end{kframe}
\end{knitrout}
\pause
\vfill
\centering
\large
\bf
\alert{The denominator degrees-of-freedom are wrong} \\
\end{frame}




\begin{frame}[fragile]
  \frametitle{Correct analysis}
\begin{knitrout}\small
\definecolor{shadecolor}{rgb}{0.878, 0.918, 0.933}\color{fgcolor}\begin{kframe}
\begin{alltt}
\hlstd{aov.correct} \hlkwb{<-} \hlkwd{aov}\hlstd{(larvae} \hlopt{~} \hlstd{Treatment} \hlopt{+} \hlkwd{Error}\hlstd{(Plot),}
                   \hlkwc{data}\hlstd{=gypsyData)}
\end{alltt}
\end{kframe}
\end{knitrout}
\pause
\begin{knitrout}\small
\definecolor{shadecolor}{rgb}{0.878, 0.918, 0.933}\color{fgcolor}\begin{kframe}
\begin{alltt}
\hlkwd{summary}\hlstd{(aov.correct)}
\end{alltt}
\begin{verbatim}
## 
## Error: Plot
##           Df Sum Sq Mean Sq F value  Pr(>F)    
## Treatment  2 215.39  107.69   57.87 0.00012 ***
## Residuals  6  11.17    1.86                    
## ---
## Signif. codes:  0 '***' 0.001 '**' 0.01 '*' 0.05 '.' 0.1 ' ' 1
## 
## Error: Within
##           Df Sum Sq Mean Sq F value Pr(>F)
## Residuals 27  13.92  0.5156
\end{verbatim}
\end{kframe}
\end{knitrout}
\end{frame}



\begin{frame}[fragile]
  \frametitle{What happens if we analyze plot-level means?}
  The \inr{aggregate} function is similar to \inr{tapply} but it
  works on entire {\tt data.frame}s. Here we get averages for each whole plot.
  \footnotesize
\begin{knitrout}
\definecolor{shadecolor}{rgb}{0.878, 0.918, 0.933}\color{fgcolor}\begin{kframe}
\begin{alltt}
\hlstd{plotData} \hlkwb{<-} \hlkwd{aggregate}\hlstd{(}\hlkwc{formula}\hlstd{=larvae} \hlopt{~} \hlstd{Treatment} \hlopt{+} \hlstd{Plot,}
                      \hlkwc{data}\hlstd{=gypsyData,} \hlkwc{FUN}\hlstd{=mean)}
\end{alltt}
\end{kframe}
\end{knitrout}
\pause
\begin{knitrout}
\definecolor{shadecolor}{rgb}{0.878, 0.918, 0.933}\color{fgcolor}\begin{kframe}
\begin{alltt}
\hlstd{plotData}
\end{alltt}
\begin{verbatim}
##   Treatment Plot larvae
## 1        Bt    1  15.50
## 2        Bt    2  13.75
## 3        Bt    3  14.00
## 4   Control    4  18.25
## 5   Control    5  18.75
## 6   Control    6  19.25
## 7   Dimilin    7  12.50
## 8   Dimilin    8  13.50
## 9   Dimilin    9  13.00
\end{verbatim}
\end{kframe}
\end{knitrout}
\end{frame}



\begin{frame}[fragile]
  \frametitle{$F$ and $p$ values are the same as before}
\begin{knitrout}\scriptsize
\definecolor{shadecolor}{rgb}{0.878, 0.918, 0.933}\color{fgcolor}\begin{kframe}
\begin{alltt}
\hlstd{aov.plot} \hlkwb{<-} \hlkwd{aov}\hlstd{(larvae} \hlopt{~} \hlstd{Treatment,} \hlkwc{data}\hlstd{=plotData)}
\hlkwd{summary}\hlstd{(aov.plot)}
\end{alltt}
\begin{verbatim}
##             Df Sum Sq Mean Sq F value  Pr(>F)    
## Treatment    2  53.85  26.924   57.87 0.00012 ***
## Residuals    6   2.79   0.465                    
## ---
## Signif. codes:  0 '***' 0.001 '**' 0.01 '*' 0.05 '.' 0.1 ' ' 1
\end{verbatim}
\end{kframe}
\end{knitrout}
\pause
\begin{knitrout}\scriptsize
\definecolor{shadecolor}{rgb}{0.878, 0.918, 0.933}\color{fgcolor}\begin{kframe}
\begin{alltt}
\hlkwd{summary}\hlstd{(aov.correct)}
\end{alltt}
\begin{verbatim}
## 
## Error: Plot
##           Df Sum Sq Mean Sq F value  Pr(>F)    
## Treatment  2 215.39  107.69   57.87 0.00012 ***
## Residuals  6  11.17    1.86                    
## ---
## Signif. codes:  0 '***' 0.001 '**' 0.01 '*' 0.05 '.' 0.1 ' ' 1
## 
## Error: Within
##           Df Sum Sq Mean Sq F value Pr(>F)
## Residuals 27  13.92  0.5156
\end{verbatim}
\end{kframe}
\end{knitrout}
\end{frame}






\begin{frame}[fragile]
  \frametitle{Issues}
%  \large
  {\bf When using using \inr{aov} with \inr{Error} term:}
  \begin{itemize}
    \item<1-> You can't use \inr{TukeyHSD}
    \item<1-> You don't get a direct estimate of $\sigma^2_B$
    \item<1-> Doesn't handle unbalanced designs well
    \item<1-> But, you can use \inr{model.tables} and \inr{se.contrast}
  \end{itemize}
%  \pause
  \vfill
  \uncover<2->{{\bf An alternative is to use \inr{lme} function in
      \inr{nlme} package}}
  \begin{itemize}
    \item<3-> Possible to get direct estimates of $\sigma^2_B$ and
      other variance parameters
    \item<3-> Handles very complex models and unbalanced designs
    \item<3-> Possible to do multiple comparisons and contrasts using the the
      \inr{glht} function in the \inr{multcomp} package. %(We will cover
%      this next time)
    \item<4-> But\dots
    \item<4-> Only works if there random effects
    \item<4-> ANOVA tables aren't as complete as \inr{aov}
  \end{itemize}
\end{frame}








\section{Using {\tt lme}}






\begin{frame}[fragile]
  \frametitle{Using the {\tt lme} function}
\begin{knitrout}
\definecolor{shadecolor}{rgb}{0.878, 0.918, 0.933}\color{fgcolor}\begin{kframe}
\begin{alltt}
\hlkwd{library}\hlstd{(nlme)}
\hlkwd{library}\hlstd{(multcomp)}
\hlstd{lme1} \hlkwb{<-} \hlkwd{lme}\hlstd{(larvae} \hlopt{~} \hlstd{Treatment,} \hlkwc{random}\hlstd{=}\hlopt{~}\hlnum{1}\hlopt{|}\hlstd{Plot,}
            \hlkwc{data}\hlstd{=gypsyData)}
\end{alltt}
\end{kframe}
\end{knitrout}
\pause
\vfill
\begin{knitrout}
\definecolor{shadecolor}{rgb}{0.878, 0.918, 0.933}\color{fgcolor}\begin{kframe}
\begin{alltt}
\hlkwd{anova}\hlstd{(lme1,} \hlkwc{Terms}\hlstd{=}\hlstr{"Treatment"}\hlstd{)}
\end{alltt}
\begin{verbatim}
## F-test for: Treatment 
##   numDF denDF  F-value p-value
## 1     2     6 57.86567   1e-04
\end{verbatim}
\end{kframe}
\end{knitrout}
\end{frame}






\begin{frame}[fragile]
  \frametitle{Variance parameter estimates}
  \small
  The first row shows the estimates of $\sigma^2_B$ and
  $\sigma_B$. The second row shows the estimates of $\sigma^2$ and $\sigma$
\begin{knitrout}
\definecolor{shadecolor}{rgb}{0.878, 0.918, 0.933}\color{fgcolor}\begin{kframe}
\begin{alltt}
\hlkwd{VarCorr}\hlstd{(lme1)}
\end{alltt}
\begin{verbatim}
## Plot = pdLogChol(1) 
##             Variance  StdDev   
## (Intercept) 0.3363889 0.5799904
## Residual    0.5155556 0.7180220
\end{verbatim}
\end{kframe}
\end{knitrout}
 \pause
 \vfill
 \centering
 \bf
  There is more random variation within whole units than among whole
  units (after accounting for treatment effects) \\
\end{frame}



\begin{frame}[fragile]
  \frametitle{Extract the plot-level random effects}
  These are the $\beta_{ij}$'s
\begin{knitrout}
\definecolor{shadecolor}{rgb}{0.878, 0.918, 0.933}\color{fgcolor}\begin{kframe}
\begin{alltt}
\hlkwd{round}\hlstd{(}\hlkwd{ranef}\hlstd{(lme1),} \hlnum{2}\hlstd{)}
\end{alltt}
\begin{verbatim}
##   (Intercept)
## 1        0.78
## 2       -0.48
## 3       -0.30
## 4       -0.36
## 5        0.00
## 6        0.36
## 7       -0.36
## 8        0.36
## 9        0.00
\end{verbatim}
\end{kframe}
\end{knitrout}
\end{frame}



\begin{frame}[fragile]
  \frametitle{Multiple comparisons}
\begin{knitrout}\scriptsize
\definecolor{shadecolor}{rgb}{0.878, 0.918, 0.933}\color{fgcolor}\begin{kframe}
\begin{alltt}
\hlstd{tuk} \hlkwb{<-} \hlkwd{glht}\hlstd{(lme1,} \hlkwc{linfct}\hlstd{=}\hlkwd{mcp}\hlstd{(}\hlkwc{Treatment}\hlstd{=}\hlstr{"Tukey"}\hlstd{))}
\end{alltt}
\end{kframe}
\end{knitrout}
\pause
\begin{knitrout}\scriptsize
\definecolor{shadecolor}{rgb}{0.878, 0.918, 0.933}\color{fgcolor}\begin{kframe}
\begin{alltt}
\hlkwd{summary}\hlstd{(tuk)}
\end{alltt}
\begin{verbatim}
## 
## 	 Simultaneous Tests for General Linear Hypotheses
## 
## Multiple Comparisons of Means: Tukey Contrasts
## 
## 
## Fit: lme.formula(fixed = larvae ~ Treatment, data = gypsyData, random = ~1 | 
##     Plot)
## 
## Linear Hypotheses:
##                        Estimate Std. Error z value Pr(>|z|)    
## Control - Bt == 0        4.3333     0.5569   7.781   <0.001 ***
## Dimilin - Bt == 0       -1.4167     0.5569  -2.544   0.0295 *  
## Dimilin - Control == 0  -5.7500     0.5569 -10.324   <0.001 ***
## ---
## Signif. codes:  0 '***' 0.001 '**' 0.01 '*' 0.05 '.' 0.1 ' ' 1
## (Adjusted p values reported -- single-step method)
\end{verbatim}
\end{kframe}
\end{knitrout}
\end{frame}








\begin{frame}[fragile]
  \frametitle{Assignment}
  \small
  To determine if salinity affects adult fish reproductive
  performance, a researcher places one pregnant female in a tank with
  one of three salinity levels: low, medium, and high, or a control
  tank. A week after birth, two offspring (fry) are measured. \\
  \vfill
  Run a nested ANOVA using \inr{aov} and \inr{lme} on the
  {\tt fishData.csv} dataset. Answer the following questions:
\begin{enumerate}[\bf (1)]
  \item What are the null and alternative hypotheses?
  \item Does salinity affect fry growth?
  \item If so, which salinity levels differ?
  \item Is there more random variation among or within experimental units?
\end{enumerate}

\vfill
\centering
\normalsize
Upload your self-contained \R~script to ELC at least one
day before your next lab \\
\end{frame}





\end{document}

